%%%%% lecture 6 %%%%%

\subsection*{Greensche Funktion für den Halbraum} 
\label{sec:section_name}
Definiere 
\[
	\mathbb{R}^n_+ := \set[(x_1,\dots,x_n) \in \mathbb{R}^n]{x_n > 0}
\]

\begin{definition}
	Zu $x \in \mathbb{R}^n$ heißt der Punkt $\bar{x} \in \mathbb{R}^n$, der definiert ist als 
	\[
		\bar{x} := (x_1, \dots, x_{n-1},-x_n) 
	\]
	der Spiegelungspunkt von $x$ bezüglich $\partial \mathbb{R}^n_+$. 
\end{definition}
	Für $x,y \in \mathbb{R}^n_+$ und $\Psi^*(y):= \Phi(\bar{x}-y)$ gilt
	\begin{enumerate}[(i)]
		\item $y \to \Psi^*(y)$ ist glatt weil $\bar{x} \in \mathbb{R}^n \setminus \mathbb{R}^n_+$ (also $\bar{x} \neq y$) und harmonisch
		\item Für $y \in \partial \mathbb{R}^n_+$ gilt
		\begin{align*}
			\abs{\bar{x}-y} &= \sqrt{(x_1-y_1)^2+ \dots + (x_{n-1}-y_{n-1})^2 + (-x_n)^2} \\
			&= \sqrt{(x_1-y_1)^2+ \dots + (x_{n-1}-y_{n-1})^2 + x_n^2} \\
			&= \abs{x-y}
		\end{align*}
		und es folgt für $y \in \partial \mathbb{R}^n_+$ \[
			-\Phi(\bar{x}-y)+  \Phi(x-y) = 0.
		\] 
		Es folgt, dass $\Psi^*$ eine Korrektorfunktion ist und wir definieren die Greensche Funktion als
		\[
			G_{\mathbb{R}^n_+}(x,y) := \Phi(x-y) - \Phi(\bar{x},y) \qquad \text{für alle }x,y \in \mathbb{R}^n_+, x \neq y
		\]
		Dann gilt
		\[
				G_{\mathbb{R}^n_+}(x,y) = \begin{cases}
					\frac{1}{n \omega_n} \left( -  \lg \abs{x-y} + \lg \abs{\bar{x}-y} \right), &\text{ falls }n=2\\
					\frac{1}{n(n-2)\omega_n} \left( \abs{x-y}^{2-n}- \abs{\bar{x}-y}^{2-n} \right), &\text{ falls }n \geq 3
				\end{cases}
		\]
		\[
			 \nabla_y G_{\mathbb{R}^n_+}(x,y) = - \frac{1}{n \omega_n} \left( \frac{y-x}{\abs{x-y}^n} - \frac{y - \bar{x}}{\abs{\bar{x}-y}^n} \right)
			 = - \frac{1}{n \omega_n \abs{x-y}^n} (-x + \bar{x})
		\]
		Außerdem gilt
		\[
			  \nabla_y G_{\mathbb{R}^n_+}(x,y) \cdot \underset{=-e_n}{\underbrace{\nu(y)}} = - \frac{2 x_n}{n \omega_n \abs{x-y}^n}
		\]
		Den Poissonkern für den Halbraum definieren wir für $x \in \mathbb{R}^n_+$ und $y \in  \partial \mathbb{R}^n_+$ durch
		\[
			K_{\mathbb{R}^n_+}(x,y) := -  \nabla_y G(x,y) \cdot \nu(y) = \frac{2x_n}{n \omega_n \abs{x-y}^n}
		\]
	\end{enumerate}

\begin{satz}[Poisson-Integralform für den Halbraum]
	Sei $g \in C^{\infty}(\mathbb{R}^{n-1})\cap L^{\infty}(\mathbb{R}^{n-1})$. Dann definiert für $x \in \mathbb{R}^n_+$
	\[
		u(x):= \int_{\partial \mathbb{R}^n_+}^{} K_{\mathbb{R}^n_+}(x,y)g(y) \,\mathrm{d}S(y)
	\]
	eine beschränkte harmonische Funktion $u \in C^2(\mathbb{R}^n_+) \cap C^0(\bar{\mathbb{R}^n_+})$ mit $u =g$ auf $ \partial \mathbb{R}^n_+$.
\end{satz}
\begin{beweis}
	Analog zum Beweis von Satz 2.26 bzw Evans Satz $38$, Satz $14$
\end{beweis}

\[
	\text{(P)} \begin{cases}
		- \Delta u = f, &\text{ in }\Omega\\
		u = g , & \text{ auf } \partial \Omega
	\end{cases}, \qquad \text{\underline{Energiemethode}}
\]
Sei $\Omega \in \mathbb{R}^n$ offen, $u \in L^2(\Omega)$, $ \nabla u \in L^2(\Omega; \mathbb{R}^n)$, $f \in L^2(\Omega)$. Wir definieren das Dirichlet-Funktional durch
\[
	I_f(u):= \frac{1}{2} \int_{\Omega}^{} \abs{ \nabla u}^2 \,\mathrm{d}x - \int_{\Omega}^{} f u \,\mathrm{d}x
\]
und 
\[
	A_g := \set[ w \in C^2(\Omega) \cap C^1(\bar{\Omega})]{w=g \text{ auf } \partial \Omega}
\]
\begin{satz}[Dirichlet-Prinzip]
	Sei $\Omega \subseteq \mathbb{R}^n$ eine beschränkte, offene Menge mit $C^1$-Rand, $f \in C^0(\partial \Omega)$. 
	Eine Funktion $u \in A_g$ ist ein Minimierer von $I_f$ auf $A_g$ genau dann, wenn $ - \Delta u = f$ in $\Omega$ und $u =g$ auf $\partial \Omega$.
	Dies bedeutet, dass $u$ eine Lösung von (P) ist.
\end{satz}

\begin{beweis}
	\begin{description}
		\item[$\Leftarrow $:] Wir nehmen an, dass $u$ eine Lösung von (P) ist und wir wollen zeigen, dass 
		\[
			I_f(u) \leq I_f(v) \qquad \forall\,  v \in A_g .
		\] 
		Sei $v \in A_g$ und $v = u + (v-u)$. Dann folgt mit partieller Integration
		\begin{align*}
			I_f(v) &= \frac{1}{2} \int_{\Omega}^{} \abs{ \nabla  u}^2 \,\mathrm{d}x + \frac{1}{2} \int_{ \Omega}^{} \abs{  \nabla (v-u)}^2 \,\mathrm{d}x 
			+ \int_{ \Omega}^{}  \nabla u  \nabla (v-u) \,\mathrm{d}x - \int_{\Omega}^{}fu \,\mathrm{d}x - \int_{\Omega}^{}f(v-u) \,\mathrm{d}x \\
			&= I_f(u) + \frac{1}{2} \int_{\Omega}^{} \abs{  \nabla (v-u)}^2 \,\mathrm{d}x 
			+ \int_{\partial \Omega}^{} \underset{\substack{=0, \\ \text{auf } \partial \Omega, \\\text{weil }v,u \in A_g}}{\underbrace{(v-u)}} 
			\nabla u \cdot \nu \,\mathrm{d}S - \int_{\Omega}^{}(\Delta u +f)(v-u) \,\mathrm{d}x \\
			&= I_f(u) + \frac{1}{2} \int_{\Omega}^{} \abs{  \nabla (v-u)}^2 \,\mathrm{d}x - \int_{\Omega}^{} (\underset{=0}{\underbrace{ \Delta u + f}})(v-u) \,\mathrm{d}x
		\end{align*}
		und insgesamt folgt 
		\[
			I_f(v) \geq I_f(u) \qquad \forall\, v \in A_g.
		\]
		Somit ist $u$ ein Minimierter von $I_f$.
		\item[$\Rightarrow $:] Wir nehmen an, dass $u$ ein Minimierer von $I_f$ ist also 
		\[
			I_f(u) \leq I_f(v) \qquad \forall\, v \in A_g.
		\]
		Insbesondere 
		\[
			I_f(-u) \leq I_f(u + t \phi) \qquad \forall\,  t \in \mathbb{R}, \varphi \in C^{\infty}_0(\Omega)
		\]
		und $u + t \varphi \in A_g$. \\
		Wir wollen nun zeigen, dass $u$ eine Lösung von (P) ist. Sei $F: \mathbb{R} \to \mathbb{R}$ mit
		\begin{align*}
			F(t) &:= I_f(u + t \varphi) \\ &= \frac{1}{2} \int_{\Omega}^{} \abs{  \nabla u}^2 \,\mathrm{d}x 
			+ \frac{t^2}{2} \int_{\Omega} \abs{ \nabla \varphi}^2 \,\mathrm{d}x + t \int_{\Omega}^{}  \nabla u \cdot  \nabla \varphi \,\mathrm{d}x
			- \int_{\Omega}^{} fu \,\mathrm{d}x - t \int_{\Omega}^{}f \varphi \,\mathrm{d}x.
		\end{align*}
		Es gilt somit $F \in C^1(\Omega)$ und außerdem
		\[
			F(0) = I_f(u) \leq I_f(u + t \varphi) = F(t) \qquad \forall\, t \in \mathbb{R}.
		\]
		Damit ist $t = 0$ ein Minimierer von $F$ und somit ist $F'(0)=0$. Dann erhalten wir auch
		\begin{align*}
			0 &= F'(t)  \Big|_{t=0}^{} \\ &= \diffd{}{t}F(t)  \Big|_{t=0}^{} \\ &= t \int_{\Omega}^{} \abs{ \nabla \varphi}^2 \,\mathrm{d}x
			+ \int_{\Omega}^{}  \nabla u \cdot  \nabla \varphi \,\mathrm{d}x - \int_{\Omega}^{} f \varphi \,\mathrm{d}x  \Big|_{t=0}^{} \\
			&= \int_{\Omega}^{}  \nabla u \cdot  \nabla \varphi \,\mathrm{d}x - \int_{\Omega}^{}f \varphi \,\mathrm{d}x
		\end{align*}
	und es folgt
	\[
		\int_{\Omega}^{}  \nabla u \cdot  \nabla \varphi \,\mathrm{d}x - \int_{\Omega}^{} f \varphi \,\mathrm{d}x = 0 \qquad \forall\, \varphi \in C^{\infty}_0(\Omega).
	\]
	Mit partieller Integration erhalten wir für alle $\varphi \in C^{\infty}_0(\Omega)$
	\[
		0 = - \int_{\Omega}^{}  \Delta u  \varphi\,\mathrm{d}x - \int_{\Omega}^{}f \varphi \,\mathrm{d}x = - \int_{\Omega}^{}(  \Delta u + f) \varphi \,\mathrm{d}x,
	\]
	 weil $ \Delta u + f$ stetig ist. Somit
	 \[
	 	\Delta u + f = 0 \qquad \text{in } \Omega,
	 \]
	 denn wäre zum Beispiel $ \Delta u (x_0) + f(x_0) > 0$ für ein $x_0 \in \Omega$. Dann folgt, weil $ \Delta u + f$ stetig ist, dass
	 $\Delta u + f > 0$ in $B_{\delta }(x_0)$ für ein kleines $\delta > 0$. 
	 Betrachte dann eine Testfunktion $\Psi_{\delta}$ mit $\Psi_{\delta } > 0$ passend mit kompaktem Träger auf $B_{\delta }(x_0)$ . 
	 Dann gilt aber
	 \[
	 	\int_{\Omega}^{}( \Delta u + f) \Psi_{\delta } \,\mathrm{d}x = \int_{B_{\delta }(x_0)}^{} ( \Delta u + f)\Psi_{\delta } \,\mathrm{d} > 0.
	 \]
	\end{description}
\end{beweis}

Der Beweis, dass dieser Minimierer wirklich existiert, benötigt die Theorie von Sobolevräumen und schwachen Lösungen. Dieses wird in der Vorlesung "Variationsrechnung" im Wintersemester behandelt. 
\newpage
\section{Wärmeleitungsgleichung} 
\label{sec:warmeleitungsgleichung}
Nun werden wir die Wärmeleitungsgleichung behandeln, die definiert ist durch
\[
	\partial_t u - \Delta u  = 0 \qquad \text{in }I \times \Omega
\]
mit $u = u(t,x)$, wobei $t>0$ eine Zeitkonstante ist und $x \in \Omega \subseteq \mathbb{R}^n$ eine Raumkonstante bezeichnet. Außerdem ist $I = (0,T]$ mit $T>0$ und in diesem Falle $ \Delta u = \Delta_x u$. Die Wärmeleitungsgleichung ist eine lineare, parabolische partielle differentialgleichung zweiter Orndung.

\begin{definition}
	Sei $\Omega \subseteq \mathbb{R}^n$, $T >0$.
	\begin{enumerate}[(i)]
		\item Der parabolische Zylinder $\Omega_T$ ist definiert als
		\[
			\Omega_T := (0,T] \times \Omega \subseteq \mathbb{R}^{n+1}
		\]
		\item der parabolische Rand von $\Omega_T$ (Mantel vom parabolischen Zylinder) ist definiert als
		\[
			\partial_p \Omega_T = ( \set{0} \times \Omega) \cup ( [0,T] \times \partial \Omega) \subset \partial \Omega .
		\]
		$\partial_p \Omega_T$ ist abgeschlossen und falls $T < \infty$, 
		so unterscheidet sich $\partial_p \Omega_T$ von $\partial \Omega_T$ genau um die Menge $\set{T} \times \Omega$.
	\end{enumerate}
\end{definition}

\begin{bemerkung}
	Da $t$ eine Zeitkoordinate ist, ist es sinnvoll, Randbedingungen nur auf $\partial_p \Omega_T$ zu fordern. 
	Es wäre unnatürlich Randbedingungen auf $\set{T} \times \Omega$ zu fordern. (Man fordert Anfangswerte aber keine Endwerte!)
\end{bemerkung}
Für eine Teilmenge $P \subseteq \mathbb{R}^{n+1}$ definieren wir für $i,j=1,\dots,n$ 
\[
	C^2_1(P):= \set[u \in C^1(P)]{ D_iD_j u \in C^0(P)}
\]

\subsection{Kalonische Funktionen} 
\label{sub:kalonische_funktionen}

\begin{definition}
	Sei $\Omega$ eine offene Teilmenge in $\mathbb{R}^n$, $T >0$ und $u \in C^2_1(\Omega_T)$. Man bezeichnet $u$ als kalonisch, falls $\partial_t u - \Delta u =0$ in 
	$\Omega_T$ gilt. Falls ledigleich die Ungleichung $\partial_t u - \Delta u \leq 0$ gilt, so nennt man sie subkalonisch und falls $\partial_t u - \Delta u \geq 0$
	gilt, superkalonisch.
\end{definition}

\begin{bemerkung}[Invarianzen der Lösungseigenschaften]
	Ist $u \in C_1^2(\Omega_T)$ kalonisch, so führen die folgenden Transformationen der Lösung wieder zu Lösungen der Wärmeleitungsgleichungen. 
	(auf dem transformierten parabolischen Zylinder). 
	\begin{enumerate}[(i)]
		\item Translationen in Zeit und Ort für $\eta >0$, $x_0 \in \mathbb{R}^n$.
		\[
			u_{(0,x_0)}(t,x): = u(t,x-x_0) \qquad , \qquad u_{(\eta,0)}(t,x) := u(t-\eta,x).
		\]
		\item Rotationen bezüglich der Raumvariablen für $R \in \text{SO}(n)$
		\[
			u_R(t,x) = u(t,Rx).
		\]
		\item Inhomogene Dilatation für $\lambda >0$.
		\[
			u_{\lambda}(t,x) := \lambda^n u ( \lambda^2 t, \lambda x)
		\]
		(Das ist auch wahr ohne $\lambda^n$. mit der angegebenen Skalierung bleibt das Integral
		\[
			\int_{\mathbb{R}^n}^{} u _{\lambda}(t,x) \,\mathrm{d}x = \int_{\mathbb{R}^n}^{} u(\lambda^2t,y) \,\mathrm{d}y
		\] erhalten, zum skalierten Zeitpunkt.)
	\end{enumerate}
\end{bemerkung}

\begin{beispiele}
	\begin{enumerate}[1)]
		\item Jede harmonische Funktion $u \in C^2(\Omega)$ ist (aufgefasst als Funktion auf $\Omega_T$) kalonisch auf $\Omega_T$. 
		Dies sind genau die kalonischen Funktionen, die invariant unter Translationen in der Zeit sind.
		\item Für $x \in \mathbb{R}$, also $n=1$
		\begin{align*}
			u(t,x) &= e^{\lambda^2t} ( c_1 \sinh(ax) + c_2 \cosh(ax)) \\
			u(t,x) &= e^{- \lambda^2t}(c_1 \sin(ax)+ c_2 \cos(ax)).
		\end{align*}
	\end{enumerate}
\end{beispiele}
