%%%% 23.05.2016 %%%%

Falls das Definitionsgebiet und die Funktion $u$ invariant unter der entsprechenden Operation sind, so ergeben sich Lösungen mit Symmetrien, die eine einfache
Differentialgleichung lösen.

\begin{enumerate}[(i)]
	\item Falls $u(t,x)=u(t-r,x)$ für alle $r \in \mathbb{R}$, dann $u(t,x)=v(x)$ für $v : \Omega \to \mathbb{R}$ und $ \Delta u = 0$.
	\item Falls $u(t,x)= u(t,x-x_0)$ für alle $x_0 \in \mathbb{R}^n$, dann $u(t,x)= w(t)$ mit $w_t(t)=0$. Also ist $w$ konstant.
	\item Falls $u(t,x)=u(t,kx)$ für alle $k \in \text{SO}(n)$, 
	dann $u(t,x)=v(t,\abs{x})$ für ein $v : \mathbb{R}^+ \times \mathbb{R}^+ \to \mathbb{R}$ und $v$ erfüllt für $r= \abs{x} >0$ und $t >0$
	\[
		v_t(t,r) - \underset{=\Delta u(t,r)}{\underbrace{\left( \partial^2_r v(t,r)+ \frac{n-1}{r} \partial_r v(t,r) \right)}}=0.
	\]

	\item Falls $u(t,x)= \lambda^n u(\lambda^2 t, \lambda x)$ für alle $\lambda > 0$ (mit der speziellen Wahl $\lambda = t^{- \frac{1}{2}}$), so existiert $\tilde v : \mathbb{R}^n \to \mathbb{R}$ mit
	\[
		u(t,x) = t^{- \frac{n}{2}} \tilde v ( t^{- \frac{1}{2}}x) \qquad \forall\, t>0, x \in \mathbb{R}^n.
	\]
	Also
	\[
		- \frac{n}{2} \tilde v(y) - \frac{1}{2} y \cdot  \nabla \tilde v(y)- \Delta \tilde v(y) = 0 \qquad \text{ in } \mathbb{R}^n
	\]
	Mit $\lambda = t^{- \frac{1}{2}}$ gilt
	\begin{align*}
		u(t,x) &= \lambda^n u(\lambda^2 t, \lambda x) \\
		&= t^{- \frac{n}{2}} u(t^{-1}t, t^{- \frac{1}{2}} x) \\
		&= t^{- \frac{n}{2}} u(1, t^{- \frac{1}{2}} x) \\
		&= t^{- \frac{n}{2}} \tilde v(t^{- \frac{1}{2}} x)
	\end{align*}
	\[
		u_t(t,x) = - \frac{n}{2} t^{- \frac{n}{2} - 1} \tilde v(t^{- \frac{1}{2}} x) + t^{- \frac{n}{2}}  \nabla \tilde v(t^{- \frac{1}{2}} x) \cdot x 
		\left( - \frac{1}{2} t^{- \frac{1}{2}-1} \right)
	\]
	Damit gilt
	\[
		\Delta_x u(t,x) = t^{- \frac{n}{2}} \Delta \tilde v(t^{- \frac{1}{2}} x) \cdot t^{-1}
	\]
	und somit mit $y = t^{- \frac{1}{2}} x \in \mathbb{R}^n$
	\begin{align*}
		u_t(t,x) - \Delta u(t,x) &= 0 \\
		-\frac{n}{2} t^{- \frac{n}{2}-1} \tilde v(t^{- \frac{1}{2}} x) + t^{- \frac{n}{2}-1}  
		\nabla  \tilde v(t^{- \frac{1}{2}} x) \cdot (-\frac{1}{2} x t^{- \frac{1}{2}}) - t^{- \frac{n}{2}-1}  \Delta \tilde v(t^{- \frac{1}{2}}x) &= 0 \\
		-\frac{n}{2} \tilde v(t^{- \frac{1}{2}}x)+  \nabla \tilde v(t^{- \frac{1}{2}} x) \cdot \left( - \frac{1}{2} x t^{- \frac{1}{2}} \right)- \Delta \tilde v(t^{- \frac{1}{2}}x) &=0 .
	\end{align*}
	
	\item Falls $u$ sowohl unter Rotationen im Raum als auch unter inhomogenen Dilatationen (wie in (iv)) ist, so existiert eine Funktion $w: \mathbb{R}^+ \to \mathbb{R}$ mit
	\[
		u(t,x)= t^{- \frac{n}{2}} w(t^{- \frac{1}{2}} \abs{x})
	\]
	für alle $t >0$ und $x \in \mathbb{R}^n$. $w$ erfüllt die folgende ODE:
	\[
		\underset{u_t}{\underbrace{-\frac{n}{2} w(s)- \frac{1}{2} s w'(s)}} - \underset{\Delta u}{\underbrace{\left( w''(s) + \frac{n-1}{s} w'(s) \right)}} 
		= 0 \qquad \forall\, s >0 
	\]
	mit $s = t^{- \frac{1}{2}} \abs{x}$. Multipliziere diese nun mit $s^{n-1}$ so ergibt sich
	\begin{align*}
		 \frac{n}{2} w(s) s^{n-1} + \frac{1}{2} s^n w'(s) + w''(s) s^{n-1} + (n-1) s^{n-2} w'(s) &= 0 \\
		\frac{1}{2} \left( w(s) s^n \right)' + \left( w'(s) s^{n-1} \right)' &=0  \\
		\frac{1}{2} w(s) s^n + w'(s) s^{n-1} = \text{Konstante} = 0
	\end{align*}
	und wir erhalten eine ODE
	\[
		w'(s) = -\frac{1}{2}w(s)s \qquad \Rightarrow \qquad w(s) = c e^{-\frac{s^2}{4}} \qquad c \in \mathbb{R}
	\]
	und somit
	\[
		u(t,x) = t^{- \frac{n}{2}} w(t^{- \frac{1}{2}} \abs{x}) = c t^{- \frac{n}{2}} e^{- \frac{\abs{x}}{4t}}
	\]
\end{enumerate}

\begin{definition}
	Die Funktion $\Psi : \mathbb{R}^+ \times \mathbb{R}^n \to \mathbb{R}$ definiert als 
	\[
		\Psi(t,x) = \frac{1}{(4 \pi t)^{\frac{n}{2}}} e^{-\frac{\abs{x}^2}{4t}}
	\]
	heißt die Fundamentallösung der Wärmeleitungsgleichung
\end{definition}
\begin{bemerkung}
	\item $ \Psi_t(t,x) - \Delta \Psi(t,x) = 0$ für alle $t>0$ und $x \in \mathbb{R}^n$.
	\item Für die Anfangswerte gilt
	\[
		\lim_{t \to 0} \Psi(t,0) = \infty \qquad , \qquad \lim_{t \to 0} \Psi(t,x) = 0 \qquad \forall\, x \neq 0, x \in \mathbb{R}^n
	\]
	\item \[
		\int_{\mathbb{R}^n}^{} \Psi(t,x) \,\mathrm{d}x = 1,
	\]
	denn mit Transformationssatz ($ y = \frac{x}{2 \sqrt{t}}$) gilt
	\begin{align*}
		\int_{\mathbb{R}^n}^{}\Psi(t,x) \,\mathrm{d}x &= \frac{1}{(4 \pi t)^{\frac{n}{2}}} \int_{\mathbb{R}^n}^{}e^{-\frac{\abs{x}^2}{4t}} \,\mathrm{d}x \\
		&= \frac{1}{(\pi)^{\frac{n}{2}}} \int_{\mathbb{R}^n}^{} e^{- \abs{y}^2} \,\mathrm{d}y \\
		&= (\pi)^{- \frac{n}{2}} \int_{\mathbb{R}^n}^{} e^{-(y_1^2 + \dots + y_n^2)} \,\mathrm{d}y_1 \dots \mathrm{d}y_n \\
		&= (\pi)^{- \frac{n}{2}} \left( \underset{= \sqrt{\pi}}{\underbrace{\int_{\mathbb{R}}^{} e^{y_1^2} \,\mathrm{d}y_1}} \right)^n = 1
	\end{align*}
\end{bemerkung}
Wir haben schon gesehen, dass wenn $u$ harmonisch auf $\Omega$ ist, dann gilt
\[
	u(x_0) = \fint_{B_r(x_0)}^{} u(x) \,\mathrm{d}x \qquad \forall\, B_r(x_0) \subset \subset \Omega.
\]
Wir wollen nun eine ähnliche Eigenschaft für kalonische Funktionen beweisen. Dafür definieren wir
\[
	B_r(x_0) = \set[x \in \mathbb{R}^n]{\abs{x-x_0}< r} = \set[x \in \mathbb{R}^n]{\Psi(x-x_0) > \Psi(re_1)},
\]
wobei $\Psi$ die Fundamentallösung der WLG ist. Für $n \geq 3$ gilt in dieser Menge
\[
	\frac{1}{n(n-2)\omega_n} \abs{x- x_0}^{2-n} > \frac{1}{n(n-2)\omega_n} r ^{2-n}
\]
und dies ist äquivalent zu $\abs{x-x_0}<r$.\\
$B_r(x_0)$ stellt genau die $\Psi(re_1)$- Superniveaumenge dar.

\begin{definition}[Wärmeleitungskugel]
	Sei $t_0 \in \mathbb{R}, x_0 \in \mathbb{R}^n$ und $r >0$. Wir definieren die Wärmeleitungskugel $W_r(t_0,x_0)$ an $(t_0,x_0)$ (nicht das Zentrum) als
	\[
		W_r(t_0,x_0) := \set[(t,x) \in \mathbb{R}^{n+1}]{ t<t_0 \text{ und } \Psi(t_0-t, x_0-x)> r^{-n}}
	\]
\end{definition}
\begin{bemerkung}
	\begin{enumerate}[(i)]
		\item $(t_0,x_0)$ gehört nicht zu der offenen Menge $W_t(t_0,x_0)$, sondern zu \\ $\partial W_r(t_0,x_0)$.
		\item Monotonie: Für $r < \tilde r$ gilt $ W_r(t_0,x_0) \subset W_{\tilde r}(t_0,x_0)$.
		\item Translationsverhalten: $W_r(t_0,x_0) = (t_0,x_0)+ W_r(0,0)$.
		\item Parabolische Reskalierung: $(t,x) \in W_r(0,0)$ ist äquivalent zu $( r^{-2}t, r^{-1}x) \in W_1(0,0)$.
		%% Begrüundung fehlt.
		\item Explizite Darstellung von $W_r(0,0)$:
		\[
			W_r(0,0) = \set[(t,x) \in \mathbb{R}^{n+1}]{t<0, \Psi(-t,-x)> r^{-n}}
		\] 
		\begin{align*}
			& \qquad \Psi(-t,-x) = \left( -4 \pi t \right)^{- \frac{n}{2}} e^{\frac{\abs{x}^2}{4t}} > r^{-n} \\
			\Leftrightarrow & \qquad e^{\frac{\abs{x}^2}{4t}} > (-4 \pi t)^{\frac{n}{2}} r^{-n}\\
			\Leftrightarrow & \qquad \frac{\abs{x}^2}{4t} > \lg \left( (-4 \pi t)^{\frac{n}{2}} r^{-n} \right) \\
			\Leftrightarrow & \qquad \frac{\abs{x}^2}{4t} > \lg (-4 \pi t)^{\frac{n}{2}} + \lg (r^{-n}) = \frac{n}{2} \lg (-4 \pi t)- n \lg ( r)
		\end{align*}
		Definiere nun $b_r: \mathbb{R}^- \times \mathbb{R}^n \to \mathbb{R}$ mit
		\[
			b_r(t,x):= \frac{\abs{x}^2}{4t} + n \lg r - \frac{n}{2} \lg(-4 \pi t)
		\]
		Damit
		\begin{align*}
			W_r(0,0) &= \set[(t,x) \in \mathbb{R}^{n+1}]{b_r(t,x)>0} \\
			&= \set[(t,x) \in \mathbb{R}^{n+1}]{- \frac{r^2}{4 \pi} < t < 0 , \abs{x}^2 < 2nt \lg \left( - \frac{r^2}{4 \pi t} \right)}
		\end{align*} \todo{fehler??}
		$W_r(0,0)$ ist beschränkt ($W_r(x_0,t_0)= (x_0,t_0)+ W_r(0,0)$). Außerdem gilt
		\[
			\partial W_r(0,0) = \set{(0,0)} \cup \set[(t,x) \in \mathbb{R}^{n+1}]{b_r(t,x)=0}.
		\]
		\item Gewichtetes Volumen: $W_r(0,0)$ hat bezüglich des gewichteten Maßes $\abs{x}^2 t^{-2} \, \mathrm{d}t \, \mathrm{d}x$ ein Volumen von $4 r^n$ also
		\[
			\int_{W_r(0,0)}^{} \frac{\abs{x}^2}{t^2} \,\mathrm{d}t \, \mathrm{d}x = 4 r^n
		\]
	\end{enumerate}
\end{bemerkung}

\begin{lemma}
	Sei $R >0$ und $u \in C^2_1(W_R(0,0))$. Definiert man $\psi : (0,R) \to \mathbb{R}$ mittels
	\[
		\psi(r) := \frac{1}{4 r^n} \int_{W_r(0,0)}^{} u(t,x) \frac{\abs{x}^2}{t^2} \,\mathrm{d}t \,\mathrm{d}x \qquad \text{ für } r \in (0,R)
	\]
	so gelten
	\begin{enumerate}[(i)]
		\item \[
			\lim_{r \to 0^+} \psi(r) = u(0,0)
		\]
		\item \[
			\psi'(r) = \frac{n}{r^{n+1}} \int_{W_r(0,0)}^{} (-u_t(t,x)+ \Delta u(t,x))b_r(t,x) \,\mathrm{d}t \,\mathrm{d}x,
		\] wobei $b_r$ die Funktion aus der obigen Bemerkung (v) ist.
	\end{enumerate}
\end{lemma}
\begin{beweis}
	\begin{enumerate}[(i)]
		\item Es gilt
		\begin{align*}
			\abs{ \psi(r) - u(0,0)} &= \abs{ \frac{1}{4r^n} \int_{W_r(0,0)}^{} (u(t,x)- u(0,0)) \frac{\abs{x}^2}{t^2} \,\mathrm{d}t \, \mathrm{d}x} \\
			& \leq \frac{1}{4 r^n} \int_{W_r(0,0)}^{} \abs{u(t,x)-u(0,0)} \frac{\abs{x}^2}{t^2} \,\mathrm{d}t \, \mathrm{d}x \\
			& \leq \sup_{W_r(0,0)} \abs{u(t,x)-u(0,0)} \to 0 \qquad \text{ für } r \to 0^+, \text{ da } u \text{ stetig ist.}
		\end{align*}
		\item Mit Transformation $t= r^2 s$ und $x = ry$ gilt
		\begin{align*}
			\psi(r) &= \frac{1}{4 r^n} \int_{W_1(0,0)}^{}u(r^2s,ry) \frac{r^2 \abs{y}^2}{r^4 s^2} r^2 r^n \,\mathrm{d}s \,\mathrm{d}y \\
			&= \frac{1}{4} \int_{W_1(0,0)}^{} u(r^2s,ry) \frac{\abs{y}^2}{s^2} \,\mathrm{d}s \,\mathrm{d}y.
		\end{align*}
		Wegen $u \in C^2_1(W_R(0,0))$ ist die Funktion $r \mapsto  u(r^2s,ry)$ ist differenzierbar für alle $(s,y) \in W_1(0,0)$. Außerdem sind
		$u(r^2s,ry)$ und $\diffd{}{r}u$ integrierbar über $W_1(0,0)$ und daraus folgt, dass die Ableitung von $\psi$ existiert mit
		\[
			\diffd{\psi}{r} = \frac{1}{4} \int_{W_1(0,0)}^{} \diffd{}{r} u(r^2s,ry) \frac{\abs{y}^2}{s^2} \,\mathrm{d}s \,\mathrm{d}y
		\]
		Berechne dazu nun
		\begin{align*}
			\diffd{}{r}u(r^2s,ry) = u_t(r^2s,ry)2rs+  \nabla u(r^2s,ry)\cdot y.
		\end{align*}
		und erhalte
		\begin{align*}
			\diffd{\psi}{r} &= \frac{1}{4} \int_{W_1(0,0)}^{} ( u_t(r^2s,ry)2rs +  \nabla u(r^2s,ry) \cdot y) \frac{\abs{y}^2}{s^2} \,\mathrm{d}s \,\mathrm{d}y \\
			&= \frac{1}{4} \int_{W_r(0,0)}^{} \left(u_t(t,x)2 \frac{t}{r} + 
			\nabla u(t,x) \cdot \frac{x}{r} \right) \frac{\abs{x}}{r^2} \frac{r^4}{t^2} \frac{\mathrm{d}t \, \mathrm{d}x}{r^2r^n}  \\
			&= \frac{1}{4r^{n+1}} \int_{W_r(0,0)}^{} \left( u_t(t,x)2t +  \nabla u(t,x) \cdot x \right) \frac{\abs{x}}{t^2} \,\mathrm{d}t \,\mathrm{d}x \\
			&= \frac{1}{r^{n+1}} \int_{W_r(0,0)}^{} \left( u_t(t,x) \frac{\abs{x}^2}{2t}
			+  \nabla u(t,x) \cdot \frac{\abs{x}^2 x}{4t^2} \right) \,\mathrm{d}t  \, \mathrm{d}x
		\end{align*}
		Es gilt
		\begin{align*}
			\diff{b_r}{t}(t,x) = - \frac{\abs{x}^2}{4t^2} - \frac{n}{2} \frac{1}{t} \qquad , \qquad  \nabla b_r(t,x) = \frac{x}{2t}
		\end{align*}
		Und wegen \[
			\frac{\abs{x}^2}{2t} = \frac{x \cdot x}{2t} =  \nabla b_r(t,x) \cdot x
		\]
		und 
		\[
			\frac{\abs{x}^2 x}{4t^2} = - \partial_tb_r(t,x)x -n \frac{x}{2t} = - \partial_t b_r(t,x) \cdot x - n  \nabla b_r(t,x)
		\]
		es gilt somit insgesamt 
		\begin{align*}
			\diffd{\psi}{r} &= \frac{1}{r^{n+1}} \int_{W_r(0,0)}^{} u_t(t,x)  \nabla b_r(t,x) \cdot x 
			+  \nabla u(t,x) \left( - \partial_t b_r(t,x) \cdot x - n  \nabla b_r(t,x) \right) \,\mathrm{d}t \,\mathrm{d}x \\
			&= \frac{1}{r^{n+1}} \int_{W_r(0,0)}^{}- b_r(t,x) \diver(u_t(t,x)x) 
			+  \nabla u(t,x) \left( - \partial_t b_r(t,x) \cdot x - n  \nabla b_r(t,x) \right) \,\mathrm{d}t \,\mathrm{d}x \\
			&= \frac{1}{r^{n+1}} \int_{W_r(0,0)}^{} \left( -b_r(t,x)  \nabla u_t(t,x) \cdot x - b_r(t,x)u_t(t,x)n \right)
			+ \left( \partial_t  \nabla u(t,x) x - n  \nabla u(t,x) \cdot  \nabla b_r(t,x) \right) \,\mathrm{d}t \,\mathrm{d}x \\
			&= \frac{1}{r^{n+1}} \int_{W_r(0,0)}^{} \left( - u_t(t,x)n b_r(t,x) + n \Delta u(t,x) b_r(t,x) \right) \,\mathrm{d}t \,\mathrm{d}x.
		\end{align*}
		Die Randintegrale verschwinden jeweils wegen $b_r(t,x) = 0$ auf $\partial W_r(0,0)$.
		%%%% 30.05.2016 %%%%
		
	\end{enumerate}
\end{beweis}