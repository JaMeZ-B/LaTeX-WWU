%%%%% lecture 25.04.2016

\begin{definition*}[$\varepsilon$-Glättung]
	Sei $\Omega$ eine offene Teilmenge im $\mathbb{R}^n$, $\varepsilon >0$ und 
	\[
		\Omega_{\varepsilon} := \set[x \in \Omega]{\dist(x,\partial \Omega)> \varepsilon}.
	\]
	Für $f \in L^1_{\text{loc}}(\Omega)$ definiert man die $\varepsilon$-Glättung $f_ \varepsilon$ von $f$ als die Faltung von $f$ mit dem Glättungskern $\eta_{\varepsilon}$, also
	\[
		f_{\varepsilon}(x):= \eta_{\varepsilon} * f := \int_{\Omega}^{}\eta_{\varepsilon}(x-y)f(y) \,\mathrm{d}y \qquad \text{für alle }x \in \Omega_{\varepsilon}.
	\]
\end{definition*}

\begin{bemerkung}[Eigenschaften von $\varepsilon$-Glättungen]
	\begin{enumerate}[1)]
		\item $f_{\varepsilon} \in C^{\infty}(\Omega _{\varepsilon})$ mit \\
		$D^{\alpha}f _{\varepsilon}(x)= \int_{\Omega}^{}D^{\alpha} \eta_{\varepsilon}(x-y)f(y) \,\mathrm{d}y$ für beliebige Multiindizes $\alpha \in \mathbb{N}_0^n$.
		\item $f_{\varepsilon} \to f$ fast überall in $\Omega$.
		\item $f_{\varepsilon} \to f$ in $L^1_{\text{loc}}(\Omega)$
	\end{enumerate}
\end{bemerkung}

\begin{satz}
	Sei $\Omega$ eine offene Teilmenge des $\mathbb{R}^n$ und $u \in C^0(\Omega)$ eine Funktion, die die sphärische Mittelwerteigenschaft erfüllt, d.h.
	\begin{equation}
		u(x_0) = \fint_{\partial B_r(x_0)}^{} u\,\mathrm{d}S \qquad \forall\, B_r(x_0) \subset \subset \Omega. 
	\end{equation}
	Dann gilt
	\begin{equation}
		u(x_0) = u_{\varepsilon}(x_0) \qquad \forall\, x_0 \in \Omega \text{ und } \varepsilon < \dist(x_0,\partial \Omega),
	\end{equation}
	wobei $u_{\varepsilon}$ die $\varepsilon$-Glättung bezeichnet. Insbesondere ist $u$ also von der Klasse $C^{\infty}(\Omega)$ und harmonisch auf $\Omega$.
\end{satz}
\begin{beweis}
	$x_0 \in \Omega$, $\varepsilon < \dist(x_0, \partial \Omega)$. \\
	Wegen der Radialsymmetrie von $\eta _{\varepsilon}$ und weil $\eta _{\varepsilon}(x_0-x)= \text{konstant}$ für alle $x \in \partial B_r(x_0)$ gilt
	\begin{align*}
		u_{\varepsilon}(x_0) &\stackrel{\hphantom{\substack{\text{Mittelwert-} \\ \text{Eigenschaft}}}}{=} \int_{B_{\varepsilon}(x_0)}^{}\eta_{\varepsilon}(x_0-y) u(y) 
		\,\mathrm{d}y \\
		&\stackrel{\hphantom{\substack{\text{Mittelwert-} \\ \text{Eigenschaft}}}}{=} \int_{0}^{\varepsilon} \int_{\partial B_r(x_0)}^{} \eta_{\varepsilon}(x_0-y)u(y)
		 \,\mathrm{d}S(y) \,\mathrm{d}r \\
		&\stackrel{\hphantom{\substack{\text{Mittelwert-} \\ \text{Eigenschaft}}}}{=} \int_{0}^{\varepsilon} 
		\left(\eta_{\varepsilon}\underset{=r}{\underbrace{(x_0-y)}} \cdot \int_{\partial B_r(x_0)}^{}u(y) \,\mathrm{d}S(y) \right) \,\mathrm{d}r \\
		&\stackrel{\hphantom{\substack{\text{Mittelwert-} \\ \text{Eigenschaft}}}}{=} \int_{0}^{\varepsilon} \left( \eta_{\varepsilon}(x_0-y)S_nr^{n-1} \cdot 
		\underset{=u(x_0)}{\underbrace{\fint_{\partial B_r(x_0)}^{}u(y) \,\mathrm{d}S(y)}} \right) \,\mathrm{d}r \\
		& \stackrel{\substack{\text{Mittelwert-} \\ \text{Eigenschaft}}}{=} \int_{0}^{\varepsilon}u(x_0) \cdot \int_{\partial B_r(x_0)}^{} \eta_{\varepsilon}(x_0-y)
		\,\mathrm{d}S(y) \,\mathrm{d}r \\
		&\stackrel{\hphantom{\substack{\text{Mittelwert-} \\ \text{Eigenschaft}}}}{=} u(x_0)
		\underset{=1}{\underbrace{\int_{B_{\varepsilon}(x_0)}^{} \eta_{\varepsilon}(x_0-x) \,\mathrm{d}x}} \\
		&\stackrel{\hphantom{\substack{\text{Mittelwert-} \\ \text{Eigenschaft}}}}{=} u(x_0)
	\end{align*}
\end{beweis}

\begin{bemerkung}
	Der Satz macht keine Annahme über die Randwerte von $u$, diese müssen nicht glatt sein (und können sogar unstetig sein).
\end{bemerkung}

Die Aussage von Satz 2.10 bleibt gültig, wenn die Funktion $u$ die Mittelwerteigenschaft auf Kugeln erfüllt.

\begin{korollar}
	Sei $\Omega \subseteq \mathbb{R}^n$ offen und $u \in C^0(\Omega)$ eine Funktion, die die Mittelwerteigenschaft auf Kugeln erfüllt, d.h.
	\begin{equation}
		u(x_0) = \fint_{B_r(x_0)}^{}u \,\mathrm{d}x \qquad \forall\, B_r(x_0) \subset \subset \Omega.
	\end{equation}
	Dann gilt
	\begin{equation}
		u(x_0) = u_{\varepsilon}(x_0) \qquad \forall\, x_0 \in \Omega, \varepsilon < \dist(x_0,\partial \Omega).
	\end{equation}
\end{korollar}

\begin{beweis}
	Es genügt zu zeigen, dass auch in diesem Fall die sphärische Mittelwerteigenschaft erfüllt ist. 
	Dazu definieren wir für eine beliebige Kugel $B_r(x_0) \subset \subset \Omega$ die Funktion $\psi: (0,r) \to \mathbb{R}$ mit
	\begin{equation}
		\psi(\rho) := \int_{\partial B_{\rho}(x_0)}^{} (u(x)-u(x_0)) \,\mathrm{d}S(x)
	\end{equation}
	$\psi$ ist stetig, weil $u$ stetig ist. 
	Außerdem gilt mit $x= x_0 + \rho y$ für alle $R \in (0,r)$
	\begin{equation}
		\int_{0}^{R} \psi(\rho) \,\mathrm{d}\rho = \int_{0}^{R} \int_{\partial B_1(0)}^{} (u(x_0+\rho y)-u(x_0)) \rho^{n-1} \,\mathrm{d}S \,\mathrm{d}\rho
		= \int_{B_R(x_0)}^{}(u(x)-u(x_0)) \,\mathrm{d}x = 0.
	\end{equation}
	Damit muss $\psi \equiv 0$ auf $(0,r)$ gelten und somit
	\[
		u(x_0) = \fint_{\partial B_r(x_0)}^{} u(x) \,\mathrm{d}S(x)
	\]
\end{beweis}
Eine einfache Folgerung ist nun, dass Harmonizität unter gleichmäßiger Konvergenz erhalten bleibt.

\begin{korollar}[Konvergenzsatz von Weierstraß]
	Sei $\Omega \subseteq \mathbb{R}^n$ offen und zusammenhängend. Sei $(u_k)_{k \in \mathbb{N}}$ eine Folge harmonischer Funktionen auf $\Omega$, die (lokal) gleichmäßig gegen eine Funktion $u$ konvergiert. Dann ist $u$ harmonisch auf $\Omega$.
\end{korollar}

\begin{beweis}
	Für jede Kugel $B_r(x_0) \subset \subset \Omega$ gilt 
	\[
		u(x_0)= \lim_{k \to \infty}u_k(x_0) = \lim_{k \to \infty} \fint_{B_r(x_0)}^{}u_k \,\mathrm{d}x = \fint_{B_r(x_0)}^{}u \,\mathrm{d}x.
	\]
	Dies gilt, weil $u$ stetig ist (gleichmäßiger Limes stetiger Funktionen). 
	Somit erfüllt $u$ die Mittelwerteigenschaft auf Kugeln und mit Korollar 2.11 folgt dann die Behauptung.
\end{beweis}

\begin{korollar}[Harnack'scher Konvergenzsatz]
	Sei $\Omega \subseteq \mathbb{R}^n$ offen und zusammenhängend. Sei $(u_k)_{k \in \mathbb{N}}$ eine monoton wachsende Folge harmonischer Funktionen. 
	Gibt es ein $x_0 \in \Omega$, so dass $(u_k(x_0))_{k \in \mathbb{N}}$ beschränkt (und damit konvergent) ist, so konvergiert $(u_k)_{k \in \mathbb{N}}$ auf jeder
	zusammenhängenden offenen Menge $V \subset \subset \Omega$ gleichmäßig gegen eine harmonische Funktion auf $\Omega$.
\end{korollar}

\begin{beweis}
	Wegen der Monotonie ist $u_k-u_j$ für $k > j$ eine nicht-negative harmonische Funktion. Sei $V \subset \subset \Omega$ offen und zusammenhängend. 
	Sei o.B.d.A $x_0 \in V$ (andernfalls können wir $ \tilde V$ mit $V \subset \tilde V \subset \subset \Omega$ mit $x_0 \in \tilde V$ konstruieren)
	
	
	\begin{align*}
		0 &\stackrel{\hphantom{x_0 \in V}}{\leq} \sup_V(u_k-u_j) \stackrel{\text{Harnack}}{\leq} c(V) \inf_V(u_k-u_j) \\
		& \stackrel{x_0 \in V}{\leq} c(V) (u_k(x_0)-u_j(x_0)) \\
		&\stackrel{\hphantom{x_0 \in V}}{\leq} c(V) \varepsilon
	\end{align*}
	Da $(u_k(x_0))_{k \in \mathbb{N}}$ nach Voraussetzung eine Cauchy-Folge ist folgt, dass $(u_k)$ eine Cauchy-Folge bezüglich der Supremumsnorm auf $V$ ist. 
	Damit ist sie gleichmäßig konvergent auf $V$. Nach Korollar 2.12 ist ihr Limes eine harmonische Funktion. \\
	Wähle $V = B_r(x_0) \subset \subset \Omega$, dann folgt
	\begin{equation}
		u_k(x_0) = \fint_{B_r(x_0)}^{}u_k \,\mathrm{d}x 
	\end{equation}
	und wegen gleichmäßiger Konvergenz folgt dann
	\begin{equation}
	u(x_0) = \fint_{B_r(x_0)}^{}u \,\mathrm{d}x.
	\end{equation}
	Somit ist $u$ harmonisch auf $\Omega$, weil $u$ stetig ist.
\end{beweis}

\begin{satz}[Hermann Weyl]
	Sei $\Omega \subseteq \mathbb{R}^n$ offen und $u \in L^1_{\text{loc}}(\Omega)$ eine Funktion, für die 
	\begin{equation}
		\int_{\Omega}^{} u \Delta \varphi \,\mathrm{d}x = 0 \qquad \forall\, \varphi \in C^{\infty}_0(\Omega)
	\end{equation}
	erfüllt ist. 
	Dann ist $u$ harmonisch in $\Omega$ (streng genommen: dann existiert eine Funktion $\tilde u$, die harmonisch ist mit $u = \tilde u$ fast überall in $\Omega$)
\end{satz}

\begin{beweis}
	Wir zeigen zunächst, dass die Glättungen $u_{\varepsilon}$ harmonisch in $\Omega_{\varepsilon}$ sind. Nach Übung gilt für $x \in \Omega_{\varepsilon}$
	\[
		\Delta u_{\varepsilon}(x) = ( \Delta \eta_{\varepsilon} * u)(x) = \int_{\Omega}^{} \Delta \eta_{\varepsilon}(x-y)u(y) \,\mathrm{d}y.
	\]
	Nach Voraussetzung mit $\varphi(y)= \eta_{\varepsilon}(x-y) \in C^{\infty}_0(\Omega)$ folgt 
	\[
		\Delta u_{\varepsilon}(x) \equiv 0 \qquad \text{in }\Omega_{\varepsilon}.
	\]
	Es gilt für fast alle $x_0 \in \Omega$ und alle $r < \dist(x_0, \partial \Omega)$ ($B_r(x_0) \subset \subset \Omega$) 
	\begin{equation}
		u(x_0) \stackrel{\substack{\text{fast überall}\\\text{Konvergenz}\\\text{von }u_{\varepsilon}}}{=}
		\lim_{\varepsilon \to 0} u_{\varepsilon}(x_0) 
		\stackrel{\substack{\text{Mittelwerteigenschaft} \\\text{der Familie }u_{\varepsilon}}}{=} \lim_{\varepsilon \to 0} 
		\fint_{B_r(x_0)}^{}u_{\varepsilon}(x) \,\mathrm{d}x 
		\stackrel{\substack{L^1_{\text{loc}}\text{-Konvergenz}\\ \text{von }u_{\varepsilon}}}{=}
		\fint_{B_r(x_0)}^{} u(x) \,\mathrm{d}x.
	\end{equation}
	Das heißt $u$ erfüllt für fast alle $x_0 \in \Omega$ die Mittelwerteigenschaft auf Kugeln. 
	Definiert man nun $\tilde u: \Omega \to \mathbb{R}$ durch
	\[
		\tilde u(x_0):= \fint_{B_r(x_0)}^{}u \,\mathrm{d}x
	\]
	mit $r:= \frac{\dist(x_0, \partial \Omega)}{2}$, so gilt 
	\begin{equation}
		u = \tilde u
	\end{equation}
	fast überall in $\Omega$. Wegen der Absolutstetigkeit der Integrale ist $\tilde u$ stetig und erfüllt die Mittelwerteigenschaft (auf Kugeln) überall in $\Omega$.
	Damit folgt die Harmonizität von $\tilde u$ aus Korollar 2.11.
\end{beweis}
Nun werden wir lokale Abschätzungen für höhere Ableitungen harmonischer Funktionen beweisen.

