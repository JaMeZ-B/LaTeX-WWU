%!TEX root = liegruppen.tex
Hier findet man einige ausgewählte Beweise und Ergänzungen, auf die der Vorlesung hingewiesen wurde und die teilweise im Rahmen von Übungsaufgaben bearbeitet wurden.
Sie stammen also auch hauptsächlich aus den Übungen und deswegen sei hier noch einmal darauf hingewiesen, dass für die Richtigkeit keine Garantie übernommen wird!

Die Reihenfolge richtet sich nach der Reihenfolge der Übungsaufgaben.

\section{Funktionalgleichung} % (fold)
\label{sec:funktionalgleichung}
Sei $\mathfrak{g}$ die Liealgebra einer Liegruppe.
Für $X,Y \in \mathfrak{g}$ gilt
\[
	\benbrace*{X,Y} = 0 \implies \exp(X+Y) = \exp(X) \exp(Y)
\]
\begin{beweis}
	Die von $X$ und $Y$ aufgespannte Unteralgebra von $\mathfrak{g}$ ist abelsch und somit auch die zugehörige zusammenhängende Lieuntergruppe $A$ von $G$.
	Betrachte nun
	\[
		\alpha(t) = \exp(t X) \exp(t Y)
	\]
	$\alpha \colon \mathbb{R} \to G$ ist ein Liegruppenhomomorphismus, da $A$ abelsch ist.
	Es gilt\footnote{mittels $(d m)_{(a,b)}(v,w) = (d R_b)_a \cdot v + (d L_a)_b \cdot w$} $\alpha'(0) = X(e) + Y(e)$ und es folgt
	\[
		\exp(t X) \exp(tY) = \exp t (X+Y)
	\]
	für alle $t$.
\end{beweis}
% section funktionalgleichung (end)


\section{Dichte Teilmenge des Torus} % (fold)
\label{sec:dichte_teilmenge_torus}
\emph{Der hier bewiesene \autoref{satz:dichte_tm_torus} wird im Beweis von \autoref{lem:242} benötigt.}\smallskip \\
Wir beginnen mit einem Hilfslemma, das wir für den Beweis des Satzes benötigen werden:

\begin{lemma}[label=lem:Tn_to_Tm]
	Ist $\varphi \colon T^n \to T^m$ ein Liegruppenhomomorphismus, so ist das Differential $(\mathd \varphi)_e \colon \mathbb{R}^n \to \mathbb{R}^m$ gegeben durch 
	\[
		(\mathd \varphi)_e(e_i) = \sum\nolimits_j k_{ji} \cdot e_j \quad \text{ mit } k_{ij} \in \mathbb{Z} 
	\]
	Ist umgekehrt $\psi \colon \mathbb{R}^n \to \mathbb{R}^m$ linear mit $\psi(e_i)=\sum_j k_{ji} e_j$ mit $k_{ij} \in \mathbb{Z}$, so existiert (genau ein) Liegruppenhomomorphismus $\overline{\psi} \colon T^n \to T^m$ mit $(\mathd \overline{\psi})_e = \psi$.
\end{lemma}
\begin{beweis}
	Für die Basisvektoren gilt $e_i \in \mathbb{Z}^n$.
	Da die Exponentialfunktion $\exp_{T^n} \colon \mathbb{R}^n \to T^n$ der Quotientenabbildung $v \mapsto v + \mathbb{Z}^n$ entspricht, folgt $\exp_{T^n}(e_i)=0$ und damit
	\[
		0 = \varphi \circ \exp_{T^n}(e_i) = \exp_{T^m} \enbrace[\big]{(\mathd \varphi)_e (e_i)} \implies (\mathd \varphi) e_i \in \mathbb{Z}^m
	\]
	Ist umgekehrt $\psi$ wie oben, so definieren wir $\tilde{\psi}$ als die Komposition 
	\[
		\begin{tikzcd}
			\mathbb{R}^n \rar["\psi"] & \mathbb{R}^m \rar["\exp"] & T^m
		\end{tikzcd}
	\]
	Für $x \in Z^n$ ist $\tilde{\psi}(x) = \sum_j k_{ji} x_i e_j + \mathbb{Z}^m = 0$.
	Aus dem Homomorphiesatz erhalten wir ein $\overline{\psi} \colon T^n \to T^m$ mit $\tilde{\psi} = \overline{\psi} \circ \pi = \overline{\psi} \circ {\exp}$.
	Weiter gilt
	\[
		(\mathd \overline{\psi})_e  e_i =\diffd{}{t}\Big|_{t=0} \overline{\psi} \enbrace[\big]{\exp(t e_i)} = \diffd{}{t}\Big|_{t=0} \tilde{\psi}(t e_i) = \sum\nolimits_j k_{ij} e_j \qedhere
	\]
\end{beweis}

\begin{satz}[label=satz:dichte_tm_torus]
	Sei $v \in \mathbb{R}^n$.
	Dann gilt
	\begin{enumerate}[(i)]
		\item $V \coloneqq \set*{\exp(tv) \given t \in \mathbb{R}}$ ist dicht in $T^n \iff v_1,\ldots ,v_n$ linear unabhängig über $\mathbb{Q}$.
		\item $S \coloneqq \set*{\exp(kv) \given k \in \mathbb{Z}}$ ist dicht in $T^n \iff 1, v_1, \ldots ,v_n$ linear unabhängig über $\mathbb{Q}$.
	\end{enumerate}
\end{satz}
\begin{beweis}
	\begin{enumerate}[(i)]
		\item Angenommen $V$ ist nicht dicht, das heißt $\overline{V}$ ist eine echte, abgeschlossene, zusammenhängende Untergruppe von $T^n$.
		Sei nun $\varphi$ die Komposition
		\[
			\begin{tikzcd}
				T^n \rar[two heads] & \sfrac{T^n}{\overline{V}} \cong T^k \rar["\pr_1"] & T^1
			\end{tikzcd}
		\]
		Dies ist offenbar ein Liegruppenhomomorphismus mit $V \subseteq \ker \varphi$.
		Es ist $v \in \ker \mathd \varphi$ und es gilt nach \autoref{lem:Tn_to_Tm}
		\[
			0 = \mathd \varphi(v) = \sum\nolimits_i k_{1i} v_i \quad  \text{ mit } k_{1i} \in \mathbb{Z} 
		\]
		Nach Konstruktion können nicht alle $k_{1i}$ Null sein und somit folgt, dass $v_1, \ldots ,v_n$ $\mathbb{Q}$-linear abhängig sein muss.
		
		Seien nun $v_1, \ldots ,v_n$ linear abhängig über $\mathbb{Q}$, also linear abhängig über $\mathbb{Z}$. 
		Damit existieren $k_i \in \mathbb{Z}$ mit $\sum_i v_i k_i =0$ aber nicht alle $0$.
		Sei $\psi \colon T^n \to T^1$ der Liegruppenhomomorphismus aus \autoref{lem:Tn_to_Tm} mit $(\mathd \psi)(x) = \sum k_i x_i$ für $x \in \mathbb{R}^n$.
		Dann gilt $v \in \ker (\mathd \psi)$ und somit $V \subseteq \ker \psi$.
		Nun folgt $\overline{V} \subseteq \ker \psi \neq T^n$.
		\item Angenommen $S$ ist \emph{nicht} dicht.
		Dann ist wieder
		\(
			\begin{tikzcd}[sep=small,cramped]
				\varphi \colon T^n \rar[two heads] & \sfrac{T^n}{\overline{S}} \cong T^k  \rar[two heads] & T^1
			\end{tikzcd}
		\)
		ein Liegruppenhomomorphismus mit $S \subseteq \ker \varphi$.
		Es gilt
		\[
			1 = \varphi \enbrace[\big]{\exp(v)} = \exp \enbrace[\big]{(\mathd \varphi) v} \implies (\mathd \varphi) v = \sum\nolimits_i k_i v_i \in \mathbb{Z}\marginnote{nicht alle $k_i=0$}
		\]
		Damit muss $1, v_1, \ldots ,v_n$ über $\mathbb{Z}$ bzw. $\mathbb{Q}$ linear abhängig sein.
		
		Sind umgekehrt $1, v_1, \ldots v_n$ linear abhängig über $\mathbb{Z}$, so finden wir $k_i \in \mathbb{Z}$ nicht alle $0$ mit $\sum_i k_i v_i \in \mathbb{Z}$.
		Sei nun wieder $\psi \colon T^n \to T^1$ der Liegruppenhomomorphismus nach \autoref{lem:Tn_to_Tm} mit Differential $(\mathd \psi) (x) = \sum_i k_i x_i$ für $x \in \mathbb{R}^n$.
		Dann gilt
		\[
			\psi \enbrace*{\exp(v)} = \exp \enbrace*{(\mathd \psi) v} = 0 \implies \exp(v) \in \ker \psi \implies S \subseteq \ker \psi
		\]
		Damit muss auch wieder $\overline{S} \subseteq \ker \psi \neq T^n$ gelten.\qedhere
	\end{enumerate}
\end{beweis}
% section dichte_teilmenge_des_torus (end)

\section{Aufgabe 3 -- Über Quaternionen} % (fold)
\label{sec:aufg3}
Wir bezeichnen mit $\mathbb{H} \coloneqq \set*{q = \alpha \cdot 1 + \beta \cdot i + \gamma \cdot j + \delta \cdot k \given \alpha, \beta, \gamma, \delta \in \mathbb{R}}$ den Schiefkörper der \Index{Quaternionen}.\footnote{Zur Erinnerung: $i^2 = j^2 = k^2 = ijk = -1$}
Seien weiter
\[
	\mathbb{R}^3 \coloneqq \Im \mathbb{H} \coloneqq \set[\big]{q = \beta \cdot i + \gamma \cdot j + \delta \cdot k \given \beta,\gamma,\delta \in \mathbb{R}}
\]
die rein imaginären Quaternionen, das heißt $q \in \mathbb{R}^3$ genau dann, wenn $-q = \overline{q}$.
Ferner bezeichnen wir mit $S^3 \coloneqq \set*{q \in \mathbb{H} \given \norm*{q}=1}$ die \Index{Einheitsquaternionen}.
Es gilt
\begin{enumerate}[(1)]
	\item Ist $q \in S^3$, so sind die Linksmultiplikation $L_q \colon \mathbb{H} \to \mathbb{H}$, $p \mapsto q \cdot p$ und die Rechtsmultiplikation $R_q \colon \mathbb{H} \to \mathbb{H}$, $p \mapsto p \cdot q$ Isometrien von $\mathbb{H}=\mathbb{R}^4$ versehen mit dem Standardskalarprodukt.
	\item Für $q \in S^3$ ist $\Ad(q) \colon \mathbb{H} \to \mathbb{H}$, $p \mapsto q \cdot p \cdot \overline{q}$ eine Isometrie von $\mathbb{H}$, die $\mathbb{R}^3$ invariant lässt.
	\item $\SO(3) = \set*{\Ad(q)|_{\mathbb{R}^3} \given q \in S^3}$ und $S^3$ ist eine zweifache Überlagerung von $\SO(3)$.
	\item $\SO(4) = \set*{L_q \circ R_{\overline{q}} \given q,\overline{q} \in S^3}$ und $S^3 \times S^3$ ist eine zweifache Überlagerung von $\SO(4)$.
\end{enumerate}
\begin{beweis}
	Wir erinnern uns zunächst, dass das Skalarprodukt auf $\mathbb{H}$ dem euklidischen auf $\mathbb{R}^4$ entspricht und weiter gilt $\skal*{x}{y} = \sfrac{1}{2} (x \overline{y} + y\overline{x})$.
	Weiter gilt für $x,y \in \mathbb{R}^3$ 
	\[
		x \perp y \implies xy= -yx
	\]
	wegen $0 = \skal*{x}{y} = \sfrac{1}{2} \enbrace[\big]{x(-y) + y(-x)} = - \sfrac{1}{2} \enbrace*{xy-yx}$.
	Außerdem sollte noch erwähnt werden, dass das Zentrum der Quaternionen genau die reellen Zahlen sind, also $Z(\mathbb{H}) = \mathbb{R}$.
	
	Sei $v = \lambda + w \in Z(\mathbb{H})$ mit $\lambda \in \mathbb{R}$ und $w \in \mathbb{R}^3$ und $p \in \mathbb{R}^3$ mit $p\neq 0$ und $p \perp w$.
	Dann gilt
	\[
		pv = vp = (\lambda + w )p = p \lambda - p w = p (\lambda-w) = p \overline{v} \implies v=\overline{v} \implies v \in \mathbb{R}
	\]
	Kommen wir nun zu den eigentlichen Aufgaben:
	\begin{enumerate}[(1)]
		\item Für $p,q \in \mathbb{H}$ gilt
		\[
			\norm*{pq}^2 = pq \overline{pq} = p q \overline{q} \overline{p} = \norm*{q}^2\, p \overline{p} = \norm*{q}^2\, \norm*{p}^2
		\]
		Also folgt die Behauptung für $q \in S^3$ direkt.
		\item Nach (1) gilt $\Ad(q) = L_q \circ R_{\overline{q}} \in \On(\mathbb{H})$ für $q \in S^3$.
		Damit erhalten wir für $\lambda \in \mathbb{R}$
		\[
			\Ad(q)(\lambda) = q \lambda \overline{q} = \lambda \norm*{q}^2 = \lambda 
		\]
		Also ist $\mathbb{R}$ $\Ad(q)$-invariant und damit auch das orthogonale Komplement $\mathbb{R}^\bot = \mathbb{R}^3$.
		\item Aus Zusammenhangsgründen muss das Bild von $\Ad(.)|_{\mathbb{R}^3} \colon S^3 \to \On(3)$ in $\SO(3)$ enthalten sein.
		Wir berechnen nun das Differential von $\Ad \colon \mathbb{H} \to \End(\mathbb{R}^3)$.
		\[
			(\mathd \Ad)_1 (q)= \diffd{}{t}\Big|_{t=1} (1 -tq) v \overline{(1- tq)} = \diffd{}{t}\Big|_{t=1} \enbrace*{v + t(qv+v \overline{q}) + t^2 q v \overline{q}} = qv + v \overline{q}
		\]
		Es gilt nun $\ker (\mathd \Ad) =0 $, denn falls $q \in \ker (\mathd {\Ad})$ gilt für $v \in \mathbb{R}^3$
		\[
			0= qv + v \overline{q} = qv - \overline{qv}  \implies q=0
		\]
		Wegen $\dim \Tmap_1 S^3 = 3 = \dim \Tmap_{\id} \SO(3)$, folgt, dass $(\mathd \Ad)_1$ ein Isomorphismus ist.
		Nach \autoref{lem:132} (2) ist damit $\Ad \colon S^3 \to \SO(3)$ eine Überlagerung.
		Wir zeigen nun noch $\ker {\Ad} = \set*{\pm 1}$, um schließen zu können, dass die Überlagerung zweiblättrig ist.
		
		Die Relation \enquote{$\supseteq$} ist klar.
		Sei andersrum $q \in \ker {\Ad} \subseteq S^3$.
		Dann ist $\Ad(q) \colon \mathbb{H} \to \mathbb{H}$ die Identität, also gilt $qv \overline{q} = v$ für alle $v \in \mathbb{H}$.
		Damit folgt
		\[
			q v = v q \implies q \in Z(\mathbb{H}) = \mathbb{R} \implies q \in S^3 \cap \mathbb{R} = \set*{\pm 1}
		\]
		\item Betrachten den Liegruppenhomomorphismus
		\mapdef{\varphi \colon S^3 \times S^3}{\SO(\mathbb{H})}{(p,q)}{L_p \circ R_{\overline{q}}}{}
		Sei $(p,q) \in \ker \varphi$, das heißt es gelte $p v \overline{q}=v$ für alle $v \in \mathbb{H}$.
		Für $v =1$ folgt $p \overline{q}=1$ und damit $p=q$; wie eben folgt $p =\pm 1$ und damit $\ker \varphi = \set*{\pm (1,1)}$.
		Es folgt wieder $\ker (\mathd \varphi)_{(1,1)}=0$ und $\varphi$ ist damit wieder eine Überlagerung.\qedhere
	\end{enumerate}
\end{beweis}
% section aufg3 (end)

\section{Aufgabe 8 -- Lieuntergruppen von $T^2$} % (fold)
\label{sec:aufg8}
Betrachte die Liegruppe $T^2= S^1 \times S^1$ und $S^1_r = \set*{(e^{i \varphi}, e^{i r \varphi}) \given \varphi \in \mathbb{R}} \subseteq T^2$.
Es gilt dann
\begin{enumerate}[a)]
	\item $S^1_r \subset T^2$ ist eine Lieuntergruppe
	\item Für $r \in \mathbb{Q}$ ist $S^1_r$ kompakt
	\item Für $r \in \mathbb{R} \setminus \mathbb{Q}$ ist $S^1_r$ dicht in $T^2$.
\end{enumerate}
\begin{beweis}
	\begin{enumerate}[a)]
		\item Wir betrachten 
		\mapdef{\exp \colon \Tmap_{(1,1)}T^2 = \mathbb{R}^2}{T^2}{(x,y)}{\enbrace*{e^{ix},e^{iy}}}{}
		Dies ist eine Überlagerung und ein Liegruppenhomomorphismus, da $T^2$ eine zusammenhängende abelsche Liegruppe ist.
		Da $\mathbb{R} \cdot (1,r) \subset \mathbb{R}^2$ eine Untergruppe ist, muss also auch $S^1_r = \exp \enbrace*{\mathbb{R} \cdot (1,r)}$ eine Lieuntergruppe von $T^2$ sein.
		Anders ausgedrückt ist $S^1_r$ das Bild des Liegruppenhomomorphismus $f_r \colon \mathbb{R}\to T^2$, $\varphi \mapsto \enbrace*{e^{i \varphi}, e^{ir \varphi}}$.
		\item Sei nun $r \in \mathbb{Q}$.
		Dann gilt
		\[
			\ker f_r = \set[\big]{\varphi \in \mathbb{R} \given \varphi \in 2 \pi \mathbb{Z}, r \cdot \varphi \in 2 \pi \mathbb{Z}}
		\]
		Wir zeigen nun, dass $\ker f_r \neq \set*{0} \iff r \in \mathbb{Q}$ gilt.
		Für die erste Implikation sei dazu $2 \pi a \in \ker f_r$ mit $a \in \mathbb{R}\setminus \set*{0}$.
		Dann muss $a \in \mathbb{Z}$ und $ra \in \mathbb{Z}$ gelten.
		Daraus folgt $r \in \mathbb{Q}$.
		
		Sei umgekehrt $r= \sfrac{m}{n}$ mit $m,n \in \mathbb{Z}$, $n \neq 0$.
		Für $\varphi \coloneqq 2 \pi n \neq 0$ gilt $\varphi \in 2 \pi \mathbb{Z}$ und $r \varphi = \sfrac{m}{n}(2 \pi n) = 2 \pi m \in 2 \pi \mathbb{Z}$.
		Also ist $\varphi \in  \ker f_r$.
		
		Es gilt nun $S^1_r \cong \sfrac{\mathbb{R}}{\ker f_r}$.
		Ist $r \in \mathbb{Q}$, so ist $\ker f_r \neq \set*{0}$ und somit $S_r^1$ kompakt.
		\item Sei nun $r$ irrational.
		Nach der Überlegung aus b) ist $f_r$ in diesem Fall injektiv, das heißt es gilt $S_r^1 \cong \mathbb{R}$, insbesondere ist $S_r^1$ nicht kompakt.
		Damit kann $S_r^1$ auch nicht abgeschlossen in $T^2$ sein und damit nach \autoref{satz:1210} keine eingebettete Lieuntergruppe.
		Da $T^2$ eine zusammenhängende Liegruppe ist, gilt stehen die zusammenhängenden Lieuntergruppen von $T^2$ in 1:1-Beziehung mit den Lieunteralgebren von $\mathbb{R}^2$ nach \autoref{satz:125}.
		Da nun der Abschluss $\overline{S_r^1}$ auch eine Lieuntergruppe ist, muss wegen dieser Bijektion aus $S_r^1 \subsetneq \overline{S_r^1}$ auch $\Tmap_e S_r^1 \subsetneq \Tmap_e \overline{S_r^1}$ gelten.
		Damit muss aber bereits $T_e \overline{S_r^1} = \mathbb{R}^2$, also auch $\overline{S_r^1} = T^2$ gelten.\qedhere
	\end{enumerate}
\end{beweis}
% subsection aufg8 (end)

\section{Aufgabe 16 -- Killingform einer Unteralgebra} % (fold)
\label{sec:einschr_killing}
Sei $\mathfrak{g}$ eine Liealgebra und $\mathfrak{h} \subset \mathfrak{g}$ eine Unteralgebra.
Dann entspricht die Killingform von $\mathfrak{h}$ nicht notwendigerweise der Einschränkung von $\mathcal{B}_{\mathfrak{g}}$: Betrachte dazu $\mathfrak{g} = \lenbrace{x,y}$ mit Lieklammer $\benbrace*{x,y} = y$ und die Lieunteralgebra $\mathfrak{h} = \lenbrace{x}$.  
Dann ist $\mathcal{B}_{\mathfrak{h}}(x,x) = 0$, denn $\ad(x)_{\mathfrak{h}} =0$.
Andererseits ist aber $\mathcal{B}_{\mathfrak{g}}(x,x) = 1$, denn $\ad(x)_\mathfrak{g} = \begin{psmallmatrix} 0 & 0 \\ 0 & 1 \end{psmallmatrix}$.

Nimmt man zusätzlich an, dass $\mathfrak{h}$ ein Ideal ist, so stimmen die beiden Killingformen aber überein:
\begin{lemma}
	Sei $\mathfrak{g}$ eine endlichdimensionale Liealgebra und $\mathfrak{h} \subset \mathfrak{g}$ ein Ideal.
	Dann gilt $\mathcal{B}_\mathfrak{g}\big|_{\mathfrak{h}} = \mathcal{B}_\mathfrak{h}$.
\end{lemma}
\begin{beweis}
	Wähle eine Basis von $\mathfrak{h}$ und erweitere diese zu einer Basis von $\mathfrak{g}$.
	Für $x \in \mathfrak{h}$ gilt dann
	\[
		\ad(x)_\mathfrak{g} = \begin{pmatrix}
			\ad(x)_\mathfrak{h} & * \\ 0 & 0
		\end{pmatrix}
	\]
	da $\mathfrak{h}$ ein Ideal ist.
	Damit ist die Matrix von $\ad(x)_\mathfrak{g} \circ \ad(y)_\mathfrak{g}$ für $x,y \in \mathfrak{h}$
	\[
		\begin{pmatrix}
			\ad(x)_\mathfrak{h} \circ \ad(y)_\mathfrak{h} & * \\ 0 & 0
		\end{pmatrix}
	\]
	Damit folgt die Behauptung.
\end{beweis}
% section einschrankung_der_killingform (end)

\section{Aufgabe 17 -- Universelle Überlagerung der speziellen linearen Gruppe} % (fold)
\label{sec:aufg17}

Sei $\pi \colon \widetilde{\SL(n,\mathbb{R})} \to \SL(n,\mathbb{R})$ die universelle Überlagerung.
$\widetilde{\SL(n,\mathbb{R})}$ ist \emph{keine} Matrixliegruppe, das heißt es existiert kein injektiver Homomorphismus $\widetilde{\SL(n,\mathbb{R})} \to \GL(N,\mathbb{C})$ für alle $N \in \mathbb{N}$.
\begin{beweis}
	Angenommen ein solcher injektiver Homomorphismus $\varphi \colon \widetilde{\SL(n,\mathbb{R})} \hookrightarrow \GL(N,\mathbb{C})$ existiert doch.
	Dann ist $f \coloneqq (\mathd \varphi)_e \colon \mathfrak{sl}(n,\mathbb{R}) \hookrightarrow \mathfrak{gl}(n,\mathbb{C})$ ein reeller Liealgebrenhomomorphismus.
	
	Betrachtet man allgemein einen reellen Liealgebrenhomomorphismus $f \colon \mathfrak{g} \to \mathfrak{h}_{\mathbb{R}}$, wobei $\mathfrak{h}$ eine komplexe Liealgebra ist, so erhält man einen $\mathbb{C}$-Liealgebrenhomomorphismus durch
	\mapdef{\tilde{f} \colon \mathfrak{g}_{\mathbb{C}}= \mathfrak{g} \otimes \mathbb{C}}{\mathfrak{h}}{x \otimes \lambda}{\lambda \cdot f(x)}{}
	Weitergehend überlegt man sich, dass durch $f \mapsto \tilde{f}$ und $f' \mapsto  f' \circ \iota$ mit $\iota\colon \mathfrak{g} \to \mathfrak{g}_{\mathbb{C}}$, $x \mapsto x \otimes 1$ eine Bijektion zwischen den reellen Liealgebrenhomomorphismen $\mathfrak{g} \to \mathfrak{h}_{\mathbb{R}}$ und den komplexen Liealgebrenhomomorphismen $\mathfrak{g}_{\mathbb{C}} \to \mathfrak{h}$ definiert wird.
	
	In unserem Fall erhalten wir also $\tilde{f} \colon \mathfrak{sl}(n,\mathbb{R})_{\mathbb{C}} \cong \mathfrak{sl}(n,\mathbb{C}) \to \mathfrak{gl}(N,\mathbb{C})$.
	Da $\mathfrak{sl}(n,\mathbb{C})$ einfach ist\footnote{$\mathfrak{sl}(n,\mathbb{C})$ besteht aus den Matrizen mit Spur $0$ und hat somit als Basis die Matrizen $e_{ij}$ mit $i\neq j$ und $e_{ii} - e_{nn}$ für $i=1,\ldots ,n-1$. Man kann nun nachrechnen, dass ein Ideal, dass ein $X\neq 0$ enthält, bereits alle diese Basisvektoren enthalten muss.}, gilt $\ker \tilde{f}=0$ oder $\ker \tilde{f} = \mathfrak{sl}(n,\mathbb{C})$.
	Letzteres ist nicht möglich, da $\tilde{f} \circ \iota = f$ ist und  $f$ injektiv war; folglich ist auch $\tilde{f}$ injektiv.
	Da $\SL(n,\mathbb{C})$ einfach zusammenhängend ist,\footnote{Die Fundamentalgruppe von $\Un(n)$ ist $\mathbb{Z}$ für alle $n$. Dies sieht man induktiv mittels der Faserung $\Un(n-1) \to \Un(n) \to \sfrac{\Un(n)}{\Un(n-1)} = S^{2n-1}$ und da $\Un(1)=S^1$. Damit ist auch $\pi_1(\GL(n,\mathbb{C}))=\mathbb{Z}$ (Deformationsretrakt). Für $\SL(n,\mathbb{C})$ betrachte die Faserung $\SL(n,\mathbb{C}) \to \GL(n,\mathbb{C}) \to \sfrac{\GL(n,\mathbb{C})}{\SL(n,\mathbb{C})}$.} existiert nach \autoref{lem:134} ein eindeutiger Liegruppenhomomorphismus $\psi \colon \SL(n,\mathbb{C}) \to \GL(N,\mathbb{C})$ mit $(\mathd \psi)_{E_n} = \tilde{f}$.
	Betrachte nun das Diagramm
	\[
		\begin{tikzcd}
			\widetilde{\SL(n,\mathbb{R})} \rar["\varphi"] \dar["\pi"] & \GL(N,\mathbb{C}) \\
			\SL(n,\mathbb{R}) \rar[hook,"\iota"] & \SL(n,\mathbb{C}) \uar["\psi"']
		\end{tikzcd}
	\]
	Da $\widetilde{\SL(n,\mathbb{R})}$ (einfach) zusammenhängend ist, genügt es, dies für die Differentiale nachzuweisen:\marginnote{$(\mathd \pi)_{E_n}$ ist Isomorphismus}
	\[
		(\mathd \varphi)_{E_n} = \mathd \enbrace*{\psi \circ \iota \circ \pi}_{E_n}
	\]
	Dies gilt bereits nach Konstruktion.
	Da nun $\pi$ aber nicht injektiv ist, kann auch auch $\varphi$ nicht injektiv sein im Widerspruch zur Annahme.
\end{beweis}
% section aufg17 (end)

\section{Aufgaben 18 \& 19 -- Killingform und Nilpotenz} % (fold)
\label{sec:aufg18_19}
In Aufgabe 18 war lediglich folgendes einfaches Lemma zu beweisen:
\begin{lemma}
	Ist $\mathfrak{g}$ eine nilpotente Liealgebra, so ist die Killingform identisch $0$.\marginnote{nilpotent $\Rightarrow \mathcal{B}\equiv 0 \Rightarrow$ auflösbar (Cartan)}
\end{lemma}
\begin{beweis}
	Da $\mathfrak{g}$ nilpotent ist, existiert ein $n \in \mathbb{N}$ mit $\mathfrak{g}^n =0$.
	Damit gilt für alle $X_1,\ldots ,X_n \in \mathfrak{g}$
	\[
		{\ad_{X_1}} \circ \ldots \circ {\ad_{X_n}} = 0
	\]
	Insbesondere gilt für $X,Y \in \mathfrak{g}$ damit $({\ad_X} \circ {\ad_Y})^n=0$, also ist $\ad_X \circ \ad_Y$ nilpotent.
	Dann muss aber die Spur $0$ sein, womit $\mathcal{B} \equiv 0$ folgt.
\end{beweis}

Wir geben nun eine reelle Liealgebra an, deren Killingform auch verschwindet, dabei aber nur auflösbar und nicht nilpotent ist.
Sei dazu $\mathfrak{g} = \Span_{\mathbb{R}} \set*{X,Y,Z} \cong \mathbb{R}^3$ und $\lambda \in \mathbb{R}\setminus \set*{0}$.
Wir definieren die Lieklammer wie folgt
\[
	\benbrace*{X,Y} \coloneqq Y- \lambda Z  \qquad \benbrace*{X,Z} \coloneqq \sfrac{1}{\lambda}Y + Z \qquad \benbrace*{Y,Z} \coloneqq 0
\]
\begin{lemma}
	Es gilt
	\begin{enumerate}[(i)]
		\item $\mathfrak{g}$ ist eine reelle, auflösbare Liealgebra, die nicht nilpotent ist.
		\item Die Killingform von $(\mathfrak{g}, \benbrace*{\cdot,\cdot })$ ist identisch null.
	\end{enumerate}
\end{lemma}
\begin{beweis}
	Zu (i): $\benbrace*{\cdot,\cdot }$ ist offenbar eine schiefsymmetrische Bilinearform.
	Wir müssen also lediglich die Jacobi-Identität nachrechnen
	\[
		\benbrace[\big]{\benbrace*{X,Y},Z} + \benbrace[\big]{\benbrace*{Z,X},Y} + \benbrace[\big]{\benbrace*{Y,Z},X} = \benbrace[\big]{Y- \lambda Z,Z} + \benbrace[\big]{- \sfrac{1}{\lambda} Y - Z , Y} + \benbrace[\big]{0,X} 
		= 0 + 0 +0 = 0
	\]
	Weiter gilt $\mathfrak{g}_1 = \benbrace*{\mathfrak{g},\mathfrak{g}} = \Span_{\mathbb{R}}\set*{Y- \lambda Z,\sfrac{1}{\lambda}Y + Z} = \Span_{\mathbb{R}}\set{Y,Z}$.
	Da $\benbrace*{Y,Z}=0$ folgt daraus $\mathfrak{g}_2=0$ und somit ist $\mathfrak{g}$ auflösbar.
	Es gilt $\mathfrak{g}_1 = \mathfrak{g}^1$ und damit
	\begin{align}
		\mathfrak{g}^2 = \benbrace*{\mathfrak{g}, \mathfrak{g}^1} = \Span_{\mathbb{R}} \set*{\benbrace*{X,\mathfrak{g}^1}, \benbrace*{Y,\mathfrak{g}^1}, \benbrace*{Z,\mathfrak{g}^1}} = \Span_{\mathbb{R}} \set[\big]{\benbrace*{X,Y}, \benbrace*{X,Z}} = \Span_{\mathbb{R}} \set*{Y,Z} = \mathfrak{g}^1
	\end{align}
	Folglich ist $\mathfrak{g}$ nicht nilpotent.
	
	\todo[inline]{fertig machen}
\end{beweis}
% section aufg18 (end)












