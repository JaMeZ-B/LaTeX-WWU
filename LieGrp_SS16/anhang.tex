%!TEX root = liegruppen.tex
Hier findet man einige ausgewählte Beweise und Ergänzungen, auf die der Vorlesung hingewiesen wurde und die teilweise im Rahmen von Übungsaufgaben bearbeitet wurden.
Sie stammen also auch hauptsächlich aus den Übungen und deswegen sei hier noch einmal darauf hingewiesen, dass für die Richtigkeit keine Garantie übernommen wird!

Die Reihenfolge richtet sich nach der Reihenfolge der Übungsaufgaben.


\section{Aufgabe 3 -- Über Quaternionen} % (fold)
\label{sec:aufg3}
Wir bezeichnen mit $\mathbb{H} \coloneqq \set*{q = \alpha \cdot 1 + \beta \cdot i + \gamma \cdot j + \delta k \given \alpha, \beta, \gamma, \delta \in \mathbb{R}}$ den Schiefkörper der \Index{Quaternionen}.
Seien weiter
\[
	\mathbb{R}^3 \coloneqq \Im \mathbb{H} \coloneqq \set[\big]{q = \beta \cdot i + \gamma \cdot j + \delta \cdot k \given \beta,\gamma,\delta \in \mathbb{R}}
\]
die rein imaginären Quaternionen, das heißt $q \notin \mathbb{R}^3$ genau dann, wenn $q = \overline{q}$.
Ferner bezeichnen wir mit $S^3 \coloneqq \set*{q \in \mathbb{H} \given \norm*{q}=1}$ die \Index{Einheitsquaternionen}.
Es gilt
\begin{enumerate}[(1)]
	\item Ist $q \in S^3$, so sind die Linksmultiplikation $L_q \colon \mathbb{H} \to \mathbb{H}$, $p \mapsto q \cdot p$ und die Rechtsmultiplikation $R_q \colon \mathbb{H} \to \mathbb{H}$, $p \mapsto p \cdot q$ Isometrien von $\mathbb{H}=\mathbb{R}^4$ versehen mit dem Standardskalarprodukt.
	\item Für $q \in S^3$ ist $\Ad(q) \colon \mathbb{H} \to \mathbb{H}$, $p \mapsto q \cdot p \cdot \overline{q}$ eine Isometrie von $\mathbb{H}$, die $\mathbb{R}^3$ invariant lässt.
	\item $\SO(3) = \set*{\Ad(q)|_{\mathbb{R}^3} \given q \in S^3}$ und $S^3$ ist eine zweifache Überlagerung von $\SO(3)$.
	\item $\SO(4) = \set*{L_q \circ R_{\overline{q}} \given q,\overline{q} \in S^3}$ und $S^3 \times S^3$ ist eine zweifache Überlagerung von $\SO(4)$.
\end{enumerate}
\begin{beweis}
	\todo{fertig machen}
\end{beweis}
% section aufg3 (end)

\section{Aufgabe 8 -- Lieuntergruppen von $T^2$} % (fold)
\label{sec:aufg8}
Betrachte $T^2= S^1 \times S^1$ und $S^1_r = \set*{(e^{i \varphi}, e^{i r \varphi}) \given \varphi \in \mathbb{R}} \subseteq T^2$.
Es gilt dann
\begin{enumerate}[a)]
	\item $S^1_r \subset T^2$ ist eine Lieuntergruppe
	\item Für $r \in \mathbb{Q}$ ist $S^1_r$ kompakt
	\item Für $r \in \mathbb{R} \setminus \mathbb{Q}$ ist $S^1_r$ dicht in $T^2$.
\end{enumerate}
\begin{beweis}
	\todo{fertig machen}
\end{beweis}
% subsection aufg8 (end)

\section{Aufgabe 17 -- Universelle Überlagerung der speziellen linearen Gruppe} % (fold)
\label{sec:aufg17}
\todo{fertig machen}
% section aufg17 (end)












