%!TEX root = ../EZT_WS1415.tex
% Author: Phil Steinhorst, p.st@wwu.de
\section{Quadratische Reste}
\label{sec:para6}

\minisec{Vorbemerkungen}
	Sei $m \in \NN$, $m > 1$. Wir untersuchen Kongruenzen über $\ZZ$ der Gestalt \marginnote{05.12. \\ \ [16]}
	\begin{align}
		& aX^2 + bX + c \kon 0 \modu m, \quad a \neq 0 \label{eq_6.1} \\ 
		\Leftrightarrow \quad & 4a^2X^2 + 4abX + 4ac \kon 0 \modu 4am \\ 
		\Leftrightarrow \quad & (2aX + b)^2 \kon b^2 - 4ac \modu 4am \label{eq_6.2} \\ 
		\Leftrightarrow \quad & \begin{cases}
			Y^2 \kon D := b^2 - 4ac \modu 4am \\
			Y \kon b \modu 2a
		\end{cases} \label{eq_6.3}
	\end{align}

\minisec{Bemerkung}
	\begin{enumerate}[1)]
		\item Für $(a,m) = 1$: \eqref{eq_6.1} ist äquivalent zu $X^2 + \frac{b}{a}X + \frac{c}{a} \kon 0 \modu m$.
		\item Für $m, a$ ungerade: \eqref{eq_6.1} ist äquivalent zu $(aX + \frac{b}{2})^2 - \enbrace*{ \enbrace*{\frac{b}{2}}^2 - ac} \kon 0 \modu am$.
	\end{enumerate}

\begin{falko} \label{F6.1}
	Die Kongruenz
	\begin{equation}
		X^2 \kon D \modu m \text{ mit } (D,m) = d = d_1^2 d_0 \text{ und } d_0 \text{ quadratfrei} \label{eq_6.4}
	\end{equation}
	ist genau dann lösbar, wenn $\enbrace*{\frac{m}{d},d_0} = 1$ und
	\begin{equation}
		X^2 \kon d_0 \frac{D}{d} \modu \frac{m}{d} \label{eq_6.5}
	\end{equation}
	lösbar ist. Hier sind $d_0 \frac{D}{d}$ und $\frac{m}{d}$ teilerfremd! (Denn $\frac{m}{d}$ prim zu $\frac{D}{d}$ und wegen $\enbrace*{\frac{m}{d},d_0} = 1$ auch zu $d_0$.)
\end{falko}

Damit ist alles reduziert auf eine Kongruenz der Gestalt
\begin{equation}
	X^2 \kon a \modu m \text{ mit } (a,m) = 1 \label{eq_6_stern}
\end{equation}

\begin{defn}[Quadratischer Rest]
	Ist \eqref{eq_6_stern} lösbar, d.h. existiert ein $b \in \ZZ$ mit $b^2 \kon a  \modu m$, so heißt $a$ ein \Index{Quadratischer Rest} (QR) modulo $m$, andernfalls heißt $a$ ein \bet{quadratischer Nichtrest} modulo $m$.
\end{defn}

\minisec{Probleme}
	\begin{enumerate}[1)]
		\item Sei $m$ gegeben. Man verschaffe sich eine Übersicht über die sämtlichen quadratischen Reste modulo $m$.
		\item Sei $a$ gegeben. Für welche (zu $a$ teilerfremden) natürlichen Zahlen $m > 1$ ist $a$ quadratischer Rest modulo $m$?
	\end{enumerate}

Problem 2) ist schwieriger und tiefer. Eine Antwort liefert das \bet{Quadratische Reziprozitätsgesetz}.\\ Zuerst Problem 1): \index{Quadratisches Reziprozitätsgesetz}

\begin{falko} \label{F6.2}
	$a$ ist quadratischer Rest modulo $m$ genau dann, wenn gilt:
	\begin{enumerate}[1)]
		\item $a$ ist quadratischer Rest modulo $p$ für jeden ungeraden Primteiler $p$ von $m$.
		\item $\begin{cases}
			a \kon 1 \modu 4, & \text{ falls } 4 | m, 8 \not| \ m \\
			a \kon 1 \modu 8, & \text{ falls } 8 | m
		\end{cases}$
	\end{enumerate}
	Ist $a$ quadratischer Rest modulo $m$, so hat \eqref{eq_6_stern} genau $2^{s+t}$ Lösungen modulo $m$; dabei ist $s$ die Anzahl der ungeraden Primteiler von $m$ und
	\begin{equation}
	\begin{aligned}
		t &= 2 \text{ für } w_2(m) \geq 3 \\
		t &= 1 \text{ für } w_2(m) = 2 \\
		t &= 0 \text{ für } w_2(m) \leq 1.
	\end{aligned}
	\end{equation}
\end{falko}

Damit ist alles reduziert auf den Fall $m = p$ mit $p \neq 2$ Primzahl. \marginnote{09.12. \\ \ [17]}
\begin{equation}
	X^2 \kon a \modu p, \quad (a,p) = 1 \label{eq_legendre}
\end{equation}

\begin{defn}[Legendresymbol]
	Sei $p \neq 2$ eine Primzahl. Der Ausdruck
	\begin{equation}
		\leg{a}{p} := \begin{cases}
			1, & \text{ falls \eqref{eq_legendre} lösbar} \\
			-1, & \text{ falls \eqref{eq_legendre} nicht lösbar} \\
		\end{cases}
	\end{equation}
	ist definiert für jedes $a \in \ZZ$ mit $(a,p) = 1$. \index{Legendresymbol}
\end{defn}

$S = \{ 1,2,\dots,p-1\}$ ist ein primes Restsystem modulo $p$ (Vertretersystem von $(\ZZ/p)^\times$). $H := \penbrace*{1, 2, \dots, \frac{p-1}{2}}$ heißt ein \bet{unteres Halbsystem} und $H' := \penbrace*{ \frac{p+1}{2}, \dots, p-2, p-1 }$ ein \bet{oberes Halbsystem}. Ist $a$ quadratischer Rest modulo $p$, so gibt es genau ein $b \in H$ mit $b^2 \kon a \modu p$. Also:

\begin{falko} \label{F6.3}
	Es gibt genau $\frac{p-1}{2}$ quadratische Reste modulo $p$ und ebenso viele quadratische Nichtreste modulo $p$.
\end{falko}

\begin{falko}[Eulersches Kriterium] \label{F6.4}
	Für jedes $a$ teilerfremd zu $p \neq 2$ gilt
	\[ \leg{a}{p} \kon a^{\frac{p-1}{2}} \modu p \]
\end{falko}

\minisec{Bemerkung}
	\begin{enumerate}[1)]
		\item \ref{F6.3} und \ref{F6.4} folgen auch sofort aus der Existenz einer Primitivwurzel modulo $p$. ($G = (\ZZ/p\ZZ)^\times$ ist zyklisch von der Ordnung $p-1$.)
		\item Aus $\leg{a}{p} \kon \varepsilon \modu p$ mit $\varepsilon \in \{1,-1\}$ folgt $\leg{a}{p} = \varepsilon$. Denn $1 \modu -1 \modu p$ ist unmöglich für $p \neq 2$.
		\item \ref{F6.4} für $a = -1$: $\leg{-1}{p} \kon (-1)^\frac{p-1}{2} \modu p$
		\begin{equation}
			\xRightarrow{\text{2)}} \leg{-1}{p} = (-1)^\frac{p-1}{2} = \begin{cases}
				1, & \text{ für } p \kon 1 \modu 4 \\
				-1, & \text{ für } p \kon 3 \modu 4
			\end{cases} \label{eq_F6.4}
		\end{equation}
		Also folgt erneut \ref{F3.8}. (\Index{1. Ergänzungssatz})
	\end{enumerate}
	
\begin{falko} \label{F6.5}
	\begin{enumerate}[(i)]
		\item Das Legendresymbol $\leg{a}{p}$ hängt von $a$ nur modulo $p$ ab.
		\item $\leg{ab}{p} = \leg{a}{p} \cdot \leg{b}{p}$ für alle $a,b$ prim zu $p$.
	\end{enumerate}
\end{falko}

\minisec{Bemerkung}
	\begin{enumerate}[1)]
		\item Das Legendresymbol vermittelt eine Abbildung
		\begin{equation}
		\begin{aligned}
			\chi\colon (\ZZ/p)^\times &\longrightarrow \{1, -1\} \\
			a \modu p &\longmapsto \leg{a}{p}
		\end{aligned}
		\end{equation}
		Diese ist ein Homomorphismus von Gruppen. $\chi(a \modu p)$ gibt quadratischen Charakter von $a \modu p$ an. Allgemein: Jeder Homomorphismus einer endlichen abelschen Gruppe $G$ in $\CC^\times$ heißt ein \Index{Charakter} von $G$.
		\item $a = \pm q_1 q_2 \dots q_s$ mit $q_i \neq p$ Primzahlen.
		\[ \leg{a}{p} = \leg{\pm 1}{p} \leg{q_1}{p} \leg{q_2}{p} \cdots \leg{q_s}{p} \]
	\end{enumerate}
	
Zur Beantwortung von Problem 2 ist wegen \ref{F6.2} nur zu fragen: Für welche Primzahlen $p \neq 2$ ist die gegebene Zahl $a$ quadratischer Rest modulo $p$? Wegen \ref{F6.5} genügt es dann weiter, für $a$ folgende Fälle zu betrachten: \begin{enumerate}[1.]
	\item $a = -1$. Schon erlegt durch den 1. Ergänzungssatz.
	\item $a = 2$. Wird erledigt durch den 2. Ergänzungssatz. \index{2. Ergänzungssatz}
	\item $a$ ist eine ungerade Primzahl $q$. Lösung durch das quadratische Reziprozitätsgesetz. \index{Quadratisches Reziprozitätsgesetz}
\end{enumerate}

\begin{falko}[Gaußsches Lemma] \label{F6.6}
	\[ \leg{a}{p} = \prod_{x \in H} \varepsilon(ax) \]
\end{falko}

\begin{falko}[2. Ergänzungssatz] \label{F6.7}
	\[ \leg{2}{p} = \begin{cases}
		+1 & \text{ für } p \kon \pm 1 \modu 8 \\
		-1 & \text{ für } p \kon \pm 5 \modu 8
	\end{cases} \] \index{2. Ergänzungssatz}
\end{falko}

\begin{satz}[Quadratisches Reziprozitätsgesetz] \label{satz_6.1}
	Für ungerade Primzahlen $p \neq q$ gilt
	\begin{equation}
		\leg{p}{q} \leg{q}{p} = (-1)^{\frac{p-1}{2} \frac{q-1}{2}} \label{eq_quadrez}
	\end{equation}
	Das bedeutet: \index{Quadratisches Reziprozitätsgesetz} \begin{enumerate}[1)]
		\item Ist eine der beiden Primzahlen $p,q$ kongruent zu $1 \modu 4$, so gilt
			\[ \leg{q}{p} = \leg{p}{q}, \]
			d.h. $q$ ist quadratischer Rest modulo $p$ genau dann, wenn $p$ quadratischer Rest modulo $q$ ist.
		\item Sind beide Primzhlen $p$ und $q$ kongruent zu $3 \modu 4$, so gilt
			\[ \leg{q}{p} = -\leg{p}{q}, \]
			d.h. $q$ ist quadratischer Rest modulo $p$ genau dann, wenn $p$ quadratischer Nichtrest modulo $q$ ist.
	\end{enumerate}
\end{satz}

\minisec{Zusatz für $p \neq 2$}
	\[ \leg{-1}{p} = (-1)^\frac{p-1}{2} \text{ nach 1. Ergänzungssatz} \]
	\[ \leg{2}{p} = (-1)^\frac{p^2-1}{8} \text{ nach 2. Ergänzungssatz} \]
	Das bedeutet:
	\begin{equation}
	\begin{aligned}
		-1 \text{ quadratischer Rest modulo } p &\Leftrightarrow p \kon 1 \modu 4 \\
		2 \text{ quadratischer Rest modulo } p &\Leftrightarrow p \kon \pm 1 \modu 8 
	\end{aligned}
	\end{equation}
	
\begin{defn}[Jacobi-Symbol]
	Seien $a, b \in \ZZ \setminus \setnull$, $b$ ungerade, $(a,b) = 1$. Definiere das \Index{Jacobi-Symbol} durch \marginnote{12.12. \\ \ [18]}
	\begin{equation}
		\leg{a}{b}_J = \prod \leg{a}{p}^{w_p(b)}
	\end{equation}
	Für $b = p$ prim ist also $\leg{a}{p}_J = \leg{a}{p}$. Daher verzichten wir auf das $J$ im Index.
\end{defn}

\minisec{Eigenschaften}
	\begin{enumerate}[(1)]
		\item $a \kon a' \modu b \Rightarrow \leg{a}{b} = \leg{a'}{b}$
		\item $\leg{a}{b} \leg{a'}{b} = \leg{aa'}{b}$ und $\leg{a}{bb'} = \leg{a}{b} \leg{a}{b'}$
		\item $\leg{x^2}{b} = 1 = \leg{a}{y^2}, \leg{ax^2}{b} = \leg{a}{b} = \leg{a}{by^2}$ für $(x,b) = (y,a) = 1$ und $y$ ungerade
		\item $a$ quadratischer Rest $\modu b \Rightarrow \leg{a}{b} = 1$
	\end{enumerate}
	
\setcounter{satz}{0}
\begin{satz}[Reziprozitätsgesetz für das Jacobi-Symbol] \label{satz_6.1a}
	Sei $b \in \ZZ$ ungerade.
		\begin{equation}
			\leg{-1}{b} = (-1)^{\frac{b-1}{2} + \frac{\sgn(b)-1}{2}} \label{eq_jacobi_E1} 	
		\end{equation}
		\begin{equation}
			\leg{2}{b} = (-1)^\frac{b^2-1}{8} = \begin{cases}
				+1 & \text{ für } b \kon \pm 1 \modu 8 \\
				-1 & \text{ für } b \kon \pm 3 \modu 8
			\end{cases} \label{eq_jacobi_E2}
		\end{equation}
		\begin{equation}
			\leg{a}{b} \leg{b}{a} = (-1)^{\frac{a-1}{2} \frac{b-1}{2} + \frac{\sgn(a)-1}{2} \frac{\sgn(b)-1}{2}} \text{ für } a \text{ ungerade und } (a,b)=1 \label{eq_jacobi_R}
		\end{equation}
\end{satz}

\minisec{Korollar}
	Sei $a \in \NN$ gegeben. Für alle ungeraden $b \in \ZZ$ prim zu $a$ hängt $\leg{a}{b}$ von $b$ nur modulo $4a$ ab, im Falle $a \kon 1 \modu 4$ sogar nur modulo $a$.
\cleardoubleemptypage