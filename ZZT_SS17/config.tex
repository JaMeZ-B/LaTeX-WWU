%!TEX root = ZZT_SS17.tex
% Author: Phil Steinhorst, p.st@wwu.de
\documentclass[%
	paper=a4,
	fontsize=10,
	DIV=13,
	BCOR=9mm,
	chapterprefix=false,
	headinclude,
	footinclude,
	headheight=20.7pt,
	footheight=18pt,
	headings=optiontohead,
	toc=bibliography,
	toc=listof,
	ngerman,
	final=true,
	twoside,
	index=totoc,
	cleardoublepage=empty
]{scrreprt}

% Basics für Codierung und Sprache
% ===========================================================
	\usepackage{scrtime}
	\usepackage{etex}
	\usepackage{shellesc}
	\usepackage[final]{graphicx}
	\usepackage[utf8]{inputenc}
	\usepackage{babel}
	\usepackage[german=quotes]{csquotes}
% ===========================================================

% Fonts und Typographie
% ===========================================================
	% \usepackage[lining,semibold]{libertine}
	%\usepackage[proportional,scaled=1.0]{erewhon}
	% \usepackage{concmath}
	% \usepackage[default]{droidserif}
	\usepackage[oldstyle]{XCharter}
	%\usepackage[rm]{merriweather}
	%\usepackage[default,regular,bold,osf]{sourceserifpro}
	
	% Sans- und Mono-Fonts
	% \usepackage[rm,light]{roboto}
	\usepackage[scale=0.90]{sourcecodepro}
	\usepackage[scale=1.00]{sourcesanspro}
	% \usepackage[sb,lining,scale=.92]{FiraSans}
	% \usepackage[lining,scale=.87]{FiraMono}
	
	\usepackage[babel=true,final,tracking=smallcaps]{microtype}
	\DisableLigatures{encoding = T1, family = tt* } % keine Ligaturen für Monospace-Fonts
	\usepackage{ellipsis}
% ===========================================================

% Farben
% ===========================================================
	\usepackage[usenames,x11names,final]{xcolor}
% ===========================================================

% Mathe-Pakete und -Einstellungen
% ===========================================================
	\usepackage{mathtools}
	\usepackage{amssymb}
	\usepackage{amsthm}
	\usepackage[bigdelims]{newtxmath}		% moderne Mathe-Font
	\allowdisplaybreaks						% seitenübergreifende Rechnungen
	% % % % % % % % % % % % % % % % % % % % % % % % % % % % % % % % %
%		PhistMath.tex											%
%		Weitere Mathe-Befehle									%
%																%
%		Author: Phil Steinhorst									%
% % % % % % % % % % % % % % % % % % % % % % % % % % % % % % % % %

% % % Buchstaben und Zahlen
\newcommand{\aff}{\mathbb{A}}
\newcommand{\CC}{\mathbb{C}}
\newcommand{\EE}{\mathbb{E}}
\newcommand{\FF}{\mathbb{F}}
\newcommand{\HH}{\mathcal{H}}
\newcommand{\KK}{\mathbb{K}}
\newcommand{\LL}{\mathbb{L}}
\newcommand{\MM}{\mathcal{M}}
\newcommand{\NN}{\mathbb{N}}
\newcommand{\OO}{\mathbb{O}}
\newcommand{\QQ}{\mathbb{Q}}
\newcommand{\RR}{\mathbb{R}}
\newcommand{\Ss}{\mathbb{S}}
\newcommand{\UU}{\mathcal{U}}
\newcommand{\ZZ}{\mathbb{Z}}
\newcommand{\oh}{\mathcal{O}}					% Landau-O
\newcommand{\ind}{1\hspace{-0,8ex}1} 			% Indikatorfunktion (Doppeleins)

% % % Abk�rzungen
\newcommand{\bewrueck}{"$\Leftarrow$":} 	% Beweis R�ckrichtung
\newcommand{\bewhin}{"$\Rightarrow$":}		% Beweis Hinrichtung
\newcommand{\setone}{\{1\}}							% Einsmenge
\newcommand{\NT}{\trianglelefteq}					% Normalteiler
\newcommand{\setzero}{\{0\}}						% Nullmenge

% % % Operatoren
\DeclareMathOperator{\Abb}{Abb}						% Menge der Abbildungen
\DeclareMathOperator{\Aut}{Aut} 					% Automorphismen
\DeclareMathOperator{\Bild}{Bild}					% Kern
\DeclareMathOperator{\Char}{char} 					% Charakteristik
\DeclareMathOperator{\Det}{\det\,\!}				% Determinante
\DeclareMathOperator{\End}{End}
\DeclareMathOperator{\ev}{ev}						% Auswertungsabb.
\DeclareMathOperator{\GL}{GL}						% allgemeine lineare Gruppe
\DeclareMathOperator{\grad}{grad}					% Grad
\DeclareMathOperator{\Hom}{Hom} 					% Homomorphismen
\DeclareMathOperator{\id}{id} 						% Identit�t
\DeclareMathOperator{\im}{im} 						% image
\renewcommand{\Im}{\operatorname{Im}}				% Imagin�rteil
\DeclareMathOperator{\Isom}{Isom}					% Isometrien
\DeclareMathOperator{\Kern}{Kern}					% Kern
\DeclareMathOperator{\LH}{LH}						% Lineare H�lle
\DeclareMathOperator{\ord}{ord} 					% Ordnung
\DeclareMathOperator{\pot}{\mathcal{P}}				% Potenzmenge
\DeclareMathOperator{\Rang}{Rang}					% Rang
\renewcommand{\Re}{\operatorname{Re}}				% Realteil
\DeclareMathOperator{\SAut}{SAut}
\DeclareMathOperator{\sgn}{sgn} 					% Signum
\DeclareMathOperator{\sign}{sign} 					% Signum
\DeclareMathOperator{\SL}{SL} 						% Spezielle lineare Gruppe
\DeclareMathOperator{\SO}{SO} 						% Spezielle orthogonale Gruppe
\DeclareMathOperator{\SU}{SU} 						% Spezielle unit�re Gruppe
\DeclareMathOperator{\Sym}{Sym} 					% Symmetrische Gruppe
\DeclareMathOperator{\Span}{span}					% span

% % % Klammerungen
\DeclarePairedDelimiter{\abs}{\lvert}{\rvert}			% Betrag
\DeclarePairedDelimiter{\ceil}{\lceil}{\rceil}			% aufrunden
\DeclarePairedDelimiter{\floor}{\lfloor}{\rfloor}		% aufrunden
\DeclarePairedDelimiter{\Norm}{\lVert}{\rVert}			% Norm
\DeclarePairedDelimiter{\sprod}{\langle}{\rangle}		% spitze Klammern
\DeclarePairedDelimiter{\enbrace}{(}{)}					% runde Klammern
\DeclarePairedDelimiter{\benbrace}{\lbrack}{\rbrack}	% eckige Klammern
\DeclarePairedDelimiter{\penbrace}{\{}{\}}				% geschweifte Klammern
\newcommand{\Underbrace}[2]{{\underbrace{#1}_{#2}}} % Underbrace als Befehl in LaTeX-Syntax (und ohne Spacing-Probleme mit nachfolgenden Operatoren...)

\newcommand{\enb}[1]{\enbrace*{#1}}
\newcommand{\penb}[1]{\penbrace*{#1}}
\newcommand{\benb}[1]{\benbrace*{#1}}


\newcommand{\stack}[2]{\makebox[1cm][c]{$\stackrel{#1}{#2}$}}
\newcommand{\ol}[1]{\overline{#1}}
\newcommand{\mat}[4]{\tensor*[^{#2}_{}]{#1}{^{#3}_{#4}}}

\newcommand\SetSymbol[1][]{\nonscript\:#1\vert\allowbreak\nonscript\:\mathopen{}}
\providecommand\given{} % to make it exist
\DeclarePairedDelimiterX\set[1]\{\}{\renewcommand\given{\SetSymbol[\delimsize]}#1}

\usepackage{stmaryrd}
% ===========================================================

% TikZ
% ===========================================================
	\usepackage{tikz}
	\usepackage{tikz-cd}					% kommutative Diagramme
	\usetikzlibrary{arrows.meta}			% mehr Pfeile!
	\tikzset{>=Latex}						% Standard-Pfeilspitze
% ===========================================================

% Seitenlayout, Kopf-/Fußzeile
% ===========================================================
	\usepackage{scrpage2}
	\pagestyle{scrheadings}
	\clearscrheadfoot 
	\setheadsepline{1pt} 					% Linie in Kopfzeile
	\automark[section]{section}				% Abschnittstitel in Kopfzeile
	\rohead{\rightmark} 
	\lehead{\rightmark} 
	\ofoot[{\Large \pagemark}]{\Large \pagemark}	% Seitenzahl in Fußzeile
	\raggedbottom
	\usepackage{setspace}					
	\onehalfspacing							% Zeilenabstand 1.5-fach
	%\setlength{\parindent}{0pt}
	%\setlength{\parskip}{0.5\baselineskip}
% ===========================================================

% Nummerierungen
% ===========================================================
	\usepackage{chngcntr}
	\counterwithout{equation}{chapter} 	% keine Kapitel in Gleichungsnummern
	\counterwithout{section}{chapter}	% keine Kapitel in Abschnittsnummern
% ===========================================================

% Biblatex
% ===========================================================
	\usepackage[%
		backend=biber,
		sortlocale=auto,
		natbib,
		hyperref,
		backref,
		backrefstyle=three+,
		style=alphabetic]{biblatex}
	\setlength{\bibitemsep}{1em}		% Abstand zwischen den Literaturangaben
	\setlength{\bibhang}{2em}			% Einzug nach jeweils erster Zeile
	\addbibresource{literature.bib}		% Literaturdatei einlesen
	\nocite{*}							% Aufführen nicht referenzierter Quellen
% ===========================================================

% Hyperref
% ===========================================================
	\usepackage[%
		hidelinks,
		pdfpagelabels,
		bookmarksopen=true,
		bookmarksnumbered=true,
		linkcolor=black,
		urlcolor=SkyBlue2,
		plainpages=false,
		pagebackref,
		citecolor=black,
		hypertexnames=true,
		pdfauthor={Phil Steinhorst},
		pdfborderstyle={/S/U},
		linkbordercolor=SkyBlue2,
		colorlinks=false,
		backref=false]{hyperref}
	\hypersetup{final}
% ===========================================================

% Marginnotes / ToDo-Notes
% ===========================================================
	\usepackage[fulladjust]{marginnote}
	\renewcommand*{\marginfont}{\itshape\footnotesize}
	\usepackage[textsize=small,color=Red1!80!OrangeRed1!80]{todonotes}
% ===========================================================

% Listen und Tabellen
% ===========================================================
	\usepackage{multicol}
	\usepackage[shortlabels]{enumitem}
	\setlist{itemsep=0pt}
	\setlist[enumerate]{font=\sffamily\bfseries}
	\setlist[itemize]{label=$\triangleright$}
	\usepackage{tabularx}
% ===========================================================

% Inhaltsverzeichnis und Querverweise
% ===========================================================
	\usepackage[tocindentauto]{tocstyle}
	% \setcounter{tocdepth}{3}				% Tiefe des Inhaltsverzeichnisses
	\usetocstyle{KOMAlike}	
	\usepackage[nameinlink]{cleveref}
% ===========================================================

% Indexerstellung
% ===========================================================
	\usepackage{makeidx}
	\newcommand{\Index}[1]{\textbf{#1}\index{#1}}
	\makeindex
	\renewcommand{\indexpagestyle}{scrheadings}
% ===========================================================

% Theorem-Pakete und -Konfiguration
% ===========================================================
	\usepackage{thmtools}
	\declaretheoremstyle[%
		headfont=\sffamily\bfseries,
		notefont=\normalfont\sffamily,
		bodyfont=\normalfont,
		headformat=\NUMBER \ \NAME \NOTE,
		headpunct={\\},
		postheadspace=1ex,
		spaceabove=15pt,spacebelow=10pt]{mainstyle}
	\declaretheoremstyle[%
		headfont=\sffamily\bfseries,
		notefont=\normalfont\sffamily,
		bodyfont=\normalfont,
		headformat=\NAME \NOTE,
		headpunct={\\},
		postheadspace=1ex,
		spaceabove=15pt,spacebelow=10pt]{nonumber}
	\declaretheoremstyle[%
		headfont=\sffamily\bfseries,
		notefont=\normalfont\sffamily,
		bodyfont=\normalfont,
		headformat=\NAME \ \NOTE,
		headpunct={\\},
		postheadspace=1ex,
		spaceabove=15pt,spacebelow=10pt]{miscstyle}
	\declaretheoremstyle[%
		headfont=\bfseries\scshape,
		bodyfont=\normalfont,
		headpunct=:,
		postheadspace=1ex,
		spacebelow=12pt,spaceabove=2pt,
		qed=\qedsymbol]{beweise}
	
	\declaretheorem[name=Definition,parent=section,style=mainstyle]{definition}
	\declaretheorem[name=Satz,sharenumber=definition,style=mainstyle]{satz}
	\declaretheorem[name=Korollar,sharenumber=definition,style=mainstyle]{korollar}
	\declaretheorem[name=Lemma,sharenumber=definition,style=mainstyle]{lemma}
	\declaretheorem[name=Proposition,sharenumber=definition,style=mainstyle]{proposition}
	\declaretheorem[name=Bemerkung,sharenumber=definition,style=mainstyle]{bemerkung}
	\declaretheorem[name=Beispiel,sharenumber=definition,style=mainstyle]{beispiel}
	
	\declaretheorem[name=Beweis,numbered=no,style=beweise]{beweis}
% ===========================================================

% Sonstiges / für Testzwecke
% ===========================================================
	\usepackage{lipsum}
% ===========================================================