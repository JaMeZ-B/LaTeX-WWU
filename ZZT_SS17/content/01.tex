%!TEX root = ../ZZT_SS17.tex
\section{Teiler, Primzahlen und der Hauptsatz der Arithmetik}
\label{sec:1}
	\subsection*{Erinnerung}
	Wir betrachten die Menge der \textbf{natürlichen Zahlen} \index{natürliche Zahl} \marginnote{[1] \\ 20.4.}
	\[
		\NN= \{0,1,2,3,\dots\},
	\]
	die Menge der \textbf{ganzen Zahlen} \index{ganze Zahl}
	\[
		\ZZ= \{0, \pm 1, \pm 2, \pm 3, \dots\}
	\]
	mit Addition $+$ und Multiplikation $\cdot$, mit Anordnung $\leq$ und dem \Index{Absolutbetrag}
	\[
		\abs{z} = \max\{z,-z\}.
	\]
	In $\NN$ ist die Subtraktion nur eingeschränkt möglich und in $\ZZ$ ist die Division nur eingeschränkt möglich.
	Aber wir können kürzen:
	\begin{gather}
		a+x = a+y \quad \Rightarrow \quad x = y \qquad \text{in } \NN \text{ und } \ZZ \\
		ax = ay \quad \Rightarrow \quad x=y \qquad \text{in } \NN \text{ und } \ZZ, a \neq 0.
	\end{gather}
	
	\subsection*{Induktion und Wohlordnung}
	\begin{satz}[Erstes Induktionsprinzip]
	\label{satz:IP1}
	Ist $S \subseteq \NN$ mit \index{erstes Induktionsprinzip} \marginnote{[2]}
	\begin{enumerate}[(i)]
		\item $0 \in S$,
		\item $s \in S \Rightarrow s+1 \in S$,
	\end{enumerate}
	dann gilt $S = \NN$.	
	\end{satz}
	
	Anschauung: Alle natürlichen Zahlen erreicht man durch Zählen, ausgehend von $0$.
	
	Aus dem ersten Induktionsprinzip folgt das \Index{Wohlordnungsprinzip}
	
	\begin{satz}[Wohlordnungsprinzip]
		\label{satz:WOP}
		Ist $S \subseteq \NN$ mit $S \neq \emptyset$, so gibt es in $S$ ein kleinstes Element $m = \min(S)$.
	\end{satz}
	
	\begin{beweis}
		Sei $S \subseteq \NN$ mit $S \neq \emptyset$.
		Sei $T = \{t \in \NN : t \leq s \text{ für alle } s \in S\}$.
		Es folgt $0 \in T$.
		Ist $s \in S$ so ist $s+1 \notin T$, denn $s+1 \nleq s$.
		Es folgt $T \neq \NN$.
		Nach dem ersten Induktionsprinzip folgt: Es gibt $t \in T$ mit $t+1 \notin T$.
		
		Nun gilt $t \in S$. Denn wäre $t \notin S$, so wäre $t < s$ für alle $s \in S$, also $t+1 \leq s$ für alle $s \in S$. Damit wäre $t+1 \in T$. \lightning
		
		Also ist $t$ ein kleinstes Element in $S$.
	\end{beweis}
	
	Aus dem Wohlordnungsprinzip folgt wiederum das \textbf{zweite Induktionsprinzip}:
	
	\begin{satz}[Zweites Induktionsprinzip]
		\label{satz:IP2}
		Ist $S \subseteq \NN$ mit
		\begin{enumerate}[(i)]
			\item $0 \in S$,
			\item $s \in S \Leftrightarrow$ für alle $t \in \NN$ mit $t < s$ ist $t \in S$,
		\end{enumerate}
		so gilt $S = \NN$.
	\end{satz}
	
	\begin{beweis}
		Angenommen, $S \subseteq \NN$ erfüllt (i) und (ii), und angenommen $S \neq \NN$.
		Betrachte $R = \NN \setminus S \neq \emptyset$.
		Sei $r := \min(R)$ (existiert nach dem Wohlordnungsprinzip). Für alle $t < r$ gilt also $t \in S$. Nach (ii) folgt somit $r \in S$. Widerspruch, da $r \in R = \NN \setminus S$.
	\end{beweis}

	Vorsicht: Die Sätze \ref{satz:IP1}, \ref{satz:WOP} und \ref{satz:IP2} gelten für $\NN$, aber nicht für $\ZZ$!
	Zum Beispiel hat $\ZZ \subseteq \ZZ$ kein kleinstes Element.
	Es gilt aber Folgendes:
	
	\begin{lemma}
		\label{lemma:1.3}
		Sei $S \subseteq \ZZ$ und sei $S \neq \emptyset$. \marginnote{[3]}
		Wenn $S$ eine untere (bzw. obere) Schranke hat, so hat $S$ auch ein kleinstes (bzw. größtes) Element.
	\end{lemma}
	
	\begin{beweis}
		Sei $k$ untere Schranke für $S$, das heißt $k \leq s$ für alle $s \in S$.
		Betrachte die Menge $S' := \{s - k : s \in S\} \subseteq \NN$, $S' \neq \emptyset$.
		Sei $l = \min(S')$.
		Dann ist $l+k$ ein Minimum für $S$.
		
		Sei nun $m$ eine obere Schranke für $S \subseteq \ZZ$, sei $S'' = \{-s : s \in S\}$.
		Dann ist $-m$ eine untere Schranke für $S''$.
		Sei $n = \min(S'')$.
		Damit ist $-n$ ein größtes Element in $S$. 
	\end{beweis}

	Wir betrachten jetzt Teilbarkeit.
	
	\begin{definition}[Teiler]
		\label{def:teiler}
		Sei $a,b \in \ZZ$.
		Wenn es ein $m \in \ZZ$ gibt mit $ma = b$, so heißt $a$ ein \Index{Teiler} von $b$. \marginnote{[4]}
		Man schreibt $a \mid b$, andernfalls $a \nmid b$. \marginnote{lies: \enquote{$a$ teilt $b$ (nicht)}}
	\end{definition}

	\begin{beispiel}
		Es gilt $3 \mid 6$ und $3 \nmid 7$.
	\end{beispiel}

	\begin{satz}[Rechenregeln für Teilbarkeit]
		\label{satz:rechenregeln}
		\begin{enumerate}[(i)]
			\item $1 \mid a$, $a \mid a$, $a \mid 0$.
			\item $a \mid b$ und $b \mid c \Rightarrow a \mid c$
			\item $a \mid b$ und $b \mid a \Rightarrow a = \pm b$
			\item $a \mid b$ und $a \mid c \Rightarrow a \mid b \pm c$
			\item $a \neq 0$ und $ab \mid ac \Rightarrow b \mid c$
		\end{enumerate}
	\end{satz}

	\begin{beweis}
		\begin{enumerate}[(i)]
			\item klar.
			\item Gilt $am = b$ und $bn = c$, so gilt $amn = $.
			\item Gilt $am = b$ und $bn = a$, so gilt $bmn = b$, und damit folgt $a = b = 0$ oder $mn = 1$, also $m,n = \pm 1$.
			\item Gilt $am = b$ und $an = c$, so folgt $am \pm an = a(m \pm n) = b \pm c$.
			\item Ist $a \neq 0$ und $abm = ac$, so folgt $bm = c$.
		\end{enumerate}
	\end{beweis}

	\begin{satz}[Teilen mit Rest]
		\label{satz:teilen mit rest}
		Sei $a,b \in \ZZ$, sei $b \neq 0$. \index{Teilen mit Rest} \marginnote{[5]}
		Dann existieren eindeutige Zahlen $r,s \in \ZZ$ mit $a = bs + r$ und $0 \leq r < \abs{b}$.
	\end{satz}

	\begin{beweis}
		\textit{Zur Eindeutigkeit:} Angenommen, $a = bs + r = bs' + r'$ mit $s,s',r,r' \in \ZZ$ und $0 \leq r,r' < \abs{b}$.
		Ohne Einschränkung sei $r' \geq r$.
		Umstellen der Gleichungen liefert
		\[
			\abs{b} > \underbrace{r' - r}_{\geq 0} = b(s - s').
		\]
		Also folgt $b \mid (r'-r)$ und $\abs{r' - r} < b$, das heißt $r' - r = 0$ und damit $s = s'$.
		
		\textit{Zur Existenz:} Sei $S = \{k \in \ZZ : k \cdot \abs{b} \leq a\}$. \marginnote{24.4.}
		Dann gilt $-\abs{a} \in S$, also $S \neq \emptyset$.
		Weiter ist $\abs{a}$ eine obere Schranke für $S$.
		Folglich existiert in $S$ ein größtes Element $s = \max(S)$.
		Es folgt $s \cdot \abs{b} \leq a$ und $(s+1) \cdot \abs{b} > a$.
		Wähle $r \in \NN$ mit $s \cdot \abs{s} + r = a$, also $s \cdot \abs{b} + r < (s-1) \cdot \abs{b}$ und damit $0 \leq r < \abs{b}$.
		Für $b>0$ ist $\abs{b} = b$ und $s \cdot b + r = a$.
		Für $b<0$ ist $\abs{b} = -b$ und $-s \cdot b + r = a$.
	\end{beweis}

	\begin{satz}
		Sei $H \subseteq \ZZ$ eine Teilmenge  mit \marginnote{[6]}
		\begin{enumerate}[(i)]
			\item $H \neq \emptyset$
			\item $x,y \in H \quad \Rightarrow \quad x-y \in H$.
		\end{enumerate}
		Dann gibt es ein eindeutig bestimmtes $d \in \NN$ mit $H = d \cdot \ZZ = \{d \cdot k : k \in \ZZ\}$.
	\end{satz}

	\begin{beweis}
		Sei $x \in H$.
		Dann ist $x-x = 0 \in H$ und $0-x = -x \in H$.
		Folglich ist für $x,y \in H$ auch $x -(-y) = x+y \in H$.
		
		Falls $H = \{0\}$, so gilt $0 \cdot \ZZ = \setzero$ und es ist eindeutig $d = 0$.
		
		Falls $H \neq \setzero$, so gibt es ein Element $h \in H$ mit $h > 0$.
		Sei $S = \{h \in H : h > 0\} \neq \emptyset$.
		Setze $d = \min(S)$.
		Ist nun $h \in H$ beliebig, schreibe $a = s\cdot d + r$ mit $0 \leq r < \abs{d}$ (Teilen mit Rest).
		Dann gilt $s \cdot d \in H$ und damit $r = h-sd \in H$.
		Nach Wahl von $d = \min(S)$ folgt $r = 0$, das heißt $h = d \cdot s \in d \cdot \ZZ$.
		
		Ist $H = d' \cdot \ZZ$ mit $d' \in \NN$, so folgt $d' = \min\{h \in H : h > 0\} = \min(S) = d$.
	\end{beweis}

	\begin{korollar}
		\label{kor:satz_1.6}
		Seien $a_1,\dots,a_n \in \ZZ, n \geq 1$ und sei
		\[
			H = \{z_1a_1 + z_2a_2 + \dots + z_na_n : z_1,\dots,z_n \in \ZZ\}
		\]
		die Menge aller ganzzahligen Linearkombinationen der Zahlen $a_1,\dots,a_n$.
		Dann gibt es genau ein $d \in \NN$ mit $H = d \cdot \ZZ$.
	\end{korollar}

	\begin{beweis}
		Es gilt $a_1 \cdot 1 + a_2 \cdot 0 + \dots + a_n \cdot 0 \in H \neq \setzero$ und $(a_1z_1 + \dots + a_nz_n) - (a_1z_1' + \dots + a_nz_n') = a_1(z_1-z_1') + \dots + a_n(z_n-z_n') \in H$.
		Nach dem vorigen Satz folgt $H = d \cdot \ZZ$ mit $d$ eindeutig.		
	\end{beweis}
	\newpage
	\begin{definition}[größter gemeinsamer Teiler]
		Die Zahl $d \in \NN$ aus Korollar~\ref{kor:satz_1.6} heißt der \textbf{größte gemeinsame Teiler} von $a_1,\dots,a_n$. \index{größter gemeinsamer Teiler}
		Wir schreiben:
		\[
			\ggT(a_1,\dots,a_n) = d.
		\]
	\end{definition}

	Der Name \enquote{größter gemeinsamer Teiler} ist berechtigt und passt zur Definition aus der Schule:
	
	\begin{lemma}
		\label{lem:1.7}
		Sei $a_1,\dots,a_n \in \ZZ$ und $d = \ggT(a_1,\dots,a_n)$. \marginnote{[7]}
		Dann gilt:
		\begin{enumerate}[(i)]
			\item $d \mid a_i$ für alle $i = 1,\dots,n$.
			\item Ist $b \in \ZZ$ mit $b \mid a_i$ für alle $i=1,\dots,n$, dann gilt $b \mid d$ (das heißt, jeder gemeinsame Teiler der $a_i$ teilt auch $d$).
		\end{enumerate}
	\end{lemma}

	\begin{beweis}
		Sei $H = \{a_1z_1 + \dots + a_nz_n : z_1,\dots,z_n \in \ZZ\} = d \cdot \ZZ$.
		\begin{enumerate}[(i)]
			\item Für jedes $h \in H$ gilt also $d \mid h$.
			Nun gilt $a_i = 0 \cdot a_1 + \dots + 0 \cdot a_{i-1} + 1 \cdot a_i + 0 \cdot a_{i+1} + \dots + 0 \cdot a_n \in H$, also $d \mid a_i$.
			\item Wenn $b \in \ZZ$ mit $b \mid a_i$ für alle $i$, so folgt $b \mid h$ für alle $h \in H$.
			Mit $d \in H$ folgt $b \mid d$.
		\end{enumerate}
	\end{beweis}

	\begin{definition}[teilerfremd, koprim]
		Die Zahlen $a_1,\dots,a_n \in \ZZ$ heißen \Index{teilerfremd} oder \Index{koprim}, falls $\ggT(a_1,\dots,a_n) = 1$. \marginnote{[8]}
	\end{definition}

	\begin{satz}
		\label{satz:1.8}
		Seien $a_1,\dots,a_n \in \ZZ$.
		Dann sind äquivalent:
		\begin{enumerate}[(i)]
			\item $\ggT(a_1,\dots,a_n) = 1$.
			\item Es gibt $z_1,\dots,z_n \in \ZZ$ mit $1 = a_1\cdot z_1 + \dots + a_n \cdot z_n$.
		\end{enumerate}
	\end{satz}

	\begin{beweis}
		\begin{description}
			\item[(i) $\Rightarrow$ (ii):] Ist $d = 1 = \ggT(a_1,\dots,a_n)$, so folgt $1 \in H = \{a_1z_1 + \dots + a_nz_n : z_1,\dots,z_n \in \ZZ\}$ nach Definition.
			\item[(ii) $\Rightarrow$ (i):] Es folgt für $d = \ggT(a_1,\dots,a_n)$, dass $d \mid 1$, also $d = 1$. 
		\end{description}
	\end{beweis}

	\begin{korollar}
		\label{kor:1.8.1}
		Sind $a,b,c \in \ZZ$ mit $\ggT(a,b) = 1 = \ggT(a,c)$, so gilt $\ggT(a,bc) = 1$.
	\end{korollar}

	\begin{beweis}
		Schreibe $ax + by = 1 = au + cv$ für geeignete $x,y,u,v \in \ZZ$ (mit vorigem Satz).
		Dann ist
		\begin{align*}
			1 &= (ax + by)(au+cv) \\
			&= a (uax + cvx + byu) + bc(yv).
		\end{align*}
		Wieder mit Satz~\ref{satz:1.8} folgt $\ggT(a,bc) = 1$.
	\end{beweis}

	\begin{korollar}
		Sei $a_1,\dots,a_n,b \in \ZZ$ mit $\ggT(a_i,b) = 1$ für alle $i = 1,\dots,n$.
		Dann gilt $\ggT(a_1\cdot a_2 \cdot \dots \cdot a_n,b) = 1$.
	\end{korollar}

	\begin{beweis}
		Mit Induktion nach $n$:
		\begin{description}
			\item[I.A.:] Für $n = 0$ und $n=1$ ist nichts zu zeigen.
			\item[I.S.:] Gelte die Behauptung für ein $n \in \NN$.
			Es folgt für $a_1,\dots,a_{n+1},b \in \ZZ$ mit $\ggT(a_i,b) = 1$, dass $\ggT(a_1 \cdot \dots \cdot a_n,b)=1$ nach Induktionsvoraussetzung.
			Zusammen mit $\ggT(a_{n+1},b) = 1$ folgt mit Korollar~\ref{kor:1.8.1} die Behauptung.
		\end{description}
	\end{beweis}

	\begin{definition}[Primzahl]
		Eine natürliche Zahl $p \geq 2$ heißt \Index{Primzahl}, wenn $\pm 1$ und $\pm p$ die einzigen Teiler von $p$ sind. \marginnote{[9]}
		Die Menge aller Primzahlen bezeichnen wir mit $\PP = \{2,3,5,7,11,13,17,\dots\}$.
	\end{definition}

	\begin{lemma}
		Für $n \in \NN$ mit $n \geq 2$ setze $p(n) := \min\{d \in \NN : d \geq 2 \text{ und } d \mid n\}$.
		Dann ist $p(n)$ stets eine Primzahl.
	\end{lemma}

	\begin{beweis}
		Angenommen, $d \mid p(n)$ mit $d \geq 2$.
		Dann folgt $d \mid n$.
		Es folgt $p(n) \leq d$, also $p(n) = d$.
	\end{beweis}

	\begin{satz}[Euklid]
		Es gibt unendlich viele Primzahlen. \marginnote{[10]}
	\end{satz}

	\begin{beweis}
		Angenommen, $\PP$ ist endlich, also $\PP= \{p_1 < p_2 < \dots < p_m\}$.
		Betrachte $n = p_1p_2p_3 \cdots p_m + 1$.
		Für alle $i = 1, \dots, m$ gilt dann $p_i \nmid n$, denn sonst wäre auch $p_i \mid p_1p_2\cdots p_n - n = 1$.
		Widerspruch zu $p_i \geq 2$.
		
		Betrachte nun $q := p(n)$.
		Es folgt $p_i \neq q$ für alle $i=1,\dots,n$, aber $q \in \PP$.
		Widerspruch.
	\end{beweis}

	\begin{thm}[Hauptsatz der Arithmetik]
		Sei $n \in \NN$ mit $n \geq 2$. \marginnote{[11]}
		Dann existieren eindeutig bestimmte Primzahlen $p_1 \leq p_2 \leq \dots \leq p_m$ mit $p_1 \cdot p_2 \cdot p_3 \cdots p_m = n$.
		Man nennt die $p_i$ die \textbf{Primfaktoren} von $n$ und das Produkt $p_1 \cdot p_2 \cdots p_n$ die \Index{Primfaktorzerlegung} von $n$.
	\end{thm}

	\begin{beweis}
		\textit{Zur Existenz:} Mit Induktion nach $n$ (Zweites Induktionsprinzip).
		\begin{description}
			\item[I.A.:] Für $n=0$ und $n=1$ ist nichts zu zeigen und $n=2$ ist Primzahl.
			\item[I.S.:] Schreibe $n = p(n) \cdot s$ mit $s \geq 1, p(n) \in \PP$.
			Ist $s = 1$, so ist $n = p(n) \in \PP$.
			Ist $s > 1$, so ist $2 \leq s < n$, weil $p(n) \geq 2$.
			Nach Induktionsvorraussetzung existieren also Primzahlen $q_1 \leq q_2 \leq \dots \leq q_t$ mit $s = q_1 \cdot q_2 \cdots q_t$.
			Damit folgt	$n = p(n) \cdot q_1 \cdot q_2 \cdots q_t$, was (nach eventuellem Umsortieren) eine Primfaktorzerlegung von $n$ ist.
		\end{description}
	
		\textit{Zur Eindeutigkeit:} Schreibe $n = p_1 \cdots p_m$ mit $p_i \in \PP$, $p_1 \leq p_2 \leq \dots \leq p_m$. \marginnote{27.4.}
		Sei $q \in \PP$.
		Sei $\ell$ die Anzahl der Primfaktoren $p_i = q$ (also $\ell = 0$, falls alle $p_i \neq q$).
		Dann folgt $n = q^\ell \cdot s$ und $s$ ist Produkt von Primzahlen $p_j$ mit $p_j \neq q$, also insbesondere mit $\ggT(q,p_j) = 1$.
		Nach Korollar~\ref{kor:1.8.1} gilt dann $\ggT(q,s) = 1$, insbesondere $q \nmid s$.
		Es folgt $q^{\ell+1} \nmid n$, denn aus $n = q^{\ell+1} \cdot t = q^{\ell} \cdot s$ folgt durch Kürzen $q \cdot t = s$.
		Damit haben wir
		\[
			\ell = \max\{k \in \NN : q^k \mid n\}.
		\]
		Diese Zahlen hängen nicht von der gegebenen Primfaktorzerlegung ab.
		Wir wir also definieren
		\[
			\nu_q(n) := \max\{k \in \NN : q^k \mid n\},
		\]
		so folgt $\{p_1,\dots,p_k\} = \{q \in \PP : \nu_q(n) \neq 0\}$ und jedes $p_i$ kommt genau mit der Vielfachheit $\nu_q(n)$ in der Primfaktorzerlegung vor.
	\end{beweis}

	\begin{definition}[$p$-adische Bewertung]
		\label{def:p-adisch}
		Für $n \in \ZZ$ und $q \in \PP$ definieren wir die $\bm{p}$\textbf{-adische Bewertung} von $n$ durch \index{p-adische Bewertung@$pq$-adische Bewertung}
		\[
			\nu_p(n) = \begin{cases}
				\infty, & \text{falls } n = 0 \\
				\max\{k \in \NN : p^k \mid n\}, & \text{falls } n \neq 0.
			\end{cases}
		\]
	\end{definition}

	Damit erhalten wir folgende Umformulierung des Hauptsatzes:
	\begin{satz}[Fundamentalsatz der Arithmetik, 2. Version]
		\label{satz:fundamentalsatz_2}
		Sei $n \in \ZZ$ mit $n \neq 0$.
		Sei
		\[
			\varepsilon(n) = \begin{cases}
				1, & \text{falls } n > 0 \\
				-1, & \text{falls } n < 0.
			\end{cases}
		\]
		Es gilt
		\[
			n = \varepsilon(n) \cdot \prod_{p \in \PP} p^{\nu_p(n)}.
		\]
		Das ist ein unendliches Produkt, in dem aber fast alle Faktoren $1$ sind.
		Also steht rechts ein Produkt aus endlich vielen Primzahlpotenzen.
	\end{satz}
	
	\begin{korollar}
		\label{kor:1.12}
		Sei $a,b \in \ZZ$. \marginnote{[12]}
		Dann sind äquivalent:
		\begin{enumerate}[(i)]
			\item $a \mid b$
			\item Für alle $q \in PP$ gilt $\nu_q(a) \leq \nu_q(b)$.
		\end{enumerate}
	\end{korollar}
	\begin{beweis}
		\begin{description}
			\item[(i) $\Rightarrow$ (ii):] Klar nach Definition von $\nu_q$, denn aus $q^k \mid a$ und $a \mid b$ folgt $q^k \mid b$.
			\item[(ii) $\Rightarrow$ (i):] Ist richtig für $b = 0$.
			Wenn aber $b \neq 0$, so folgt die Behauptung aus Satz~\ref{satz:fundamentalsatz_2}.
		\end{description}
	\end{beweis}

	\begin{lemma}
		\label{lemma:1.13}
		Sei $a,b \in \ZZ$ und $p \in \PP$. \marginnote{[13]}
		Gilt $p \mid ab$, so folgt $p \mid a$ oder $p \mid b$.
	\end{lemma}

	\begin{beweis}
		Angenommen, $p \nmid a$ und $p \nmid b$.
		Dann ist $\ggT(p,a) = 1 = \ggT(p,b)$ mit $p \in \PP$, also $\ggT(p,ab) = 1$ nach \ref{kor:1.8.1} und damit $p \nmid ab$.
	\end{beweis}

	Wir machen als Anwendung einen kleinen Exkurs über \textbf{vollkommene Zahlen}. \index{vollkommene Zahl}
	Eine Zahl $n \in \NN$ heißt \textbf{vollkommen}, falls $n$ die Summe aller echten positiven Teiler von $n$ ist.
	In der Antike interessiert man sich aus Gründen der Mystik für solche Zahlen.
	Zum Beispiel ist $6 = 1 + 2 + 3$ vollkommen und $8 \neq 1 + 2 + 4$ nicht vollkommen.
	\newpage
	\begin{definition}
		Für $n \in \NN, n \geq 1$ sei \marginnote{[14]}
		\[
			\sigma(n) = \sum_{\substack{k \geq 0 \\ k \mid n}} k
		\]
		die Summe aller positiven Teiler von $n$.
		Weiter sei
		\[
			\tau(n) = \sum_{\substack{k \geq 0} \\ k \mid n} 1
		\]
		die Anzahl aller positiven Teiler von $n$.
	\end{definition}
	Also gilt $n \in \PP \Leftrightarrow \tau(n)$ und $n$ vollkommen $\Leftrightarrow \sigma(n) = 2n$.
	
	\begin{center}
			\begin{tabular}{|ccc|}
			\hline 
			$\bm{n}$ & $\bm{\tau(n)}$ & $\bm{\sigma(n)}$ \\ 
			\hline 
			1 & 1 & 1 \\ 
			\hline 
			2 & 2 & 3 \\ 
			\hline 
			3 & 2 & 4 \\ 
			\hline 
			4 & 3 & 7 \\ 
			\hline 
			5 & 2 & 6 \\ 
			\hline 
			6 & 4 & 12 \\ 
			\hline 
		\end{tabular}
	\end{center}

	\begin{lemma}
		Es gilt:
		\[
			\tau(n) = \prod_{q \in \PP} (1+ \nu_q(n))
		\]
	\end{lemma}

	\begin{beweis}
		Folgt direkt aus Korollar~\ref{kor:1.12}.
	\end{beweis}
	
	Hieraus folgt unmittelbar: Ist $\ggT(a,b) = 1$ für $a,b \in \NN, a,b \geq 1$, so gilt $\tau(ab) = \tau(a) \cdot \tau(b)$.
	
	\begin{lemma}
		Es gilt
		\[
			\sigma(n) = \prod_{q \in \PP} \sigma \enb{q^{\nu_q(n)}} = \prod_{q \in \PP} \frac{q^{1+ \nu_q(n)}-1}{q-1}.
		\]
	\end{lemma}

	\begin{beweis}
		Mit Korollar~\ref{kor:1.12} und der geometrischen Summefolgt
		\begin{align*}
			\sigma(n) &= \sum_{\ell_q = 0}^{\nu_q(n)} \prod_{q \in \PP} q^{\ell_q} = \prod_{q \in \PP} \sum_{\ell_q = 0}^{\nu_q(n)} q^{\ell_q} \\
			&= \prod_{q \in \PP} \sigma \enb{q^{\nu_q(n)}} = \prod_{q \in \PP} \frac{q^{\nu_q(n)+1}-1}{q-1}.
		\end{align*}
	\end{beweis}

	Daraus folgt: Wenn $\ggT(a,b) = 1$ und $a,b \geq 1$, so gilt $\sigma(ab) = \sigma(a) \cdot \sigma(b)$.
	
	\begin{thm}[Euklid, Euler]
		\label{thm:1.15}
		Sei $n \geq 2$ gerade. \marginnote{[15]}
		Dann sind äquivalent:
		\begin{enumerate}[(i)]
			\item $n$ ist vollkommen.
			\item $n = 2^{k-1} \cdot (2^k - 1)$ für ein $k \geq 2$ und $2^k - 1 \in \PP$.
		\end{enumerate}
	\end{thm}

	\begin{beweis}
		\begin{description}
			\item[(ii) $\Rightarrow$ (i):] Angenommen, $n = 2^{k-1} \cdot q$ mit $k \geq 2$, $q = 2^k-1 \geq 3$. \marginnote{4.5.}
			Dann:
			\begin{align*}
				\sigma(n) &= \sigma \enb{2^{k-1}} \cdot \sigma(q) = \frac{2^k-1}{2-1} \cdot (1+q) \\
				&= (2^k-1) \cdot 2^k = 2 \cdot n.
			\end{align*}  
			\item[(i) $\Rightarrow$ (ii):] Schreibe $n = 2^{k-1} \cdot m$ mit $m$ ungerade.
			Dann ist $k \geq 2$, da $n$ gerade ist.
			Also folgt, da $n$ vollkommen ist:
			\[
				\sigma(n) = 2n = \frac{2^k-1}{2-1} \cdot \sigma(m) = 2^k \cdot m.
			\] 
			Es folgt $2^k \mid \sigma(m)$, das heißt $\sigma(n) = 2^k \cdot l$ für ein $l \in \NN$, und damit:
			\begin{align*}
				2n = 2^k \cdot m &= (2^k - 1) \cdot 2^k \cdot l \\
				m &= (2^k - 1) \cdot l.
			\end{align*}
			Nun ist $l=1$, denn sonst sind $1, 2^k-1$ und $l$ verschiedene Teiler von $m$ und damit $\sigma(m) \geq 1+(2^k-1)+l+(2^k-1)\cdot l > 2^k \cdot l = \sigma(m)$.
			Also ist $m = 2^k-1$ und damit $\sigma(m) = m+1$, also $m \in \PP$.
		\end{description}
	\end{beweis}

	\begin{bemerkung}
		\begin{enumerate}[(1)]
			\item Es ist unbekannt, ob es ungerade vollkommene Zahlen gibt (Wenn, dann müssen sie größer sein als $10^{1500}$). \marginnote{[16]}
			\item Primzahlen der Form $2^k-1$ heißen \textbf{Mersennesche Primzahlen}. \index{Mersenne-Primzahl}
			Zur Zeit sind $49$ Mersennesche Primzahlen bekannt.
			Ob es unendlich viele Mersennesche Primzahlen gibt, ist unbekannt.
			\item Wenn $2^k-1$ eine Primzahl ist, dann ist auch $k$ eine Primzahl, denn:
			Wenn $k = uv$ mit $u,v > 1, u,v \in \NN$, so ist
			\[
				2^{uv} - 1 = (2^u)^v - 1 = \underbrace{\enb{\sum\limits_{j=0}^{v-1} 2^j}}_{>1} \cdot \underbrace{(2^u-1)}_{>1} \notin \PP.
			\]
			Aber: $2^{11} - 1 = 23 \cdot 89$.
			Das Kriterium ist nur notwendig, aber nicht hinreichend.
		\end{enumerate}
	\end{bemerkung}

	\subsection*{Euklids Algorithmus}
	\begin{lemma}
		Sei $a,b,m \in \ZZ$. \marginnote{[17]}
		Dann gilt $\ggT(a,b) = \ggT(a,b+ma)$.
	\end{lemma}

	\begin{beweis}
		Sei $K = \{ax+by : x,y \in \ZZ\}$ und $L = \{au+(b+ma)v : u,v \in \ZZ\}$.
		Dann gilt $K = L$, denn:
		\begin{align*}
			ax + by &= au + (b+ma) \cdot v \\
			\Leftrightarrow \quad ax + by &= a \cdot (u+mv) + bv
		\end{align*}
		ist immer lösbar mit $v = y$.
		Also gilt $\ggT(a,b) = \ggT(a,b+ma)$.
	\end{beweis}

	\begin{quote}
		Wenn $b$ aber $a$ nicht misst und man nimmt von $a,b$ abwechselnd immer das Kleinere vom Größeren weg, dann muss schließlich eine Zahl übrigbleiben, die alle vorangegangenen misst.
		
		{\footnotesize --- Euklid}
	\end{quote}
	\newpage
	\begin{algo}
		Input: ganze Zahlen $a,b$ -- Output: $\ggT(a,b)$ \index{Euklidischer Algorithmus}
		
\begin{lstlisting}
	ggT(a,b){
		if a=0 return |b|					$\textcolor{black!50}{\ggT(0,b) = |b|}$
		if b=0 return |a|					$\textcolor{black!50}{\ggT(a,0) = |a|}$
		a $\leftarrow$ |a|
		b $\leftarrow$ |b|								$\textcolor{black!50}{\ggT(a,b) = \ggT(|a|,|b|)}$
		while a $\neq$ b {
			if a > b:	a $\leftarrow$ a-b				$\textcolor{black!50}{\ggT(a,b) = \ggT(a-b,b)}$
			if b > a:	b $\leftarrow$ b-a				$\textcolor{black!50}{\ggT(a,b) = \ggT(a,b-a)}$
		}
		return a
	}
\end{lstlisting}
		In jedem Durchlauf der \texttt{while}-Bedingung wird $|a-b|$ echt verringert.
		Also sind $a$ und $b$ nach endlich vielen Durchläufen gleich.
		Dann gilt $a=b=\ggT(a,b)$.
		
		Beachte: An arithmetischen Operationen wird nur Subtraktion benutzt.
	\end{algo}

	\begin{bemerkung}
		Der euklidische Algorithmus wird oft anders formuliert: \marginnote{[18]}
		Gegeben sind ganze Zahlen $r_0, r_1$ mit $r_1 \neq 0$.
		Gesucht ist der $\ggT(r_1,r_0)$.
		Verwende Teilen mit Rest, vgl. Satz~\ref{satz:teilen mit rest}.
		Schreibe $r_i = s_i \cdot r_{i+1} + r_{i+2}$ mit $0 \leq r_{i+2} < |r_{i+1}|$:
		
		\[
			\begin{array}{rclcl}
				r_0 = s_0 \cdot r_1 + r_2 & \text{mit} & 0 \leq r_2 < |r_1| && \textcolor{black!50}{\ggT(r_0,r_1)} \\
				r_1 = s_1 \cdot r_2 + r_3 & \text{mit} & 0 \leq r_3 < |r_2| && \textcolor{black!50}{= \ggT(r_1,r_2)} \\
				r_2 = s_2 \cdot r_3 + r_4 & \text{mit} & 0 \leq r_4 < |r_3| && \textcolor{black!50}{= \ggT(r_2,r_3)} \\
				 & \vdots & && \textcolor{black!50}{\vdots} \\
				r_k = s_k \cdot r_{k+1} + r_{k+2} & \text{mit} & 0 \leq r_{k+2} < |r_{k+1}| && = \textcolor{black!50}{\ggT(r_k,r_{k+1})}\\
				r_{k+1} = s_{k+1} \cdot \textcolor{red}{r_{k+2}} + 0 &&&& \textcolor{black!50}{=\ggT(r_{k+1},r_{k+2}) =} \ \textcolor{red}{r_{k+2}}
			\end{array}
		\]
		
		Rückwärtseinsetzen liefert $d = \ggT(r_0,r_1)$ als ganzzahlige Linearkombination aus $r_0$ und $r_1$:
		\begin{align*}
			r_k - s_k \cdot r_{k+1} &= r_{k+2} = d \\
			r_{k-1} - s_{k-1} \cdot r_k &= r_{k+1} \\
			&\vdots \\
			r_0 - s_0 \cdot r_1 &= r_2
		\end{align*}
	\end{bemerkung}

	\begin{beispiel}
		Berechne $d = \ggT(343,280)$ und löse $343x + 280y = d$ mit $x,y \in \ZZ$.
		
		\begin{align*}
			343 &= 1 \cdot 280 + 63 \\
			280 &= 4 \cdot 63  + 28 \\
			63 &= 2 \cdot 28 + 7 \\
			28 &= 4 \cdot 7 + 0
		\end{align*}
		
		Also ist $d = 7$.
		Jetzt rückwärts:
		\begin{align*}
			7 &= 63 - 2 \cdot 28 \\
			&= 63 - 2 \cdot (280 - 4 \cdot 63) \\
			&= 9 \cdot 63 + (-2) \cdot 280 \\
			&= 9 \cdot (343 - 280) + (-2) \cdot 280 \\
			&= 9 \cdot 343 + (-11) \cdot 280
		\end{align*}
	\end{beispiel}
\cleardoubleoddemptypage