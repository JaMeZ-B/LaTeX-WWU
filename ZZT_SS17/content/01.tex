%!TEX root = ../ZZT_SS17.tex
\section{Teiler, Primzahlen und der Hauptsatz der Arithmetik}
\label{sec:1}
	\subsection{Erinnerung}
	Wir betrachten die Menge der \textbf{natürlichen Zahlen} \index{natürliche Zahl} \marginnote{[1] \\ 20.4.}
	\[
		\NN= \{0,1,2,3,\dots\},
	\]
	die Menge der \textbf{ganzen Zahlen} \index{ganze Zahl}
	\[
		\ZZ= \{0, \pm 1, \pm 2, \pm 3, \dots\}
	\]
	mit Addition $+$ und Multiplikation $\cdot$, mit Anordnung $\leq$ und dem \Index{Absolutbetrag}
	\[
		\abs{z} = \max\{z,-z\}.
	\]
	In $\NN$ ist die Subtraktion nur eingeschränkt möglich und in $\ZZ$ ist die Division nur eingeschränkt möglich.
	Aber wir können kürzen:
	\begin{gather}
		a+x = a+y \quad \Rightarrow \quad x = y \qquad \text{in } \NN \text{ und } \ZZ \\
		ax = ay \quad \Rightarrow \quad x=y \qquad \text{in } \NN \text{ und } \ZZ, a \neq 0.
	\end{gather}
	
	\subsection{Induktion und Wohlordnung}
	\begin{satz}[Erstes Induktionsprinzip]
	\label{satz:IP1}
	Ist $S \subseteq \NN$ mit \index{erstes Induktionsprinzip} \marginnote{[2]}
	\begin{enumerate}[(i)]
		\item $0 \in S$,
		\item $s \in S \Rightarrow s+1 \in S$,
	\end{enumerate}
	dann gilt $S = \NN$.	
	\end{satz}
	
	Anschauung: Alle natürlichen Zahlen erreicht man durch Zählen, ausgehend von $0$.
	
	Aus dem ersten Induktionsprinzip folgt das \Index{Wohlordnungsprinzip}
	
	\begin{satz}[Wohlordnungsprinzip]
		\label{satz:WOP}
		Ist $S \subseteq \NN$ mit $S \neq \emptyset$, so gibt es in $S$ ein kleinstes Element $m = \min(S)$.
	\end{satz}
	
	\begin{beweis}
		Sei $S \subseteq \NN$ mit $S \neq \emptyset$.
		Sei $T = \{t \in \NN : t \leq s \text{ für alle } s \in S\}$.
		Es folgt $0 \in T$.
		Ist $s \in S$ so ist $s+1 \notin T$, denn $s+1 \nleq s$.
		Es folgt $T \neq \NN$.
		Nach dem ersten Induktionsprinzip folgt: Es gibt $t \in T$ mit $t+1 \notin T$.
		
		Nun gilt $t \in S$. Denn wäre $t \notin S$, so wäre $t < s$ für alle $s \in S$, also $t+1 \leq s$ für alle $s \in S$. Damit wäre $t+1 \in T$. \lightning
		
		Also ist $t$ ein kleinstes Element in $S$.
	\end{beweis}
	
	Aus dem Wohlordnungsprinzip folgt wiederum das \textbf{zweite Induktionsprinzip}:
	
	\begin{satz}[Zweites Induktionsprinzip]
		\label{satz:IP2}
		Ist $S \subseteq \NN$ mit
		\begin{enumerate}[(i)]
			\item $0 \in S$,
			\item $s \in S \Leftrightarrow$ für alle $t \in \NN$ mit $t < s$ ist $t \in S$,
		\end{enumerate}
		so gilt $S = \NN$.
	\end{satz}
	
	\begin{beweis}
		Angenommen, $S \subseteq \NN$ erfüllt (i) und (ii), und angenommen $S \neq \NN$.
		Betrachte $R = \NN \setminus S \neq \emptyset$.
		Sei $r := \min(R)$ (existiert nach dem Wohlordnungsprinzip). Für alle $t < r$ gilt also $t \in S$. Nach (ii) folgt somit $r \in S$. Widerspruch, da $r \in R = \NN \setminus S$.
	\end{beweis}

	Vorsicht: Die Sätze \ref{satz:IP1}, \ref{satz:WOP} und \ref{satz:IP2} gelten für $\NN$, aber nicht für $\ZZ$!
	Zum Beispiel hat $\ZZ \subseteq \ZZ$ kein kleinstes Element.
	Es gilt aber Folgendes:
	
	\begin{lemma}
		\label{lemma:1.3}
		Sei $S \subseteq \ZZ$ und sei $S \neq \emptyset$. \marginnote{[3]}
		Wenn $S$ eine untere (bzw. obere) Schranke hat, so hat $S$ auch ein kleinstes (bzw. größtes) Element.
	\end{lemma}
	
	\begin{beweis}
		Sei $k$ untere Schranke für $S$, das heißt $k \leq s$ für alle $s \in S$.
		Betrachte die Menge $S' := \{s - k : s \in S\} \subseteq \NN$, $S' \neq \emptyset$.
		Sei $l = \min(S')$.
		Dann ist $l+k$ ein Minimum für $S$.
		
		Sei nun $m$ eine obere Schranke für $S \subseteq \ZZ$, sei $S'' = \{-s : s \in S\}$.
		Dann ist $-m$ eine untere Schranke für $S''$.
		Sei $n = \min(S'')$.
		Damit ist $-n$ ein größtes Element in $S$. 
	\end{beweis}

	Wir betrachten jetzt Teilbarkeit.
	
	\begin{definition}[Teiler]
		\label{def:teiler}
		Sei $a,b \in \ZZ$.
		Wenn es ein $m \in \ZZ$ gibt mit $ma = b$, so heißt $a$ ein \Index{Teiler} von $b$. \marginnote{[4]}
		Man schreibt $a \mid b$, andernfalls $a \nmid b$. \marginnote{lies: \enquote{$a$ teilt $b$ (nicht)}}
	\end{definition}

	\begin{beispiel}
		Es gilt $3 \mid 6$ und $3 \nmid 7$.
	\end{beispiel}

	\begin{satz}[Rechenregeln für Teilbarkeit]
		\label{satz:rechenregeln}
		\begin{enumerate}[(i)]
			\item $1 \mid a$, $a \mid a$, $a \mid 0$.
			\item $a \mid b$ und $b \mid c \Rightarrow a \mid c$
			\item $a \mid b$ und $b \mid a \Rightarrow a = \pm b$
			\item $a \mid b$ und $a \mid c \Rightarrow a \mid b \pm c$
			\item $a \neq 0$ und $ab \mid ac \Rightarrow b \mid c$
		\end{enumerate}
	\end{satz}

	\begin{beweis}
		\begin{enumerate}[(i)]
			\item klar.
			\item Gilt $am = b$ und $bn = c$, so gilt $amn = $.
			\item Gilt $am = b$ und $bn = a$, so gilt $bmn = b$, und damit folgt $a = b = 0$ oder $mn = 1$, also $m,n = \pm 1$.
			\item Gilt $am = b$ und $an = c$, so folgt $am \pm an = a(m \pm n) = b \pm c$.
			\item Ist $a \neq 0$ und $abm = ac$, so folgt $bm = c$.
		\end{enumerate}
	\end{beweis}

	\begin{satz}[Teilen mit Rest]
		\label{satz:teilen mit rest}
		Sei $a,b \in \ZZ$, sei $b \neq 0$. \index{Teilen mit Rest} \marginnote{[5]}
		Dann existieren eindeutige Zahlen $r,s \in \ZZ$ mit $a = bs + r$ und $0 \leq r < \abs{b}$.
	\end{satz}

	\begin{beweis}
		\textit{Zur Eindeutigkeit:} Angenommen, $a = bs + r = bs' + r'$ mit $s,s',r,r' \in \ZZ$ und $0 \leq r,r' < \abs{b}$.
		Ohne Einschränkung sei $r' \geq r$.
		Umstellen der Gleichungen liefert
		\[
			\abs{b} > \underbrace{r' - r}_{\geq 0} = b(s - s').
		\]
		Also folgt $b \mid (r'-r)$ und $\abs{r' - r} < b$, das heißt $r' - r = 0$ und damit $s = s'$.
		
		\textit{Zur Existenz:} Sei $S = \{k \in \ZZ : k \cdot \abs{b} \leq a\}$. \marginnote{24.4.}
		Dann gilt $-\abs{a} \in S$, also $S \neq \emptyset$.
		Weiter ist $\abs{a}$ eine obere Schranke für $S$.
		Folglich existiert in $S$ ein größtes Element $s = \max(S)$.
		Es folgt $s \cdot \abs{b} \leq a$ und $(s+1) \cdot \abs{b} > a$.
		Wähle $r \in \NN$ mit $s \cdot \abs{s} + r = a$, also $s \cdot \abs{b} + r < (s-1) \cdot \abs{b}$ und damit $0 \leq r < \abs{b}$.
		Für $b>0$ ist $\abs{b} = b$ und $s \cdot b + r = a$.
		Für $b<0$ ist $\abs{b} = -b$ und $-s \cdot b + r = a$.
	\end{beweis}

	\begin{satz}
		Sei $H \subseteq \ZZ$ eine Teilmenge  mit \marginnote{[6]}
		\begin{enumerate}[(i)]
			\item $H \neq \emptyset$
			\item $x,y \in H \quad \Rightarrow \quad x-y \in H$.
		\end{enumerate}
		Dann gibt es ein eindeutig bestimmtes $d \in \NN$ mit $H = d \cdot \ZZ = \{d \cdot k : k \in \ZZ\}$.
	\end{satz}

	\begin{beweis}
		Sei $x \in H$.
		Dann ist $x-x = 0 \in H$ und $0-x = -x \in H$.
		Folglich ist für $x,y \in H$ auch $x -(-y) = x+y \in H$.
		
		Falls $H = \{0\}$, so gilt $0 \cdot \ZZ = \setzero$ und es ist eindeutig $d = 0$.
		
		Falls $H \neq \setzero$, so gibt es ein Element $h \in H$ mit $h > 0$.
		Sei $S = \{h \in H : h > 0\} \neq \emptyset$.
		Setze $d = \min(S)$.
		Ist nun $h \in H$ beliebig, schreibe $a = s\cdot d + r$ mit $0 \leq r < \abs{d}$ (Teilen mit Rest).
		Dann gilt $s \cdot d \in H$ und damit $r = h-sd \in H$.
		Nach Wahl von $d = \min(S)$ folgt $r = 0$, das heißt $h = d \cdot s \in d \cdot \ZZ$.
		
		Ist $H = d' \cdot \ZZ$ mit $d' \in \NN$, so folgt $d' = \min\{h \in H : h > 0\} = \min(S) = d$.
	\end{beweis}

	\begin{korollar}
		\label{kor:satz_1.6}
		Seien $a_1,\dots,a_n \in \ZZ, n \geq 1$ und sei
		\[
			H = \{z_1a_1 + z_2a_2 + \dots + z_na_n : z_1,\dots,z_n \in \ZZ\}
		\]
		die Menge aller ganzzahligen Linearkombinationen der Zahlen $a_1,\dots,a_n$.
		Dann gibt es genau ein $d \in \NN$ mit $H = d \cdot \ZZ$.
	\end{korollar}

	\begin{beweis}
		Es gilt $a_1 \cdot 1 + a_2 \cdot 0 + \dots + a_n \cdot 0 \in H \neq \setzero$ und $(a_1z_1 + \dots + a_nz_n) - (a_1z_1' + \dots + a_nz_n') = a_1(z_1-z_1') + \dots + a_n(z_n-z_n') \in H$.
		Nach dem vorigen Satz folgt $H = d \cdot \ZZ$ mit $d$ eindeutig.		
	\end{beweis}
	\newpage
	\begin{definition}[größter gemeinsamer Teiler]
		Die Zahl $d \in \NN$ aus Korollar~\ref{kor:satz_1.6} heißt der \textbf{größte gemeinsame Teiler} von $a_1,\dots,a_n$. \index{größter gemeinsamer Teiler}
		Wir schreiben:
		\[
			\ggT(a_1,\dots,a_n) = d.
		\]
	\end{definition}

	Der Name \enquote{größter gemeinsamer Teiler} ist berechtigt und passt zur Definition aus der Schule:
	
	\begin{lemma}
		\label{lem:1.7}
		Sei $a_1,\dots,a_n \in \ZZ$ und $d = \ggT(a_1,\dots,a_n)$. \marginnote{[7]}
		Dann gilt:
		\begin{enumerate}[(i)]
			\item $d \mid a_i$ für alle $i = 1,\dots,n$.
			\item Ist $b \in \ZZ$ mit $b \mid a_i$ für alle $i=1,\dots,n$, dann gilt $b \mid d$ (das heißt, jeder gemeinsame Teiler der $a_i$ teilt auch $d$).
		\end{enumerate}
	\end{lemma}

	\begin{beweis}
		Sei $H = \{a_1z_1 + \dots + a_nz_n : z_1,\dots,z_n \in \ZZ\} = d \cdot \ZZ$.
		\begin{enumerate}[(i)]
			\item Für jedes $h \in H$ gilt also $d \mid h$.
			Nun gilt $a_i = 0 \cdot a_1 + \dots + 0 \cdot a_{i-1} + 1 \cdot a_i + 0 \cdot a_{i+1} + \dots + 0 \cdot a_n \in H$, also $d \mid a_i$.
			\item Wenn $b \in \ZZ$ mit $b \mid a_i$ für alle $i$, so folgt $b \mid h$ für alle $h \in H$.
			Mit $d \in H$ folgt $b \mid d$.
		\end{enumerate}
	\end{beweis}

	\begin{definition}[teilerfremd, koprim]
		Die Zahlen $a_1,\dots,a_n \in \ZZ$ heißen \Index{teilerfremd} oder \Index{koprim}, falls $\ggT(a_1,\dots,a_n) = 1$. \marginnote{[8]}
	\end{definition}

	\begin{satz}
		\label{satz:1.8}
		Seien $a_1,\dots,a_n \in \ZZ$.
		Dann sind äquivalent:
		\begin{enumerate}[(i)]
			\item $\ggT(a_1,\dots,a_n) = 1$.
			\item Es gibt $z_1,\dots,z_n \in \ZZ$ mit $1 = a_1\cdot z_1 + \dots + a_n \cdot z_n$.
		\end{enumerate}
	\end{satz}

	\begin{beweis}
		\begin{description}
			\item[(i) $\Rightarrow$ (ii):] Ist $d = 1 = \ggT(a_1,\dots,a_n)$, so folgt $1 \in H = \{a_1z_1 + \dots + a_nz_n : z_1,\dots,z_n \in \ZZ\}$ nach Definition.
			\item[(ii) $\Rightarrow$ (i):] Es folgt für $d = \ggT(a_1,\dots,a_n)$, dass $d \mid 1$, also $d = 1$. 
		\end{description}
	\end{beweis}

	\begin{korollar}
		\label{kor:1.8.1}
		Sind $a,b,c \in \ZZ$ mit $\ggT(a,b) = 1 = \ggT(a,c)$, so gilt $\ggT(a,bc) = 1$.
	\end{korollar}

	\begin{beweis}
		Schreibe $ax + by = 1 = au + cv$ für geeignete $x,y,u,v \in \ZZ$ (mit vorigem Satz).
		Dann ist
		\begin{align*}
			1 &= (ax + by)(au+cv) \\
			&= a (uax + cvx + byu) + bc(yv).
		\end{align*}
		Wieder mit Satz~\ref{satz:1.8} folgt $\ggT(a,bc) = 1$.
	\end{beweis}

	\begin{korollar}
		Sei $a_1,\dots,a_n,b \in \ZZ$ mit $\ggT(a_i,b) = 1$ für alle $i = 1,\dots,n$.
		Dann gilt $\ggT(a_1\cdot a_2 \cdot \dots \cdot a_n,b) = 1$.
	\end{korollar}

	\begin{beweis}
		Mit Induktion nach $n$:
		\begin{description}
			\item[I.A.:] Für $n = 0$ und $n=1$ ist nichts zu zeigen.
			\item[I.S.:] Gelte die Behauptung für ein $n \in \NN$.
			Es folgt für $a_1,\dots,a_{n+1},b \in \ZZ$ mit $\ggT(a_i,b) = 1$, dass $\ggT(a_1 \cdot \dots \cdot a_n,b)=1$ nach Induktionsvoraussetzung.
			Zusammen mit $\ggT(a_{n+1},b) = 1$ folgt mit Korollar~\ref{kor:1.8.1} die Behauptung.
		\end{description}
	\end{beweis}

	\begin{definition}[Primzahl]
		Eine natürliche Zahl $p \geq 2$ heißt \Index{Primzahl}, wenn $\pm 1$ und $\pm p$ die einzigen Teiler von $p$ sind. \marginnote{[9]}
		Die Menge aller Primzahlen bezeichnen wir mit $\PP = \{2,3,5,7,11,13,17,\dots\}$.
	\end{definition}

	\begin{lemma}
		Für $n \in \NN$ mit $n \geq 2$ setze $p(n) := \min\{d \in \NN : d \geq 2 \text{ und } d \mid n\}$.
		Dann ist $p(n)$ stets eine Primzahl.
	\end{lemma}

	\begin{beweis}
		Angenommen, $d \mid p(n)$ mit $d \geq 2$.
		Dann folgt $d \mid n$.
		Es folgt $p(n) \leq d$, also $p(n) = d$.
	\end{beweis}

	\begin{satz}[Euklid]
		Es gibt unendlich viele Primzahlen. \marginnote{[10]}
	\end{satz}

	\begin{beweis}
		Angenommen, $\PP$ ist endlich, also $\PP= \{p_1 < p_2 < \dots < p_m\}$.
		Betrachte $n = p_1p_2p_3 \cdots p_m + 1$.
		Für alle $i = 1, \dots, m$ gilt dann $p_i \nmid n$, denn sonst wäre auch $p_i \mid p_1p_2\cdots p_n - n = 1$.
		Widerspruch zu $p_i \geq 2$.
		
		Betrachte nun $q := p(n)$.
		Es folgt $p_i \neq q$ für alle $i=1,\dots,n$, aber $q \in \PP$.
		Widerspruch.
	\end{beweis}

	\begin{thm}[Hauptsatz der Arithmetik]
		Sei $n \in \NN$ mit $n \geq 2$. \marginnote{[11]}
		Dann existieren eindeutig bestimmte Primzahlen $p_1 \leq p_2 \leq \dots \leq p_m$ mit $p_1 \cdot p_2 \cdot p_3 \cdots p_m = n$.
		Man nennt die $p_i$ die \textbf{Primfaktoren} von $n$ und das Produkt $p_1 \cdot p_2 \cdots p_n$ die \Index{Primfaktorzerlegung} von $n$.
	\end{thm}

	\begin{beweis}
		\textit{Zur Existenz:} Mit Induktion nach $n$ (Zweites Induktionsprinzip).
		\begin{description}
			\item[I.A.:] Für $n=0$ und $n=1$ ist nichts zu zeigen und $n=2$ ist Primzahl.
			\item[I.S.:] Schreibe $n = p(n) \cdot s$ mit $s \geq 1, p(n) \in \PP$.
			Ist $s = 1$, so ist $n = p(n) \in \PP$.
			Ist $s > 1$, so ist $2 \leq s < n$, weil $p(n) \geq 2$.
			Nach Induktionsvorraussetzung existieren also Primzahlen $q_1 \leq q_2 \leq \dots \leq q_t$ mit $s = q_1 \cdot q_2 \cdots q_t$.
			Damit folgt	$n = p(n) \cdot q_1 \cdot q_2 \cdots q_t$, was (nach eventuellem Umsortieren) eine Primfaktorzerlegung von $n$ ist.
		\end{description}
	
		\textit{Zur Eindeutigkeit:} Schreibe $n = p_1 \cdots p_m$ mit $p_i \in \PP$, $p_1 \leq p_2 \leq \dots \leq p_m$.
		Sei $q \in \PP$.
		Sei $l$ die Anzahl der Primfaktoren $p_i = q$.
	\end{beweis}
\cleardoubleoddemptypage