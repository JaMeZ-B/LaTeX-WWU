%!TEX root = ../ZZT_SS17.tex
\begin{abstract}
	\section*{Vorwort}
	\label{sec:preface}
	Der vorliegende Text ist eine Mitschrift zur Vorlesung \textit{Zahlen und elementare Zahlentheorie}, gelesen von Prof. Dr. Linus Kramer an der WWU Münster im Sommersemester 2017. Der Inhalt entspricht weitestgehend den Vorlesungsnotizen, welche auf der Vorlesungswebsite bereitgestellt werden. Dieses Werk ist daher keine Eigenleistung des Autors und wird nicht vom Dozenten der Veranstaltung korrekturgelesen. Für die Korrektheit und Vollständigkeit des Inhalts wird keinerlei Garantie übernommen. Bemerkungen, Korrekturen und Ergänzungen kann man folgenderweise loswerden:
	\begin{itemize}
		\item persönlich durch Überreichen von Notizen oder per E-Mail
		\item durch Abändern der entsprechenden \TeX-Dateien und Versand per E-Mail an mich
		\item direktes Mitarbeiten via GitLab. Dieses Skript befindet sich im \texttt{LaTeX-WWU}-Repository von Jannes Bantje:
		\begin{center}
			\url{https://gitlab.com/JaMeZ-B/LaTeX-WWU}
		\end{center}
	\end{itemize}

	\subsection*{Literatur}
	\label{sub:literatur}
	\begin{itemize}
		\item \textsc{Schmidt}: Einführung in die elementare Zahlentheorie (\href{http://link.springer.com/book/10.1007/978-3-540-45974-3}{Springer-Link}) \cite{Schmidt}
		\item \textsc{Bundschuh}: Einführung in die Zahlentheorie (\href{http://www.springerlink.com/content/978-3-540-76490-8}{Springer-Link}) \cite{Bundschuh}
		\item \textsc{Dudley}: Elementary number theory \cite{Dudley}
		\item \textsc{Apostol}: Introduction to analytic number theory \cite{Apostol}
		\item \textsc{Remmert, Ullrich}: Elementare Zahlentheorie \cite{RemmertUllrich}
		\item \textsc{Weil}: Number theory for beginners \cite{Weil}
		\item \textsc{Landau}: Vorlesungen über Zahlentheorie \cite{Landau}
		\item \textsc{Jacobson}: Basic algebra I \cite{Jacobson}
	\end{itemize}

	\subsection*{Kommentar des Dozenten}
	\label{sub:kommentar}
	In der Vorlesung beschäftigen wir uns mit Zahlen.
	Im ersten Teil der Vorlesung wiederholen wir ganz kurz den Aufbau des Zahlsystems: die ganzen, die rationalen, die reellen und die komplexen Zahlen.
	Wir werden uns dann mit irrationalen und mit transzendenten Zahlen beschäftigen und zum Beispiel beweisen, dass die Zahl $\pi = 3.14159\dots$ transzendent ist.
	Dazu benötigen wir Primzahlen und den Hauptsatz der Arithmetik.
	Im zweiten Teil der Vorlesung wird es dann vor allem um ganze Zahlen, um Kongruenzen und das Rechnen mit Kongruenzklassen gehen.
	Nebenher wiederholen wir dabei einige aus den Anfängervorlesungen vertraute algebraische Strukturen wie Gruppen, Ringe und Körper.
	
	Zielgruppe der Vorlesung sind Studentinnen und Studenten im 2-Fach-Bachelor Mathematik.

	\subsection*{Vorlesungswebsite}
	\label{sub:link}
	Das handgeschriebene Skript sowie weiteres Material findet man unter folgendem Link:
	\begin{center}
		\url{http://wwwmath.uni-muenster.de/u/ag_kramer/index.php?name=VorlesungZahlenUndZahlentheorie17&menu=teach&lang=de}
	\end{center}
	
	\vfill
	\begin{flushright}
		Phil Steinhorst \\
		p.st@wwu.de
	\end{flushright}
\end{abstract}