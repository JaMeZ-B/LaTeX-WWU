%!TEX root = ZZT_SS17.tex
% Author: Phil Steinhorst, p.st@wwu.de

% Abk�rzungen
% ===========================================================
	\newcommand{\CC}{\mathbb{C}}
	\newcommand{\EE}{\mathbb{E}}
	\newcommand{\FF}{\mathbb{F}}
	\newcommand{\HH}{\mathcal{H}}
	\newcommand{\KK}{\mathbb{K}}
	\newcommand{\LL}{\mathbb{L}}
	\newcommand{\NN}{\mathbb{N}}
	\newcommand{\QQ}{\mathbb{Q}}
	\newcommand{\RR}{\mathbb{R}}
	\newcommand{\ZZ}{\mathbb{Z}}
	\newcommand{\oh}{\mathcal{O}}				% Landau-O
	\newcommand{\ind}{1\hspace{-0,8ex}1} 		% Indikatorfunktion (Doppeleins)
	\newcommand{\bewrueck}{"$\Leftarrow$":} 	% Beweis R�ckrichtung
	\newcommand{\bewhin}{"$\Rightarrow$":}		% Beweis Hinrichtung
	\newcommand{\setone}{\{1\}}					% Einsmenge
	\newcommand{\NT}{\trianglelefteq}			% Normalteiler
	\newcommand{\setzero}{\{0\}}				% Nullmenge
	\newcommand{\ol}[1]{\overline{#1}}
	\newcommand{\wt}[1]{\widetilde{#1}}
	\newcommand{\wh}[1]{\widehat{#1}}
% ===========================================================

% Operatoren
% ===========================================================
	\DeclareMathOperator{\Abb}{Abb}				% Menge der Abbildungen
	\DeclareMathOperator{\Bild}{Bild}			% Bild
	\DeclareMathOperator{\Char}{char} 			% Charakteristik
	\DeclareMathOperator{\Det}{\det\,\!}		% Determinante mit Subskript
	\DeclareMathOperator{\End}{End}				% Endomorphismen
	\DeclareMathOperator{\GL}{GL}				% allgemeine lineare Gruppe
	\DeclareMathOperator{\Hom}{Hom} 			% Homomorphismen
	\DeclareMathOperator{\id}{id} 				% Identit�t
	\DeclareMathOperator{\im}{im} 				% image
	\renewcommand{\Im}{\operatorname{Im}}		% Imagin�rteil
	\DeclareMathOperator{\Kern}{Kern}			% Kern
	\DeclareMathOperator{\LH}{LH}				% Lineare H�lle
	\DeclareMathOperator{\ord}{ord} 			% Ordnung
	\DeclareMathOperator{\pot}{\mathcal{P}}		% Potenzmenge
	\DeclareMathOperator{\Rang}{Rang}			% Rang
	\DeclareMathOperator{\rk}{rk}				% rank
	\renewcommand{\Re}{\operatorname{Re}}		% Realteil
	\DeclareMathOperator{\sgn}{sgn} 			% Signum
	\DeclareMathOperator{\sign}{sign} 			% Signum
	\DeclareMathOperator{\SL}{SL} 				% Spezielle lineare Gruppe
	\DeclareMathOperator{\SO}{S\oh} 			% Spezielle orthogonale Gruppe
	\DeclareMathOperator{\SU}{S\UU} 			% Spezielle unit�re Gruppe
	\DeclareMathOperator{\Sym}{Sym} 			% Symmetrische Gruppe
% ===========================================================

% Klammerungen und �hnliches
% ===========================================================
	\DeclarePairedDelimiter{\absolut}{\lvert}{\rvert}		% Betrag
	\DeclarePairedDelimiter{\ceiling}{\lceil}{\rceil}		% aufrunden
	\DeclarePairedDelimiter{\Floor}{\lfloor}{\rfloor}		% aufrunden
	\DeclarePairedDelimiter{\Norm}{\lVert}{\rVert}			% Norm
	\DeclarePairedDelimiter{\sprod}{\langle}{\rangle}		% spitze Klammern
	\DeclarePairedDelimiter{\enbrace}{(}{)}					% runde Klammern
	\DeclarePairedDelimiter{\benbrace}{\lbrack}{\rbrack}	% eckige Klammern
	\DeclarePairedDelimiter{\penbrace}{\{}{\}}				% geschweifte Klammern
	\newcommand{\Underbrace}[2]{{\underbrace{#1}_{#2}}} 	% bessere Unterklammerungen
	% Kurzschreibweisen f�r Faule und Code-Vervollst�ndigung
	\newcommand{\abs}[1]{\absolut*{#1}}
	\newcommand{\ceil}[1]{\ceiling*{#1}}
	\newcommand{\flo}[1]{\Floor*{#1}}
	\newcommand{\no}[1]{\Norm*{#1}}
	\newcommand{\sk}[1]{\sprod*{#1}}
	\newcommand{\enb}[1]{\enbrace*{#1}}
	\newcommand{\penb}[1]{\penbrace*{#1}}
	\newcommand{\benb}[1]{\benbrace*{#1}}
% ===========================================================

% Sonstiges
% ===========================================================
	\newcommand{\stack}[2]{\makebox[1cm][c]{$\stackrel{#1}{#2}$}}
% ===========================================================