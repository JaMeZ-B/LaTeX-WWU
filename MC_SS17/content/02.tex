%!TEX root = ../MC_SS17.tex
\section{Systemmodellierung durch Automaten}
\label{sec:para2}
\nextlecture

Der erste Schritt für das Model-Checking ist die Modellierung des Systems. Dies ist in der Praxis oft schwierig, da das Modell detailliert genug sein muss um interessante Eigenschaften zu erfassen, gleichzeitig aber auch abstrakt genug sein muss, um automatische Verifikation zu ermöglichen.

\subsection{Kripke-Strukturen und Transitionssyteme}
\begin{bsp}
	\mbox{}
	\begin{center}
		\begin{tikzpicture}
			\node[state, initial, label = right:{\small \itshape Start}](1) at (0,0) {$1$};
			\node[state, label = right:{\small \itshape Münze}](2) at (0,-2) {$2$};
			\node[state, label = right:{\small \itshape Kaffee, Milch}](3) at (-2,-3) {$3$};
			\node[state, label = right:{\small \itshape Tee, Zucker}](4) at (2,-3) {$4$};
			\node[state, label = right:{\small \itshape Tee, Milch}](5) at (0,-4) {$5$};
			
			\path[->] 	
			(1) edge [bend left=10] (2)
			(2) edge [bend right=10] (3)
			(2) edge [bend left=10] (4)
			(2) edge [bend left=10] (5)
			(3) edge [bend left=10] (1)
			(4) edge [bend right=10] (1)
			(5) edge [bend left=25] (1);
		\end{tikzpicture}
	\end{center}

	\begin{itemize}
		\item $AP = \{\textit{Start}, \textit{Münze}, \textit{Kaffee}, \textit{Tee}, \textit{Milch}, \textit{Zucker}\}$
		\item $S = \{1,2,3,4,5\}$
		\item $S_0 = \{1\}$
		\item $R = \{(1,2),(2,3),(2,4),(2,5),(3,1),(4,1),(5,1)\}$
		\item $L:S\rightarrow2^{AP}$
		\item $L(1) = \{\textit{Start}\}, L(2)=\{\textit{Münze}\}, \dots$
	\end{itemize}
\end{bsp}

\begin{defn}[Kripke-Struktur]
	Sei $AP$ eine (in der Regel) endliche Menge atomarer Propositionen. Eine Kripe-Struktur $\mathcal{K} = (S, S_0, R, L)$ über $AP$ besteht aus
	\begin{itemize}
		\item eine endliche Zustandsmenge $S$
		\item eine Menge von Anfangszuständen $S_0 \subseteq S$
		\item eine Transitionsrelation $R \subseteq S \times S$
		\item eine Annotation $L:S \rightarrow 2^{AP}$ von Zuständen mit atomaren Präpositionen
	\end{itemize}
\end{defn}

\begin{bsp}[Digitaluhr]
	\mbox{}
	\begin{center}
		\begin{tikzpicture}
			\node[](-1) at (0,0){$\dots$};
			\node[state, label = below:{\small \itshape 10 Uhr, 30 Min}](0) at (2,0) {$10:30$};
			\node[state, label = below:{\small \itshape 10 Uhr, 31 Min}](1) at (6,0) {$10:31$};
			\node[state, label = below:{\small \itshape 10 Uhr, 32 Min}](2) at (10,0) {$10:32$};
			\node[](+1) at (12,0){$\dots$};
			
			\path[->] 	
			(-1) edge [bend left=10] (0)
			(0) edge [bend left=10] (1)
			(1) edge [bend left=10] (2)
			(2) edge [bend left=10] (+1);
		\end{tikzpicture}
	\end{center}

	\begin{itemize}
		\item $S = \left \{(h,m) \mid h \in \{0, \dots, 23\}, m \in \{0, \dots, 59\} \right \}$
		\item $S_0 = S $ oder vielleicht $S_0 = \{(0,0)\}$
		\item $R = \{\left((h,m),(h',m')\right) \mid m' = (m+1) \mod 60, m' \neq 0 \Rightarrow h'=h, m' = 0 \Rightarrow h' = (h+1) \mod 24\}$
		\item $L:S\rightarrow 2^{AP}, L\left((h,m)\right) = \{\textit{h Uhr, m Min}\}$
	\end{itemize}
\end{bsp}

Oft verwendet man auch benannte Transitionen:

\begin{bsp}[Modulo3-Zähler]
	\mbox{}
	\begin{center}
		\begin{tikzpicture}
			\node[initial, state](0) at (0,0) {$0$};
			\node[state](1) at (4,0) {$1$};
			\node[state](2) at (2,-2) {$2$};
			
			\path[->] 	
			(0) edge [bend left=10] node [align=center,fill=white] 
			{$\textit{inc}$} (1)
			(1) edge [bend left=15] node [align=center,fill=white] 
			{$\textit{inc}$} (2)
			(2) edge [bend left=15] node [align=center,fill=white] 
			{$\textit{inc}$} (0)
			(0) edge [bend left=15] node [align=center,fill=white] 
			{$\textit{dec}$} (2)
			(2) edge [bend left=15] node [align=center,fill=white] 
			{$\textit{dec}$} (1)
			(1) edge [bend left=10] node [align=center,fill=white] 
			{$\textit{dec}$} (0);
		\end{tikzpicture}
	\end{center}

	\begin{itemize}
		\item $S = \{0,1,2\}$
		\item $S_0 = \{0\}$
		\item $\textit{Act} = \{\textit{inc}, \textit{dec}\}$
		\item $\{(0,\textit{inc},1),(1,\textit{inc},2),(2,\textit{inc},0),(0,\textit{dec},2),(2,\textit{dec},1),(1,\textit{dec},0)\}$
	\end{itemize}
\end{bsp}

\begin{defn}[Gelabeltes Transitionssystem]
	Ein gelabeltes Transitionssystem (labeled Transitionssystem, LTS) ist eine Struktur $\mathcal{T} = (S, S_0, \textit{Act}, R)$, wobei
	\begin{itemize}
		\item $S$ eine endliche Zustandsmenge
		\item $S_0 \subseteq S$ eine endliche Menge von Startzuständen
		\item $\textit{Act}$ eine endliche Menge von Aktionen (Transitionslabeln)
		\item $R \subseteq S \times \textit{Act} \times S$ eine Transitionsrelation
	\end{itemize}
\end{defn}

\begin{bsp}[Digi-Code]
	Wir betrachten einen Türöffner mit drei Tasten $A,B$ und $C$. Die Tür öffnet mit dem Code $\mathit{ABA}$
	\begin{center}
		\begin{tikzpicture}
			\node[initial, state](1) at (0,0) {$1$};
			\node[state, label = below:{\small \itshape SeenA}](2) at (2,0) {$2$};
			\node[state, label = below:{\small \itshape SeenAB}](3) at (4,0) {$3$};
			\node[state, label = below:{\small \itshape Open}](4) at (6,0) {$4$};
			
			\path[->] 	
			(1) edge [bend left] node [align=center,fill=white] 
			{$A$} (2)
			(2) edge [bend left] node [align=center,fill=white] 
			{$B$} (3)
			(3) edge [bend left] node [align=center,fill=white] 
			{$A$} (4)
			(2) edge [bend left = 25] node [align=center,fill=white] 
			{$C$} (1)
			(3) edge [bend left = 75] node [align=center,fill=white] 
			{$B,C$} (1)
			(1) edge [loop above] node [align=center] 
			{$B,C$} (1)
			(2) edge [loop above] node [align=center] 
			{$A$} (2);
		\end{tikzpicture}
	\end{center}

	\begin{itemize}
		\item $S=\{1,2,3,4\}$
		\item $S_0 = {1}$
		\item $\textit{Act} =  \{A,B,C\}$
		\item $R = \{(1,A,2), (2, A, 2), \dots\}$
		\item[] Ergänzt mit atomaren Präpositionen:
		\item $AP = \{\textit{SeenA}, \textit{SeenB}, \textit{Open}\}$
		\item $L:S\rightarrow2^{AP} \text{ mit } L(1) = \emptyset, L(2) = \{\textit{SeenA}\}, L(3) = \{\textit{SeenB}\}, L(4) = \{\textit{Open}\}$
	\end{itemize}
\end{bsp}

\begin{defn}[Kripke-Transitionssystem]
	Sei $AP$ eine Menge atomarer Propositionen. Ein Kripke-Transitionssystem ($KTS$) über $AP$ ist eine Struktur $\mathcal{K} = (S, S_0, \textit{Act}, R, L)$, wobei
	\begin{itemize}
		\item $S$ eine endliche Zustandsmenge
		\item $S_0 \subseteq S$ Menge der Anfangszustände
		\item $\textit{Act}$ endliche Menge von Transitionslabeln (Aktionen)
		\item $R \subseteq S \times \textit{Act} \times S$ Transitionsrelation
		\item $L:S\rightarrow2^{AP}$ Annotation der Zustände mit atomaren Präpositionen
	\end{itemize}
\end{defn}

\subsection{Modellierung von Schaltwerken}

\begin{bsp}[Modulo-8-Zähler]
	\mbox{}
	\begin{center}
		\begin{tikzpicture}[small circuit symbols,
		%every circuit symbol/.style={logic gate IEC symbol color=black},
		circuit ee IEC,
		circuit logic US,]
		 %TODO schoener umsetzen, aber derzeit keine Zeit
		\node [draw, fill = white, text height = 3 cm, text width = 1 cm](takt) at (0,0){};
		\node[] (v2) at (0,1) {$v_2$};
		\node[] (v1) at (0,0) {$v_1$};
		\node[] (v0) at (0,-1) {$v_0$};
		\node [xor gate={info'={\tiny XOR}}](v1xor) at (5,0) {};
		\node [xor gate={info'={\tiny XOR}}](v2xor) at (4,2) {};
		\node [not gate={info'={\tiny NOT}}](not) at (3,-1) {};
		\node [and gate={info'={\tiny AND}}, rotate = 90](and) at (1.5,1) {};
		
		\draw[-] (takt.east |- v0.east) -- (not.west);
		
		\draw[-] (not.east) -- +(1,0) -- +(1, -1.5) -- +(-4.5, -1.5) -- +(-4.5, 0) -- (takt.west |- v0.west);
		
		\draw[-] (v1xor.east) -- +(1,0) -- +(1, -3.0) -- +(-7, -3.0) -- +(-7, 0) -- (takt.west |- v1.west);
		
		\draw[-] (v2xor.east) -- +(1,0) -- +(1, 1) -- +(-6, 1) -- +(-6, -1) -- (takt.west |- v2.west);
		
		\draw[-] (takt.east |- v1.east) -- ($(v1xor.west) - (1,0)$) -- ($(v1xor.west) - (1,-0.1)$) -- ($(v1xor.west) - (0,-0.1)$); 
		
		\draw[-] (takt.east |- v2.east) -- ($(v2.east) + (0.5,0)$) -- ($(v2.east) + (0.5,1.1)$) -- ($(v2xor.west) + (0,0.1)$);
		
		\draw[-] (and.east) -- ($(and.east) + (0,0.541)$) -- ($(v2xor.west) - (0,0.1)$);
		
		\draw[-] ($(and.west) - (0.1,0)$) -- +(0,-0.68) node[circle, fill = black, minimum size=0.15cm, inner sep = 0pt]{};
		
		\draw[-] ($(and.west) + (0.1,0)$) -- +(0,-1.68) node[circle, fill = black, minimum size=0.15cm, inner sep = 0pt]{};
		
		\draw[-] ($(v0.east) + (0.5,0)$) node[circle, fill = black, minimum size=0.15cm, inner sep = 0pt]{} -- + (0,0.7) -- +(3.5, 0.7) -- +(3.5,0.9) -- ($(v1xor.west) - (0,0.1)$); 
		\end{tikzpicture}
	\end{center}
	\textbf{Modellierung durch Kripke-Struktur}
	\begin{itemize}
		\item $S = \{(v_0, v_1, v_2) \mid v_i \in \mathbb{B}\} = \mathbb{B}^3, \mathbb{B} = \{0,1\}$
		\item $S_0 = \{(0,0,0)\}$
		\item $R = \{ \left((v_0,v_1,v_2), (v_0', v_1', v_2')\right) \mid v_0' = \bar{v_0}, v_1' = v_0 \oplus v_1, v_2' = (v_0 \wedge v_1) \oplus v_2\}$ (aus Schaltwerk ablesen)
		\item $AP = \{\textit{Reg}_0, \textit{Reg}_1, \textit{Reg}_2\}$
		\item $L\left((v_0, v_1, v_2)\right) = \{\textit{Reg}_0 \mid v_0 = 1\} \cup \{\textit{Reg}_1 \mid v_1 = 1\} \cup \{\textit{Reg}_2 \mid v_2 = 1\}$
	\end{itemize}

	\textbf{Visualisierung der Kripke-Struktur}
	
	\begin{center}
		\begin{tikzpicture}[scale=0.72]
			\node[initial, state](000) at (0,0) {\footnotesize $(0,0,0)$};
			\node[state](100) at (2,-1) {\footnotesize $(1,0,0)$};
			\node[state](010) at (3,-3)  {\footnotesize $(0,1,0)$};
			\node[state](110) at (2,-5) {\footnotesize $(1,1,0)$};
			\node[state](001) at (0,-6) {\footnotesize $(0,0,1)$};
			\node[state](111) at (-2,-1) {\footnotesize $(1,1,1)$};
			\node[state](011) at (-3,-3) {\footnotesize $(0,1,1)$};
			\node[state](101) at (-2,-5) {\footnotesize $(1,0,1)$};
			
			\path[->] 	
			(000) edge [bend left = 5] (100)
			(100) edge [bend left = 5] (010)
			(010) edge [bend left = 5] (110)
			(110) edge [bend left = 5] (001)
			(001) edge [bend left = 5] (101)
			(101) edge [bend left = 5] (011)
			(011) edge [bend left = 5] (111)
			(111) edge [bend left = 5] (000);
		\end{tikzpicture}
	\end{center}
\end{bsp}

\subsubsection*{Allgemeines Schaltwerk}
\todo{Grafik einfügen}

\subsubsection*{Modellierung durch ein KTS}
$\mathcal{K} = (S, S_0, \textit{Act}, R, L)$ mit
\begin{itemize}
	\item $S = \mathbb{B}^n$
	\item $S_0 = \{(0, \dots, 0)\}$ (je nach Anwendung)
	\item $\textit{Act} = \mathbb{B}^k \times \mathbb{B}^l$ (in jeder Transition wird eine Eingabe aus $\mathbb{B}^k$ gelesen und eine Ausgabe aus $\mathbb{B}^l$ ausgegeben)
	\item $R = \{(x, (i,o), x) \mid f(i,x) = (o, x')\}$
	\item $AP = \{R_1, \dots R_n\}$
	\item $L\left((x_1,\dots,x_2)\right) = \{R_i \mid x_i = 1, i=1,\dots,n\}$
\end{itemize}

\nextlecture
\subsection{Automaten mit Variablen über endlichen Wertebereichen}

\cleardoubleoddemptypage