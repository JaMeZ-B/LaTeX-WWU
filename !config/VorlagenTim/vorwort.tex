\section*{Vorwort --- Mitarbeit am Skript}
Dieses Dokument ist eine Mitschrift aus der Vorlesung \enquote{\fach, \Semester}, gelesen von \prof. Der Inhalt entspricht weitestgehend dem Tafelanschrieb. Für die
Korrektheit des Inhalts übernehme ich keinerlei Garantie! Für Bemerkungen und Korrekturen -- und seien es nur Rechtschreibfehler -- bin ich sehr dankbar. 
Korrekturen bitte durch persönliches Ansprechen oder per Mail an \url{keil.menden@web.de}.
%\begin{itemize}
%	\item Persönliches Ansprechen in der Uni oder Mails an keil.menden@web.de (gerne auch mit annotieren PDFs) \url{https://github.com/JaMeZ-B/latex-wwu}.
%	\item \emph{Direktes} Mitarbeiten am Skript: Den Quellcode poste ich auf GitHub (siehe oben), also stehen vielfältige Möglichkeiten der Zusammenarbeit zur Verfügung:
%	Zum Beispiel durch Kommentare am Code über die Website und die Kombination Fork + Pull Request. Wer sich verdient macht oder ein Skript zu einer Vorlesung, die 
%	ich nicht besuche, beisteuern will, dem gewähre ich gerne auch Schreibzugriff.
	
%	Beachten sollte man dabei, dass dazu ein Account bei \url{github.com} notwendig ist, der allerdings ohne Angabe von persönlichen Daten angelegt werden kann. 
%	Wer bei GitHub (bzw. dem zugrunde liegenden Open-Source-Programm \enquote{\texttt{git}}) -- verständlicherweise -- Hilfe beim Einstieg braucht, dem helfe ich gerne 
%	weiter. Es gibt aber auch zahlreiche empfehlenswerte Tutorials im Internet.\footnote{zB. \url{https://try.github.io/levels/1/challenges/1}, ist auf Englisch, aber dafür 
%	interaktives LearningByDoing}
%	\item \emph{Indirektes} Mitarbeiten: \TeX-Dateien per Mail verschicken. 
	
%	Dies ist nur dann sinnvoll, wenn man einen ganzen Abschnitt ändern möchte (zB. einen alternativen Beweis geben), da ich die Änderungen dann per Hand einbauen muss! Ich freue mich aber auch über solche Beiträge!
%\end{itemize}