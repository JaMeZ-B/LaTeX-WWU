%!TEX root = ../AnaTopGeo_SS14/ana_top_geo.tex
\usepackage{scrtime} % KOMA, Uhrzeit ermoeglicht
\usepackage{etoolbox}
\usepackage{letltxmacro}
\usepackage{ifthen}

%--Farbdefinitionen
%-- muss vor tikz geladen werden
\usepackage[usenames, table, x11names]{xcolor}
\definecolor{dark_gray}{gray}{0.45}
\definecolor{light_gray}{gray}{0.6}
\definecolor{fb10_blue}{cmyk}{0.8,0.4,0.13,0.07}
\usepackage[final]{graphicx}

%--Zum Zeichnen
%-- muss vor polyglossia bzw. babel geladen werden
\usepackage{tikz}
\usepackage{tikz-cd}
\usetikzlibrary{external}
\tikzset{>=latex}
\usetikzlibrary{shapes,arrows.meta,intersections}
\usetikzlibrary{calc,3d}
\usetikzlibrary{decorations.pathreplacing,decorations.markings, decorations.pathmorphing}
\usetikzlibrary{angles}

%-- Konfiguration von tikzexternalize
\tikzexternalize[prefix=tikz/,up to date check=diff]
\pgfkeys{/pgf/images/include external/.code=\includegraphics{#1}}
\tikzset{external/system call={lualatex \tikzexternalcheckshellescape -halt-on-error -interaction=batchmode --shell-escape -jobname "\image" "\texsource"}}


%-- tikzexternalize fuer tikzcd deaktivieren, da inkompatibel
\AtBeginEnvironment{tikzcd}{\tikzexternaldisable}
\AtEndEnvironment{tikzcd}{\tikzexternalenable}

%-- um Inkompatibilitaeten von quotes und polyglossia bzw. babel zu vermeiden
\tikzset{
  every picture/.append style={
    execute at begin picture={\shorthandoff{"}},
    execute at end picture={\shorthandon{"}}
  }
}
\usetikzlibrary{quotes}
\usepackage{pgfplots}
\usepgfplotslibrary{colormaps}
% \pgfplotsset{compat=1.12}



%-- Mathesymbole etc.
\usepackage[T1]{fontenc}
\renewcommand{\rmdefault}{zpltlf}
\usepackage{mathtools} % beinhaltet amsmath
\mathtoolsset{showonlyrefs, centercolon,showmanualtags}
\newtagform{brackets}[\textbf]{[}{]}
\usetagform{brackets}
\usepackage{fix-cm}
% \usepackage{wasysym} % zusätzliche Symbole
\usepackage[bbgreekl]{mathbbol}
\usepackage{amssymb} % zusätzliche Symbole
% \usepackage{stmaryrd} % für Blitz
\usepackage{nicefrac} % schräge Brüche
\usepackage{faktor}
\newcommand{\Faktor}[1]{\faktor[\textstyle]{#1}}
\usepackage{xfrac}
\usepackage{cancel}
\usepackage{mathdots} % Verbesserung von Punkten wie zB \ldots
\usepackage[bb=px]{mathalfa}
\usepackage{centernot}

%-- einzelne Symbole aus dem mathx-package
% \DeclareFontFamily{U}{mathx}{\hyphenchar\font45}
% \DeclareFontShape{U}{mathx}{m}{n}{<-> mathx10}{}
% \DeclareSymbolFont{mathx}{U}{mathx}{m}{n}
% \DeclareMathSymbol{\bigplus}{1}{mathx}{"90}

%-- 
\usepackage{silence}
\WarningFilter{latexfont}{Size substitutions with differences}
\WarningFilter{latexfont}{Font shape `U/bbold/m/n' in size}

\DeclareSymbolFont{bbold}{U}{bbold}{m}{n}
\DeclareSymbolFontAlphabet{\mathbbold}{bbold}
\newcommand{\ind}{\mathbbold{1}} % charakteristische-Funktion-Eins

\def\mathul#1#2{\color{#1}\underline{{\color{black}#2}}\color{black}} %farbiges Untersteichen im Mathe-Modus
\renewcommand{\le}{\leqslant}
\renewcommand{\ge}{\geqslant}


%-- coloured frame box
\usepackage{adjustbox}
\newcommand{\cfbox}[2]{%
	\ifmmode
	\mathchoice{\adjustbox{cfbox=#1}{$\displaystyle#2$}}{\adjustbox{cfbox=#1}{$\textstyle#2$}}{\adjustbox{cfbox=#1}{$\scriptstyle#2$}}{\adjustbox{cfbox=#1}{$\scriptscriptstyle#2$}}
	\else
	\adjustbox{cfbox=#1}{#2}
	\fi
}



%-- Alles was mit Schrift und XeTeX zu tun hat
\usepackage[euler-digits]{eulervm} % vor fontspec laden!
\usepackage[no-math]{fontspec}
\usepackage{polyglossia} % moderner babel-ersatz
\setmainlanguage[spelling=new,babelshorthands=true]{german}
\shorthandoff{"}
\setotherlanguage{english}
\defaultfontfeatures{Mapping=tex-text, WordSpace={1.2}, Ligatures={Required,Common,Contextual},Extension=.otf} %

%-- Alle hier aufgeführten Schriftarten sind in jeder TeXlive-Installation vorhanden!!
%-- Hauptschriftart festlegen
% \setmainfont{texgyrepagella}[Extension=.otf,UprightFont=*-regular,BoldFont=*-bold,ItalicFont=*-italic,BoldItalicFont=*-bolditalic,ItalicFeatures={Style=Historic}]
\setmainfont{TeXGyrePagellaX}[UprightFont=*-Regular,BoldFont=*-Bold,ItalicFont=*-Italic,BoldItalicFont=*-BoldItalic,ItalicFeatures={Style=Historic},Ligatures={Required,Common,Contextual,Historic}]
% \setmainfont[BoldFont={* Bold}, ItalicFont={* Light Italic},SmallCapsFont={Linux Libertine O}, SmallCapsFeatures={Letters=SmallCaps}]{Source Sans Pro}


%-- Serifenlose Schriftart festlegen
% \setsansfont{Roboto}[Scale=MatchUppercase,Extension=.ttf,UprightFont=*-Regular,BoldFont=*-Bold,ItalicFont=*-RegularItalic,BoldItalicFont=*-MediumItalic]
% \setsansfont{RobotoSlab}[Scale=MatchUppercase,Extension=.ttf,UprightFont=*-Regular,BoldFont=*-Bold,ItalicFont=*-Thin,SmallCapsFont=*-Regular]
% \setsansfont{FiraSans}[Scale=MatchUppercase,Extension=.otf, UprightFont=*-Regular, BoldFont=*-SemiBold, ItalicFont=*-RegularItalic, BoldItalicFont=*-MediumItalic]
% \setsansfont{SourceSansPro}[Scale=MatchUppercase,Extension=.otf,UprightFont=*-Regular,BoldFont=*-Semibold,ItalicFont=*-LightIt,BoldItalicFont=*-SemiboldIt]
% \setsansfont{texgyreadventor}[Scale=MatchUppercase, Extension=.otf, UprightFont=*-regular, BoldFont=*-bold, ItalicFont=*-italic,BoldItalicFont=*-bolditalic]
% \setsansfont{Kurier}[Scale=MatchUppercase, Extension=.otf, UprightFont=*Medium-Regular, BoldFont=*Heavy-Regular, ItalicFont=*Medium-Italic,BoldItalicFont=*Heavy-Italic]
% \setsansfont{Iwona}[Scale=MatchUppercase, Extension=.otf, UprightFont=*Medium-Regular, BoldFont=*Heavy-Regular, ItalicFont=*Medium-Italic,BoldItalicFont=*Heavy-Italic]
% \setsansfont{LinBiolinum}[Scale=MatchUppercase, Extension=.otf, UprightFont=*_R, BoldFont=*_RB, ItalicFont=*_RI,BoldItalicFont=*_RBO]
% \setsansfont{texgyrebonum}[Scale=MatchUppercase,Extension=.otf, UprightFont=*-regular, BoldFont=*-bold, ItalicFont=*-italic, BoldItalicFont=*-bolditalic]
\setsansfont{texgyreadventor}[Scale=MatchUppercase, UprightFont=*-regular, BoldFont=*-bold, ItalicFont=*-italic, BoldItalicFont=*-bolditalic]
% \setsansfont{Raleway}[Scale=MatchUppercase,Extension=.otf, UprightFont=*-Regular, BoldFont=*-Bold, ItalicFont=*-Regular-Italic, BoldItalicFont=*-Bold-Italic]
% \setsansfont{Syntax Next LT W1G}



% \newfontfamily{\geometric}{texgyreheros}[Scale=MatchLowercase, Extension=.otf, UprightFont=*-regular, BoldFont=*-bold, ItalicFont=*-italic]
% \addtokomafont{disposition}{\geometric}
\setmonofont{SourceCodePro}[Scale=0.9,UprightFont=*-Regular, BoldFont=*-Semibold, ItalicFont=*-Light]
\usepackage{xltxtra}
\usepackage{fontawesome}
% \usepackage[euler-digits]{eulervm}
\usepackage[final]{microtype}


\usepackage[draft=false]{scrlayer-scrpage} % wie fancyhdr, nur optimiert auf KOMA-Skript, leicht andere Syntax

\flushbottom


%--Mixed
\usepackage[shortlabels,inline]{enumitem}
\setlist[itemize,1]{label=\faCaretRight}
\setlist[enumerate]{font=\bfseries}
\setlist[description]{font=\normalfont\bfseries}
\usepackage[autostyle,german=quotes,english=british]{csquotes}
\usepackage{makeidx}
\usepackage{wrapfig}
\usepackage{float}
\usepackage[margin=10pt, font=small, labelfont={sf, bf}, format=plain, indention=1em]{caption}
\captionsetup[wrapfigure]{name=Abb. }
\usepackage{stackrel}
\usepackage{multicol}
\usepackage{booktabs}
% \usepackage{calc}


%--Unterstreichung
\usepackage[normalem]{ulem}
\setlength{\ULdepth}{1.8pt}

%--Indexverarbeitung
\newcommand{\bet}[1]{\textbf{\emph{#1}}}
\newcommand{\Index}[1]{\bet{#1}\index{#1}}
\makeindex
\setindexpreamble{{\noindent\sffamily\small Die \emph{Seitenzahlen} sind mit Hyperlinks versehen und somit anklickbar}  \par \bigskip}
\renewcommand{\indexpagestyle}{scrheadings}


%--Marginnote & todonotes
\deffootnote[1.5em]{1.5em}{1.5em}{\textsuperscript{\thefootnotemark}\ }
\usepackage[fulladjust]{marginnote}
\renewcommand*{\marginfont}{\itshape\footnotesize}
\usepackage[textsize=small]{todonotes}
\usepackage{ragged2e}
\renewcommand*{\raggedleftmarginnote}{\RaggedLeft}
\renewcommand*{\raggedrightmarginnote}{\RaggedRight}

%--Todonotes mit tikz/externalize kompatibel machen
\LetLtxMacro{\oldtodo}{\todo}
\renewcommand{\todo}[2][]{\tikzexternaldisable\oldtodo[#1]{#2}\tikzexternalenable}
\LetLtxMacro{\oldmissingfigure}{\missingfigure}
\renewcommand{\missingfigure}[2][]{\tikzexternaldisable\oldmissingfigure[{#1}]{#2}\tikzexternalenable}


% -- BibLaTeX 
\usepackage[%
	backend=biber,
	sortlocale=auto,
	natbib,
	hyperref,
	backref,
	style=alphabetic
	]%
{biblatex}
\renewcommand*{\mkbibnamelast}[1]{%
  \ifmknamesc{\textsc{#1}}{#1}}

\renewcommand*{\mkbibnameprefix}[1]{%
  \ifboolexpr{ test {\ifmknamesc} and test {\ifuseprefix} }
    {\textsc{#1}}
    {#1}}

\def\ifmknamesc{%
  \ifboolexpr{ test {\ifcurrentname{labelname}}
               or test {\ifcurrentname{author}}
               or ( test {\ifnameundef{author}} and test {\ifcurrentname{editor}} ) }}
\addbibresource{../!config/quellen.bib}

%--Konfiguration von Hyperref pdfstartview=FitH, 
\usepackage[hidelinks, pdfpagelabels,  bookmarksopen=true, bookmarksnumbered=true, linkcolor=black, urlcolor=SkyBlue2, plainpages=false,pagebackref, citecolor=black, hypertexnames=true, pdfauthor={Jannes Bantje}, pdfborderstyle={/S/U}, linkbordercolor=SkyBlue2, colorlinks=false,final,backref=false]{hyperref}

\usepackage[nameinlink,noabbrev]{cleveref}

\newcommand{\appendLink}[1]{#1\,\faExternalLink}
\newcommand{\hrefsym}[2]{\href{#1}{\texttt{\appendLink{#2}}}}
\newcommand{\hrefsymX}[2]{\href{#1}{\appendLink{#2}}}
\newcommand{\hrefsymmail}[2]{\href{#1}{\texttt{\faEnvelopeO\,#2}}}
\renewcommand{\url}[1]{\hrefsym{#1}{\nolinkurl{#1}}}



% -- QR-Codes
\usepackage{qrcode} %-- hinter hyperref laden!

%--Römische Zahlen
\newcommand{\RM}[1]{\MakeUppercase{\romannumeral #1{}}}

%%--Blitz
\usepackage{marvosym}
\newcommand{\light}{\text{\raisebox{-.3ex}{\Large\Lightning}}}

%-- Underbrace als Befehl in LaTeX-Syntax (und ohne Spacingprobleme mit nachfolgenden Operatoren...)
\newcommand{\Underbrace}[2]{{\underbrace{#1}_{#2}}}
\newcommand{\Underbracket}[2]{{\underbracket[0.7pt][2pt]{#1}_{#2}}}
\newcommand{\Overbracket}[2]{{\overbracket[0.7pt][2pt]{#1}^{#2}}}

%-- Ersatz für \mathring
\newcommand*\interior[1]{\overset{\smash{\raisebox{-0.18ex}{$\scriptstyle\circ$}}}{#1}}

%-- Definitionen von weiteren Mathe-Befehlen
\DeclareMathOperator{\re}{Re} % Realteil
\let\Re\relax
\DeclareMathOperator{\Re}{Re} % Realteil
\DeclareMathOperator{\im}{im} % Bild
\let\Im\relax
\DeclareMathOperator{\Im}{Im} % Bild
\DeclareMathOperator{\id}{id} % identische Abbildung
\DeclareMathOperator{\conj}{conj} % Konjugation
\DeclareMathOperator{\sgn}{sgn} % Signum
\DeclareMathOperator{\End}{End} % Endomorphismen
\DeclareMathOperator{\Hom}{Hom} % Homomorphismen
\DeclareMathOperator{\Iso}{Iso} % Isomorphismen
\DeclareMathOperator{\Span}{span} % Span
\DeclareMathOperator{\coker}{coker}
\DeclareMathOperator{\Tr}{Tr} % Spur,Trace
\DeclareMathOperator{\pr}{pr} % Projektion
\DeclareMathOperator{\diag}{diag} % Projektion
\DeclareMathOperator{\Rg}{Rg} % Rang
\DeclareMathOperator{\const}{const} % konstante Abbildung
\DeclareMathOperator{\Spur}{Spur}
\DeclareMathOperator{\Arg}{Arg}

\DeclareMathOperator{\grad}{grad} % Gradient
\DeclareMathOperator{\dive}{div} % Gradient
\DeclareMathOperator{\rot}{rot} % Rotation


\DeclareMathOperator{\Char}{char} % Charakteristik
\DeclareMathOperator{\Aut}{Aut} % Automorphismen
\DeclareMathOperator{\ab}{ab} % Abelisierung

% -- Zum Finetuning von Befehlen
\makeatletter
\newcommand{\raisemath}[1]{\mathpalette{\raisem@th{#1}}}
\newcommand{\raisem@th}[3]{\raisebox{#1}{$#2#3$}}
\makeatother

\makeatletter
\newcommand{\killDescendersM}[1]{\mathpalette{\killD@scendersM{#1}}}
\newcommand{\killD@scendersM}[2]{\raisebox{0pt}[\height][0pt]{$#2#1$}}
\makeatother


\newcommand{\sing}{{\raisemath{1.1pt}{\scriptscriptstyle\mathrm{sing}}}}
\newcommand{\pt}{\mathrm{pt}}
\DeclareMathOperator{\Zyl}{Zyl}
\DeclareMathOperator{\Tel}{Tel}
\newcommand{\op}{\mathrm{op}}
\DeclareMathOperator{\Sp}{Sp}
\DeclareMathOperator{\Keg}{Keg}
\newcommand{\slashedi}{i\hspace{-3.5pt}/}
\newcommand{\cupp}{\smallsmile}
\newcommand{\capp}{\smallfrown}
\DeclareMathOperator*{\colim}{colim}
\DeclareMathOperator{\PD}{PD}
\newcommand{\lf}{\mathrm{lf}}
\DeclareMathOperator{\sig}{sig}
\DeclareMathOperator{\Tor}{Tor}
\DeclareMathOperator{\Ext}{Ext}
\DeclareMathOperator{\AW}{AW}
\DeclareMathOperator{\Proj}{Proj}
\DeclareMathOperator{\Gr}{Gr}
\DeclareMathOperator{\res}{res}
\DeclareMathOperator{\Spec}{Spec}
\DeclareMathOperator{\co}{co}
\DeclareMathOperator{\ch}{ch}
\DeclareMathOperator{\wOp}{w}
\DeclareMathOperator{\Ar}{Ar}
\newcommand{\acts}{\curvearrowright}
\DeclareMathSymbol{\bbDelta}{\mathord}{bbold}{"01}
\newcommand{\DDelta}{\bbDelta}
\DeclareMathOperator{\Star}{Star}
\DeclareMathOperator{\Link}{Link}
\DeclareMathOperator{\EPK}{EPK}
\newcommand{\action}{\mathrel{\curvearrowright}}
\DeclareMathOperator{\Vol}{Vol}

\newcommand{\D}{\ensuremath{\mathrm{D}\mkern-1.0mu}}
\newcommand{\mathd}{\ensuremath{\mathrm{d}\mkern-1.0mu}}
\newcommand{\Tmap}{\ensuremath{\mathrm{T}\mkern-0.85mu}}
\newcommand{\opL}{\ensuremath{\mathrm{L}\mkern-0.6mu}}


\DeclareMathOperator{\supp}{supp} %Träger
\DeclareMathOperator{\GL}{GL}
\DeclareMathOperator{\Diff}{Diff}


%-- Liegruppen
\DeclareMathOperator{\SO}{SO}
\DeclareMathOperator{\Ad}{Ad}
\DeclareMathOperator{\ad}{ad}
\DeclareMathOperator{\On}{O}
\DeclareMathOperator{\Un}{U}
\DeclareMathOperator{\SU}{SU}
\DeclareMathOperator{\Mat}{Mat}
% \DeclareMathOperator{\Der}{Der}
\DeclareRobustCommand{\Der}{\mathop{\mathfrak{der}}}
\DeclareMathOperator{\SL}{SL}
\DeclareMathOperator{\Graph}{Graph}
\DeclareMathOperator{\Int}{Int}
% \DeclareMathOperator{\intAlg}{int}
\DeclareRobustCommand{\intAlg}{\mathop{\mathfrak{int}}}
\DeclareMathOperator{\aut}{aut}
\DeclareMathOperator{\Rad}{Rad}
\DeclareMathOperator{\Nil}{Nil}
\DeclareMathOperator{\rad}{rad}
\DeclareMathOperator{\nil}{nil}
\DeclareMathOperator{\Ric}{Ric}
\DeclareMathOperator{\ric}{ric}
\newcommand{\bi}{\mathrm{bi}}
\DeclareMathOperator{\Isom}{Isom}
\DeclareMathOperator{\Sym}{Sym}

% -- Befehle für FunkAna und Operatoralgebren
\DeclareMathOperator{\tr}{tr}
\newcommand{\w}{\mkern1mu\mathrm{w}}
\newcommand{\sa}{\mathrm{sa}}
\newcommand{\vb}{\mathrm{v\mkern-2.5mu.b\mkern-1.5mu.}} % vollständig beschränkt
\newcommand{\so}{\mathrm{\mkern.3mu s\mkern-1.4mu.\mkern-.6mu o\mkern-1.7mu.}} % \newcommand{\so}{\mathrm{s.o.}}
\newcommand{\solim}{\so\text{-}\mkern-0.8mu\lim}
\newcommand{\wo}{\mathrm{w\mkern-3mu.\mkern-.4mu o\mkern-1.7mu.}}
\newcommand{\Top}[1]{\mathcal{T}_{\mkern-2.3mu #1}}
\newcommand{\weakT}[1]{\ensuremath{\mathcal{T}_{#1}^{\mkern+1.0mu\text{\raisebox{0.4ex}{$\mathrm{w}$}}}}}
\newcommand{\weakTstar}[1]{\ensuremath{\mathcal{T}_{#1}^{\mkern+1.0mu\text{\raisebox{0.4ex}{$\mathrm{w}$}}^*}}}
\newcommand{\TWeakStar}{\Top{\w^*}}
\newcommand{\TWeakOp}{\Top{\wo}}
\newcommand{\Tso}{\Top{\so}}
\newcommand{\finSub}{\subset\mkern-0.7mu \subset}
\DeclareMathOperator{\Inv}{Inv}
\DeclareMathOperator{\dist}{dist}
\newcommand{\simm}{{\hspace{-1.6pt}\raisemath{0.5pt}{\sim}}}
\newcommand{\plus}{{\hspace{-1.6pt}+}}
\DeclareMathOperator{\ev}{ev}
\DeclareMathOperator{\Alg}{Alg}
\DeclareMathOperator{\her}{her}
\newcommand{\subher}{\subset_{\her}}
\newcommand{\grenzw}[1]{\xrightarrow{\minwidthbox{#1}{1.4em}}}
\newcommand{\grenzwl}[1]{\xleftarrow{\minwidthbox{#1}{1.4em}}}
\newcommand{\grenzwIn}[1]{\grenzw{\raisemath{-2pt}{#1}}}
\newcommand{\MyTo}[1]{\tikzexternaldisable\mathbin{\tikz[baseline] \draw[-to,line width=.4pt] (0ex,0.94ex) -- (#1,0.94ex);}\tikzexternalenable}
\newcommand{\dlim}{%
    \mathchoice
      {\lim\limits_{\MyTo{4.2ex}}}% \displaystyle
      {\lim\limits_{\MyTo{2.8ex}}}% \textstyle
      {\lim\limits_{\MyTo{2.3ex}}}% \scriptstyle
      {\lim\limits_{\MyTo{2.3ex}}}% \scriptscriptstyle
}
\newcommand{\Dlim}{\killDescendersM{\dlim}}
\DeclareMathOperator{\sep}{sep}
\DeclareMathOperator{\diam}{diam}
\DeclareMathOperator{\conv}{conv}
\DeclareMathOperator{\Prim}{Prim}
\DeclareMathOperator{\hull}{hull}
\DeclareMathOperator{\red}{red}
\DeclarePairedDelimiterX\bra[1]{\langle}{\rvert}{#1\,}
\DeclarePairedDelimiterX\ket[1]{\lvert}{\rangle}{\,#1}
\DeclarePairedDelimiterX\bracket[2]{\langle}{\rangle}{#1\,\delimsize\vert\,#2}
\newcommand{\tensormax}{\mathbin{\otimes_{\max}}}
\newcommand{\tensormin}{\mathbin{\otimes_{\min}}}
\DeclareMathOperator{\Ped}{Ped}
\newcommand{\alg}{\mathrm{alg}}
\DeclareMathOperator{\CPC}{CPC}
\DeclareMathOperator{\CP}{CP}
\DeclareMathOperator{\UPC}{UPC}
\newcommand{\DeltaOp}{\mathbin{\Delta}}
\newcommand{\kernedP}{\mathcal{P}\mkern-2mu}
\newcommand{\Pinfty}{\kernedP_{\infty}}
\DeclareMathOperator{\Groth}{Groth}
\DeclareMathOperator{\rk}{rk}
\newcommand{\MvN}{\mathrm{MvN}}

% -- Kategorien
\DeclareMathOperator{\Mor}{Mor}
\DeclareMathOperator{\Obj}{Obj}
\newcommand{\TOP}{\textsc{Top}}
\newcommand{\HTOP}{\textsc{HTop}}
\newcommand{\VR}{\textsc{VR}}
\newcommand{\MOD}{\textsc{Mod}}
\newcommand{\Mod}[1]{#1\text{-}\MOD}
\newcommand{\MONOIDE}{\textsc{Monoide}}
\newcommand{\SET}{\textsc{Set}}
\newcommand{\MAN}{\textsc{Man}}
\newcommand{\GRUPPEN}{\textsc{Gruppen}}
\newcommand{\ABELGRUPPEN}{\textsc{Abel.Gruppen}}
\newcommand{\ABEL}{\textsc{Abel}}
\newcommand{\KAT}{\textsc{Kat}}
\newcommand{\FUN}{\textsc{Fun}}
\newcommand{\SIMP}{\textsc{Simp}}
\newcommand{\VEKT}{\textsc{Vekt}}
\newcommand{\CH}{\textsc{Ch}}
\newcommand{\CSTARUN}{C^*\text{-}\textsc{Alg}^{\raisemath{-2.5pt}{1}}}
\newcommand{\CSTAR}{C^*\text{-}\textsc{Alg}}
\newcommand{\AB}{\textsc{Ab}}

% \DeclareMathOperator{\cell}{cell}
\newcommand{\cell}{{\raisemath{1.1pt}{\scriptscriptstyle\mathrm{cell}}}}
\DeclarePairedDelimiter{\homologieklasse}{\llbracket}{\rrbracket}
\newcommand{\rand}[1]{\ensuremath{\partial^{\scriptscriptstyle #1}}}


% -- theorem packages
\usepackage{amsthm}
\usepackage{thmtools,thm-restate}
\usepackage{mdframed}
\renewcommand{\listtheoremname}{Übersicht aller Aussagen}

% -- Theoreme als PDF-Lesezeichen
\usepackage{bookmark}
\bookmarksetup{open,numbered}
\makeatletter
\newcommand*{\theorembookmark}{%
  \bookmark[
    dest=\@currentHref,
    rellevel=1,
    keeplevel,
  ]{%
    \thmt@thmname\space\csname the\thmt@envname\endcsname
    \ifx\thmt@shortoptarg\@empty
    \else
      \space(\thmt@shortoptarg)%
    \fi
  }%
}   
\makeatother

% -- Definition der einzelnen Umgebungen
\declaretheoremstyle[%
	headfont=\sffamily\bfseries,
	notefont=\normalfont\sffamily\scshape,
	bodyfont=\normalfont,
	headformat=\NUMBER\ \NAME\NOTE,
	headpunct=.,
	postheadspace=1em,
	spaceabove=15pt,spacebelow=10pt,
	shaded={bgcolor=gray!20},
	% mdframed={%
	% 	backgroundcolor=gray!20,
	% 	innertopmargin=0pt,
	% 	innerleftmargin=0pt,
	% 	% roundcorner=5pt,
	% 	innerbottommargin=0pt,
	% 	splittopskip = 0pt,
	% 	% skipbelow=6pt,
	% 	% skipbelow=6pt,
	% 	topline=false,bottomline=false,leftline=false,rightline=false
	% },
	postheadhook=\theorembookmark]%
{mainstyle}
\declaretheoremstyle[%
	headfont=\sffamily\bfseries,
	notefont=\normalfont\sffamily\scshape,
	bodyfont=\normalfont,
	headformat=\NUMBER\ \NAME\NOTE,
	headpunct=.,
	postheadspace=1em,
	spaceabove=15pt,spacebelow=10pt,
	shaded={bgcolor=fb10_blue!20},
	postheadhook=\theorembookmark]%
{mainstyle_blue}
\declaretheoremstyle[%
	headfont=\sffamily\bfseries,
	notefont=\normalfont\sffamily\scshape,
	bodyfont=\normalfont,
	headformat=\NUMBER\ \NAME\NOTE,
	headpunct=.,
	postheadspace=1em,
	spaceabove=15pt,spacebelow=10pt,
	postheadhook=\theorembookmark]%
{mainstyle_unshaded}
\declaretheoremstyle[%
	headfont=\sffamily\bfseries,
	notefont=\normalfont\sffamily\scshape,
	bodyfont=\normalfont,
	headformat=\NUMBER\NAME\NOTE,
	headpunct=.,
	postheadspace=1em,
	spaceabove=15pt,spacebelow=10pt,
	% shaded={bgcolor=gray!20},
	postheadhook=\theorembookmark]%
{mainstyle_unnumbered}
\declaretheoremstyle[%
	headfont=\sffamily\bfseries,
	notefont=\normalfont\sffamily\scshape,
	bodyfont=\normalfont,
	headformat=swapnumber,
	headpunct=.,
	postheadspace=1em,
	spaceabove=15pt,spacebelow=10pt,
	shaded={bgcolor=gray!20},
	postheadhook=\theorembookmark,
	qed=\qedsymbol]%
{mainstyleB}
\declaretheorem[name=Definition,parent=section,style=mainstyle_blue]{definition}
\declaretheorem[name=Definition \& Proposition,refname=Proposition,sharenumber=definition,style=mainstyle_blue]{definitionP}
\declaretheorem[name=Definition,numbered=no,style=mainstyle_unnumbered]{definition*}
\declaretheorem[name=Theorem,sharenumber=definition,style=mainstyle]{theorem}
\declaretheorem[name=Theorem,numbered=no,style=mainstyle_unnumbered]{theorem*}
\declaretheorem[name=Proposition,sharenumber=definition,style=mainstyle,refname=Proposition]{proposition}
\declaretheorem[name=Lemma,sharenumber=definition,style=mainstyle]{lemma}
\declaretheorem[name=Satz,sharenumber=definition,style=mainstyle,refname=Satz]{satz}
\declaretheorem[name=Satz,sharenumber=definition,style=mainstyle_unshaded]{satzUnshaded}
\declaretheorem[name=Definition,sharenumber=definition,style=mainstyle_unshaded]{definitionUnshaded}
\declaretheorem[name=Satz,numbered=no,style=mainstyle_unnumbered]{satz*}
\declaretheorem[name=Korollar,sharenumber=definition,style=mainstyle,refname=Korollar]{korollar}
\declaretheorem[name=Korollar,sharenumber=definition,style=mainstyleB,refname=Korollar]{korollarB}
\declaretheorem[name=Frage,numbered=no,style=mainstyle_unnumbered]{frage}
\declaretheorem[name=Frage,sharenumber=definition,style=mainstyle_unshaded]{frageA}
\declaretheorem[name=Erinnerung,sharenumber=definition,style=mainstyle_unshaded]{erinnerungA}
\declaretheorem[name=Ausblick,sharenumber=definition,style=mainstyle_unshaded]{ausblick}
\declaretheorem[name=Konvention,sharenumber=definition,style=mainstyle]{konvention}
\declaretheorem[name=Notation,sharenumber=definition,style=mainstyle_unshaded]{notation}
\declaretheorem[name=Bemerkung,sharenumber=definition,style=mainstyle_unshaded,refname=Bemerkung]{bemerkung}
\declaretheorem[name=Bemerkung,numbered=no,style=mainstyle_unnumbered]{bemerkung*}
\declaretheorem[name=Beispiel,sharenumber=definition,style=mainstyle_unshaded,refname=Beispiel]{beispiel}
\declaretheorem[name=Beispiel,numbered=no,style=mainstyle_unnumbered]{beispiel*}
\declaretheorem[name=Exkurs,numbered=no,style=mainstyle_unnumbered]{exkurs*}

% english versions
\declaretheorem[name=Remark,sharenumber=definition,style=mainstyle_unshaded]{remark}
\declaretheorem[name=Remark,numbered=no,style=mainstyle_unnumbered]{remark*}
\declaretheorem[name=Example,sharenumber=definition,style=mainstyle_unshaded]{example}
\declaretheorem[name=Corollary,sharenumber=definition,style=mainstyle]{corollary}

% -- Beweise
\declaretheoremstyle[headfont=\bfseries\scshape,bodyfont=\normalfont,headpunct=:,postheadspace=1em,spacebelow=12pt,spaceabove=2pt,qed=\qedsymbol]{beweise}
\declaretheoremstyle[headfont=\bfseries\scshape,bodyfont=\normalfont,headpunct=:,postheadspace=1em,spacebelow=12pt,spaceabove=2pt]{beweisskizze}
\declaretheoremstyle[headfont=\sffamily\bfseries,bodyfont=\normalfont,headpunct=:,postheadspace=1em,spacebelow=10pt,spaceabove=10pt]{bemerkungen}
\declaretheorem[name=Beweis,numbered=no,style=beweise]{beweis}
\let\proof\relax
\declaretheorem[name=Proof,numbered=no,style=beweise]{proof}
\declaretheorem[name=Sketch of Proof,numbered=no,style=beweisskizze]{sketch}

\declaretheorem[name=Übung,numbered=no,style=bemerkungen]{uebung}
\declaretheorem[name=Erinnerung,numbered=no,style=bemerkungen]{erinnerung}


%--Skalarprodukt (3 Varianten)
\DeclarePairedDelimiterX\sprod[2]{\langle}{\rangle}{#1\,\delimsize\vert\,#2}
\DeclarePairedDelimiterX\skal[2]{\langle}{\rangle}{#1\,,\,#2}

\makeatletter
\DeclareFontFamily{OMX}{MnSymbolE}{}
\DeclareSymbolFont{MnLargeSymbols}{OMX}{MnSymbolE}{m}{n}
\SetSymbolFont{MnLargeSymbols}{bold}{OMX}{MnSymbolE}{b}{n}
\DeclareFontShape{OMX}{MnSymbolE}{m}{n}{
    <-6>  MnSymbolE5
   <6-7>  MnSymbolE6
   <7-8>  MnSymbolE7
   <8-9>  MnSymbolE8
   <9-10> MnSymbolE9
  <10-12> MnSymbolE10
  <12->   MnSymbolE12
}{}
\DeclareFontShape{OMX}{MnSymbolE}{b}{n}{
    <-6>  MnSymbolE-Bold5
   <6-7>  MnSymbolE-Bold6
   <7-8>  MnSymbolE-Bold7
   <8-9>  MnSymbolE-Bold8
   <9-10> MnSymbolE-Bold9
  <10-12> MnSymbolE-Bold10
  <12->   MnSymbolE-Bold12
}{}

\let\llangle\@undefined
\let\rrangle\@undefined
\DeclareMathDelimiter{\llangle}{\mathopen}%
                     {MnLargeSymbols}{'164}{MnLargeSymbols}{'164}
\DeclareMathDelimiter{\rrangle}{\mathclose}%
                     {MnLargeSymbols}{'171}{MnLargeSymbols}{'171}
\makeatother

\DeclarePairedDelimiterX\sskal[2]{\llangle}{\rrangle}{#1\,,\,#2}

%--Betrag, Gaußklammer
\DeclarePairedDelimiter{\abs}{\lvert}{\rvert}
\DeclarePairedDelimiter{\floor}{\lfloor}{\rfloor}
\DeclarePairedDelimiter{\ceil}{\lceil}{\rceil}

%--Norm
\DeclarePairedDelimiter\norm{\Vert}{\Vert}


%--Umklammern
\DeclarePairedDelimiter\enbrace{(}{)}
\DeclarePairedDelimiter\benbrace{[}{]}
\DeclarePairedDelimiter\bbenbrace{[\![}{]\!]}
\DeclarePairedDelimiter\lenbrace{<}{>}
\DeclarePairedDelimiter\angbrace{\langle}{\rangle}
\newcommand{\ssbrace}[1]{{\scriptscriptstyle\enbrace{#1}}}
\newcommand{\ssbbrace}[1]{{\scriptscriptstyle\benbrace{#1}}}

%--Mengen
\newcommand\SetSymbol[1][]{\nonscript\:#1\vert\allowbreak\nonscript\:\mathopen{}}
\providecommand\given{} % to make it exist
\DeclarePairedDelimiterX\set[1]\{\}{\renewcommand\given{\SetSymbol[\delimsize]}#1}

%--Abbildungsdefinition
\newcommand{\mapdef}[5]{%
	\[
		\begin{array}{rcl}
			\textstyle #1 &\xrightarrow{\minwidthbox{#5}{2em}} & \textstyle #2 \\[0.5ex]
			\textstyle #3 &\xmapsto{\minwidthbox{\mbox{ }}{2em}} & \textstyle #4
		\end{array}
	\]
}

%--modifiziertes Stackrel
\newcommand{\StackText}[2]{\stackrel{\mbox{\scriptsize #1}}{#2}}
\newcommand{\StackTextClap}[2]{\stackrel{\mathclap{\mbox{\scriptsize #1}}}{#2}}

%--Befehl für eine Box mit Mindestweite
\DeclareRobustCommand{\minwidthbox}[2]{%
  \ifmmode
    \expandafter\mathmakebox
  \else
    \expandafter\makebox
  \fi
  [\ifdim#2<\width\width\else#2\fi]{#1}%
}

%--Befehl zum Einkreisen
\newcommand*\circled[1]{\tikzexternaldisable\tikz[baseline=(char.base)]{\node[shape=circle,draw,inner sep=2pt] (char) {#1};}\tikzexternalenable}


%--Differential
\newcommand{\diff}[2]{\ensuremath{\frac{{\partial #1}}{{\partial #2}} }}
\newcommand{\diffd}[2]{\ensuremath{\frac{\mathd #1}{\mathd #2} }}

%--Inhaltsverzeichnis
\usepackage[tocindentauto]{tocstyle}
\usetocstyle{KOMAlike}	
\shorthandon{"}
		
%--Punkte (als letztes laden)
\usepackage{ellipsis}
