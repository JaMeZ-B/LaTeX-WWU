% % % % % % % % % % % % % % % % % % % % % % % % % % % % % % % % %
%		PhistScript.tex											%
%		XeLaTeX-Konfiguration für Skripte						%
%																%
%		Author: Phil Steinhorst									%
% % % % % % % % % % % % % % % % % % % % % % % % % % % % % % % % %

\RequirePackage[thinlines]{easybmat} %-- muss aufgrund eines Bugs vor etex und tikz geladen werden
\documentclass[a4paper, twoside, headsepline, index=totoc,toc=listof, fontsize=9pt, cleardoublepage=empty, headinclude, DIV=13, BCOR=13mm, titlepage]{scrartcl}

\usepackage{scrtime} 	% KOMA, Uhrzeit ermoeglicht
\usepackage{scrpage2} 	% wie fancyhdr, nur optimiert auf KOMA-Skript, leicht andere Syntax
\usepackage{etoolbox}

\usepackage[usenames, table, x11names]{xcolor}	% Farben. Vor TikZ laden!

% % % TikZ-Pakete % % %
\usepackage{tikz}
\usepackage{tikz-cd}
\usetikzlibrary{external}
\tikzset{>=latex}
\usetikzlibrary{shapes,arrows,intersections}
\usetikzlibrary{calc,3d}
\usetikzlibrary{decorations.pathreplacing,decorations.markings}
\usetikzlibrary{angles}
\tikzexternalize[prefix=tikz/, up to date check=diff]	% -shell-escape-Flag nötig!

% % % verwende LuaLaTeX, wegen dynamischer Speicherallokation
\tikzset{external/system call={xelatex \tikzexternalcheckshellescape 
    -halt-on-error -interaction=batchmode -jobname "\image" "\texsource"}}
    
% % % tikzexternalize für tikzcd deaktivieren, da inkompatibel
\AtBeginEnvironment{tikzcd}{\tikzexternaldisable}
\AtEndEnvironment{tikzcd}{\tikzexternalenable}

% % % um Inkompatibilitaeten von quotes und polyglossia bzw. babel zu vermeiden
\tikzset{
  every picture/.append style={
    execute at begin picture={\shorthandoff{"}},
    execute at end picture={\shorthandon{"}}
  }
}
\usetikzlibrary{quotes}
\usepackage{pgfplots}

% % % Mathe-Zeugs
\usepackage{mathtools} 						% beinhaltet amsmath
\mathtoolsset{showonlyrefs, centercolon}
\usepackage{wasysym}
\usepackage{amssymb} 						% zusätzliche Symbole
\usepackage{latexsym} 						% zusätzliche Symbole
\usepackage{stmaryrd} 						% für Blitz
\usepackage{nicefrac} 						% schräge Brüche
\usepackage{cancel} 						% Befehle zum Durchstreichen
\usepackage{extarrows}						% mehr Pfeile
\usepackage{mathdots}
\DeclareSymbolFont{bbold}{U}{bbold}{m}{n}
\DeclareSymbolFontAlphabet{\mathbbold}{bbold}
\newcommand{\mathds}[1]{\mathbb{#1}} 		% um Kompatibilitaet mit frueheren Benutzung von dsfont herzustellen
\def\mathul#1#2{\color{#1}\underline{{\color{black}#2}}\color{black}} %farbiges Untersteichen im Mathe-Modus

% % % XeTeX & Fonts
\usepackage{mathspec} 						% beinhaltet fontspec 
\usepackage{polyglossia} 					% babel-Ersatz
\setmainlanguage[spelling=new,babelshorthands=false]{german}
\newcommand\glqq{"}
\newcommand\grqq{"}
\defaultfontfeatures{Mapping=tex-text, WordSpace={1.4}} %
\setmainfont{SourceSansPro}[%
	Extension=.otf,
	UprightFont=*-Regular,
	BoldFont=*-Semibold,
	ItalicFont=*-LightIt,
	BoldItalicFont=*-SemiboldIt,
	SmallCapsFont={LinLibertine_R.otf},
	SmallCapsFeatures={Letters=SmallCaps}]
\setsansfont{SourceSansPro}[%
	Extension=.otf,
	UprightFont=*-Regular,
	BoldFont=*-Semibold,
	ItalicFont=*-LightIt,
	BoldItalicFont=*-SemiboldIt,
	SmallCapsFont={LinLibertine_R.otf},
	SmallCapsFeatures={Letters=SmallCaps}]
\setmonofont{SourceCodePro}[Scale=0.9,Extension=.otf,UprightFont=*-Regular, BoldFont=*-Semibold, ItalicFont=*-Light] 
\usepackage{xltxtra}
\usepackage{fontawesome}

% % % Misc.
\usepackage[neverdecrease]{paralist}
\usepackage[german=quotes]{csquotes}
\usepackage{booktabs}
\usepackage{wrapfig}
\usepackage{float}
\usepackage[margin=10pt, font=small, labelfont=sf, format=plain, indention=1em]{caption}
\captionsetup[wrapfigure]{name=Abb. }
\usepackage{stackrel}
\usepackage{ifthen}
\usepackage{multicol}
%\flushbottom

% % % Unterstreichung
\usepackage[normalem]{ulem}
\setlength{\ULdepth}{1.8pt}

% % % Indexverarbeitung
\usepackage{makeidx}
\newcommand{\bet}[1]{\textbf{#1}}
\newcommand{\Index}[1]{\textbf{#1}\index{#1}}
\makeindex
\renewcommand{\indexpagestyle}{scrheadings}

% % % Marginnote & todonotes
\usepackage{marginnote}
\renewcommand*{\marginfont}{\color{gray} \footnotesize }
\usepackage[textsize=small]{todonotes}
\makeatletter
\renewcommand{\todo}[2][]{\tikzexternaldisable\@todo[#1]{#2}\tikzexternalenable}
\makeatother

% % % Konfiguration von Hyperref pdfstartview=FitH, 
\usepackage[hidelinks, pdfpagelabels,  bookmarksopen=true, bookmarksnumbered=true, linkcolor=black, urlcolor=RoyalBlue3, plainpages=false, hypertexnames=false, citecolor=black, hypertexnames=true, pdfauthor={\verfasser}, pdfborderstyle={/S/U}, linkbordercolor=RoyalBlue3, colorlinks=true, unicode, pdfencoding=auto]{hyperref}

\newcommand{\RM}[1]{\MakeUppercase{\romannumeral #1{}}} 	% Römische Zahlen
\usepackage{ellipsis}										% Punkte

% % % % % % % % % % % % % % % % % % % % % % % % % % % % % % % % %
%		PhistMath.tex											%
%		Weitere Mathe-Befehle									%
%																%
%		Author: Phil Steinhorst									%
% % % % % % % % % % % % % % % % % % % % % % % % % % % % % % % % %

% % % Buchstaben und Zahlen
\newcommand{\aff}{\mathbb{A}}
\newcommand{\CC}{\mathbb{C}}
\newcommand{\FF}{\mathbb{F}}
\newcommand{\HH}{\mathbb{H}}
\newcommand{\KK}{\mathbb{K}}
\newcommand{\NN}{\mathbb{N}}
\newcommand{\OO}{\mathbb{O}}
\newcommand{\QQ}{\mathbb{Q}}
\newcommand{\RR}{\mathbb{R}}
\newcommand{\ZZ}{\mathbb{Z}}
\newcommand{\bigO}{\mathcal{O}}					% Landau-O
\newcommand{\ind}{1\hspace{-0,8ex}1} 			% Indikatorfunktion (Doppeleins)

% % % Abk�rzungen
\newcommand{\ab}[1]{\overline{#1}}					% Abschluss
\newcommand{\assoz}{\mathrel{\hat=}}				% assoziiert
\newcommand{\bewrueck}{\glqq$\Leftarrow$\grqq:} 	% Beweis R�ckrichtung
\newcommand{\bewhin}{\glqq$\Rightarrow$\grqq:}		% Beweis Hinrichtung
\newcommand{\borel}{\mathfrak{B}}					% Borelsche Sigma-Algebra
\newcommand{\maxid}{\mathfrak{m}}					% maximales Ideal
\newcommand{\setone}{\{1\}}							% Einsmenge
\newcommand{\leb}{\lambda \hspace{-0,95ex}\lambda}	% Lebesgue-Ma� (Doppel-Lambda)
\newcommand{\Lp}{\mathcal{L}}						% L^p-R�ume
\newcommand{\NT}{\trianglelefteq}					% Normalteiler
\newcommand{\setnull}{\{0\}}						% Nullmenge
\newcommand{\weak}{\rightharpoonup}					% schwache Konvergenz
\newcommand{\weaks}{\overset{*}{\rightharpoonup}}	% schwache *-Konvergenz
\newcommand{\salg}{\mathfrak{A}}					% Sigma-Algebra (Skript-A)
\newcommand{\zyklot}[1]{#1^{(\infty)}}				% zyklotomische Erweiterung


% % % Operatoren
\DeclareMathOperator{\Alt}{Alt} 					% Alternierende n-Linearform
\DeclareMathOperator{\Aut}{Aut} 					% Automorphismen
\DeclareMathOperator{\Bil}{Bil} 					% Bilinearformen
\DeclareMathOperator{\bild}{Bild} 					% Bild
\DeclareMathOperator{\Char}{char} 
\DeclareMathOperator{\dom}{dom} 					% Domain
\DeclareMathOperator{\diam}{diam}					% Durchmesser
\DeclareMathOperator{\dist}{dist} 					% Distanz
\DeclareMathOperator{\eqs}{\mathrel{\widehat{=}}}	% entspricht
\DeclareMathOperator{\diver}{div} 					% Gradient
\DeclareMathOperator{\EPK}{EPK} 					% Einpunktkompaktifizierung
\DeclareMathOperator{\End}{End} 					% Endomorphismen
\DeclareMathOperator{\esssup}{esssup}				% essentielles Supremum
\DeclareMathOperator{\Gal}{Gal}	 					% Galoisgruppe
\DeclareMathOperator{\ggT}{ggT} 					% ggT
\DeclareMathOperator{\GL}{GL}						% allgemeine lineare Gruppe
\DeclareMathOperator{\grad}{grad} 					% Gradient
\DeclareMathOperator{\Grad}{Grad} 					% Grad
\DeclareMathOperator{\Hess}{Hess} 					% Hesse-Matrix
\DeclareMathOperator{\Hom}{Hom} 					% Homomorphismen
\DeclareMathOperator{\id}{id} 						% identische Abbildung
\DeclareMathOperator{\im}{im} 						% image
\DeclareMathOperator{\Jac}{Jac} 					% Jacobson-Radikal
\DeclareMathOperator{\Kern}{Kern}					% Kern
\DeclareMathOperator{\kgV}{kgV} 					% kgV
\DeclareMathOperator{\Koker}{Koker} 				% Kokern
\DeclareMathOperator{\Cov}{Cov} 					% Kovarianz
\DeclareMathOperator{\Mod}{Mod} 					% Moduln
\DeclareMathOperator{\modu}{mod} 					% Modulo
\DeclareMathOperator{\ord}{ord} 					% Ordnung
\DeclareMathOperator{\der}{\partial}				% Partielle Ableitung
\DeclareMathOperator{\pot}{\mathcal{P}}				% Potenzmenge
\DeclareMathOperator{\prlim}{\varprojlim\limits}	% projektiver Limes
\DeclareMathOperator{\Quot}{Quot}					% Quotientenring
\DeclareMathOperator{\Rang}{Rang} 					% Rang
\DeclareMathOperator{\rot}{rot} 					% Rotation
\DeclareMathOperator{\sgn}{sgn} 					% Signum
\DeclareMathOperator{\Spec}{Spec} 					% Spektrum
\DeclareMathOperator{\SL}{SL} 						% Spezielle lineare Gruppe
\DeclareMathOperator{\SO}{SO} 						% Spezielle orthogonale Gruppe
\DeclareMathOperator{\SU}{SU} 						% Spezielle unit�re Gruppe
\DeclareMathOperator{\Spur}{Spur} 					% Spur
\DeclareMathOperator{\supp}{supp} 					% Tr�ger
\DeclareMathOperator{\Sym}{Sym} 					% Symmetrische Gruppe
\DeclareMathOperator{\tr}{tr} 						% trace

% % % Klammerungen
\DeclarePairedDelimiter{\abs}{\lvert}{\rvert}		% Betrag
\DeclarePairedDelimiter{\ceil}{\lceil}{\rceil}		% aufrunden
\DeclarePairedDelimiter{\floor}{\lfloor}{\rfloor}	% aufrunden
\DeclarePairedDelimiter{\sprod}{\langle}{\rangle}	% spitze Klammern
\DeclarePairedDelimiter{\enbrace}{(}{)}				% runde Klammern
\DeclarePairedDelimiter{\benbrace}{\lbrack}{\rbrack}			% eckige Klammern
\DeclarePairedDelimiter{\penbrace}{\{}{\}}			% geschweifte Klammern

% % % Norm
\DeclarePairedDelimiter\doppelstrich{\Vert}{\Vert}
\newcommand{\norm}[2][\relax]{
\ifx#1\relax \ensuremath{\doppelstrich*{#2}}
\else \ensuremath{\doppelstrich*{#2}_{#1}}
\fi}							% selbst definierte Mathe-Befehle

\setlength\parindent{0pt}             % ohne Einrueckung
\renewcommand{\baselinestretch}{1.2}
% % % Konfiguration von scrheadings
\setheadsepline{1pt}[\color{black}]
\pagestyle{scrheadings}
\clearscrheadfoot

% % % Kopf-/Fußzeilenlayout
\providecommand{\shortFach}{\fach}
\ohead{\small \leftmark}
\ofoot[{ \small \thepage}]{ \small \thepage} 
\automark{section}

% Metadaten
\title{\fach}
\author{\verfasser}
\subtitle{\untertitel}
\date{\semester}

% Inhaltsverzeichnis
\usepackage[tocindentauto]{tocstyle}
\usetocstyle{KOMAlike}	

% ntheorem package
\usepackage[hyperref]{ntheorem}			% Lade ntheorem mit hyperref-Workaround
\theoremstyle{break}					% Zeilenumbruch nach Theoremkopf
\theoremheaderfont{\usekomafont{disposition}} % Benutze die gleiche Schriftart, wie für \section \subsection etc.
\theorembodyfont{\normalfont}			% nicht kursiv
\theorempreskipamount0.5cm				% Abstand vor Theorem
\theorempostskipamount0.5cm				% Abstand nach Theorem

% Workaround für Tabulatoren im Theorem-Verzeichnis
\makeatletter
\def\thm@@thmline@name#1#2#3#4{%
        \@dottedtocline{-2}{0em}{2.3em}%
                   {\makebox[\widesttheorem][l]{#1 \protect\numberline{#2}}#3}%
                   {#4}}
\@ifpackageloaded{hyperref}{
\def\thm@@thmline@name#1#2#3#4#5{%
    \ifx\\#5\\%
        \@dottedtocline{-2}{0em}{2.3em}%
            {\makebox[\widesttheorem][l]{#1 \protect\numberline{#2}}#3}%
            {#4}
    \else
        \ifHy@linktocpage\relax\relax
            \@dottedtocline{-2}{0em}{2.3em}%
                {\makebox[\widesttheorem][l]{#1 \protect\numberline{#2}}#3}%
                {\hyper@linkstart{link}{#5}{#4}\hyper@linkend}%
        \else
            \@dottedtocline{-2}{0em}{2.3em}%
                {\hyper@linkstart{link}{#5}%
                  {\makebox[\widesttheorem][l]{#1 \protect\numberline{#2}}#3}\hyper@linkend}%
                    {#4}%
        \fi
    \fi}
}
\makeatother
\newlength\widesttheorem
\AtBeginDocument{
  \settowidth{\widesttheorem}{Proposition 10.10\quad}	% Referenzstring für die Breite der ersten Spalte
}

\newcommand{\qed}{\ifmmode \tag*{$\square$} \else \hfill $\square$ \fi} % qed-Symbol.