%!TEX root = ../madsen_weiss.tex
%!TEX TS-program = xelatex
%!TEX TS-options = -shell-escape
\section{Homological stability III. -- Connectivity}
In this talk we will fill the gap in the proof of the stability theorem, i.e. proving the connectivity of the curve complexes $\mathcal{O}^1$ and $\mathcal{O}^2$.
To be precise we will prove the following theorem.

\begin{theorem}[label=connec_main]
	$\mathcal{O}(S,b_0,b_1)$ is $(g-2)$-connected.
\end{theorem}

We start by defining a few more complexes larger than $\mathcal{O}^i(S)$.

Let $\Delta \subset \partial S$ be a non-empty set of points.
An arc $A$ with boundary in $\Delta$ is called \emph{trivial}, if $S \setminus a$ consists of two components on of which is a disc intersecting $\Delta$ only in the boundary of $a$.

\begin{definition}
	The \Index{full arc complex} $\mathcal{A}(S,\Delta)$ is the simplical complex with vertices isotopy classes of non-trivial arcs with boundary in $\Delta$.
\end{definition}
A $p$-simplex in $\mathcal{A}(S,\Delta)$ is a collection of isotopy classes of arcs $\angbrace*{a_0, \ldots ,a_p}$ which can be represented by a collection of $p+1$ arcs with disjoint interiors.

Given two disjoint subsets $\Delta_0, \Delta_1 \subset \partial S$ we define a subcomplex $\mathcal{B}(S,\Delta_0,\Delta_1) \subset \mathcal{A}(S,\Delta_0 \cup \Delta_1)$ by restricting to arcs with one boundary point in $\Delta_0$ and one in $\Delta_1$.
This is obviously a subcomplex.

The next property we require is that the arcs need to be non-seperating.
Therefore we let 
\[
	\mathcal{B}_0(S,\Delta_0, \Delta_1) \subset \mathcal{B}(S,\Delta_0, \Delta_1) 
\]
be the subcomplex consisting of \emph{non-seperating collections}, meaning simplices $\sigma = \angbrace{a_0, \ldots ,a_p}$ such that the complement of the arcs $a_0, \ldots ,a_p$ in $S$ is connected.
With this definition $\mathcal{O}(S,b_0,b_1)$ is the subcomplex of $\mathcal{B}_0(S,\set*{b_0},\set*{b_1})$ where the ordering of the arcs in each simplex in $b_0$ is opposite to the ordering of the arcs at $b_1$ (compare with\todo{ref first talk of this chapter}).

The proof of \autoref{connec_main} will use the following inclusions
\[
	\begin{tikzcd}
		\mathcal{A}(S,\Delta) & \mathcal{B}(S, \Delta_0, \Delta_1) \lar["i_1"',hook'] & \mathcal{B}_0(S,\Delta_0,\Delta_1) \lar["i_2"',hook'] & \mathcal{O}(S,b_0,b_1) \lar["i_3"',hook']
	\end{tikzcd}
\]
We will prove that $\mathcal{A}(S,\Delta)$ is contractible and deduce bounds for the connectivity of the complexes step by step.
Altough we are interested in the case $\Delta=\set*{b_0,b_1}$ in the end, we will need the more general complexes along the way.\footnote{because cutting along an arc may result in several copies of the boundary points}

\begin{theorem}
	$\mathcal{A}(S,\Delta)$ is contractible as long as $S$ is neither a disk nor an annulus with $\Delta$ included in a single component of $\partial S$ in which case it is $(q+2r-7)$-connected, where $q=\abs*{\Delta}$ and $r=1,2$ is the number of boundary components of $S$.
\end{theorem}

To prove this theorem we will need the following lemma:

\begin{lemma}
	Suppose $\mathcal{A}(S,\Delta) \neq \emptyset$ and $\Delta'$ is obtained form $\Delta$ by adding an extra point in a component of $\partial S$ already containing a point of $\Delta$.
	If $\mathcal{A}(S,\Delta)$ is $d$-connected, then $\mathcal{A}(S,\Delta')$ is $(d+1)$-connected.
\end{lemma}
\begin{sketch}
	Suppose $\Delta' = \Delta \cup \set*{p'}$ and find $p \in \Delta$ closest to $p'$ in $\partial S$.
	Define two arcs $I$ and $I'$ as in \cref{fig:arcs_lemma_d_dplus1}.
	\begin{figure}[thb]
		\centering
		\begin{tikzpicture}
			\draw (0,0) rectangle (2,2);
		\end{tikzpicture}
		\caption{Definition of the non-trivial arcs $I$ and $I'$ near the added point $p'$.}\label{fig:arcs_lemma_d_dplus1}
	\end{figure}
	We get a decomposition
	\[
		\mathcal{A}(S,\Delta') = \Star(I) \cup_{\Link(I)} X
	\]
	with $X$ being the subcomplex of $\mathcal{A}(S,\Delta')$ of collections of arcs not containing I.
	The labour-intensive part of the proof is now to construct a deformation retraction $X \to \Star(I')$ (see \textcite{wahlHomological} for the details).
	Hence $X$ is contractible.
	Using the fact, that $\mathcal{A}(S,\Delta)$ is isomorphic to $\Link(I)$ we get the desired result: 
	If $d \ge 0$ Seifert-van Kampen implies, that $\mathcal{A}(S,\Delta')$ is simply connected and therefore we may use homology to check $(d+1)$-connectivity (Hurewicz).
	This easily follow via Mayer-Vietoris.
\end{sketch}