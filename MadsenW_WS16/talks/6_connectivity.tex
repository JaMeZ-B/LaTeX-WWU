%!TEX root = ../madsen_weiss.tex
%!TEX TS-program = xelatex
%!TEX TS-options = -shell-escape
\renewcommand{\setminus}{\backslash}
\section{Homological stability III. -- Connectivity}
In this talk we will fill the gap in the proof of the stability theorem, i.e. proving the connectivity of the curve complexes $\mathcal{O}^1$ and $\mathcal{O}^2$.
To be precise we will prove the following theorem.

\begin{restatable}[{name=[Connectivity]}]{theorem}{connectivity}\label{connec_main}
	$\mathcal{O}(S,b_0,b_1)$ is $(g-2)$-connected.
\end{restatable}

\begin{remark*}
	Just a quick reminder of a few definitions and conventions we need along the way:
	\begin{itemize}[itemsep=0pt]
		\item \enquote{$(-1)$-connected} means non-empty by convention
		\item For a simplex $\sigma$ in a simplical complex we define
		\(
			\Star(\sigma)
		\)
		as the subcomplex consisting of all simplices that contain $\sigma$ together with their faces.
		\item We also define $\Link(\sigma) \subset \Star(\sigma)$ as the subcomplex of the star consisting of all simplices disjoint from $\sigma$.
	\end{itemize}
\end{remark*}

We start by defining a few more complexes larger than $\mathcal{O}^i(S)$.

Let $\Delta \subset \partial S$ be a non-empty set of points.
An arc $A$ with boundary in $\Delta$ is called \emph{trivial}, if $S \setminus a$ consists of two components on of which is a disc intersecting $\Delta$ only in the boundary of $a$.

\begin{definition*}
	The \Index{full arc complex} $\mathcal{A}(S,\Delta)$ is the simplical complex with vertices isotopy classes of non-trivial arcs with boundary in $\Delta$.
\end{definition*}
A $p$-simplex in $\mathcal{A}(S,\Delta)$ is a collection of isotopy classes of arcs $\angbrace*{a_0, \ldots ,a_p}$ which can be represented by a collection of $p+1$ arcs with disjoint interiors.

Given two disjoint subsets $\Delta_0, \Delta_1 \subset \partial S$ we define a subcomplex $\mathcal{B}(S,\Delta_0,\Delta_1) \subset \mathcal{A}(S,\Delta_0 \cup \Delta_1)$ by restricting to arcs with one boundary point in $\Delta_0$ and one in $\Delta_1$.
This is obviously a subcomplex.

The next property we require is that the arcs need to be non-seperating.
Therefore we let 
\[
	\mathcal{B}_0(S,\Delta_0, \Delta_1) \subset \mathcal{B}(S,\Delta_0, \Delta_1) 
\]
be the subcomplex consisting of \emph{non-seperating collections}, meaning simplices $\sigma = \angbrace{a_0, \ldots ,a_p}$ such that the complement of the arcs $a_0, \ldots ,a_p$ in $S$ is connected.
With this definition $\mathcal{O}(S,b_0,b_1)$ is the subcomplex of $\mathcal{B}_0(S,\set*{b_0},\set*{b_1})$ where the ordering of the arcs in each simplex in $b_0$ is opposite to the ordering of the arcs at $b_1$ (compare: first talk of the section about homological stability).

The proof of \autoref{connec_main} will use the following inclusions
\[
	\begin{tikzcd}
		\mathcal{A}(S,\Delta) & \mathcal{B}(S, \Delta_0, \Delta_1) \lar["i_1"',hook'] & \mathcal{B}_0(S,\Delta_0,\Delta_1) \lar["i_2"',hook'] & \mathcal{O}(S,b_0,b_1) \lar["i_3"',hook']
	\end{tikzcd}
\]
We will prove that $\mathcal{A}(S,\Delta)$ is contractible and deduce bounds for the connectivity of the complexes step by step.
Although we are interested in the case $\Delta=\set*{b_0,b_1}$ in the end, we will need the more general complexes along the way.\footnote{because cutting along an arc may result in several copies of the boundary points}

\begin{theorem}[label=thm:connec_full]
	$\mathcal{A}(S,\Delta)$ is contractible as long as $S$ is neither a disk nor an annulus with $\Delta$ included in a single component of $\partial S$ in which case it is $(q+2r-7)$-connected, where $q=\abs*{\Delta}$ and $r=1,2$ is the number of boundary components of $S$.
\end{theorem}

To prove this theorem we will need the following lemma:

\begin{lemma}[label=lem:d_dplus1]
	Suppose $\mathcal{A}(S,\Delta) \neq \emptyset$ and $\Delta'$ is obtained form $\Delta$ by adding an extra point in a component of $\partial S$ already containing a point of $\Delta$.
	If $\mathcal{A}(S,\Delta)$ is $d$-connected, then $\mathcal{A}(S,\Delta')$ is $(d+1)$-connected.
\end{lemma}
\begin{sketch}
	Suppose $\Delta' = \Delta \cup \set*{p'}$ and find $p \in \Delta$ closest to $p'$ in $\partial S$.
	Define two arcs $I$ and $I'$ as in \cref{fig:arcs_lemma_d_dplus1}.
	\begin{figure}[thb]
		\centering
		\begin{tikzpicture}
			\draw (0,0) rectangle (2,2);
		\end{tikzpicture}
		\caption{Definition of the non-trivial arcs $I$ and $I'$ near the added point $p'$.}\label{fig:arcs_lemma_d_dplus1}
	\end{figure}
	We get a decomposition
	\[
		\mathcal{A}(S,\Delta') = \Star(I) \cup_{\Link(I)} X
	\]
	with $X$ being the subcomplex of $\mathcal{A}(S,\Delta')$ of collections of arcs not containing I.
	The labour-intensive part of the proof is now to construct a deformation retraction $X \to \Star(I')$ (see \textcite{wahlHomological} for the details; the argument is similar to the one in the proof of \autoref{thm:connec_full}).
	Hence $X$ is contractible.
	Using the fact, that $\mathcal{A}(S,\Delta)$ is isomorphic to $\Link(I)$ we get the desired result: 
	If $d \ge 0$ Seifert-van Kampen implies, that $\mathcal{A}(S,\Delta')$ is simply connected and therefore we may use homology to check $(d+1)$-connectivity (Hurewicz).
	This easily follows via Mayer-Vietoris.\todo{vlt das Bild übernehmen?}
\end{sketch}

\begin{proof}[{name={of \autoref{thm:connec_full}}}]
	Let's handle the two exceptional cases first (disk and cylinder with $\Delta$ in a single component of $\partial S$):
	\begin{figure}[htb]
		\centering
		\missingfigure{disk and cylinder with sufficiently large set $\Delta$ and the corresponding arcs}
		\caption{Non-emptiness of the arc complex of certain surfaces}\label{fig:non-empty}
	\end{figure}
	As one sees in \cref{fig:non-empty} $\mathcal{A}(D^2,\Delta)$ is non-empty (there are non-trivial arcs connecting points in $\Delta$) as soon as $\Delta$ has four points; similarly $\mathcal{A}(S^1 \times [0,1],\Delta)$ is non-empty if $\Delta$ consists of two points in the same boundary component.
	The claim is, that $\mathcal{A}(S,\Delta)$ is $(q+2r -7)$-connected in either of those two cases.
	Plugging in $q=4$ and $r=1$ we get $(-1)$-connectivity which we just showed in case of $S$ being a disk.
	For the cylinder ($q=2$ and $r=2$) we get $2 +4 -7 =-1$ and therefore the theorem holds for a cylinder with the minimal choice of $\Delta$ s.th. $\mathcal{A}(S^1\times [0,1],\Delta) \neq \emptyset$ as well.
	As adding a point to $\Delta$ obviously increases $q$ by one, the theorem holds for every choice of $\Delta$ by \autoref{lem:d_dplus1}.
	
	We may now assume that we are not in one of these two exceptional cases.
	Because of \autoref{lem:d_dplus1} we may further assume that in the general case $\Delta$ has at most one point in each boundary component of $S$ to prove contractibility of $\mathcal{A}(S,\Delta)$. 
	We first notice, that $\mathcal{A}(S,\Delta)$ is not empty: 
	\begin{itemize}
		\item If $\Delta$ has at least two points, any arc connecting two points in different boundary components is non-trivial (because it does not disconnect the surface).
		\item If $\Delta$ consists of just a single point, we know that $S$ has at least genus one or at least three boundary components.
		\Cref{fig:non-empty-delta-one} gives examples of non-trivial arcs in those cases.
	\end{itemize}
	\begin{figure}[htb]
		\centering
		\missingfigure{non-trivial arcs for pair of pants and genus one surface}
		\caption{Non-emptiness of $\mathcal{A}(S,\Delta)$, if $\Delta$ is a singleton}\label{fig:non-empty-delta-one}
	\end{figure}
	To prove contractibility of $\mathcal{A}(S,\Delta)$ we fix a point $p \in \Delta$ and an arc $a$ of $\mathcal{A}(S,\Delta)$ with a least one endpoint at $p$ and proceed by
	constructing a retraction to $\Star(a)$.
	To do so we need to fix a germ of $a$ at $p$ as well.
	
	Given a $q$-simplex $\sigma$ of $\mathcal{A}(S,\Delta)$ we choose representing arcs $a_0, \ldots ,a_q$ with minimal and transverse intersection with $a$.
	If there are $k$ intersections at germs of arcs $\gamma_1, \ldots ,\gamma_k$, we are going to define $k$ intermediate simplices $r_1(\sigma), \ldots ,r_k(\sigma)$ to get to the desired retraction.
	For $\gamma_i$ let $a_{j_i}$ be the arc $\gamma_i$ is a germ of.

	If $a_i$ intersects $a$ at the point $x$ we define $L(a_i)$ and $R(a_i)$ by cutting $a_i$ at $x$ and connecting the new endpoints to the boundary via $a$.
	Pushing the arcs around a little they become disjoint of $a$ and each other.
	If $L(a_i)$ or $R(a_i)$ is trivial we forget it.
	Since $\Delta$ has at most one point in each boundary component and therefore trivial arcs have to be a loop at a single point in $\Delta$, the only case in which this constructions leads to trivial arcs is the one shown in \cref{fig:surg_trivial}.
	\Cref{fig:surg_trivial} further shows that $L(a_i)$ and $R(a_i)$ cannot be be trivial at the same time.
	\begin{figure}[hbt]
		\centering
		\missingfigure{Surgery creating a trivial arc (figure 12 in \cite{wahlHomological})}
		\caption{A situation in which surgery creates a trivial arc}\label{fig:surg_trivial}
	\end{figure}
	
	We define $r_i(\sigma)$ as the $(q+ l_i)$-simplex with $1 \le l_i \le i + 2$ consisting of the vertices $L^{\epsilon_i(l)}(a_l)$ and $R^{\epsilon_i(l)}(a_l)$ (if non-trivial)  where 
	\begin{itemize}
		\item $\epsilon_i(l) \in \set*{0, \ldots ,k}$ is the number of germs $\gamma_j$ included in $a_l$ with $j < i$\todo{shouldn't this be $\le$? Otherwise $r_1(\sigma)$ is ill-defined} and
		\item $L^0(a_l)=R^0(a_l)=a_l$ is a single vertex.
	\end{itemize}
	We want a map $f \colon I \times \mathcal{A}(S,\Delta) \to \mathcal{A}(S,\Delta)$ with the properties $f(s,\Star(a)) = \id_{\Star(a)}$, $f(1,\mathcal{A}(S,\Delta)) \subset \Star(a)$ and $f(0,\mathcal{A}(S,\Delta)) = \id_{\mathcal{A}(S,\Delta)}$.
	
	\todo[inline]{complete the proof}
\end{proof}

To state the next theorem -- which is the most intricate one -- we need to define \emph{pure} and \emph{impure edges}:
The two disjoint sets $\Delta_0, \Delta_1$ define a decomposition of $\partial S$ into vertices (the set $\Delta_0 \cup \Delta_1$), edges and circles without vertices.\todo{draw a picture}
\begin{itemize}
	\item An edge is called \emph{pure} if both of its endpoints lie in either $\Delta_0$ or $\Delta_1$; it is called \emph{impure} otherwise.
	The number of impure edges is always even (easy induction argument).
	\item A boundary component of $S$ containing points of $\Delta_0 \cup \Delta_1$ is called \emph{pure}, if it is composed of pure edges, i.e. the point are either all in $\Delta_0$ or all in $\Delta_1$.
\end{itemize}


Note that a priori this definition has nothing to do with the arcs we are interested in as it only involves $\Delta_0$ and $\Delta_1$. 

\begin{theorem}
	The complex $\mathcal{B}(S,\Delta_0,\Delta_1)$ is $(4g +r +r ' +l +m -6)$-connected, where $g$ ist the genus of $S$, $r$ is the number of boundary components, $r'$ the number of components of $\partial S$ containing points of $\Delta_0 \cup \Delta_1$, $l$ is half the number of \emph{impure edges} and $m$ is the number of \emph{pure edges}.
\end{theorem}

To prove this one first eliminates a few cases before using the general inductive argument.


\begin{lemma}[label=lem:44]
	If $S$ has at least one pure boundary component, then $\mathcal{B}(S,\Delta_0,\Delta_1)$ is contractible.
\end{lemma}

To prove this one uses the same technique as in the proof of \autoref{thm:connec_full}, tough in this case only $L(a_i)$ or $R(a_i)$ is kept after the surgery process.

\begin{lemma}[label=lem:45]
	If $S$ has at least one pure edge between a pure and an impure one in a boundary component of $S$, then $\mathcal{B}(S,\Delta_0,\Delta_1)$ is contractible.
\end{lemma}

One proceeds analogous to the proof auf \autoref{lem:d_dplus1}, by providing a contraction form the complex $\mathcal{B}(S,\Delta_0,\Delta_1)$ to the contractible complex $\Star{I'}$.

\begin{lemma}[label=lem:46]
	If the complex is non-empty, adding a pure edge between two impure edges increases the connectivity of $\mathcal{B}(S,\Delta_0,\Delta_1)$ by one.
\end{lemma}

This can be proven precisely as in \autoref{lem:d_dplus1}, while the proof of the next statement is a variation of said lemma.

\begin{lemma}[label=lem:47]
	When $S$ has at least one impure edge and the complex is non-empty, adding a boundary component to $S$ disjoint from $\Delta$ increases the connectivity of $\mathcal{B}(S,\Delta_0,\Delta_1)$ by one.
\end{lemma}

After eliminating a few cases via \cref{lem:44,lem:45,lem:46} and establishing a stepping stone by proving \cref{lem:47} the general argument to get the connectivity of $\mathcal{B}(S,\Delta_0,\Delta_1)$ is an induction over the lexicographically ordered set of triples $(g,r,q)$.

\begin{theorem}[label=thm:48]
	If $\Delta_0, \Delta_1$ are two disjoint non-empty sets of points in $\partial S$, then the complex $\mathcal{B}_0(S,\Delta_0,\Delta_1)$ is $(2g+r'-3)$-connected, for $g$ and $r'$ as above.
\end{theorem}


\connectivity*
\begin{proof}
	We first notice, that the theorem is true for $g=0$ as being $-2$-connected is a void property and true for $g=1$ as well, since $\mathcal{O}(S,b_0,b_1)$ is not empty in that case.
	We proceed with an induction on $g$ an may assume $g\ge 2$.
	
	Let $k \le g-2$ and suppose we have a continuous map $f \colon S^k \to \mathcal{O}(S,b_0,b_1)$, which we may assume to be simplical by choosing a PL triangulation\footnote{the link of every simplex is a piecewise-linear sphere} on $S^k$ and applying the simplical approximation theorem.
	\Cref{thm:48} implies, that there is an extension $\hat{f} \colon D^{k+1} \to \mathcal{B}_0(S,\set*{b_0},\set*{b_1})$ since 
	\[
		2g + r' -3 \ge g-2
	\]
	By simplical approximation we may further assume, that $\hat{f}$ is simplical as well.
	We will try to modify $\hat{f}$ appropriately so its image lies in $\mathcal{O}(S,b_0,b_1)$.
	
	Let $\sigma$ be a simplex of $\mathcal{B}_0(S,\set*{b_0},\set*{b_1})$ with $\sigma = \lenbrace{a_0, \ldots , a_p}$ such that the arcs are ordered $a_0 < a_1 < \ldots < a_p$ in the anticlockwise ordering at $b_0$.
	We can write 
	\[
		\sigma = \sigma^g * \sigma^b
	\]
	uniquely, where $\sigma^g = \lenbrace{a_0, \ldots ,a_i}$ with $i$ maximal such that the clockwise ordering at $b_1$ starts with $a_0, \ldots , a_i$.
	With this notation $\sigma = \sigma^g$ implies that $\sigma$ is a simplex of $\mathcal{O}(S,b_0,b_1)$ (\emph{good}) and we call $\sigma$ \emph{purely bad} if $\sigma^g$ is empty.
	Vertices are always good.
	
	We claim that $S\setminus \sigma$ has genus at least $g-p$ if $\sigma$ is a purely bad $p$-simplex.
	\begin{itemize}[itemsep=0pt]
		\item If $b_0,b_1$ are in the different boundary components, this is true regardless of the fact, that $\sigma$ is purely bad:\marginnote{$\chi(S) = 2 - 2g -r$ and $\chi(S\backslash \sigma) = \chi(S)+ p + 1$}
		Cutting along the first arc of $\sigma$ reduces just the number of boundary components and does not affect the genus.
		Using the Euler characteristic we verify that removing subsequent arcs can reduce the genus only by one for each arc. 
		
		(We already used this in the first talk on homological stability; see \textcite[Prop.2.2]{wahlHomological}).
		\item If $b_0,b_1$ are in the same boundary component, we need the \enquote{badness}:
		Since $\sigma$ is bad, there are two arcs $a_i, a_j$ of $\sigma$ which are ordered clockwise at both $b_0$ and $b_1$.
		Cutting along both arcs leaves us with a surface which has the same number of boundary components as before.\todo{draw a picture?}
		By the Euler characteristic $S \setminus (a_i \cup a_j)$ has genus $g-1$.
		The remaining $p-1$ arcs of $\sigma$ can each reduce the genus by at most one.
	\end{itemize}
	We obviously want to remove the purely bad simplices from the image of $\hat{f}$.
	Let $\sigma$ be a maximal simplex of $D^{k+1}$ such that $\hat{f}(\sigma)$ is purely bad.
	Now let $b_0'$ be the copy of $b_0$ lying between the boundary component containing $b_0$ and the first arc of $\sigma$ at $b_0$ (in anticlockwise ordering) and analogously let $b_1'$ be the copy of $b_1$ between the boundary containing $b_1$ and the first arc at $b_1$ (in clockwise ordering).\todo{another picture}
	Because $\sigma$ is maximal the link of $\sigma$ is mapped to $\mathcal{O}(S\setminus \hat{f}(\sigma),b_0',b_1')$.
	
	Let $\sigma$ be a $p$-simplex.
	Then $\hat{f}(\sigma)$ is a $p'$-simplex with $p' \le p$ and by the previous claim $S \setminus \hat{f}(\sigma)$ has at least genus $g- p' \ge g- p$.
	\begin{itemize}[itemsep=0pt]
		\item Since we chose a PL triangulation of $D^{k+1}$ the $\Link(\sigma)$ is a sphere of dimension $k-p$.
		As $k \le g-2$, we have $k-p \le g -p -2$.
		\item As $\hat{f}(\sigma)$ is bad, it cannot be a vertex, so $p' \ge 2$ and $S \setminus \hat{f}(\sigma)$ has genus $\tilde{g} < g$ by the claim above.
	\end{itemize}
	Plugging all of this together we get
	\[
		k-p \le g- p -2 \le g- p' -2 \le \tilde{g} -2
	\]
	Therefore we may apply the induction hypothesis to the restriction of $\hat{f}$ to $\Link(\sigma)$ an get a extension $F$ of $\hat{f}|_{\Link(\sigma)}$ on a $(k+1-p)$-disk $K$.
	\todo[inline]{prove this!}
\end{proof}