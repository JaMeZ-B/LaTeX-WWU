%!TEX root = ../madsen_weiss.tex
%!TEX TS-program = xelatex
%!TEX TS-options = -shell-escape
\section{Surfaces, mapping class groups and surface bundles}
\todo[inline]{incomplete at the moment}



The classification of surfaces by \textcite[chap.~9, thm.~3.11]{hirschDifferential} goes as follows (not the same wording as in the talk):
\begin{theorem}[{name={classification of surfaces}}]
	Two connected compact surfaces are diffeomorphic if and only if they have the same Euler characteristic and the same number of boundary components.
\end{theorem}

Let's turn to the definition of the mapping class group: 

\begin{definition}
	Let $\Sigma$ be an oriented compact surface. 
	We define\marginnote{equiv.: $f|_U = {\id}$ for a neighbourhood of $\partial \Sigma$}
	\[
		\Diff_{\partial}^+ \coloneqq \set[\big]{f \colon \Sigma \to \Sigma \given f \text{ is an orientation preserving diffeo and } f|_{\partial \Sigma} = {\id}}
	\]
\end{definition}

\begin{remark}
	Given the $C^\infty$-topology $\Diff_\partial^+(\Sigma)$ becomes a topological group.
	This topology has the following property
	\[
		f_n \to f \iff \text{ all derivatives converge uniformly} 
	\]	
\end{remark}

\begin{definition}
	Let $\Sigma$ be an compact oriented surface.
	We define the \Index{mapping class group} of $\Sigma$ as\marginnote{this is a group, because $\Diff_\partial^+$ is a topological group}
	\[
		\Gamma(\Sigma) \coloneqq \pi_0 \enbrace*{\Diff_\partial^+(\Sigma)}
	\]
	and for simplicity we set $\Gamma_{g,r} \coloneqq \Gamma \enbrace*{\Sigma_{g,r}}$.
\end{definition}

\begin{theorem}[{name={Earle-Eells}}]
	If $g \ge 2$ or $r \ge 1$ then $\Gamma_{g,r} \simeq \Diff_\partial^+(\Sigma_{g,r})$ with $\Diff_\partial^+(\Sigma_{g,r}) \to \Gamma_{g,r}$ being the projection.
\end{theorem}

