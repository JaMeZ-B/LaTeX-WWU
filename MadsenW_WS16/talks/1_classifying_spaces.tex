%!TEX root = ../madsen_weiss.tex
%!TEX TS-program = xelatex
%!TEX TS-options = -shell-escape
\section{Fibre bundles and their classifying spaces}
In this talk we will get to know the classifying space $BG$ of a group $G$ and we will try to understand, why we care about $H_*(BG)$.

\begin{definition}[{name=[fibre bundle]}]
	A \Index{fibre bundle} with fibre $F$ is a surjective map $\pi \colon E \to X$ that is locally trivial, i.e. for every $x \in X$ there is an open neighbourhood $U \subseteq X$ and a homeomorphism $h \colon \pi^{-1}(U) \to U \times F$ such that the following diagramm commutes
	\[
		\begin{tikzcd}[row sep=large]
			\pi^{-1}(U) \rar["h","\cong"'] \dar["\pi"] & U \times F \dlar["\pr_U"] \\
			U
		\end{tikzcd}
	\]
\end{definition}

The local triviality ensures that the preimage $\pi^{-1}(x)$ of every point $x \in X$ is homeomorphic to the fibre $F$.

\begin{definition}[{name=[G-principal bundle]}]
	Let $G$ be a topological group.
	A \bet{$G$-principal bundle}\index{G-principal bundle@$G$-principal bunde} over a space $X$ is a fibre bundle $\pi \colon E \to X$ together with a free, continuous right $G$-action on the total space $E$, such that 
	\begin{itemize}
		\item $\pi(y.g) = \pi(y)$ for all $y \in E$ and $g \in G$,
		\item $G$ acts transitively on each fibre,
		\item the local trivializations can be chosen to be $G$-equivariant.\marginnote{not 100\% sure, but may be a consequence of the previous points}
	\end{itemize}
	
\end{definition}

\begin{remark}
	Since $G$ acts transitively and freely the map 
	\mapdef{G}{\pi^{-1}(x)}{g}{p.g}{\cong}
	ist a homeomorphism for a fixed $p \in \pi^{-1}(x)$, $x \in X$.
\end{remark}

\begin{example}
	\begin{enumerate}[(i)]
		\item Let $\pi \colon Y \to X$ be a Galois-covering.
		This means that the group of deck transformation $G=\operatorname{Deck}(\pi)$ acts transitively on the fibres $\pi^{-1}(x)$.
		Therefore we get a $G$-principal bundle.
		In particular taking the universal cover $Y= \tilde{X}$ we get a $\pi_1(X,x_0)$-principal bundle.
		
		This applies to the well known covering $\mathbb{R} \to S^1$, $\varphi \mapsto e^{i \varphi}$ as well.
		\item Let $G= \Un(1) =S^1 = \SO(2)$ and $E = S^{2n+1} \subseteq \mathbb{C}^{n+1}$.
		Define a $G$-action by $(v,g) \mapsto v \cdot g = g \cdot v$.
		Since $\mathbb{C}P^n = \sfrac{E}{\Un(1)}$ we see that $S^{2n+1} \to \mathbb{C}P^n$ is a $\Un(1)$-principal bundle.
	\end{enumerate}
\end{example}

One may ask how principal bundles are connected to vector bundles. 
Let $\pi \colon V \to X$ be a $\mathbb{R}$-vector bundle, i.e. $V_x = \pi^{-1}(x)$ is a  $n$-dimensional vector space.
We now construct the corresponding \Index{frame bundle} $\operatorname{Fr}(V)$ by setting
\[
	\operatorname{Fr}(V) = \coprod_{x \in X} \Iso(\mathbb{R}^n, V_x)
\]
using the obvious mapping to $X$. 
The group $\GL(n,\mathbb{R})$ acts on $\operatorname{Fr}(V)$ by $f.g \coloneqq f \circ g$.
Using a \enquote{suitable} topology $\operatorname{Fr}(V)$ becomes a $\GL(n,\mathbb{R})$-principal bundle.

\begin{definition}[{name=[bundle map]}]
	Let $P \to X$ and $Q \to Y$ be $G$-principal bundles and $f \colon X \to Y$ a continuous map.
	A \Index{bundle map} over $f$ is a $G$-equivariant map $\hat{f} \colon P \to Q$ such that
	\[
		\begin{tikzcd}
			P \rar["\hat{f}"] \dar["\pi_P"] & Q \dar["\pi_Q"] \\
			X \rar["f"] & Y
		\end{tikzcd}
	\] 
\end{definition}

\begin{remark}[label=rmk:id_bundle]
	If $f = {\id_X}$ then $\hat{f}$ is bijective and $\hat{f}^{-1}$ is continuous.
\end{remark}

\begin{definition}[{name=[pullback bundle]}]
	Let $\pi \colon P \to X$ be a $G$-principal bundle and $f \colon Y \to X$ be continuous.
	The \Index{pullback bundle} is defined by
	\[
		f^*P = \set[\big]{(y,p) \in Y \times P \given f(y)= \pi(p)}
	\]
	with $G$-action given by $(y,p).g \coloneqq (y,p.g)$.
	$f^* P$ is a $G$-principal bundle over $Y$ with fibre $\pi^{-1}(f(y))$ over $y \in Y$.
	We get the following commutative diagramm
	\[
		\begin{tikzcd}
			f^* P \rar \dar & P \dar \\
			Y \rar["f"] & X
		\end{tikzcd}
	\]
\end{definition}

\begin{remark}
	Given two $G$-principal bundles $Q \to Y$ and $P \to X$ and a continuous map $f \colon Y \to X$ we get 
	\[
		\begin{tikzcd}[sep=small]
			Q \ar[rr,"\hat{f}"]  \ar[dd] \drar[dashed,"\exists!"] & & P \ar[dd] \\
			& f^* P \urar \dlar & \\
			Y \ar[rr,"f"] & & X
		\end{tikzcd}
	\]
	for every bundle map $\hat{f}$ over $f$.
	In particular the previous \autoref{rmk:id_bundle} implies $Q \cong f^* P$.
\end{remark}

\begin{definition}[{name={Borel construction}}]
	Let $\pi \colon P \to X$ be a $G$-principal bundle and $F$ a left $G$-space.
	Define a right $G$-action on $P \times F$ by
	\[
		(p,f) .g \coloneqq \enbrace*{p.g, g^{-1}.f}
	\]
	The orbit space is denoted by $P \times_G F \coloneqq \sfrac{P \times F}{G}$.
	The invariant map $\pi$ induces $q \colon P \times_G F \to X$ and we get
	\[
		q^{-1}(x) = \sfrac{\pi^{-1}(x) \times F}{G} \cong F
	\]
	where the homeomorphism is given by $f \mapsto \benbrace*{(p,f)}$ for a fixed $p \in \pi^{-1}(x)$.
\end{definition}

\begin{example}
	Let $V$ be a vector bundle.
	Then $V$ can be recovered from the frame bundle constructed earlier by the Borel construction; to be precise there is a isomorphism of vector bundles $V \cong \operatorname{Fr}(V) \times_{\GL(n,\mathbb{R})} \mathbb{R}^n$.
	The Isomorphism becomes clear, once one realises that the map
	\mapdef{\operatorname{Fr}(V) \times \mathbb{R}^n}{V}{(f,v)}{f(v)}{}
	maps $\enbrace*{f.g, g^{-1}. v}$ to $f(v)$ as well for every $f, \in \operatorname{Fr}(V)$, $v \in \mathbb{R}^n$ and $g \in \GL(n,\mathbb{R})$.
\end{example}









