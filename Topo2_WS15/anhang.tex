%!TEX root = topologie_2.tex
Hier findet man einige ausgewählte Beweise und Ergänzungen, auf die der Vorlesung hingewiesen wurde und die teilweise im Rahmen von Übungsaufgaben bearbeitet wurden.
Sie stammen also auch hauptsächlich aus den Übungen und deswegen sei hier noch einmal darauf hingewiesen, dass für die Richtigkeit keine Garantie übernommen wird!

Die Reihenfolge richtet sich nach dem Auftreten der Verweise in der Vorlesung.

\subsection{Singuläre Kohomologie ist eine Kohomologietheorie} % (fold)
\label{sub:sing_kohomo}
\emph{Dies sind die fehlenden Beweise von \autoref{satz:110}, die in den Übungsaufgaben 4 von Blatt 1 und 3 von Blatt 2 behandelt wurden.}
\begin{enumerate}[i),itemsep=1.5pt]
	\item \textbf{Dimensionsaxiom:} Es gilt $H^n_\sing(\set{\pt};V) = V$, falls $n=0$ ist und sonst $0$.\index{Dimensionsaxiom}
	\item \textbf{Paarfolge:} Es gibt eine natürliche Transformation $\partial^*\colon H^*(A;V) \to H^{*+1}(X,A;V)$ sodass für jedes Paar\index{Paarfolge} 
	\[
		\begin{tikzcd}
			0 \rar & H^0(X,A;V) \rar & H^0(X;V) \rar & H^0(A;V) \rar["\partial"] & H^1(X,A;V) \rar & \ldots 
		\end{tikzcd}
	\]
	eine lange exakte Folge ist.
	\item \textbf{Ausschneidung:} Sei $L \subseteq A$ mit $\overline{L} \subseteq \mathring{A}$. 
	Dann induziert die Inklusion\index{Ausschneidung} $i \colon (X \setminus L, A \setminus L) \hookrightarrow (X,A)$ einen Isomorphismus $i^* \colon H^*(X,A;V) \to H^*(X \setminus L, A \setminus L;V)$.
\end{enumerate}
Die Homotopieinvarianz wurde bereits in der Vorlesung gezeigt.
\begin{beweis}[name={\cite[S. 199 ff]{Hatcher}}]
	\leavevmode
	\begin{enumerate}[(i)]
		\item Wir wissen schon (Topologie \RM{1}, Bsp. 5.8), dass für den singulären Kettenkomplex von $\set{\pt}$ gilt
		\[
			\begin{tikzcd}
				\mathbb{Z} & \mathbb{Z} \lar["0"'] & \mathbb{Z} \lar["\id"'] & \mathbb{Z} \lar["0"'] & \ldots \lar 
			\end{tikzcd}
		\]
		Der zugehörige Kokettenkomplex mit Koeffizienten in $V$ hat nun die Form
		\[
			\begin{tikzcd}
				V \rar["0"] & V \rar["\id"] & V \rar["0"] & V \rar & \ldots 
			\end{tikzcd}
		\]
		Damit folgt direkt die Behauptung.
		\item Wir dualisieren zunächst die kurze exakte Sequenz 
		\[
			\begin{tikzcd}
				0 \rar & C_n^\sing(A) \rar["i"] & C_n^\sing (X) \rar["j"] & C_n^\sing(X,A) \rar & 0
			\end{tikzcd}
		\]
		und überlegen uns, dass die resultierende Sequenz
		\[
			\begin{tikzcd}
				0 & C^n_\sing(A) \lar & C^n_\sing(X) \lar["i^*"'] & C^n_\sing(X,A) \lar["j^*"'] & 0 \lar
			\end{tikzcd}
		\]
		auch wieder kurz exakt ist:
		$i^*$ ist surjektiv, da $i^*$ eine Kokette auf ihre Einschränkung auf singuläre $n$-Simplizes in $A$ abbildet und man eine Kokette in $C^n(A)$ durch $0$ offensichtlich auf singuläre $n$-Simplizes auf ganz $X$ fortsetzten lässt.
		Der Kern von $i^*$ besteht aus Koketten in $C^n(X)$, die auf Simplizes in $A$ den Wert $0$ annehmen.
		Da diese aber das gleiche sind, wie Homomorphismen $C_n(X,A) = \sfrac{C_n(X)}{C_n(A)}\to V$, folgt die Exaktheit in der Mitte der Sequenz. Beachte dabei, dass man $C^n(X,A;V)$ als die Menge der Funktionen von singulären $n$-Simplizes in $X$ nach $V$ auffassen kann, die auf singulären Simplizes mit Bild in $A$ verschwinden, denn die Basis von $C_n(X)$ ist die disjunkte Vereinigung der singulären Simplizes, deren Bild in $A$ enthalten ist, mit denen, deren Bild nicht in $A$ enthalten ist.
		Damit ist die Injektivität von $j^*$ auch klar. 
		Mit dem Schlangenlemma folgt die Behauptung.
		\item Wir beweisen die Aussage hier noch einmal mit dem universellen Koeffizienten-Theorem, in der Vorlesung wurde die Aussage in \autoref{korr:ausschneidung_sing_koho} bewiesen.
		
		Wir wenden die Natürlichkeit des universellen Koeffizienten-Theorems (siehe \autoref{univ_koeff_space}) an:
		\[
			\hspace{-1.2em}\begin{tikzcd}[column sep=1.3em]
				0 \rar & \Ext \enbrace[\big]{H_{n-1}(X\setminus L,A \setminus L),V} \rar & H^n(X\setminus L,A \setminus L;V) \rar & \Hom \enbrace[\big]{H_n(X \setminus L,A\setminus L),V} \rar & 0 \\
				0 \rar & \Ext \enbrace[\big]{H_{n-1}(X,A),V} \rar \uar["(i_*)^*","\cong"'] & H^n(X,A;V) \uar["i^*"] \rar & \Hom \enbrace[\big]{H_n(X,A),V} \rar \uar["(i_*)^*","\cong"'] & 0 
			\end{tikzcd}
		\]
		Dabei erhalten wir die beiden Isomorphismen aus dem Ausschneidungssatz für Homologie.
		Mit dem Fünfer-Lemma folgt, dass auch die mittlere Abbildung ein Isomorphismus ist.\qedhere
	\end{enumerate}
\end{beweis}
% subsection sing_kohomo (end)

\subsection{Ringe, sodass Untermoduln freier Moduln frei sind} % (fold)
\label{sub:untermod_frei}
\emph{Dies war Übungsaufgabe 4 von Blatt 2.}\smallskip\\
Es sei $R$ ein Ring mit der Eigenschaft, dass Untermoduln freier $R$-Moduln wieder frei sind.
Dann ist $R$ ein Hauptidealring.

Es gilt auch die Umkehrung.\footnote{siehe z.B. \url{http://math.stackexchange.com/a/162958/196134}}
\begin{beweis}
	Sei $r \in R \setminus \set*{0}$.
	Dann ist $r R$ als Untermodul von $R$ frei und damit auch torsionsfrei.
	Da $r$ beliebig war, ist $R$ also ein Integritätsbereich.
	
	Sei nun $I \subsetneq R$ ein Ideal.
	Wir können $I$ wieder als Untermodul des $R$-Moduls $R$ auffassen und wissen damit, dass $I$ eine Basis hat.
	Eine solche kann aber nur ein Element haben, da zwei Ringelemente in einem kommutativen Ring nicht linear unabhängig sein können (für $a \neq b \in R$ gilt $ab -ba=0$).
	Damit ist $I$ ein Hauptideal und $R$ ein Hauptidealring.
\end{beweis}
% subsection untermod_frei (end)


\subsection{Relative Cup-Produkte, Blatt 4 Aufgabe 3} % (fold)
\label{sub:relative_cup_produkte}
\emph{Dies komplettiert \autoref{bem:rel_cup}.}\smallskip\\
Seien $X$ ein topologischer Raum und $U$, $V$ offen in $X$.
\begin{enumerate}[(a)]
	\item Sei $C^*(X, U+V) \subset C^*(X)$ der Unterkomplex der Koketten, die auf Ketten in $U$ und auf Ketten in $V$ verschwinden. 
	Dann ist die Inklusion $C^*(X,U \cup V) \hookrightarrow C^*(X,U + V)$ eine Kettenhomotopieäquivalenz.
	\item Konstruiere ein $\cupp$-Produkt $\cupp \colon H^*(X,U) \otimes H^*(X,V) \to H^*(X, U\cup V)$ derart, dass folgendes Diagramm kommutiert
	\[
		\begin{tikzcd}
			H^*(X,U) \otimes H^*(X,V) \rar["\cupp"]\dar & H^*(X, U \cup V) \dar \\
			H^*(X) \otimes H^*(X) \rar["\cupp"] & H^*(X)
		\end{tikzcd}
	\]
\end{enumerate}
\begin{beweis}
	\leavevmode
	\begin{enumerate}[(a)]
		\item $\mathcal{U}=\set*{U,V}$ ist eine offene Überdeckung von $U \cup V$ und somit ist $C_*(U+V)=C_*^{\mathcal{U}}(U\cup V)$.
		Aus Topologie \RM{1} ist bekannt, dass $C_*(U + V) \hookrightarrow C_*(U \cup V)$ einen Isomorphismus in Homologie induziert, also eine Kettenhomotopieäquivalenz ist nach \autoref{satz:5:kettenhomotopie}.
		Folgendes Diagramm kommutiert und hat kurze exakte Zeilen:
		\[
			\begin{tikzcd}
				0 \rar & C_*(U+V)\dar[hook] \rar & C_*(X) \dar[equal] \rar & \sfrac{C_*(X)}{C_*(U +V)} \dar \rar & 0 \\
				0 \rar & C_*(U \cup V) \rar & C_*(X) \rar & C_*(X, U \cup V) \rar & 0
			\end{tikzcd}
		\]
		Nach dem Fünfer-Lemma induziert die rechte Abbildung auch einen Isomorphismus in Homologie und ist somit nach \autoref{satz:5:kettenhomotopie} auch eine Kettenhomotopieäquivalenz.
		Damit ist nach Anwenden des $\Hom$-Funktors auch $C^*(X, U \cup V) \to \Hom \enbrace*{\sfrac{C_*(X)}{C_*(U +V)}, \mathbb{Z}} = C^*(X, U + V)$ eine Kokettenhomotopieäquivalenz.
		\item Betrachte folgendes kommutatives Diagramm:
		\[
			\begin{tikzcd}
				C^p(X,U) \otimes C^q(X,V) \dar \rar["\cupp"] & C^{p+q}(X,U+V) \dar \\
				C^p(X) \otimes C^q(X) \rar["\cupp"] & C^{p+q}(X) 
			\end{tikzcd}
		\]
		Da auch für das obige Cup-Produkt die Formel $d^{p+q}(\alpha \cupp \beta) = d^p(\alpha) \cupp \beta + (-1)^p \cdot \alpha \cupp d^q(\beta)$ gilt, induziert die obere Zeile ein Cup-Produkt
		\[
			H^*(X,U) \otimes H^*(X,V) \longrightarrow H^*(X, U+ V) \StackText{a)}{\cong} H^*(X, U \cup V)
		\]
		und wir erhalten das gewünschte Cup-Produkt in Kohomologie. \qedhere
	\end{enumerate}
\end{beweis}
In dem Beweis haben wir die Offenheit von $U$ und $V$ nur gebraucht, damit $C_*(U + V) \hookrightarrow C_*(U \cup V)$ einen Isomorphismus in Homologie induziert.
Dies kann aber auch in anderen Fällen eintreten, zum Beispiel wenn $U$ und $V$ Unterkomplexe des CW-Komplexes $X$ sind.
% subsection relative_cup_produkte (end)

\subsection{Paarung ist Isomorphismus, wenn $H_{n-1}(X)$ frei ist} % (fold)
\label{sub:paarung_isomorphismus_frei}
\emph{Dies war Aufgabe 3 von Blatt 3 und spielt in \autoref{bem:paarung_iso_koerper} eine Rolle.}\smallskip\\
Sei $X$ ein topologischer Raum und $V$ eine abelsche Gruppe.
Angenommen $H_{n-1}(X)$ ist frei für ein $n \in \mathbb{N}$.
Dann ist der kanonische Epimorphismus aus \autoref{eig_kohomo_to_hom_homo}
\[
	f \colon H^n(X;V) \longrightarrow \Hom_\mathbb{Z} \enbrace[\big]{H_n(X),V}
\]
ein Isomorphismus.
\begin{beweis}
	Es sei $[\alpha] \in \ker f$, das heißt $\alpha(a)=0$ für alle $a \in \ker d_n$.
	Da $\alpha$ auf dem Kern von $d_n$ trivial ist, faktorisiert $\alpha$ über $\im d_n \subseteq C_{n-1}(X)$ mittels $\beta_0 \colon \im d_n \to V$, also $\alpha = \beta_0 \circ d_n$.
	Wir wollen $\beta_0$ nun auf ganz $C_{n-1}(X)$ fortsetzen.
	Da $H_{n-1}(X)$ nach Voraussetzung frei ist, spaltet 
	\[
		\begin{tikzcd}
			\im d_n \rar[hook] & \ker d_{n-1} \rar[two heads] & H_{n-1}(X)
		\end{tikzcd}
	\]
	und es folgt $\ker d_{n-1} \cong \im d_n \oplus H_{n-1}(X)$.
	Da $C_{n-1}(X) \cong \im d_{n-1} \oplus \ker d_{n-1}$ ist, folgt
	\[
		C_{n-1}(X) \cong \im d_{n-1} \oplus \im d_n \oplus H_{n-1}(X)
	\]
	Damit können wir $\beta_0$ trivial zu $\beta \colon C_{n-1}(X) \to V$ fortsetzen und es folgt $\alpha= \beta \circ d_n = \pm d^{n-1} \beta$ und somit $[\alpha]=0$ in Kohomologie.
\end{beweis}
% subsection paarung_ist_isomorphismus_wenn_h__n_1_x_frei_ist (end)

\subsection{Fundamentalklassen unter Randabbildung der Paarfolge} % (fold)
\label{sub:fundamentalklasse_rand}
\emph{Dies war Aufgabe 4 von Blatt 9 und wird beim Beweis der Poincaré-Dualität für Mannigfaltigkeiten mit Rand benötigt, siehe \autoref{satz:poincare_rand}.}\smallskip\\
Sei $W$ eine kompakte orientierte $(n+1)$-Mannigfaltigkeit mit Rand $\partial W$ und Orientierungsklasse $\mu_W \in H_{n+1}(W,\partial W)$.
Sei $\partial \colon H_{n+1}(W,\partial W) \to H_n(\partial W)$ die Randabbildung aus der Paarsequenz.
Dann ist $\partial \mu_W$ eine Orientierungsklasse von $\partial W$.
\begin{beweis}[{name={\cite[\RM{14}. Th.~7.6]{Massey}}}]
	Wir müssen zeigen, dass $\partial(\mu_W)|_x$ für jedes $x \in \partial W$ ein Erzeuger von $H_n(\partial W, \partial W \setminus \set*{x})$ ist.
	Per Definition existiert ein Kartengebiet $x \in U \subseteq W$ und eine Karte, von der wir annehmen dürfen, dass $x$ auf $(0,\ldots ,0) \in \mathbb{R}^{n+1}_+$ abgebildet wird.
	Wir identifizieren die Punkte in $U$ mit ihren Bildern.
	Dann sind die Punkte in $\partial W \cap U$ durch die Gleichung $x_{n+1}=0$ bestimmt.
	Sei $y \in U$ der Punkt mit den Koordinaten $(0,\ldots ,0,1)$.
	Wir definieren Teilmengen von $U$
	\[
		A := \set*{(x_1, \ldots ,x_{n+1}) \in U \given \sum\nolimits_i x_i^2 \le 4 } \qquad B := \set*{(x_1,\ldots ,x_{n+1}) \in A \given \sum\nolimits_i x_i^2 < 4 \text{ und } x_{n+1}>0 }
	\]
	und $C := A \cap \partial W$.
	Betrachte nun das folgende kommutative Diagramm\marginnote{dabei $\iota \colon \partial W \hookrightarrow (\partial W,\partial W \setminus \set*{x})$ die Inklusion}
	\[
		\begin{tikzcd}[row sep=4em, column sep=5.2em]
			H_{n+1} \enbrace[\big]{W,\partial W}\rar["\partial' = \iota_*\circ \partial"] \dar["\rho"] \ar[dr] & H_{n} \enbrace[\big]{\partial W, \partial W \setminus \set*{x}}\drar & H_n \enbrace[\big]{C,C\setminus \set*{x}} \dar["5"] \lar["6"']\\
			H_{n+1} \enbrace[\big]{W,W\setminus \set*{y}} & H_{n+1} \enbrace[\big]{W,W \setminus B}\lar["1"']  \rar["\partial_1"] & H_n \enbrace[\big]{W \setminus B,(W \setminus B) \setminus \set*{x} } \\
			H_{n+1} \enbrace[\big]{A,A\setminus \set*{y}}\uar["2"] & H_{n+1} \enbrace[\big]{A,A \setminus B} \uar \lar["3"'] \rar["\partial_2"] & H_n \enbrace[\big]{A \setminus B, (A \setminus B) \setminus \set*{x}} \uar["4"]
		\end{tikzcd}
	\]
	Die Homomorphismen $\partial',\partial_1, \partial_2$ sind dabei Randabbildungen aus den entsprechenden Tripeln, alle anderen sind von Inklusionen induziert.
	Die Kommutativität folgt dann auf der Ebene von Raumpaaren beziehungsweise aus der Natürlichkeit der Randabbildung.
	Mit den üblichen Methoden rechnet man nach, dass die Homomorphismen $1-6$, sowie $\partial_2$ Isomorphismen sind.
	
	Wir wissen nun, dass $\rho(\mu_W)$ ein Erzeuger von $H_{n+1} (W,W\setminus \set*{y})$ ist, also muss auch $\partial'(\mu_W)$ ein Erzeuger sein.
\end{beweis}

% subsection fundamentalklasse_rand (end)


\subsection{Dimensionsargument in \autoref{satz:sig_rand_4k+1_mfkt}} % (fold)
\label{sub:dimensionsargument_rand_mfkt}
Es seien $f \colon V \to W$ eine lineare Abbildung, wobei $V$ und $W$ endlichdimensional sind. 
Mit $V^*$ und $W^*$ bezeichnen wir die Dualräume.
Dann gilt
\[
	\dim \ker f = \dim \ker f^*
\]
\begin{beweis}
	Für eine Teilmenge $U \subseteq V$ bezeichnet $U^0 := \set*{\varphi \in V^* \given \varphi(x)=0 \,\forall x \in U}$ den Annullator von $U$.
	In der linearen Algebra zeigt man\footnote{z.B. auf StackExchange \url{http://math.stackexchange.com/questions/611575/proof-dimension-of-annihilator} und \url{http://math.stackexchange.com/questions/1344904/show-that-ker-hatt-textann-textrange-t}}
	\begin{enumerate}[(i)]
		\item $\dim U^0 = \dim V - \dim U$ und
		\item $\ker f^* = (\im f)^0$
	\end{enumerate}
	Mit $U= \im f$ folgt also
	\(
		\dim \ker f = \dim V - \dim \im f = \dim (\im f)^0 = \dim \ker f^* 
	\) wie gewünscht.
\end{beweis}
% subsection dimensionsargument_in_autoref_satz_sig_rand_4k_1_mfkt (end)

\subsection{Linksexakte Funktoren und der Kern} % (fold)
\label{sub:exakte_funktoren_und_der_kern}
\emph{Dies war Übungsaufgabe 2 von Blatt 8.}\smallskip \\
Sei $R$ ein Ring, $\mathcal{A}$ die Kategorie der $R$-Moduln und $F \colon \mathcal{A} \to \mathcal{A}$ ein Funktor, sodass die induzierte Abbildung $\Hom(M,N) \to \Hom \enbrace*{F(M), F(N)}$ für alle Moduln $M,N$ ein Homomorphismen von abelschen Gruppen ist. 
Ein solcher Funktor heißt \bet{additiv}\index{Funktor!additiver}.

Die folgenden Aussagen äquivalent:
\begin{enumerate}[(i)]
	\item $F$ schickt exakte Folgen $ \begin{tikzcd}[cramped,sep=small]
		0 \rar & M' \rar["f"] & M \rar["g"] & M''
	\end{tikzcd}$ auf kurze exakte Folgen
	\[
		\begin{tikzcd}
			0 \rar & F(M') \rar & F(M) \rar & F(M'')
		\end{tikzcd}
	\]
	\item Wie (i) nur mit 
	\(
		\begin{tikzcd}[cramped,sep=small]
			0 \rar & M' \rar & M \rar & M'' \rar & 0
		\end{tikzcd}
	\)
	\item Für jeden Homomorphismus $f \colon M \to M''$ von $R$-Moduln ist die kanonische von $\iota \colon \ker f \hookrightarrow M$ induzierte Abbildung $F(\ker f) \to \ker F(f)$ ein Isomorphismus.
\end{enumerate}
\begin{beweis}
	Da $F$ additiv ist, gilt $F(0)=0$ sowohl für Homomorphismen als auch für Moduln. 
	Die Implikation (i)$\Rightarrow$(ii) ist klar.
	Für (ii)$\Rightarrow$(iii) betrachten wir die exakte Sequenz\marginnote{$F(\iota) \colon F(\ker f) \to \ker F(f)$}
	\[
		\begin{tikzcd}
			0 \rar & \ker f \rar[hook,"\iota"] & M \rar["f"] & M''
		\end{tikzcd}
	\]
	Sie bleibt exakt nach Anwenden von $F$ und somit folgt $\ker F(f) = \im F(\iota) \cong F(\ker f)$, wobei der Isomorphismus wegen der Injektivität von $F(\iota)$ gilt.
	
	Sei nun eine exakte Folge wie in (i) gegeben und es gelte (iii).
	Da $F(0)=0$ folgt direkt $\im f \subset \ker g$ aus $g \circ f =0$.
	Nach (iii) gilt $\ker F(f) \cong F(\ker f) = 0$, womit wir die Injektivität von $F(f)$ erhalten.
	Der Isomorphismus $M' \cong \im f$ bleibt durch Anwenden von $F$ bestehen und es gilt
	\[
		F(M') \cong F(\im f) = F(\ker g) \cong \ker F(g)
	\]
	wobei der letzte Isomorphismus durch $F(\iota)$ für $\iota \ker g \to M$ gegeben ist.
	Die gesamte Komposition ist dann durch $F(\iota \circ f) = F(f)$ gegeben und es folgt $\ker F(g) \subseteq \im F(f)$.
\end{beweis}
% subsection exakte_funktoren_und_der_kern (end)

\subsection{Einschränkung und Erweiterung der Skalare} % (fold)
\label{sub:einschrankung_und_erweiterung_der_skalare}
\emph{Diese Ausführungen sind dem englischen Wikipedia-Artikel \enquote{Change of rings}\footnote{\url{https://en.wikipedia.org/wiki/Change_of_rings}} entnommen.}\smallskip\\
Seien $S$ und $R$ Ringe und $f \colon S \to R$ ein Ringhomomorphismus.\marginnote{Moduln meint wieder Linksmoduln}
Wir wollen untersuchen, wie man $S$-Moduln in $R$-Moduln überführt und umgekehrt.
\begin{description}
	\item[Einschränkung der Skalare]
	Sei $M$ ein $R$-Modul.\index{Einschränkung der Skalare}
	Dann können wir $M$ als $S$-Modul auffassen durch die $S$-Wirkung $s.m = f(s) \cdot m$ für $m \in M$.
	Dies definiert offenbar auch einen Funktor, denn ein $R$-Homomorphismus wird mit dieser Definition automatisch zu einem $S$-Morphismus.
	Für $S=\mathbb{Z}$ erhält man den Vergissfunktor $\Mod{R} \to \ABELGRUPPEN$.
	\item[Erweiterung der Skalare] 
	Wir wollen nun einen $S$-Modul zu einem $R$-Modul machen.\index{Erweiterung der Skalare}
	Sei $N$ ein $S$-Modul.
	Betrachte das Tensorprodukt
	\[
		_R N := R \otimes_S N
	\]
	Dabei fallen wir $R$ mittels $f$ als $S$-Rechtsmodul auf (damit wird $R$ zu einem $(R,S)$-Bimodul).
	$_R N$ ist ein $R$-Modul mittels $r \cdot (R' \otimes n) = rr' \otimes n$.
	Offensichtlich erhalten wir auch hier wieder einen Funktor, indem wir für $\varphi \colon N \to N'$ 
	\[
		_R \varphi \colon {_R N} \longrightarrow {_R N'}
	\]
	durch $_R \varphi = \id_R \otimes \varphi$ definieren.
	\item[Adjunktion der beiden Funktoren] 
	Wir konstruieren nun einen Homomorphismus von abelschen Gruppen $F \colon \Hom_S(N,M) \to \Hom_R({_R N},M)$. 
	Sei also $\varphi \in \Hom_S(N,M)$, wobei wir $M$ mittels Einschränkung der Skalare als $S$-Modul auffassen.
	Wir setzen
	\[
		F(\varphi) \colon {_R N} \longrightarrow M \qquad r \otimes m \longmapsto r \cdot \varphi(m)
	\]
	Dies definiert offensichtlich einen $R$-Homomorphismus.
	
	Wir nehmen nun zusätzlich an, dass $R$ und $S$ Einheiten besitzen und $f$ diese erhält.
	Unter diesen Vorraussetzungen können wir nun einen zu $F$ inversen Homomorphismus 
	\[
		G \colon \Hom_R({_R N},M) \longrightarrow \Hom_S(N,M)
	\]
	angeben.
	$\psi \in \Hom_R({_R N},M)$ bilden wir dazu auf die Komposition
	\[
		\begin{tikzcd}[sep=2.8em]
			N \rar & S \otimes_S N \rar["f \otimes {\id_N}"] & R \otimes_S N = {_R N} \rar["\psi"] & M
		\end{tikzcd}
	\]
	ab, wobei die erste Abbildung der kanonische Isomorphismus $n \mapsto 1 \otimes n$ ist.
	Man macht sich schnell klar, dass $F$ und $G$ tatsächlich inverse zueinander sind.
	Desweiteren kann man sich auch noch überlegen, dass dieser Isomorphismus natürlich ist, da er nur von $f$ abhängt.
	Damit ist die Erweiterung der Skalare linksadjungiert zur Einschränkung der Skalare.
\end{description}
Statt \enquote{zu Fuß} nachzurechnen, dass dieser Isomorphismus existiert,\marginnote{ich habe das nicht durchgerechnet, bin mir aber ziemlich sicher, dass das stimmt} hätte man sich auch überlegen können, dass dies nur ein Spezialfall der \href{https://en.wikipedia.org/wiki/Tensor-hom_adjunction}{Tensor-Hom-Adjunktion}\footnote{\url{https://en.wikipedia.org/wiki/Tensor-hom_adjunction}} ist, siehe dazu auch Beweis von \autoref{univ_koeff_space}.
% subsection einschrankung_und_erweiterung_der_skalare (end)









