%!TEX root = topologie_2.tex
Hier findet man einige ausgewählte Beweise und Ergänzungen, auf die der Vorlesung hingewiesen wurde und die teilweise im Rahmen von Übungsaufgaben bearbeitet wurden.
Sie stammen also auch hauptsächlich aus den Übungen und deswegen sei hier noch einmal darauf hingewiesen, dass für die Richtigkeit keine Garantie übernommen wird!

Die Reihenfolge richtet sich nach dem Auftreten der Verweise in der Vorlesung.

\subsection{Singuläre Kohomologie ist eine Kohomologietheorie} % (fold)
\label{sub:sing_kohomo}
\emph{Dies sind die fehlenden Beweise von \autoref{satz:110}, die in den Übungsaufgaben 4 von Blatt 1 und 3 von Blatt 2 behandelt wurden.}
\begin{enumerate}[i),itemsep=1.5pt]
	\item \textbf{Dimensionsaxiom:} Es gilt $H^n_\sing(\set{\pt};V) = V$, falls $n=0$ ist und sonst $0$.\index{Dimensionsaxiom}
	\item \textbf{Paarfolge:} Es gibt eine natürliche Transformation $\partial^*\colon H^*(A;V) \to H^{*+1}(X,A;V)$ sodass für jedes Paar\index{Paarfolge} 
	\[
		\begin{tikzcd}
			0 \rar & H^0(X,A;V) \rar & H^0(X;V) \rar & H^0(A;V) \rar["\partial"] & H^1(X,A;V) \rar & \ldots 
		\end{tikzcd}
	\]
	eine lange exakte Folge ist.
	\item \textbf{Ausschneidung:} Sei $L \subseteq A$ mit $\overline{L} \subseteq \mathring{A}$. 
	Dann induziert die Inklusion\index{Ausschneidung} $i \colon (X \setminus L, A \setminus L) \hookrightarrow (X,A)$ einen Isomorphismus $i^* \colon H^*(X,A;V) \to H^*(X \setminus L, A \setminus L;V)$.
\end{enumerate}
Die Homotopieinvarianz wurde bereits in der Vorlesung gezeigt.
\begin{beweis}[name={\cite[S. 199 ff]{Hatcher}}]
	\leavevmode
	\begin{enumerate}[(i)]
		\item Wir wissen schon (Topologie \RM{1}, Bsp. 5.8), dass für den singulären Kettenkomplex von $\set{\pt}$ gilt
		\[
			\begin{tikzcd}
				\mathbb{Z} & \mathbb{Z} \lar["0"'] & \mathbb{Z} \lar["\id"'] & \mathbb{Z} \lar["0"'] & \ldots \lar 
			\end{tikzcd}
		\]
		Der zugehörige Kokettenkomplex mit Koeffizienten in $V$ hat nun die Form
		\[
			\begin{tikzcd}
				V \rar["0"] & V \rar["\id"] & V \rar["0"] & V \rar & \ldots 
			\end{tikzcd}
		\]
		Damit folgt direkt die Behauptung.
		\item Wir dualisieren zunächst die kurze exakte Sequenz 
		\[
			\begin{tikzcd}
				0 \rar & C_n^\sing(A) \rar["i"] & C_n^\sing (X) \rar["j"] & C_n^\sing(X,A) \rar & 0
			\end{tikzcd}
		\]
		und überlegen uns, dass die resultierende Sequenz
		\[
			\begin{tikzcd}
				0 & C^n_\sing(A) \lar & C^n_\sing(X) \lar["i^*"'] & C^n_\sing(X,A) \lar["j^*"'] & 0 \lar
			\end{tikzcd}
		\]
		auch wieder kurz exakt ist:
		$i^*$ ist surjektiv, da $i^*$ eine Kokette auf ihre Einschränkung auf singuläre $n$-Simplizes in $A$ abbildet und man eine Kokette in $C^n(A)$ durch $0$ offensichtlich auf singuläre $n$-Simplizes auf ganz $X$ fortsetzten lässt.
		Der Kern von $i^*$ besteht aus Koketten in $C^n(X)$, die auf Simplizes in $A$ den Wert $0$ annehmen.
		Da diese aber das gleiche sind, wie Homomorphismen $C_n(X,A) = \sfrac{C_n(X)}{C_n(A)}\to V$, folgt die Exaktheit in der Mitte der Sequenz. Beachte dabei, dass man $C^n(X,A;V)$ als die Menge der Funktionen von singulären $n$-Simplizes in $X$ nach $V$ auffassen kann, die auf singulären Simplizes mit Bild in $A$ verschwinden, denn die Basis von $C_n(X)$ ist die disjunkte Vereinigung der singulären Simplizes, deren Bild in $A$ enthalten ist, mit denen, deren Bild nicht in $A$ enthalten ist.
		Damit ist die Injektivität von $j^*$ auch klar. 
		Mit dem Schlangenlemma folgt die Behauptung.
		\item Wir beweisen die Aussage hier noch einmal mit dem universellen Koeffizienten-Theorem, in der Vorlesung wurde die Aussage in \autoref{korr:ausschneidung_sing_koho} bewiesen.
		
		Wir wenden die Natürlichkeit des universellen Koeffizienten-Theorems (siehe \autoref{univ_koeff_space}) an:
		\[
			\hspace{-1.2em}\begin{tikzcd}[column sep=1.3em]
				0 \rar & \Ext \enbrace[\big]{H_{n-1}(X\setminus L,A \setminus L),V} \rar & H^n(X\setminus L,A \setminus L;V) \rar & \Hom \enbrace[\big]{H_n(X \setminus L,A\setminus L),V} \rar & 0 \\
				0 \rar & \Ext \enbrace[\big]{H_{n-1}(X,A),V} \rar \uar["(i_*)^*","\cong"'] & H^n(X,A;V) \uar["i^*"] \rar & \Hom \enbrace[\big]{H_n(X,A),V} \rar \uar["(i_*)^*","\cong"'] & 0 
			\end{tikzcd}
		\]
		Dabei erhalten wir die beiden Isomorphismen aus dem Ausschneidungssatz für Homologie.
		Mit dem Fünfer-Lemma folgt, dass auch die mittlere Abbildung ein Isomorphismus ist.\qedhere
	\end{enumerate}
\end{beweis}
% subsection sing_kohomo (end)

\subsection{Relative Cup-Produkte, Blatt 4 Aufgabe 3} % (fold)
\label{sub:relative_cup_produkte}
\emph{Dies komplettiert \autoref{bem:rel_cup}.}\smallskip\\
Seien $X$ ein topologischer Raum und $U$, $V$ offen in $X$.
\begin{enumerate}[(a)]
	\item Sei $C^*(X, U+V) \subset C^*(X)$ der Unterkomplex der Koketten, die auf Ketten in $U$ und auf Ketten in $V$ verschwinden. 
	Dann ist die Inklusion $C^*(X,U \cup V) \hookrightarrow C^*(X,U + V)$ eine Kettenhomotopieäquivalenz.
	\item Konstruiere ein $\cupp$-Produkt $\cupp \colon H^*(X,U) \otimes H^*(X,V) \to H^*(X, U\cup V)$ derart, dass folgendes Diagramm kommutiert
	\[
		\begin{tikzcd}
			H^*(X,U) \otimes H^*(X,V) \rar["\cupp"]\dar & H^*(X, U \cup V) \dar \\
			H^*(X) \otimes H^*(X) \rar["\cupp"] & H^*(X)
		\end{tikzcd}
	\]
\end{enumerate}
\begin{beweis}
	\leavevmode
	\begin{enumerate}[(a)]
		\item $\mathcal{U}=\set*{U,V}$ ist eine offene Überdeckung von $U \cup V$ und somit ist $C_*(U+V)=C_*^{\mathcal{U}}(U\cup V)$.
		Aus Topologie \RM{1} ist bekannt, dass $C_*(U + V) \hookrightarrow C_*(U \cup V)$ einen Isomorphismus in Homologie induziert, also eine Kettenhomotopieäquivalenz ist nach \autoref{satz:5:kettenhomotopie}.
		Folgendes Diagramm kommutiert und hat kurze exakte Zeilen:
		\[
			\begin{tikzcd}
				0 \rar & C_*(U+V)\dar[hook] \rar & C_*(X) \dar[equal] \rar & \sfrac{C_*(X)}{C_*(U +V)} \dar \rar & 0 \\
				0 \rar & C_*(U \cup V) \rar & C_*(X) \rar & C_*(X, U \cup V) \rar & 0
			\end{tikzcd}
		\]
		Nach dem Fünfer-Lemma induziert die rechte Abbildung auch einen Isomorphismus in Homologie und ist somit nach \autoref{satz:5:kettenhomotopie} auch eine Kettenhomotopieäquivalenz.
		Damit ist nach Anwenden des $\Hom$-Funktors auch $C^*(X, U \cup V) \to \Hom \enbrace*{\sfrac{C_*(X)}{C_*(U +V)}, \mathbb{Z}} = C^*(X, U + V)$ eine Kokettenhomotopieäquivalenz.
		\item Betrachte folgendes kommutatives Diagramm:
		\[
			\begin{tikzcd}
				C^p(X,U) \otimes C^q(X,V) \dar \rar["\cupp"] & C^{p+q}(X,U+V) \dar \\
				C^p(X) \otimes C^q(X) \rar["\cupp"] & C^{p+q}(X) 
			\end{tikzcd}
		\]
		Da auch für das obige Cup-Produkt die Formel $d^{p+q}(\alpha \cupp \beta) = d^p(\alpha) \cupp \beta + (-1)^p \cdot \alpha \cupp d^q(\beta)$ gilt, induziert die obere Zeile ein Cup-Produkt
		\[
			H^*(X,U) \otimes H^*(X,V) \longrightarrow H^*(X, U+ V) \StackText{a)}{\cong} H^*(X, U \cup V)
		\]
		und wir erhalten das gewünschte Cup-Produkt in Kohomologie. \qedhere
	\end{enumerate}
\end{beweis}
In dem Beweis haben wir die Offenheit von $U$ und $V$ nur gebraucht, damit $C_*(U + V) \hookrightarrow C_*(U \cup V)$ einen Isomorphismus in Homologie induziert.
Dies kann aber auch in anderen Fällen eintreten, zum Beispiel wenn $U$ und $V$ Unterkomplexe des CW-Komplexes $X$ sind.
% subsection relative_cup_produkte (end)
