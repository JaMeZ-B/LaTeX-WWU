%!TEX root = ./WTG.tex
\section{Erstes Kapitel}

\begin{definition}[Netzwerk] 
	\marginnote{Mit anderen Worten: Ein Netzwerk ist ein positiv gewichteter Graph.} 
	Ein Netzwerk $G = (V,E,c)$ ist ein Graph $\enb{V,E}$ mit entweder gerichteten oder ungerichteten Kanten, d.h.
	\begin{align}
		E \subseteq \set{\enb{x,y} \given x,y \in V} && \text{beziehungsweise} && E \subseteq \set{\set{x,y} \given x,y \in V},
	\end{align}
	zusammen mit einer Gewichtsfunktion $c:E \to \Real^+$.
\end{definition}

Ist nun $\enb{X_n}_n$ eine Irrfahrt auf $\enb{V,E}$ mit Übergangswahrscheinlichkeiten $P = \enb{p_{xy}}$, dann wollen wir daraus ein Netzwerk konstruieren. Dazu nehmen wir an, dass die Irrfahrt \emph{reversibel} bezüglich eines Maßes $\pi$ auf $V$ ist.
\begin{gather}
	\pi\enb{x}p_{xy} = \pi\enb{y} p_{yx}
\end{gather}
Dabei muss $\pi$ kein Wahrscheinlichkeitsmaß sein.

\begin{bemerkung}
	Das Maß $\pi$ ist stationär für die Markovkette.
\end{bemerkung}
\begin{beweis}
	Zu zeigen: $\pi \cdot \p = \pi$. 
	\begin{gather}
		\marginnote{$y\sim x:=$ alle $y$ die zu $x$ benachbart sind.}
		\pi\enb{x} = \pi\enb{x} \sum\limits_{y \sim x} p_{xy} = \sum\limits_{y \sim x} \pi(x)p_{xy} = \sum\limits_{y\sim x} \pi(y)p_{yx}.
	\end{gather}
\end{beweis}
Wählt man nun $c(x,y):=\pi(x)p_{xy}$ so gilt $c(x,y)=\pi(x)p_{xy} = \pi(y)p_{yx} =  c(y,x)$.
Das Kantengewicht ist also unabhängig von der Orientierung der Kante. Zusammenfassend haben wir nun festgestellt, dass eine reversible Irrfahrt bezüglich $\pi$ auf $(V,E)$ ein Netzwerk mit Kantengewicht $c(x,y) = \pi(x)\p_{xy} = c(y,x)$ induziert.

Umgekehrt existiert es zu einem gegebenen Netzwerk $(V,E,c)$ eine Irrfahrt auf $(V,E)$, mit $p_{xy} \approx c(x,y)$. Genauer: 
\begin{gather}	
p_{xy} = \frac{1}{\sum\limits_{z \sim x}} \cdot c(x,y) = \frac{c(x,y)}{\pi(x)}.
\end{gather}
Gilt $c(x,y) = c(y,x)$ für alle $x,y$, so ist die Irrfahrt bezüglich $\pi$ reversibel, denn 
\begin{gather}
	\pi(x)p_{xy} = c(x,y) = c(y,x) = \pi(y)p_{yx} 
\end{gather}

\begin{beispiel}[Gambler's Run] \todo{das geht schöner}
	Sei $X_n$ eine Irrfahrt auf $0, \dots, n$ und	
	\begin{gather}
		\prop{X_n = i+1 \given X_{n-1} = i} = p = 1 - \prop{X_n = i-1 \given X_{n-1} = i}.
	\end{gather}
	Wie kommt man nun auf die Kantengewichte?
	\begin{gather}
		\prop{X_{n+1} = i+1 \given X_n = i} = \frac{c(i,i+1)}{c(i,i+1)+c(i,i-1)} = \frac{\enb{\frac{p}{q}}^i}{\enb{\frac{p}{q}}^i + \enb{\frac{p}{q}}^{i-1}} = \frac{\frac{p}{q}}{\frac{p}{q} +1} = p
	\end{gather}
	Sei nun $A,Z \subseteq V, A \cap Z = \setzero$. Was ist nun $\prop{\tau_A < \tau_Z}[][x]?$ (Wobei $\tau_1$ die erste Treffzeit der Menge $A$ ist und $\p_x$ die Wahrscheinlichkeit, bei Start in $x$). Sei $F(x) = \prop{\tau_A < \tau_Z}[][x]$, dann ist $F|_A = 1$ und $F|_Z = 0$. Ist $x \notin A \cup Z$, dann ist $\set{x,y} \in E$ 
	\begin{equation}
		F(x) = \sum\limits_{y \sim x} p_{xy} F(y) = \frac{1}{\pi(x)} \sum\limits_{y \sim x} c(x,y)F(y) \tag{*} \label{eqn:GamblersRun}
	\end{equation}
	Im einfachsten Fall (Simple Random Walk) ist $c(x,y) = 1$ und $\pi(x) = \deg(x)$ \marginnote{$\deg(x)$ ist der Grad von $x$} und wir haben
	\begin{equation}
		F(x) = \frac{1}{\deg(x)} \sum\limits_{y \sim x} F(y),
	\end{equation}
	also ist $F(x)$ der Mittelwert all seiner Nachbarn. 
\end{beispiel}

\begin{definition}
	Eine Funktion $F$, welche \eqref{eqn:GamblersRun} erfüllt, heißt harmonisch in $x$ bezüglich der Gewichte $c$. Ist $F$ harmonisch für alle $x \in V$ so heißt es nur harmonisch.
\end{definition}
Harmonische Funktionen erfüllen einige nützliche Prinzipien, Dazu sei für $W \subseteq V$
\begin{align}
	\delta W := \set{y \notin W \given \exists x \in W: y \sim x} && \overline{W} = W \cup \delta W
\end{align}
\begin{satz}[Maximumsprinzip]
	Sei $G(V,E,c)$ endlich oder unendlich und $H = (V(H),E(H),c)$ ein endliches zusammenhängendes Teilnetzwerk. Sei $f: V \to \Real$ harmonisch auf $H$. Falls dann gilt
	\begin{align}
		\max\limits_{x \in V(H)} f(x) = \sup\limits_G f
	\end{align}
	so gilt auf $V(H)$:
	\begin{align}
		f \equiv \sup f 
	\end{align}
\end{satz}
\begin{beweis}
	Sei $K := \set{f(x) = \sup f \given x \in V}$. Ist $x \in K \cap V(H)$, so gilt
	\begin{equation}
		\sup f = f(x) = \frac{1}{\pi(x)} \sum\limits_{y \sim x} c(x,y) f(y) \Rightarrow f(y) = f(x)
	\end{equation}
	für alle $y \sim x$. Das heißt, mit $x \in K$ sind auch alle $y \sim x$ in $K$. Da $V(H)$ endlich und zusammenhängend ist, kann man dies iterieren und kommt so bis $V(H) \cup \delta V(H) = \overline{V}(H)$.
\end{beweis}
\todo{was steht hier genau? (Eindeutigkeitsprinzip)} 
\begin{korollar}[Eindeutigkeitsprinzip]
	Sei $G = (V,E,c)$ beliebig, $W \subseteq V$ endlich. Falls $f,g: V \to Real$ harmonisch auf $W$ sind, so folgt
	\begin{align}
		f|_{V\backslash W} = g|_{V\backslash W} \Rightarrow f = g
	\end{align}
\end{korollar}
\begin{beweis}
	Die Funktion $h = f\cdot g$ ist harmonisch auf $W$ und $h=0$ auf $V\backslash W$. Wähle $x \in W$, sodass $h(x) > 0$ maximal ist. Wähle die Zusammenhangskomponente in der $x$ liegt. \todo{Beweis nochmal checken. } $h \equiv h(x)$ auf $W_0 \cup \delta W_0$. Aber $\delta W \subseteq V \backslash W$ und dort ist $h(x) = 0 \lightning \Rightarrow h \leq 0$ überall. 
	
	Das gleiche stimmt für $\overline{h} = g-f \leq 0$.
\end{beweis}

\begin{korollar}
	Sei $V$ endlich, $A,Z \subseteq V, v \cap Z = \setzero$. $G \colon V \to \Real$ erfüllen $G|_A = 1, G|_Z \equiv 0$ $G$ harmonisch azf $V \backslash (A \cap Z)$. Dann ist $G(x) = F(x) = \prop{\tau_A < \tau_Z}[][x]$
\end{korollar}
\begin{beweis}
	Folgt aus dem Eindeutigkeitsprinzip, wenn man $W = V\backslash (A \cap Z)$
\end{beweis}

\begin{satz}[Existenzprinzip]
	Sei $V$ endlich oder unendlich, $G = (V,E,c)$. Sei $W \nsubseteq V$ und $f_0 \colon V_0 \backslash W \to \Real$ beschränkt. Dann gibt es eine Funktion $f \colon V \to \Real$, sodass $f|_(v \backslash W) = f_0$ und $f|_W$ harmonisch.
\end{satz}

\begin{beweis}
	Starte die Netzwerk-Irrfahrt $(X_n)$, d.h. die deren \todo{text schwer lesbar} Übergangswahrscheinlichkeiten durch $c(x,y)$ bestimmt sind. Sei $X$ die ZV, die angibt, wann $(X_n)$ die Menge $V \backslash W$ trifft, falls dies überhaupt der Fall ist. Sei
	\begin{gather}
		Y = 
		\begin{cases}
			f_0(X), & \text{falls} (X_n) \text{ die Menge } V \backslash W \text{ trifft } \\
			0, & \text{ sonst}
		\end{cases}
	\end{gather}
	Behauptung: $\EW{Y}$ ist eine harmonische Fortsetzung von $f_0$
\end{beweis}
\begin{uebung}
	Berechnen Sie im Gambler's Run die Wahrscheinlichkeit links oder rechts anzuschlagen, sowie die erwartete Treffzeit.
\end{uebung}
\todo[inline]{missing figure}
Wir bertrachten jetzt ein Netzwerk als ein elektrisches Netzwerk, d.h. $c(x,y)$ sei die Leitfähigkeit und $\frac{1}{c(x,y)} = r(x,y)$ sei der Widerstand der Kante $(x,y)$. Wir legen in einem $a$ eine Spannung $v(a)$ an und leiten einen Strom $i$ durch das Netztwer, das in $Z \subseteq V$ abfließt ($v(z) = 0, \forall z \in Z$). Es gelten die klassischen Regeln der Elektrizitätslehre:

\begin{regel}[Ohmsches Gesetz]
	\label{regel:ohm}
	\begin{align}
		\forall x\sim y: v(x)-v(y) = i(x,y) \cdot r(x,y) = \frac{i(x,y)}{c(x,y)}
	\end{align}
\end{regel}

\begin{regel}[Kirchhoffsches Gesetz]
	\label{regel:kirch}
	\begin{align}
		\forall x \notin \set{a} \cup Z: \sum\limits_{y \sim x} i(x,y) = 0
	\end{align}
\end{regel}

Aus diesen Regeln lassen sich einige Konsequenzen ableiten:
\begin{itemize}
	\item da $c(x,y) = c(y,x) \Rightarrow i(x,y) = c(x,y)\big(v(x) -v(y)\big) = -c(x,y)\big(v(y)- v(x)\big) = - i(y,x)$
	\item da $c(x,y) > 0 \Rightarrow \Big({i(x,y) > 0 \Leftrightarrow v(x) > v(y)}\Big)$
	\item ist $x \notin A \cup Z$, so gilt nach \ref{regel:kirch} 
		\begin{align}
			0 = \sum\limits_{y \sim x} i(x,y) = \sum\limits_{y \sim x} \big[v(x)-v(y)\big]c(x,y)
		\end{align}
		Und wegen $c(x,y) = p_{xy} \cdot \pi(x)$, sowie im darauffolgenden Schritt $\sum\limits_{y \sim x} p_{xy} = 1$, folgt
		\begin{align}
			\Rightarrow \sum\limits_{y\sim x} v(y)c(x,y) = v(x)\pi(x) \sum\limits_{y \sim x} p_{xy} \\
			\Rightarrow v(x) = \frac{1}{\pi(x)} \sum\limits_{y \sim x} v(y) c(x,y)
		\end{align}
\end{itemize}



\todo[inline]{Hier fehlt noch ein großer Teil, der nachgetext werden muss (ab Korolloar 1.2)}
\newpage



