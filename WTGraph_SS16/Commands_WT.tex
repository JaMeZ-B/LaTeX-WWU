\DeclareMathOperator{\med}{med}					%median	
\DeclareMathOperator{\Bin}{Bin}					%Binomialvert.
\DeclareMathOperator{\BetaV}{Beta}				%Betavert.

\newcommand{\gauss}[1]{\left\lfloor#1\right\rfloor}	%Floor

\newcommand{\RG}{\ensuremath{G\enbrace{n,p}}} 	%Random Graph

%cases Umgebung anpassbar
\makeatletter
\renewcommand{\env@cases}[1][@{}l@{\quad}l@{}]{%
  \let\@ifnextchar\new@ifnextchar
  \left\lbrace
  \def\arraystretch{1.2}%
  \array{#1}%
}
\makeatother

%Anführungszeichen

\newcommand{\qte}[1]{\glqq #1\grqq}

%%Wkeitsbefehl definieren
%\newcommand{\propNEW}[2][\relax]{
%\ifx#3\relax
%	\ifx#2\relax
%		\ifx#1\relax
%			\ensuremath{\mathbb{P}}
%		\else
%			\ensuremath{\mathbb{P}\enb{#1}}
%		\fi
%%	\else
%%		\ifx#1\relax
%%			\ensuremath{\mathbb{P}_{#2}}
%%		\else
%%			\ensuremath{\mathbb{P}_{#2}\enb{#1}}
%%		\fi
%	\fi
%\else
%	\ifx#2\relax
%%		\ifx#1\relax
%%			\ensuremath{\mathbb{P}^{#3}
%%		\else
%%			\ensuremath{\mathbb{P}^{#3}\enb{#1}}
%%		\fi
%%	\else
%%		\ifx#1\relax
%%			\ensuremath{\mathbb{P}^{#3}_{#2}}
%%		\else
%%			\ensuremath{\mathbb{P}^{#3}_{#2}\enb{#1}}
%%		\fi
%	\fi			
%\fi
%}


\NewDocumentCommand{\propEckig}{m o o}{
	\IfNoValueTF{#2}
		{
			\IfNoValueTF{#3}
				{\ensuremath{\mathbb{P}\enb{\renewcommand\given{\;\middle\vert\;} #1}}}
				{\ensuremath{\mathbb{P}_{#3}\enb{\renewcommand\given{\;\middle\vert\;} #1}}}
		}
		{
			\IfNoValueTF{#3}
				{\ensuremath{\mathbb{P}^{#2}\enb{\renewcommand\given{\;\middle\vert\;} #1}}}
				{\ensuremath{\mathbb{P}^{#2}_{#3}\enb{\renewcommand\given{\;\middle\vert\;} #1}}}	
		}
	}
\NewDocumentCommand{\propRund}{m o o}{
	\IfNoValueTF{#2}
	{
		\IfNoValueTF{#3}
		{\ensuremath{\mathbb{P}\enb{\renewcommand\given{\;\middle\vert\;} #1}}}
		{\ensuremath{\mathbb{P}_{#3}\enb{\renewcommand\given{\;\middle\vert\;} #1}}}
	}
	{
		\IfNoValueTF{#3}
		{\ensuremath{\mathbb{P}^{#2}\enb{\renewcommand\given{\;\middle\vert\;} #1}}}
		{\ensuremath{\mathbb{P}^{#2}_{#3}\enb{\renewcommand\given{\;\middle\vert\;} #1}}}	
	}
}	


	
\newcommand{\prop}[1]{\propRund{#1}}
\newcommand{\propE}[1]{\propEckig{#1}}

\newcommand{\p}{\mathbb{P}}
%%Wkeitsbefehl definieren
%\newcommand{\prop}[1]{
%	\ifx#1\relax \ensuremath{\mathbb{P}}
%	\else \propRund{#1}
%	\fi}

%Wkeitsbefehl definieren Eckig
%\newcommand{\propE}[1]{
%	\ifx#1\relax \ensuremath{\mathbb{P}}
%	\else \ensuremath{\mathbb{P}\bracketeckig{#1}}	
%	\fi}

%EW definieren
\newcommand{\EW}[1]{
	\ifx#1\relax \ensuremath{\mathbb{E}}
	\else \ensuremath{\mathbb{E}\enb{#1}}	
	\fi}

%Var definieren
\newcommand{\Var}[1]{
	\ifx#1\relax \ensuremath{\mathbb{V}}
	\else \ensuremath{\mathbb{V}\enb{#1}}	
	\fi}

%Cov definieren
\newcommand{\Cov}[1]{
	\ifx#1\relax \ensuremath{Cov}
	\else \ensuremath{Cov\enb{#1}}	
	\fi}

%Cov definieren eckig
\newcommand{\CovE}[1]{
	\ifx#1\relax \ensuremath{Cov}
	\else \ensuremath{Cov\benb{#1}}	
	\fi}

%EW definieren Eckig
\newcommand{\EWE}[1]{
	\ifx#1\relax \ensuremath{\mathbb{E}}
	\else \ensuremath{\mathbb{E}\benb{#1}}	
	\fi}

%Var definieren Eckig
\newcommand{\VarE}[1]{
	\ifx#1\relax \ensuremath{\mathbb{V}}
	\else \ensuremath{\mathbb{V}\benb{#1}}	
	\fi}

%asymptotisch fast sicher
\newcommand{\afs}{\ifmmode \mathrm{\ a.f.s.} \else a.f.s. \fi}

%Landau big O
\newcommand{\bigO}[1]{\mathcal{O}\enb{#1}}
%Reele Zahlen
\newcommand{\Real}{\mathbb{R}}
\newcommand{\Nat}{\mathbb{N}}
\newcommand{\weight}[1]{\mathcal{C}\enb{#1}}
