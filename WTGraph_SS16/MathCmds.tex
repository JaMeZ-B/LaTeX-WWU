% % % % % % % % % % % % % % % % % % % % % % % % % % % % % % % % %
%		PhistMath.tex											%
%		Weitere Mathe-Befehle									%
%																%
%		Author: Phil Steinhorst									%
% % % % % % % % % % % % % % % % % % % % % % % % % % % % % % % % %

% % % Buchstaben und Zahlen
\newcommand{\aff}{\mathbb{A}}
\newcommand{\CC}{\mathbb{C}}
\newcommand{\EE}{\mathbb{E}}
\newcommand{\FF}{\mathbb{F}}
\newcommand{\HH}{\mathcal{H}}
\newcommand{\KK}{\mathbb{K}}
\newcommand{\LL}{\mathbb{L}}
\newcommand{\MM}{\mathcal{M}}
\newcommand{\NN}{\mathbb{N}}
\newcommand{\OO}{\mathbb{O}}
\newcommand{\QQ}{\mathbb{Q}}
\newcommand{\RR}{\mathbb{R}}
\newcommand{\Ss}{\mathbb{S}}
\newcommand{\UU}{\mathcal{U}}
\newcommand{\ZZ}{\mathbb{Z}}
\newcommand{\oh}{\mathcal{O}}					% Landau-O
\newcommand{\ind}{1\hspace{-0,8ex}1} 			% Indikatorfunktion (Doppeleins)

% % % Abk�rzungen
\newcommand{\bewrueck}{"$\Leftarrow$":} 	% Beweis R�ckrichtung
\newcommand{\bewhin}{"$\Rightarrow$":}		% Beweis Hinrichtung
\newcommand{\setone}{\{1\}}							% Einsmenge
\newcommand{\NT}{\trianglelefteq}					% Normalteiler
\newcommand{\setzero}{\{0\}}						% Nullmenge

% % % Operatoren
\DeclareMathOperator{\Abb}{Abb}						% Menge der Abbildungen
\DeclareMathOperator{\Aut}{Aut} 					% Automorphismen
\DeclareMathOperator{\Bild}{Bild}					% Kern
\DeclareMathOperator{\Char}{char} 					% Charakteristik
\DeclareMathOperator{\Det}{\det\,\!}				% Determinante
\DeclareMathOperator{\End}{End}
\DeclareMathOperator{\ev}{ev}						% Auswertungsabb.
\DeclareMathOperator{\GL}{GL}						% allgemeine lineare Gruppe
\DeclareMathOperator{\grad}{grad}					% Grad
\DeclareMathOperator{\Hom}{Hom} 					% Homomorphismen
\DeclareMathOperator{\id}{id} 						% Identit�t
\DeclareMathOperator{\im}{im} 						% image
\renewcommand{\Im}{\operatorname{Im}}				% Imagin�rteil
\DeclareMathOperator{\Isom}{Isom}					% Isometrien
\DeclareMathOperator{\Kern}{Kern}					% Kern
\DeclareMathOperator{\LH}{LH}						% Lineare H�lle
\DeclareMathOperator{\ord}{ord} 					% Ordnung
\DeclareMathOperator{\pot}{\mathcal{P}}				% Potenzmenge
\DeclareMathOperator{\Rang}{Rang}					% Rang
\renewcommand{\Re}{\operatorname{Re}}				% Realteil
\DeclareMathOperator{\SAut}{SAut}
\DeclareMathOperator{\sgn}{sgn} 					% Signum
\DeclareMathOperator{\sign}{sign} 					% Signum
\DeclareMathOperator{\SL}{SL} 						% Spezielle lineare Gruppe
\DeclareMathOperator{\SO}{SO} 						% Spezielle orthogonale Gruppe
\DeclareMathOperator{\SU}{SU} 						% Spezielle unit�re Gruppe
\DeclareMathOperator{\Sym}{Sym} 					% Symmetrische Gruppe

% % % Klammerungen
\DeclarePairedDelimiter{\abs}{\lvert}{\rvert}			% Betrag
\DeclarePairedDelimiter{\ceil}{\lceil}{\rceil}			% aufrunden
\DeclarePairedDelimiter{\floor}{\lfloor}{\rfloor}		% aufrunden
\DeclarePairedDelimiter{\Norm}{\lVert}{\rVert}			% Norm
\DeclarePairedDelimiter{\sprod}{\langle}{\rangle}		% spitze Klammern
\DeclarePairedDelimiter{\enbrace}{(}{)}					% runde Klammern
\DeclarePairedDelimiter{\benbrace}{\lbrack}{\rbrack}	% eckige Klammern
\DeclarePairedDelimiter{\penbrace}{\{}{\}}				% geschweifte Klammern
\newcommand{\Underbrace}[2]{{\underbrace{#1}_{#2}}} % Underbrace als Befehl in LaTeX-Syntax (und ohne Spacing-Probleme mit nachfolgenden Operatoren...)

\newcommand{\enb}[1]{\enbrace*{#1}}
\newcommand{\penb}[1]{\penbrace*{#1}}
\newcommand{\benb}[1]{\benbrace*{#1}}

\newcommand{\stack}[2]{\makebox[1cm][c]{$\stackrel{#1}{#2}$}}
\newcommand{\ol}[1]{\overline{#1}}
\newcommand{\mat}[4]{\tensor*[^{#2}_{}]{#1}{^{#3}_{#4}}}

\newcommand\SetSymbol[1][]{\nonscript\:#1\vert\allowbreak\nonscript\:\mathopen{}}
\providecommand\given{} % to make it exist
\DeclarePairedDelimiterX\set[1]\{\}{\renewcommand\given{\SetSymbol[\delimsize]}#1}

\usepackage{stmaryrd}