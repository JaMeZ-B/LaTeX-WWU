%!TEX root = operatoralgebren2.tex
Hier findet man einige ausgewählte Beweise und Ergänzungen, auf die der Vorlesung hingewiesen wurde und die teilweise im Rahmen von Übungsaufgaben bearbeitet wurden.
Sie stammen also auch hauptsächlich aus den Übungen und deswegen sei hier noch einmal darauf hingewiesen, dass für die Richtigkeit keine Garantie übernommen wird!

Die Reihenfolge richtet sich nach dem Auftreten der Verweise in der Vorlesung.

\subsection{Injektivität der Einbettung des Tensorproduktes von beschränkten Operatoren} % (fold)
\label{sub:injek_tensor_beschraenkte_operatoren}
\emph{Dies beweist die Injektivität der Abbildung aus \autoref{bem:16} und war Aufgabe 3 von Blatt 1.}\smallskip

Seien $\mathcal{H}_1$ und $\mathcal{H}_2$ Hilberträumen.
Dann ist die kanonische Abbildung
\mapdef{\iota \colon \mathcal{B}(\mathcal{H}_1) \odot \mathcal{B}(\mathcal{H}_2)}{\mathcal{B}(\mathcal{H}_1 \otimes \mathcal{H}_2)}{a \otimes b}{\enbrace[\big]{\xi \otimes \mu \mapsto a(\xi) \otimes b(\mu)}}{}
injektiv, liefert also eine Einbettung.
\begin{beweis}
	Ein beliebiges Element in $\mathcal{B}(\mathcal{H}_1) \odot \mathcal{B}(\mathcal{H}_2)$ ist von der Form $\sum_{i} S_i \otimes T_i$ mit
	$S_i \in \mathcal{B}(\mathcal{H}_1)$ und $T_i \in \mathcal{B}(\mathcal{H}_2)$.
	Ohne Beschränkung der Allgemeinheit können wir annehmen, dass die $S_i$ linear unabhängig sind.
	Sei also $\iota \enbrace*{\sum_{i} S_i \otimes T_i}=\sum_{i} S_i \otimes T_i=0$.
	Für $v_1,w_1 \in \mathcal{H}_1$ und $v_2,w_2 \in \mathcal{H}_2$ gilt
	\begin{align}
		\skal*{\enbrace*{\sum\nolimits_{i} S_i \otimes T_i}(v_1 \otimes v_2)}{w_1 \otimes w_2} &= \skal*{\sum\nolimits_i S_i(v_1) \otimes T_i(v_2)}{w_1 \otimes w_2}\\
		&= \sum\nolimits_i \skal*{S_i(v_1)}{w_1} \cdot \skal*{T_i(v_2)}{w_2} \\
		&= \skal*{\enbrace*{\sum\nolimits_i \skal*{T_i (v_2)}{w_2} \cdot S_i}(v_1)}{w_1}_{\mathcal{H}_1}
	\end{align}
	Damit erhalten wir 
	\begin{align}
		\sum\nolimits_{i} S_i \otimes T_i =0 &\iff \skal*{\enbrace*{\sum\nolimits_{i} S_i \otimes T_i}(v_1 \otimes v_2)}{w_1 \otimes w_2} =0 & \forall v_1,w_1 \in \mathcal{H}_1 v_2,w_2 \in \mathcal{H}_2 \\
		&\iff \skal*{\enbrace*{\sum\nolimits_i \skal*{T_i (v_2)}{w_2} \cdot S_i}(v_1)}{w_1}_{\mathcal{H}_1} =0 & \forall v_1,w_1 \in \mathcal{H}_1 v_2,w_2 \in \mathcal{H}_2\\
		&\iff \sum\nolimits_i \skal*{T_i(v_2)}{w_2} \cdot S_i =0 & \forall v_2,w_2 \in \mathcal{H}_2 \\
		&\iff \skal*{T_i(v_2)}{w_2} =0 & \forall v_2,w_2 \in \mathcal{H}_2 \\
		&\iff T_i =0 & \forall i
	\end{align}
	Es hat genügt dies auf Elementartensoren zu zeigen, denn $\Span \set*{v_1 \otimes v_2 \given v_1 \in \mathcal{H}_1 , v_2 \in \mathcal{H}_2}$ ist dicht in $\mathcal{H}_1 \otimes \mathcal{H}_2$.
\end{beweis}
% subsection injek_tensor_beschraenkte_operatoren (end)












