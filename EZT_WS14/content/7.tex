\section{Fermatsche und Mersennesche Primzahlen}
\label{sec:para7}
	
\minisec{Bemerkung}
	$2^k + 1$ prim $\Rightarrow$ $k$ ist Potenz von $2$ (vgl. Aufgabe 4). \marginnote{19.12. \\ \ [20]}
	
Für $n \in \NN$ sei
\[ F_n := 2^{2^n} + 1 \]
die $n$-te Fermatsche Zahl.

\begin{enumerate}[(1)]
	\item $F_0 = 3, F_1 = 5, F_2 = 17, F_3 = 257$ und $F_4 = 65537$ sind \bet{Fermatsche Primzahlen}. \index{Fermatsche Primzahl}
\end{enumerate}

\minisec{Satz}
	Ein regelmäßiges $n$-Eck ist mit Zirkel und Lineal genau dann konstruierbar, wenn $n$ von der Gestalt $n = 2^\nu p_1 p_2 \cdots p_r$ mit $r \in \NN_0$, paarweise verschiedenen Fermatschen Primzahlen $p_1,\dots,p_r$ und beliebigem $\nu \in \NN_0$ ist.
	
\begin{enumerate}[(1)] \setcounter{enumi}{1}
	\item $F_5$ ist keine Primzahl, sondern durch $641$ teilbar.
	\item Für $n \geq 3$ hat jeder Primteiler von $F_n$ die Gestalt $p = t 2^{n+2} + 1$.
	\item Auch $F_6$ ist nicht prim, sondern teilbar durch $274177 = 1071 \cdot 2^9 + 1$.
	\item $F_n$ ist nicht prim für $5 \leq n \leq 32$.
	\item Ob $F_n$ für $n = 33, 34, 35$ prim ist oder nicht, ist unbekannt. Für $n = 36$ ist wieder bekannt, dass $F_n$ nicht prim ist, ebenso für $n = 37, 38, 39$. Der nächste offene Fall ist $n = 40$.
	\item Derzeit ist von $271$ Fermatzahlen bekannt, dass sie nicht prim sind\footnote{vor 10 Jahren waren es erst $217$. Vor 40 Jahren war noch unsicher, je entscheiden zu können, ob $F_{17}$ prim ist oder nicht.}. Deren größte ist $F_{2747497}$; sie hat den Teiler $57 \cdot 2^{2747499}$. (Stand Mai 2013)
	\item Wie gesagt ist $F_{20}$ nicht prim, aber es ist kein Primteiler von $F_{20}$ bekannt; das gleiche für $F_{24}$.
\end{enumerate}

Offene (unlösbare?) Probleme:
\begin{enumerate}[1)]
	\item Gibt es unter den $F_n$ nur endlich viele Primzahlen? Man kennt bisher nur die fünf in (1).
	\item Sind unendlich viele $F_n$ nicht prim?
	\item Sind alle $F_n$ quadratfrei? Oder wenigstens unendlich viele?
\end{enumerate}

\begin{falko} \label{F7.1}
	Es gilt $F_{n+1} - 2 = \prod_{k=0}^{n} F_k$. Insbesondere sind $F_n,F_m$ für $m > n$ teilerfremd.
\end{falko}

Für welche ungeraden $n$ ist $n$-Teilung des Kreises mit Zirkel und Lineal möglich? Bisheriger Rekord:\\
$n = 3 \cdot 15 \cdot 17 \cdot 257 \cdot 65537 = F_5 - 2 = 2^{32} - 1 = 4.294.967.295$.

\minisec{Bemerkung}
	Aus F\ref{F7.1} folgt, dass es unendlich viele Primzahlen gibt: Jedes $F_n$ hat einen Primteiler $q_n$. Nach F\ref{F7.1} ist aber $q_n \neq q_m$ für $n \neq m$. \\
	Für die $n$-te Primzahl $p_n$ gilt $p_n \leq F_{n-2} = 2^{2^{n-2}}+1$ (sehr schlechte Abschätzung, aber nicht schlechter als die in §\ref{sec:para1} aus dem Beweis von Euklid: $p_n \leq 2^{2^{n-1}}$.)
	
\begin{falko}[Pépin-Test] \label{F7.2}
	Sei $n \geq 2$ und $g$ sei eine ganze, zu $F_n = 2^{2^n} + 1$ teilerfremde Zahl mit $\leg{g}{F_n}_J = -1$. Dann sind äquivalent: \begin{enumerate}[(i)]
		\item $F_n$ ist prim \index{Pépin-Test}
		\item $g^{\frac{F_n-1}{2}} \kon -1 \modu F_n$
		\item $\ord(g \modu F_n) = F_n - 1 = 2^{2^n}.$
	\end{enumerate}
\end{falko}

\minisec{Bemerkung}
	Den Test kann man zum Beispiel mit $g = 3, g= 5, g=10$ anwenden. Denn $F_n = 2^{2^n} + 1 \kon 2^{2 \cdot 2^{n-1}} \kon 1 + 1 \kon 2 \modu 3$ und $\leg{3}{F_n} = \leg{F_n}{3} = \leg{2}{3} = -1$. \\
	$F_n = 2^{4 \cdot 2^{n-2}} + 1 \kon 1 + 1 \kon 2 \modu 5$, $\leg{5}{F_n} = \leg{F_n}{5} = \leg{2}{5} = -1$, $(10, F_n) = 1$ und \\ $\leg{10}{F_n} = \leg{2}{F_n} \leg{5}{F_n} = \leg{5}{F_n} = -1$.

\minisec{Bemerkung}
	$2^k - 1$ prim $\Rightarrow k$ prim (Aufgabe 1).
	\[ M_p := 2^p - 1, p \text{ prim} \]

\begin{enumerate}[(1)]
	\item $M_2 = 3, M_3 = 7, M_5 = 31, M_7 = 127$ sind \bet{Mersennesche Primzahlen}. \index{Mersennesche Primzahl}
	\item $M_{11}$ ist keine Primzahl: $M_{11} = 2^{11} - 1 = 2047 = 23 \cdot 89$.
	\item $M_{13}, M_{17}, M_{19}$ sind prim.
	\item Für $23 \leq p \leq 100$ ist nur $M_{31}, M_{61}, M_{89}$ prim.
	\item Mersenne hat fünf Fehler gemacht: die Primzahlen $M_{61}, M_{89}, M_{107}$ übersehen und $M_{67}, M_{257}$ als prim behauptet.
	\item Für $100 \leq p \leq 257$ sind nur $M_{107}, M_{127}$ prim. \\
	$M_{127}$ bis 1951 größte bekannte Primzahl. 30. Januar 1952: $M_{521}$ ist prim! Zwei Stunden später: $M_{607}$ auch! \\
	$p$ Mersennesche Primzahl $\Rightarrow M_p$ prim? \\
	$M_3, M_7, M_{31}, M_{127}$ prim, aber $M_{8191} = M_{M_{13}}$ keine Primzahl.
	\item Für $p \leq 12000$ sind unter den ca. 1250 vielen $M_p$ genau 23 Primzahlen. $M_{11213}$ ist die größte.
	\item Bislang sind 48 Mersennesche Primzahlen bekannt (Stand Oktober 2014); die größte ist $M_{57885161}$. Sie hat 17.425.170 Stellen.
\end{enumerate}

\setcounter{countfalko}{3}
\begin{falko} \label{F7.4}
	Sei $p$ eine Primzahl mit $p \kon 3 \modu 4$. Dann gilt:
	\[2p + 1 \text{ Primzahl } \Leftrightarrow 2p+1 \text{ teilt } M_p. \]
	Ist also $2p+1$ eine Primzahl mit $p$ wie oben und $p \neq 3$, so ist $M_p$ keine Primzahl.
\end{falko}

\minisec{Bemerkung}
	Jeder Primteiler $q$ von $M_p$, $p \geq 3$, hat die Gestalt $q = 2kp + 1$.
	
\begin{falko}[Lucas-Test] \label{F7.5}
	Definiere Folge natürlicher Zahlen $s_1, s_2, \dots$ durch $s_1 = 4, s_{n+1} = s_n^2 - 2$. Dann gilt: \index{Lucas-Test}
	\[ M_p \text{ prim } \Leftrightarrow s_{p-1} \kon 0 \modu M_p \]
\end{falko}

\setcounter{countfalko}{2}
\begin{falko} \label{F7.3}
	Sei $p$ ein Primteiler einer Fermatschen Zahl $F_n$ bzw. einer Mersenne-Zahl $M_q$. Genau dann geht auch $p^2$ in $F_n$ bzw. $M_q$ auf, wenn die Bedingung \marginnote{06.01. \\ \ [21]}
	\begin{equation}
		2^{p-1} \kon 1 \modu p^2 \label{eq_W}
	\end{equation}
	erfüllt ist.
\end{falko}

Eine Primzahl, die \eqref{eq_W} erfüllt, heißt eine \Index{Wieferich-Primzahl}. Bisher sind nur die Wieferich-Primzahlen 1093 und 3511 bekannt; bis $10^{15}$ gibt es keine weiteren. Die Primzahlen $p = 1093$ und $p = 3511$ kommen aber wegen $1093 = 1 + 273 \cdot 2^2$ und $3511 = 1 + 1755 \cdot 2^1$ nicht als Primteiler einer Fermatzahl $F_n$ in Frage. Und auch nicht als Primteiler einer Mersenne-Zahl $M_q$, denn sonst wäre $q$ -- wegen $2^q \kon 1 \modu p$ -- ein Primteiler von $p-1$, also wegen $1093 - 1 = 2^2 \cdot 3 \cdot 7 \cdot 13$ und $3511 - 1 = 2 \cdot 3^3 \cdot 5 \cdot 13$ eine der Primzahlen $q = 2,3,5,7$ oder $13$. Für alle diese $q$ ist aber $M_q$ prim (und verschieden von 1093 und 3511).

\minisec{Bemerkung}
	Die \Index{Fermatsche Vermutung} besagt: Die Gleichung
	\begin{equation}
		X^n + Y^n = Z^n \label{eq_fermat}
	\end{equation}
	besitzt für $n \geq 3$ keine Lösung in natürlichen Zahlen, d.h. es gibt kein Tripel $(a,b,c) \in \NN^3$ mit $a^n+b^n=c^n$. Um dies für alle $n \geq 3$ zu etablieren, genügt es offenbar zu zeigen, dass \eqref{eq_fermat} im Falle einer Primzahl $n = p \geq 3$ keine Lösung $(a,b,c) \in \NN^3$ besitzt sowie im Falle $n = 4$. Den Fall $n = 4$ konnte bereits Fermat erledigen (F\ref{F7.3} auf Seite~\pageref{F7.3}). \\
	Im Falle $n = p \geq 3$ prim besagt der so genannte erste Fall der Fermatschen Vermutung, dass die Gleichung \eqref{eq_fermat} keine Lösung $(a,b,c) \in \NN^3$ hat, bei der keine der Zahlen $a,b,c$ durch $p$ teilbar ist. Dies, so konnte Wieferich zeigen, gilt unter der Voraussetzung, dass $p$ die Bedingung $2^{p-1} \not\kon 1 \modu p^2$ erfüllt.

\subsection{Zur Bedeutung der Mersenneschen Primzahlen}
\begin{defn}[vollkommene Zahl]
	Eine natürliche Zahl $n$ heißt $\Index{vollkommen}$, wenn sie gleich der Summe ihrer natürlichen Teiler $d < n$ ist. Wir definieren für $n \in \NN$ die Summation über alle natürlichen Teiler von $n$:
	\[ \sigma(n) := \sum_{d \mid n} d \]
\end{defn}

$n \in \NN$ ist genau dann vollkommen, wenn $\sigma(n) = 2n$. Ist $\sigma(n) < 2n$, so heißt $n$ \Index{defizient}. Andernfalls heißt $n$ \Index{abundant}.
\begin{enumerate}[(1)]
	\item $\sigma(p^k) = 1 + p + p^2 + \dots + p^k = \frac{p^{k+1}-1}{p-1} < 2p^k$, defizient
	\item $(n_1,n_2) = 1 \Rightarrow \sigma(n_1n_2) = \sigma(n_1) \sigma(n_2)$ \\
	$d_1 \mid n_1, d_2 \mid n_2 \Rightarrow d_1 d_2 \mid n_1 n_2$ \\
	$d \mid n_1 n_2 \Rightarrow d=d_1d_2$ mit $d_i \mid n_i$
	$\sum\limits_{d \mid n_1n_2} d = \sum\limits_{d_1 \mid n_1, d_2 \mid n_2} d_1d_2 = \sum\limits_{d_1 \mid n_1} d_1 \cdot \sum\limits_{d_2 \mid n_2} d_2$
\end{enumerate}

\setcounter{countfalko}{5}
\begin{falko}	\label{F7.6}
	Hat $n$ die Gestalt $n = 2^{k-1} (2^k-1)$ und ist dabei $2^k - 1$ eine Primzahl, so ist $n$ vollkommen.
\end{falko}

\begin{falko}	\label{F7.7}
	Die geraden vollkommenen Zahlen $n$ sind genau die Zahlen der Gestalt $n = 2^{p-1} M_p$ mit $M_p$ prim.
\end{falko}

\minisec{Problem (ungelöst)}
	Gibt es ungerade vollkommene Zahlen?
	
\minisec{Bemerkung}
	Man weiß: $n$ ungerade mit weniger als 8 Primteilern ist nicht vollkommen. \\
	Jedes ungerade $n \leq 10^{200}$ nicht vollkommen.
\newpage