%!TEX root = ana_top_geo.tex

\subsection{Ausführlicher Beweis zu \cref{lem:kpt-schnitte}} % (fold)
\label{sub:kpt-schnitte}
Sei $X$ ein Hausdorffraum. Dann ist $X$ genau dann kompakt, wenn gilt: Hat eine Familie $\mathcal{A}$ von abgeschlossenen Teilmengen von $X$ die endliche 
Durchschnittseigenschaft, so gilt 
\[
	\bigcap_{A \in \mathcal{A}} A \not= \emptyset.
\]
\begin{beweis}
	Für die erste Implikation sei $X$ kompakt und $\mathcal{A}$ eine Familie von abgeschlossenen Mengen mit der endlichen Durchschnittseigenschaft.
	Angenommen $\bigcap_{A \in \mathcal{A}} A = \emptyset$.
	Dann gilt
	\[
		X = X \setminus \bigcap_{A \in \mathcal{A}} A = \bigcup_{A \in \mathcal{A}} X \setminus A.
	\]
	Nun ist $\mathcal{U} \coloneqq \set*{X \setminus A \given A \in \mathcal{A}}$ eine offene Überdeckung von $X$ und da $X$ kompakt ist, existiert $\mathcal{A}_0 \subset \mathcal{A}$ endlich, sodass
	\[
		X = \bigcup_{A \in \mathcal{A}_0} X \setminus A = X \setminus \underbrace{\bigcap_{A \in \mathcal{A}_0 } A }_{\neq \emptyset} \quad \light
	\]
	Für die umgekehrte Implikation sei nun $\mathcal{U} = \set{U_i}_{i \in I}$ eine offene Überdeckung von $X$.
	Angenommen für jede endliche Teilmenge $J \subseteq I$ gilt $X \neq \bigcup_{i \in J} U_i$.
	Betrachte nun $\mathcal{A} =  \set{X \setminus U_i}_{i \in I}$. Dann gilt nach Annahme
	\[
		\bigcap_{i \in J} X \setminus U_i = X \setminus \bigcup_{i \in J} U_i \neq \emptyset.
	\]
	Also hat $\mathcal{A}$ die endliche Durchschnittseigenschaft. Nach Vorraussetzung gilt dann
	\[
		\emptyset \not= \bigcap_{i \in I} X \setminus U_i = X \setminus \underbrace{\bigcup_{i \in I} U_i}_{= X} \quad \light \qedhere
	\]
\end{beweis}

