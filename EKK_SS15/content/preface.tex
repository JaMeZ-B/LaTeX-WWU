\section*{Vorwort}
\label{sec:preface}
	Der vorliegende Text ist eine inhaltliche Aufbereitung zur Vorlesung Elliptische Kurven und Kryptographie, gelesen von PD Dr. Karin Halupczok an der WWU Münster im Sommersemester 2015. Der Inhalt entspricht weitestgehend dem handschriftlichen Skript, welches auf der Vorlesungswebsite bereitsgestellt wird. Dieses Werk ist daher keine Eigenleistung des Autors und wird nicht von der Dozentin der Veranstaltung korrekturgelesen. Für die Korrektheit des Inhalts wird keinerlei Garantie übernommen. Bemerkungen, Korrekturen und Ergänzungen kann man folgenderweise loswerden:
	\begin{itemize}
		\item persönlich durch Überreichen von Notizen oder per E-Mail
		\item durch Abändern der entsprechenden TeX-Dateien und Versand per E-Mail an mich
		\item direktes Mitarbeiten via GitHub. Dieses Skript befindet sich im \texttt{latex-wwu}-Repository von Jannes Bantje:
		\begin{center}
			\url{https://github.com/JaMeZ-B/latex-wwu}
		\end{center}
	\end{itemize}

\subsection*{Literatur}
\label{sub:lit}
\begin{itemize}
	\item Blake, Seroussi, Smart: Elliptic curves in cryptography
	\item Menezes, van Oorschot, Vanstone: Handbook of applied cryptography
	\item Silverman: The arithmetic of elliptic curves
	\item Silverman: A friendly introduction to number theory, chap. 40-45
	\item Washington: Elliptic curves, number theory and cryptography
	\item Werner: Elliptische Kurven in der Kryptographie
\end{itemize}

\subsection*{Kommentar der Dozentin}
In der Vorlesung beschäftigen wir uns mit den arithmetischen und geometrischen Eigenschaften elliptischer Kurven sowie deren Anwendungen in der Kryptographie. Dabei werden wir auch einen Vergleich mit Anwendungen der elementaren Zahlentheorie in der Kryptographie ziehen. Wir verfolgen eine elementare Herangehensweise, d.h. Kenntnisse der algebraischen Geometrie und der Funktionen- oder Zahlentheorie werden nicht benötigt. Es genügen die Vorkenntnisse aus den Grundvorlesungen.

\subsection*{Vorlesungswebsite}
\label{sub:link}
Das handgeschriebene Skript sowie weiteres Material findet man unter folgendem Link:
\begin{center}
	\url{\homepage}
\end{center}

\subsection*{Titelbild}
\label{sub:titlepic}
Das fehlt noch. Über Ideen und Anregungen freue ich mich sehr!

\vfill
\begin{flushright}
	Phil Steinhorst \\
	p.st@wwu.de
\end{flushright}
\newpage