%!TEX root = ../LA2.tex
\section{Eine Anwendung der Jordan-Normalform in der Analysis}
\label{sec:2.10}

In vielen physikalischen Anwendungen ist es notwendig, System von \textbf{Differentialgleichungen} der Form \index{Differentialgleichung}
\begin{align*}
	y_1'(t) &= b_{11} \cdot y_1(t) + b_{12} \cdot y_2(t) + \cdots + b_{1n} \cdot y_n(t) \\
	y_2'(t) &= b_{21} \cdot y_1(t) + b_{22} \cdot y_2(t) + \cdots + b_{2n} \cdot y_n(t) \\
	 & \qquad \qquad \qquad \vdots \\
	y_n'(t) &= b_{n1} \cdot y_1(t) + b_{n2} \cdot y_2(t) + \cdots + b_{nn} \cdot y_n(t)
\end{align*}
zu betrachten, wobei $y_1,\dots,y_n\colon \RR \rightarrow \RR$ differenzierbare reelle Funktionen sind und $b_{ij} \in \RR$ für alle $1 \leq i,j \leq n$.
Ist dann $B = (b_{ij}) \in M(n \times n, \RR)$, so können wir das System auch in der Kurzschreibweise
\[
	y'(t) = B \cdot y(t), y(t) := (y_1(t),y_2(t),\dots,y_n(t))^T
\]
formulieren, wobei dann $y'(t) := (y_1'(t),\dots,y_n'(t))^T$ die komponentenweise Ableitung von $y\colon \RR \rightarrow \RR^n$ bezeichnet.
Definieren wir die Konvergenz von Vektoren durch die Konvergenz der Komponenten, so ist $y \colon \RR \rightarrow \RR^n$ differenzierbar in $t$ genau dann, wenn
\[
	y'(t) = \lim\limits_{h \rightarrow 0} \frac{1}{h} (y(t+h)-y(t))
\]
existiert.
In der Regel wird zusätzlich zum oben gegebenen System von Differentialgleichungen noch eine Anfangsbedingung
\[
	y(t_0) = y_0 = (y_{10},\dots,y_{n0})^T
\]
vorgegeben, wobei $t_0 \in \RR$ eine gegebene Anfangszeit ist.
Gesucht sind also alle komponentenweise differenzierbaren Funktionen $y \colon \RR \rightarrow \RR^n$ mit $y' = B \cdot y$ und $y(t_0) = y_0$.
Eine Gleichung der Form $y' = By$ heißt auch \textbf{(homogenes) System von linearen Differentialgleichungen erster Ordnung mit konstanten Koeffizienten}.

\begin{beispiel}
	\label{bsp:10.1}
	Ein Zahlen-Beispiel:
	Gesucht sind alle differenzierbaren Funktionen $y_1,y_2,y_3\colon \RR \rightarrow \RR$ mit
	\begin{align*}
		y_1'(t) &= y_1(t) + 2y_2(t) - y_3(t) \\
		y_2'(t) &= -y_2(t) + y_3(t) \\
		y_3'(t) &= 3y_1(t) + y_2(t)
	\end{align*}
	und der Anfangsbedingung $y_1(0) = 1, y_2(0) = 0, y_3(0)=2$.
	Ist dann $B = \begin{pmatrix}
		1 & 2 & -1 \\
		0 & -1 & 1 \\
		3 & 0 & 1
	\end{pmatrix}$, so ist das obige System von Differentialgleichungen gegeben durch die Gleichung $y' = By$, $y(0) = (1,0,2)^T$, wobei $y = (y_1,y_2,y_3)^T \colon \RR \rightarrow \RR^3$ die gesuchte $\RR^3$-wertige Funktion ist.
\end{beispiel}

\begin{bemerkung}
	\label{bem:}
	\mbox{} \\[-1.4cm]
	\begin{enumerate}[(i)]
		\item Tatsächlich ist es nützlich, sich nicht auf den Fall reeller Funktionen zu beschränken, sondern auch komplexwertige Funktionen und komplexe Matrizen zuzulassen.
		Ist dabei $f\colon \RR \rightarrow \CC$, so ist $f$ differenzierbar in $t \in \RR$, wenn $f'(t) := \lim\limits_{h \rightarrow 0} \frac{f(t+h) - f(t)}{h}$ existiert.
		Ist $f = g + ih$ und $g,h \colon \RR \rightarrow \RR$, so ist $f$ genau dann differenzierbar in $t$, wenn $g$ und $h$ in $t$ differenzierbar sind, und dann gilt $f'(t) = g'(t) + ih'(t)$.
		Mit diesem Verständnis für die Differenzierbarkeit komplexwertiger Funktionen auf $\RR$ können wir jetzt auch Systeme der Form
		\[
			y' = By, y(t_0)=y_0
		\]
		mit $B \in M(n \times n,\CC)$ und $y_0 \in \CC^n$ betrachten.
		Die gesuchte Lösung ist dann eine Funktion $y\colon \RR \rightarrow \CC^n$.
		Wir werden im Folgenden diesen allgemeineren Fall betrachten.
		\item Durch die Substitution $t \mapsto t-t_0$ können wir uns immer auf den Fall $t_0 = 0$ beschränken:
		Ist $\wt{y} \colon \RR \rightarrow \CC^n$ mit $\wt{y}' = B \wt{y}$ und $\wt{y}(0) = y_0$ und setzen wir $y(t) := \wt{y}(t-t_0)$, so folgt sofort aus den Ableitungsregeln, dass $y' = By$ mit $y(t_0) = y_0$.
	\end{enumerate}
\end{bemerkung}

Im eindimensionalen Fall ist es sehr leicht, eine Lösung der Differentialgleichung $y'(t) = b \cdot y(t), y(0) = y_0$ anzugeben.
Nachrechnen zeigt sofort, dass $y(t) = e^{tb} y_0$ eine Lösung dieser Gleichung ist.
Mit etwas mehr Mühe sieht man dann sogar ein, dass dies die einzige Lösung der Gleichung ist.
Wir werden gleich sehen, dass eine ähnliche Formel auch für Systeme gilt.

\begin{definition}[Exponentialfunktion für Matrizen]
	\label{def:10.3}
	Für $B \in M(n \times n,\CC)$ definieren wir
	\[
		\exp(B) := \sum\limits_{k=0}^\infty \frac{1}{k!} B^k,
	\]
	mit $B^0 := E$. \index{Exponentialfunktion}
\end{definition}

\begin{bemerkung}
	\label{bem:10.4}
	\mbox{} \\[-1.4cm]
	\begin{enumerate}[(i)]
		\item Konvergenz der Reihe bedeutet, dass jede Komponente $b_{ij}(n)$ der entsprechenden Folge der Partialsummen $B(n) := \sum_{1}^{k!} B^{k} \in M(n \times n,\CC)$ konvergiert.
		Man kann zeigen (wir werden dies später nachholen), dass die Reihe in \autoref{def:10.3} in diesem Sinne immer konvergiert!
		\item Ist $B \in M(n \times n,\RR)$ eine reelle Matrix, so ist nach (i) jede Komponente von $\exp(B)$ ein Grenzwert einer rellen Folge, und damit ist auch $\exp(B) \in M(n \times n,\RR)$.
		\item Ist $A = C^{-1} BC$ für ein $C \in \GL(n,\CC)$, so folgt
		\begin{align*}
			A^{k} &= (C^{-1}BC)\cdot (C^{-1}BC) \cdots (C^{-1}BC) \\
			&= C^{-1} B (CC^{-1}) B (CC^{-1}) \cdots (CC^{-1})BC \\
			&= C^{-1} B E B E \cdots E BC = C^{-1} B^k C
		\end{align*}
		für jedes $k \in \NN_0$, und damit folgt:
		
		\[
			\exp(A) = \sum\limits_{k=0}^{\infty} \frac{1}{k!} A^k = \sum_{k=0}^{\infty} \frac{1}{k!} C^{-1} B^{k} C = C^{-1} \enb{\sum_{k=0}^{\infty} \frac{1}{k!} B^{k}} C = C^{-1} \exp(B) C.
		\]
	\end{enumerate}
\end{bemerkung}

Ähnlich wie im Komplexen erhält man eine überaus nützliche Funktionalgleichung:

\begin{lemma}
	\label{lemma:10.5}
	Seien $A,B \in M(n \times n,\CC)$ mit $AB = BA$.
	Dann gilt $\exp(A+B) = \exp(A) + \exp(B)$.
\end{lemma}

\begin{beweis}[Skizze]
	Sind $A,B \in M(n \times n,K)$ mit $AB = BA$, so folgt mit dem üblichen Induktionsbeweis \todo{komische Referenz} die Binomische Formel
	\[
		(A+B)^{k} = \sum_{j=0}^{k} \binom{k}{j} A^{j} \cdot B^{k-j}
	\]
	für alle $k \in \NN_0$.
	Mit dem Cauchy-Produkt für absolut konvergente Reihen (in einer entsprechenden Version für Matrixreihen) folgt dann
	\begin{align*}
		\exp(A+B) &= \sum_{k=0}^{\infty} \frac{1}{k!} (A+B)^k = \sum_{k=0}^{\infty} \frac{1}{k!} \enb{\sum_{j=0}^{k} A^{j} B^{k-j}} \\
		&= \sum_{k=0}^{\infty} \enb{\sum_{j=0}^{k} \enb{\frac{1}{j!} A^{j}} \cdot \enb{\frac{1}{(k-j)!} B^{k-j}}} = \enb{\sum_{j=0}^{\infty} \frac{1}{j!} A^{j}} \cdot \enb{\sum_{k=0}^{\infty} \frac{1}{k!} B^{k}} \\
		&= \exp(A) \cdot \exp(B) \qedhere
	\end{align*}
\end{beweis}

Ist $B \in M(n \times n,\CC)$, so wollen wir nun die matrixwertige Funktion
\begin{align*}
	Y\colon \RR &\longrightarrow \CC^{n} \\
	t &\longmapsto \exp(tB)
\end{align*}
betrachten.
Wir definieren Differenzierbarkeit von matrixwertigen Funktionen wie für vektorwertige Funktionen durch die Differenzierbarkeit aller Komponenten.
Dies ist wieder äquivalent zur komponentenweisen Existenz des Grenzwertes
\[
	Y'(t) = \lim\limits_{h \rightarrow 0} \frac{1}{h}(Y(t+h)-Y(t)).
\]

\begin{satz}
	\label{satz:10.6}
	Seien $B \in M(n \times n,\CC)$, $Y(t) := \exp(tB)$ und $y_0 \in \CC^{n}$.
	Dann gelten:
	\begin{enumerate}[(i)]
		\item $Y'(t) = B \cdot \exp(tB) = B \cdot Y(t)$.
		\item $y\colon \RR \rightarrow \CC^{n}, y(t) := \exp(tB) \cdot y_0$ ist eine Lösung des Systems $y' = B \cdot y, y(0) = y_0$.
	\end{enumerate}
\end{satz}

\begin{beweis}
	Betrachten wir die Funktion $Y(t) = \exp(tB) = \sum_{k=0}^{\infty} \frac{1}{k!} t^{k} B^{k}$, so stellen wir fest, dass jede Komponente $Y(t)_{ij}$ von $Y(t)$ durch eine Potenzreihe $\sum_{k=0}^{\infty} d_{ij}(k)t^{k}$ gegeben ist, wobei $d_{ij}(k)$ die $i,j$-te Komponente der Matrix $\frac{1}{k!} B^{k}$ ist.
	Da wir (bis jetzt ohne Beweis) davon ausgehen, dass die Exponentialreihe für alle $tB$ konvergiert, haben alle diese Reihen unendlichen Konvergenzradius, und wir dürfen auf ganz $\RR$ gliedweise differenzieren \todo{komische Referenz}.
	Damit folgt:
	
	\[
		Y'(t) = \sum_{k=1}^{\infty} \frac{1}{k!} kt^{k-1} B^{k} = B \cdot \enb{\sum_{k=1}^{\infty} \frac{1}{(k-1)!} t^{k-1} B^{k-1}} = B \cdot \enb{\sum_{k=0}^{\infty} \frac{1}{k!} t^{k} B^{k}} = B \cdot \exp(tB) = B \cdot Y(t).
	\]
	Mit (i) erhalten wir für $y(t) = Y(t) \cdot y_0$:
	\begin{align*}
		y'(t) &= \lim\limits_{h \rightarrow 0} \frac{1}{h} (y(t+h)-y(t)) = \lim\limits_{h \rightarrow 0} \frac{1}{h} (Y(t+h) \cdot y_0 - Y(t) \cdot y_0) \\
		&= \enb{\lim\limits_{h \rightarrow \infty} \frac{1}{h}(Y(t+h)-Y(t))} \cdot y_0 = B \cdot Y(t) \cdot y_0 = B \cdot y(t)
	\end{align*}
	und $y(0) = \exp(0\cdot B) \cdot y_0 = E \cdot y_0 = y_0$. \qedhere
\end{beweis}

\begin{bemerkung}
	\label{bem:10.7}
	\mbox{} \\[-1.4cm]
	\begin{enumerate}[(i)]
		\item Man kann auch hier zeigen, dass die im Satz angegebene Lösung die einzige Lösung des Systems $y' = By$ mit der Anfangsbedingung $y(0) = y_0$ ist.
		Wir werden dies später vielleicht noch tun.
		\item Da $\exp(tB) \in M(n \times n,\RR)$, wenn $B \in M(n \times n,\RR)$ (siehe \autoref{bem:10.4}), folgt:
		Sind $B$ und $y_0$ reell, so ist auch $y(t)$ für alle $t \in \RR$ reell.
	\end{enumerate}
\end{bemerkung}

\begin{anwendung}[Berechnung der Funktion]
	\label{anw:10.8}
	Wir wollen nun überlegen, wie wir die Funktion $Y(t) = \exp(tB)$ für $B \in M(n \times n,\CC)$ mit Hilfe einer Jordan-Normalform von $B$ explizit berechnen können.
	Da jedes Polynom über $\CC$ in Linearfaktoren zerfällt, können wir nach \autoref{satz:9.6} eine Matrix $C \in \GL(n,\CC)$ finden (und im Prinzip auch berechnen), sodass $A = C^{-1}BC$ eine Jordan-Matrix ist.
	Dann folgt $tB = CtAC^{-1}$ und $\exp(tB) = C \exp(tA)C^{-1}$.
	Es genügt also zu wissen, wie die Funktion $\exp(tA)$ für jede Jordan-Matrix $A$ berechnet werden kann.
	
	Sei nun $A = \begin{pmatrix}
		J_1 & & \\
		& \ddots & \\
		& & J_m
	\end{pmatrix}$ eine beliebige Jordan-Matrix mit den Jordan-Kästen $J_1,\dots,J_m$.
	Nach den Rechenregeln für Blockmatrizen gilt dann
	\[
		A^{k} = \begin{pmatrix}
			J_1^{k} & & \\
			& \ddots & \\
			& & J_m^{k}
		\end{pmatrix} \quad \text{und} \quad \exp(tA) = \begin{pmatrix}
			\exp(tJ_1) & & \\
			& \ddots & \\
			& & \exp(tJ_m)
		\end{pmatrix}.
	\]
	Es genügt also zu wissen, wie die Matrix $\exp(tJ)$ zu berechnen ist, wenn $J$ ein Jordan-Kasten zum Eigenwert $\lambda$ ist.
	Dann gilt:
	
	\[
		J = \begin{pmatrix}
		\lambda & 1 & 0 & \cdots & 0 \\ 
		0 & \lambda & 1 & \cdots & 0 \\ 
		\vdots &  & \ddots & \ddots & \vdots \\ 
		0 & \cdots & 0 & \lambda & 1 \\ 
		0 & \cdots & \cdots & 0 & \lambda
		\end{pmatrix} = \lambda E + N \text{ mit } N:= \begin{pmatrix}
		0 & 1 & 0 & \cdots & 0 \\ 
		0 & 0 & 1 & \cdots & 0 \\ 
		\vdots &  & \ddots & \ddots & \vdots \\ 
		0 & \cdots & 0 & 0 & 1 \\ 
		0 & \cdots & \cdots & 0 & 0
		\end{pmatrix}.
	\]
	Da $\lambda E \cdot N = N \cdot \lambda E$, folgt mit \autoref{lemma:10.5}, dass
	\[
		\exp(tJ) = \exp(t\lambda E + tN) = \exp(t \lambda E) \cdot \exp(tN).
	\]
	Wegen $(t\lambda E)^{k} = (t\lambda)^{k}E$ folgt
	\[
		\exp(t\lambda E) = \sum_{k=0}^{\infty} \frac{1}{k!} (t \lambda E)^{k} = \enb{\sum_{k=0}^{\infty} \frac{1}{k!} (t\lambda)^{k}} E = e^{t\lambda} E.
	\]
	Eine leichte Rechnung (benutze \autoref{lemma:9.9}) ergibt ferner, dass $N^{l} = (0, \dots, 0,e_1,\dots,e_{k-l})$, wobei die ersten $l$ Einträge aus Nullspalten bestehen und dann die ersten $k-l$ Einheitsvektoren des $\CC^{k}$ als weitere Spalten der Matrix $N^{l}$ auftreten.
	Insbesondere gilt $N^{k} = 0$.
	Einsetzen in die Exponentialreihe ergibt (mit $N^0 = E$):
	\[
		\exp(tN) =
		\begin{pmatrix}
			1 & t      & \frac{t^2}{2} & \frac{t^3}{3!} & \cdots         & \frac{t^{k-1}}{(k-1)!} & \frac{t^{k}}{k!}       \\
			0 & 1      & t             & \frac{t^2}{2}  & \frac{t^3}{3!} & \cdots                 & \frac{t^{k-1}}{(k-1)!} \\
			  & \ddots & \ddots        & \ddots         &                &                        &  \\
			  &        & \ddots        & \ddots         & \ddots         &                        &  \\
			  &        &               & 0              & 1              & t                      & \frac{t^2}{2}          \\
			  &        &               &                & 0              & 1                      & t                      \\
			  &        &               &                &                & 0                      & 1
		\end{pmatrix}.
	\]
	Zusammen folgt
	\[
		\exp(tJ) = e^{t\lambda} \exp(tN) =
		\begin{pmatrix}
			e^{t\lambda} & te^{t\lambda} & \frac{t^2e^{t\lambda}}{2} & \frac{t^3e^{t\lambda}}{3!} & \cdots                     & \frac{t^{k-1}e^{t\lambda}}{(k-1)!} & \frac{t^{k}e^{t\lambda}}{k!}       \\
			0            & e^{t\lambda}  & te^{t\lambda}             & \frac{t^2e^{t\lambda}}{2}  & \frac{t^3e^{t\lambda}}{3!} & \cdots                             & \frac{t^{k-1}e^{t\lambda}}{(k-1)!} \\
			             & \ddots        & \ddots                    & \ddots                     &                            &                                    &  \\
			             &               & \ddots                    & \ddots                     & \ddots                     &                                    &  \\
			             &               &                           & 0                          & e^{t\lambda}               & te^{t\lambda}                      & \frac{t^2e^{t\lambda}}{2}          \\
			             &               &                           &                            & 0                          & e^{t\lambda}                       & te^{t\lambda}                      \\
			             &               &                           &                            &                            & 0                                  & e^{t\lambda}
		\end{pmatrix}
	\]
\end{anwendung}
\newpage
\begin{beispiel}
	\label{bsp:10.9}
	Sei $B := \begin{pmatrix}
		0 & 1 & 0 & 1 & 1 \\
		0 & 0 & 2 & 0 & 1 \\
		0 & 0 & 0 & 2 & 1 \\
		0 & 0 & 0 & 2 & 2 \\
		0 & 0 & 0 & 0 & 2
	\end{pmatrix}$.
	Dann ist $C := \begin{pmatrix}
	2 & 0 & 0 & 4 & -1 \\ 
	0 & 2 & 0 & 4 & -1 \\ 
	0 & 0 & 1 & 4 & 0 \\ 
	0 & 0 & 0 & 4 & 1 \\ 
	0 & 0 & 0 & 0 & 2
	\end{pmatrix}$ eine Transformationsmatrix mit
	
	\[
		A = C^{-1}BC = \enb{\begin{BMAT}(e)[4.3pt]{ccccc}{ccccc}
		0 & 1 & 0 & 0 & 0 \\ 
		0 & 0 & 1 & 0 & 0 \\ 
		0 & 0 & 0 & 0 & 0 \\ 
		0 & 0 & 0 & 2 & 1 \\ 
		0 & 0 & 0 & 0 & 2
		\addpath{(0,2,|)uuurrrdddlll}
		\addpath{(3,0,|)uurrddll}
		\end{BMAT}}. 
	\]
	Damit folgt $\exp(tB) = C \exp(tA)C^{-1}$ mit
	\[
		\exp(tA) = \begin{pmatrix}
			1 & t & \frac{t^2}{2} & 0      & 0       \\
			0 & 1 & t             & 0      & 0       \\
			0 & 0 & 1             & 0      & 0       \\
			0 & 0 & 0             & e^{2t} & te^{2t} \\
			0 & 0 & 0             & 0      & e^{2t}
		\end{pmatrix}.
	\]
	Nachrechnen ergibt:
	\[
		Y(t) := \exp(tB) = \begin{pmatrix}
		1 & t & t^2 & -1-t-t^2+e^{2t} & 1+t+\frac{t^2}{2} -e^{2t} + 2te^{2t} \\ 
		0 & 1 & 2t & -1-2t+e^{2t} & 1+t-e^{2t}+2te^{2t} \\ 
		0 & 0 & 1 & -1+e^{2t} & \frac{1}{2} - \frac{e^{2t}}{2} + 2te^{2t} \\ 
		0 & 0 & 0 & e^{2t} & 2te^{2t} \\ 
		0 & 0 & 0 & 0 & e^{2t}
		\end{pmatrix}. 
	\]
	Für das System $y' = By, y(0) = (1,0,2,0,1)^T$ ergibt sich die eindeutig bestimmte Lösung
	\[
		y(t) = Y(t) y_0 = \begin{pmatrix}
		2+t+\frac{5t^2}{2} - e^{2t} + 2te^{2t} \\ 
		1+3t-e^{2t}+2te^{2t} \\ 
		\frac{5}{2} - \frac{e^{2t}}{2} + 2te^{2t} \\ 
		2te^{2t} \\ 
		e^{2t}
		\end{pmatrix}.
	\]
\end{beispiel}
\newpage