\section{Gruppen, Ringe, Körper}
\begin{definition}[Gruppe]
	\label{def:I.4.1}
	\mbox{} \\[-1.4cm]
	\begin{enumerate}[(i)]
		\item Eine \Index{Gruppe} $(G,*)$ besteht aus einer Menge $G$ zusammen mit einer Verknüpfung $*\colon G \times G \rightarrow G, (x,y) \mapsto x*y$ mit folgenden Eigenschaften:
		\begin{enumerate}[a)]
			\item $*$ ist \textbf{assoziativ}, das heißt es gilt $(x*y)*z = x * (y*z)$ für alle $x,y,z \in G$. \index{Assoziativität}
			\item Es existiert ein \textbf{neutrales Element} $1 \in G$ für $*$, das heißt für alle $x \in G$ gilt $1 * x = x * 1 = x$. \index{Neutrales Element}
			\item Zu jedem $x \in G$ existiert ein \textbf{inverses Element} $x^{-1} \in G$ mit $x * x^{-1} = x^{-1} * x = 1$. \index{Inverses Element}
		\end{enumerate}
		\item Eine Gruppe $(G,*)$ heißt \textbf{abelsche Gruppe}, falls zusätzlich das Kommutativgesetz gilt, das heißt für alle $x,y \in G$ gilt $x*y = y*x$. \index{Gruppe!abelsch}
	\end{enumerate}
\end{definition}

\setcounter{definition}{3}
\begin{beispiel}[Symmetrische Gruppe]
	\label{bsp:I.4.4}
	Sei $X$ eine Menge. Die Gruppe $S(X) := \{f \colon X \rightarrow X : f \text{ ist bijektiv}\}$ mit der Komposition $\circ$ als Verknüpfung heißt \textbf{symmetrische Gruppe} von $X$.
	Besitzt $X$ mehr als zwei Elemente, so ist $(S(X),\circ)$ nicht abelsch. Für $X = \{1,2,\dots,n\}$ schreiben wir $S_n := S(X)$. \index{symmetrische Gruppe}
\end{beispiel}

\begin{definition}[Ring]
	\label{def:I.4.5}
	Ein \Index{Ring} $(R,+,\cdot)$ besteht aus einer abelschen Gruppe $(R,+)$ zusammen mit einer zusätzlichen Verknäpfung $\cdot \colon R \times R \rightarrow R, (x,y) \mapsto x \cdot y$, sodass Folgendes gilt:
	\begin{enumerate}[(i)]
		\item Ist $0 \in R$ das neutrale Element für $(R,+)$, so gilt $0 \cdot x = x \cdot 0 = 0$ für alle $x \in R$.
		\item Die Verknüpfung $\cdot$ ist assoziativ.
		\item Es gelten die \textbf{Distributivgesetze}, das heißt für alle $x,y,z \in R$ gilt $(x+y) \cdot z = x\cdot z + y \cdot z$ und $x\cdot(y+z) = x \cdot y + x \cdot z$. \index{Distributivität}
	\end{enumerate}
	Ist $\cdot$ kommutativ, so heißt $R$ ein \text{kommutativer Ring}.
	Existiert zusätzlich ein neutrales Element $1 \in R$ für $\cdot$, so heißt $R$ ein \textbf{unitaler Ring} oder \textbf{Ring mit Eins}.
	Die Menge $R^* := \{x \in R : \text{es existiert ein } x^{-1} \in R \text{ mit } x \cdot x^{-1} = x^{-1} \cdot x = 1\}$ heißt dann die \Index{Einheitengruppe} von $R$. \index{Ring!unital} \index{Ring!mit Eins} \index{Ring!kommutativ}
\end{definition}

\setcounter{definition}{6}
\begin{definition}[Schiefkörper, Körper]
	\label{def:I.4.7}
	Sei $R$ ein Ring mit Eins.
	Gilt $R^* = R \setminus \setzero$, so heißt $R$ ein \Index{Schiefkörper}.
	Ist $R$ zusätzlich kommutativ, so heißt $R$ ein \Index{Körper}.
\end{definition}

\setcounter{definition}{9}
\begin{definition}[Komplexe Zahlen]
	\label{def:I.4.10}
	Sei $\CC$ die Menge der formalen Summen der Gestalt $x+iy$ mit $x,y \in \RR$, also
	\[
		\CC:= \{x+iy : x,y \in \RR\}.
	\]
	\newpage
	Vermöge $i^2 := -1$ und der Verknüpfungen
	\begin{align*}
		(a+ib) + (c+id) &:= (a+c) + i(b+d) \\
		(a+ib) \cdot (c+id) &:= (ac-bd) + i(bc+ad)
	\end{align*}
	ist $(\CC,+,\cdot)$ ein Körper -- der Körper der \textbf{komplexen Zahlen}. \index{komplexe Zahl}
	Wir definieren weiter für $z = a+ib \in \CC$:
	\begin{enumerate}[a)]
		\item $\Re(z) := a$ -- der \Index{Realteil} von $z$.
		\item $\Im(z) := b$ -- der \Index{Imaginärteil} von $z$.
		\item $\ol{z} := a-ib$ -- das \textbf{komplex Konjugierte} von $z$. \index{komplex Konjugiertes}
		\item $\abs{z} := \sqrt{a^2+b^2}$ -- der \Index{Betrag} von $z$.
	\end{enumerate}
\end{definition}

\setcounter{definition}{11}
\begin{lemma}
	\label{def:I.4.12}
	Für alle $z = a+ib, w = c+id \in \CC$ gilt:
	\begin{enumerate}[(i)]
		\item $\abs{z}^2 = z \ol{z}$ und $\abs{zw} = \abs{z} \cdot \abs{w}$.
		\item Ist $z \neq 0$, so auch $\abs{z}$, und es gilt $z^{-1} = \frac{1}{\abs{z}^2} \ol{z}$.
	\end{enumerate}
\end{lemma}

\begin{bemerkung}[Polardarstellung komplexer Zahlen]
	\label{bem:I.4.13}
	Ist $z = a+ib \in \CC$, so existiert ein $\varphi \in [0,2\pi]$ mit $\cos(\varphi) = \frac{a}{\abs{z}}$ und $\sin(\varphi) = \frac{b}{\abs{z}}$, das heißt es gilt
	\[
		z = \abs{z} \cdot (\cos \varphi + i \cdot \sin \varphi).
	\]
	\begin{center}
			\begin{tikzpicture}[scale=1,>=Latex]
			\draw [very thick,->] (-1.2,0) -- (3,0);
			\draw [very thick,->] (0,-1.2) -- (0,2);
			\draw [color=darkgray] (0,0) circle (1);
			\draw (.1,1.5) -- (-.1,1.5) node[left]{$b$};
			\draw (.1,1) -- (-.1,1) node[left]{$1$};
			
			\draw (1,.1) -- (1,-.1) node[below]{$1$};
			\draw (2,.1) -- (2,-.1) node[below]{$a$};
			
			\coordinate (N) at (0,0);
			\coordinate (Z) at (2,1.5);
			\coordinate (A) at (2,0);
			
			\draw [dashed] (A) -- (Z) node[color=red, anchor=south west]{$z=a+ib$};
			\draw [thick,color=red] (N) -- (Z) node[fill,circle,inner sep=1.5pt]{};
			\draw pic["$\varphi$",draw=black,angle eccentricity=.7,angle radius=.8cm]{angle=A--N--Z};
			\draw pic["$\bullet$",draw=black,angle eccentricity=.5,angle radius=.6cm]{angle=Z--A--N};
		\end{tikzpicture}
	\end{center}
\end{bemerkung}
\newpage