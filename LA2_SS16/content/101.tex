%!TEX root = ../LA2_SS16.tex
\section{Lineare Algebra I (Zusammenfassung)} % (fold)
\label{cha:1}
\subsection{Aussagenlogik und vollständige Induktion}
\label{sec:1.1}
	Das Herz der Mathematik bilden Beweise:
	Jede Aussage, jeder Lehrsatz muss \enquote{bewiesen} werden, das heißt durch logische Verknüpfungen aus einigen Grundaxiomen hergeleitet werden.
	Leider ist eine Darstellung des Axiomensystems sehr aufwendig und an dieser Stelle nicht durchführbar (mehr dazu in der Vorlesung Logische Grundlagen).
	Daher werden wir uns hier auf Aussagen stützen, die etwa aus dem Schulunterricht bekannt sind, und werden diese, wenn nötig, ergänzen.
	
	Frage: Was ist ein mathematischer Beweis?
	
	Sei $B$ eine mathematische Aussage (wie etwa $\sqrt{2} > 1$ oder $\sqrt{2} \in \QQ = \penb{\frac{m}{l} : m \in \ZZ, n \in \NN}$).
	Wir wollen beweisen, dass $B$ wahr ist.
	Zum Beweis von $B$ finde eine bereits als wahr bekannte Aussage $A$ und zeige $A \Rightarrow B$ (aus $A$ folgt $B$).
	Oft gelingt dies nicht in einem Schritt, sondern wir zeigen eine Kette von Folgerungen:
	\[
		A_1 \Rightarrow A_2 \Rightarrow A_3 \Rightarrow \cdots \Rightarrow A_l \Rightarrow B
	\]
	Die Aussagen $A_1,\dots,A_l$ sind hierbei in der Regel logische Verknüpfungen einer Vielzahl anderer Aussagen.
	
	Wichtig: Eine mathematische Aussage ist entweder wahr oder unwahr -- niemals beides!
	
\begin{definition}[Logische Verknüfung von Aussagen]
	\label{def:I.1.1}
	Es seien $A,B$ mathematische Aussagen. \index{logische Verknüpfung}
	Wir definieren:
	\begin{enumerate}[(i)]
		\item Und: $A \wedge B$ ist genau dann wahr, wenn beide Aussagen $A$ und $B$ wahr sind.
		\item Oder: $A \vee B$ ist genau dann wahr, wenn mindestens eine der Aussagen $A,B$ wahr ist.
		\item Nicht: $\neg A$ ist genau dann wahr, wenn $A$ nicht wahr ist.
		\item Folgt: $A \Rightarrow B$ ist genau dann wahr, wenn eine der Aussagen $A \wedge B, \neg A$ wahr ist (also genau dann, wenn $(A \wedge B) \vee \neg A$ wahr ist).
		\item Äquivalent: $A \Leftrightarrow B$ ist genau dann wahr, wenn $A \wedge B$ wahr oder $\neg A \wedge \neg B$ wahr ist (also genau dann, wenn beide Aussagen wahr oder beide Aussagen unwahr sind).
	\end{enumerate}
\end{definition}

Es gilt: $[A \Leftrightarrow B] \Leftrightarrow [(A \Rightarrow B) \wedge (B \Rightarrow A)] \Leftrightarrow [(A \wedge B) \vee (\neg A \wedge \neg B)]$.
Die Verknüpfungen $\wedge,\vee,\neg,\Rightarrow,\Leftrightarrow$ lassen sich gut in einer \Index{Wahrheitstafel} veranschaulichen:

\begin{center}
	\begin{tabular}{ccccccc}
	$\bm{A}$ & $\bm{B}$ & $\bm{A \wedge B}$ & $\bm{A \vee B}$ & $\bm{A \Rightarrow B}$ & $\bm{A \Leftrightarrow B}$ & $\bm{\neg A}$ \\ 
	\hline w & w & w & w & w & w & f \\ 
	\hline w & f & f & w & f & f & f \\ 
	\hline f & w & f & w & w & f & w \\ 
	\hline f & f & f & f & w & w & w \\ 
\end{tabular} 
\end{center}

Wir werden später fast nie die Zeichen $\wedge, \vee, \neg$ benutzen, sondern die entsprechenden Worte \enquote{und}, \enquote{oder}, \enquote{nicht}.
Wir benötigen auch Versionen von \enquote{und} und \enquote{oder} für Systeme von Aussagen:

\begin{definition}[Allquantor, Existenzquantor]
	\label{def:I.1.2}
	Sei $\{A_i : i \in I\}$ ein System von Aussagen ($I$ ist beliebige Indexmenge).
	Wir setzen: \index{Allquantor} \index{Existenzquantor}
	\begin{enumerate}[(i)]
		\item Für alle: Die Aussage $[\forall i \in I \text{ gilt } A_i]$ ist genau dann wahr, wenn alle Aussagen $A_i$ wahr sind.
		\item Es gibt: Die Aussage $[\exists i \in I \text{ mit } A_i]$ ist genau dann wahr, wenn mindestens eine der Aussagen $A_i$ wahr ist.
	\end{enumerate}
\end{definition}

Beachte: $\forall$ ist eine Form von $\wedge$ für \enquote{viele} Aussagen und $\exists$ ist eine Form von $\vee$ für viele Aussagen.
Manchmal findet man auch die Quantoren
\[
	\bigwedge_{i \in I} A_i \ (\Leftrightarrow \ \forall i \in I \text{ gilt } A_i) \qquad \text{bzw.} \qquad \bigvee_{i \in I} A_i \ (\Leftrightarrow \ \exists i \in I \text{ mit } A_i).
\]
Ich benutze diese aber eher selten!

Allein aus den Definitionen von $\wedge$, $\vee$, etc. kann man schon neue wahre Aussagen herleiten, zum Beispiel:
\[
	A \wedge B \Rightarrow A \vee B \qquad \text{und} \qquad (A \Leftrightarrow B) \Rightarrow (A \rightarrow B).
\]
Ebenso gelten
\[
	A \wedge (B \vee c) \Leftrightarrow (A \wedge B) \vee (B \wedge C) \quad \text{und} \quad A \vee (B \wedge C) \Leftrightarrow (A \vee B) \wedge (A \vee C)
\]
Beachte: $\wedge, \vee, \neg$ haben Vorrang vor $\Rightarrow$ und $\Leftrightarrow$ (ähnlich wie Punktrechnung vor Strichrechnung)
Wir überprüfen zum Beispiel die erste Aussage mit Hilfe einer Wahrheitstafel:

\begin{center}
	\begin{tabular}{cccccccc}
	$A$ & $B$ & $C$ & $A \wedge B$ & $A \wedge C$ & $B \wedge C$ & $A \wedge (B \vee C)$ & $(A \wedge B) \vee (A \wedge C)$ \\ 
	\hline 
	w & w & w & w & w & w & w & w \\ 
	\hline 
	w & w & f & w & f & w & w & w \\ 
	\hline 
	w & f & w & f & w & w & w & w \\ 
	\hline 
	w & f & f & f & f & f & f & f \\ 
	\hline 
	f & w & w & f & f & w & f & f \\ 
	\hline 
	f & w & f & f & f & w & f & f \\ 
	\hline 
	f & f & w & f & f & w & f & f \\ 
	\hline 
	f & f & f & f & f & f & f & f \\ 
\end{tabular} 
\end{center}

Aus den letzten beiden Spalten folgt dann, dass $A \wedge (B \vee C)$ genau dann wahr ist, wenn $(A \wedge B) \vee (A \wedge C)$ wahr ist. 
Die Aussage $A \vee (B \wedge C) \Leftrightarrow (A \vee B) \wedge (A \vee C)$ prüft man analog.
Im Allgemeinen sind aber die zu beweisenden Aussagen wesentlich komplexer und nicht oder nur schwer mit Wahrheitstafeln zu beweisen!

\begin{beispiel}
	\label{bsp:I.1.3}
	Sei $\NN:= \{1,2,3,\dots\}$ die Menge der natürlichen Zahlen.
	Für $n,m \in \NN$ sagen wir $n > m$, falls ein $k \in \NN$ existiert mit $n = m+k$.
	Wir behaupten nun die folgende Aussage:
	Sind $n,m,l,r \in \NN$, so gelten:
	\begin{enumerate}[(i)]
		\item Gilt $n > l$ und $m > r$, so folgt $n+m > l+r$ \\
		$[\forall n,m,l,r \in \NN \text{ gilt: } (n > l) \wedge (m > r) \Rightarrow n+m > l+r]$
		\item Gilt $n > l$ und $m > r$, so folgt $n \cdot m > l \cdot r$ \\
		$[\forall n,m,l,r \in \NN \text{ gilt: } (n > l) \wedge (m > r) \Rightarrow n \cdot m > l \cdot r]$
	\end{enumerate}
\end{beispiel}

\begin{beweis}
	\begin{enumerate}[(i)]
		\item Da $n > l$ und $m > r$, folgt nach Definition von \enquote{$>$}, dass $k_1, k_2 \in \NN$ existieren mit $n = l+k_1, m = r+k_2$.
		Aus den Rechenregeln in $\NN$ folgt hieraus mit $k = k_1 + k_2$:
		\[
			n+m = (l+k_1) + (r+k_2) = (l+r)+(k_1+k_2) = (l+r) + k.
		\]
		Nach Definition von \enquote{$>$} folgt also $n+m > l+r$.
		\item Übungsaufgabe. 
	\end{enumerate}
\end{beweis}

Frage: Auf welche \enquote{wahren Aussagen} haben wir uns im obigen Beweis gestützt?

\begin{satz}[Methode des indirekten Beweises]
	\label{satz:I.1.4}
	Sei wieder $B$ eine Aussage, die wir beweisen wollen.
	Beim indirekten Beweis suchen wir eine unwahre Aussage $A$ und zeigen, dass $\neg B \Rightarrow A$ wahr ist.
	Wenn dies gelingt, so ist $B$ eine wahre Aussage, denn wir haben die Äquivalenz
	
	\begin{equation}
		(\neg B \Rightarrow A) \Leftrightarrow (\neg A \Rightarrow B) \label{eq:I.1.4}
	\end{equation}
	
	und nach Voraussetzung ist $\neg A$ wahr.
\end{satz}

\begin{beweis}
	Wir beweisen \eqref{eq:I.1.4} durch Wahrheitstafel:
	\begin{center}
		\begin{tabular}{cccccc}
			$A$ & $B$ & $\neg A$ & $\neg B$ & $\neg B \Rightarrow A$ & $\neg A \Rightarrow B$ \\ 
			\hline 
			w & w & f & f & w & w \\ 
			\hline 
			w & f & f & w & w & w \\ 
			\hline 
			f & w & w & f & w & w \\ 
			\hline 
			f & f & w & w & f & f \\ 
		\end{tabular} \\
		
	\end{center}
\end{beweis}

\begin{beispiel}
	\label{bsp:I.1.5}
	Es gilt die Aussage $\sqrt{2} \notin \QQ$, das heißt, $\sqrt{2}$ lässt sich nicht als Bruch $\frac{n}{m}$ schreiben mit $n \in \ZZ, m \in \NN$.
\end{beispiel}

\begin{beweis}
	Sei $B$ die Aussage $\sqrt{2} \notin \QQ$.
	Wir zeigen, dass aus $\neg B$ eine absurde Aussage folgt.
	Dann ist $B$ wahr.
	Wir nehmen an, dass $\neg B$ wahr ist, also $\sqrt{2} \in \QQ$.
	Dann existiert ein $n \in \ZZ$ und $m \in \NN$ mit $\sqrt{2} = \frac{n}{m}$.
	Durch Kürzen des Bruchs $\frac{n}{m}$ können wir erreichen, dass $n$ oder $m$ ungerade sind (genau genommen müsste man dies getrennt beweisen).
	Dann folgt $2 = (\sqrt{2})^2 = \enb{\frac{n}{m}}^2 = \frac{n^2}{m^2}$, also $2m^2 = n^2$.
	Insbesondere folgt: $n^2$ ist eine gerade Zahl.
	Dann ist auch $n$ eine gerade Zahl (sonst wäre $n^2$ als Produkt ungerader Zahlen ungerade).
	Damit existiert ein $l \in \NN$ mit $n = 2l$.
	Es folgt $m$ ungerade und $2 = \frac{n^2}{m^2} = \frac{4l^2}{m^2}$, also $m^2 = 2l^2$.
	Wie oben folgt, dass auch $m$ gerade ist.
	Wir haben nun gezeigt:
	\[
		\sqrt{2} \in \QQ \Rightarrow \Underbrace{[\exists m \in \NN \text{ mit } m \text{ gerade } \wedge m \text{ ungerade}]}{\text{unwahre Aussage!}} 
	\]
\end{beweis}

Für die Durchführung indirekter Beweise ist es wichtig, Aussagen korrekt zu verneinen.
Hier einige Grundregeln:

\begin{lemma}
	\label{lemma:I.1.6}
	Es gelten:
	\begin{enumerate}[(i)]
		\item $\neg(A \wedge B) \Leftrightarrow \neg A \vee \neg B$
		\item $\neg(A \vee B) \Leftrightarrow \neg A \wedge \neg B$
		\item $\neg(\forall i \in I \text{ gilt } A_i) \Leftrightarrow \exists i \in I \text{ mit } \neg A_i$
		\item $\neg(\exists i \in I \text{ mit } A_i) \Leftrightarrow \forall i \in I \text{ gilt } \neg A_i$
	\end{enumerate}
\end{lemma}

\begin{beispiel}
	\label{bsp:I.1.7}
	\begin{enumerate}[(i)]
		\item Die Verneinung der Aussage \enquote{Franz ist mutig und reich} ist \enquote{Frank ist nicht mutig oder nicht reich}.
		\item Die Verneinung von \enquote{Alle Menschen sind dumm} wäre \enquote{Es gibt einen Menschen, der nicht dumm ist}.
	\end{enumerate}
\end{beispiel}

Wir kommen nun zu einem wichtigen Beweisprinzip, das es un in guten Fällen erlaubt, unendlich viele Aussagen gleichzeitig zu beweisen. Wir setzen voraus, dass die natürlichen Zahlen $\NN = \{1,2,3,\dots\}$ und $\NN_0 = \{0,1,2,\dots\}$ und die Rechenregeln für die Addition auf $\NN$ bekannt sind.
Wichtigste Eigenschaft von $\NN$ ist, dass für jede Zahl $n \in \NN$ genau ein Nachfolger $n+1 \in \NN$ existiert, und wenn wir mit $1$ starten, so durchlaufen wir mit $1,2,3,\dots, n, n+1, \dots$ jede natürliche Zahl genau ein mal! (\Index{Peano-Axiom})

\begin{satz}[Prinzip der vollständigen Induktion]
	\label{satz:I.1.8}
	Für alle $n \in \NN$ sei $A_n$ eine Aussage. \index{vollständige Induktion}
	Ferner gelte:
	\begin{description}
		\item[I.A.:] $A_1$ ist wahr (Induktionsanfang)
		\item[I.S.:] Für jedes $n \in \NN$ gilt: $A_n \Rightarrow A_{n+1}$ (Induktionsschluss).
	\end{description}
	Dann ist jede Aussage $A_n$ wahr.
\end{satz}

Die Idee ist natürlich klar:
Um zu sehen, dass $A_n$ wahr ist, betrachten wir die Schlusskette
\[
	A_1 \Rightarrow A_2 \Rightarrow A_3 \Rightarrow \dots \Rightarrow A_{n-1} \Rightarrow A_n.
\]
Nach Induktionsanfang ist $A_1$ wahr und nach Induktionsschluss ist jede Folgerung $A_j \Rightarrow A_{j+1}$ in der Kette wahr.
Damit ist auch $A_n$ wahr.

Alternatives Argument:
Angenommen, es existiert ein $n \in \NN$ mit $A_n$ nicht wahr.
Dann existiert ein kleinstes $j \in \{1,\dots,n\}$ mit $A_j$ nicht wahr.
Dann ist $j > 1 $ und $A_{j-1}$ ist wahr.
Da $A_{j-1} \Rightarrow A_j$ gilt, ist auch $A_j$ wahr.
Dann gilt $\neg A_j \wedge A_j$, was nicht geht.

\begin{beispiel}[Gaußsche Summenformel]
	\label{bsp:I.1.9}
	Sind $a_1,\dots,a_n$ Zahlen, so setzen wir
	\[
		\sum_{i=1}^{n} a_i = a_1 + a_2 + \dots a_l.
	\]
	Mit dieser Notation gilt für alle $n \in \NN$ und $a_i = i$ die folgende Aussage $A_n$:
	\[
		\sum_{i=1}^{n} = \frac{n(n+1)}{2}
	\]
\end{beispiel}

\begin{beweis}
	\begin{description}
		\item[I.A.:] Sei $n=1$.
		Dann gilt $\sum_{i=1}^{1} = 1 = \frac{1(1+1)}{2}$, also ist $A_1$ wahr.
		\item[I.S.:] Sei $n \in \NN$ fest gewählt und $A_n$ wahr.
		Dann folgt
		\begin{align*}
			\sum_{i=1}^{n+1} i &= \sum_{i=1}^{n} + (n+1) \stackrel{A_n}{=} \frac{n(n+1)}{2} + n+1 = \frac{n(n+1)+2(n+1)}{2} \\
			&= \frac{n^2 +n +2n +2}{2} = \frac{(n+1)(n+2)}{2} = \frac{(n+1)((n+1)+1)}{2},
		\end{align*}
		und damit ist dann auch die Aussage $A_{n+1}$ wahr, das heißt $A_n \Rightarrow A_{n+1}$ ist bewiesen. 
	\end{description}
\end{beweis}

Die Annahme \enquote{$A_n$ ist wahr} im Induktionsschritt nennt man auch Induktionsannahme.
Es ist nützlich, im Beweis stets zu kennzeichnen, wo die Annahme eingeht!

\begin{bemerkung}
	\label{bem:I.1.10}
	Das Induktionsprinzip funktioniert auch, wenn wir ein $n_0 \in \ZZ$ und Aussagen $A_{n_0}, A_{n_0+1}, A_{n_0+2},\dots$ haben.
	Dann müssen wir zeigen:
	\begin{description}
		\item[I.A.:] $A_{n_0}$ ist wahr.
		\item[I.S.:] Für alle $n \geq n_0$ gilt $A_n \Rightarrow A_{n+1}$.
	\end{description}
	Wir erhalten dann die Schlusskette $A_{n_0} \Rightarrow A_{n_0+1} \Rightarrow A_{n_0+2} \Rightarrow \dots \Rightarrow A_{n-1} \Rightarrow A_n$.	
	Ferner kann auch der Induktionsschluss
	
	\[
		\text{Für alle } n \geq n_0 \text{ gilt } A_{n_0} \wedge \dots \wedge A_n \Rightarrow A_{n+1}
	\]
	
	gesetzt werden.
	Wie sieht dann die Schlusskette für $A_n$ aus?
\end{bemerkung}

\begin{beispiel}
	\label{bsp:I.1.11}
	Das Induktionsprinzip wird oft auch für rekursive Definitionen genutzt, um zum Beispiel so etwas wie $x^n := x \cdot x \cdots x$ zu vermeiden!
	
	Beispiel: Wir definieren $x^0 := 1$, und ist $x^n$ bereits definiert für $n \geq 0$, so definieren wir $x^{n+1} := (x^n) \cdot x$.
	
	Wir definieren $0! = 1$, und ist $n!$ für $n \in \NN_0$ bereits definiert, so setzen wir $(n+1)! := n! \cdot (n+1)$.
	Damit ist $n! = 1 \cdot 2 \cdot 3 \cdot \dots \cdot n$.
\end{beispiel}

\begin{definition}[Binomialkoeffizient]
	\label{def:I.1.12}
	Für $k,n \in \NN_0$ setze
	\[
		\binom{n}{k} := \begin{cases}
			\frac{n!}{k!(n-k)!}, & \text{ falls } k \leq n \\
			0, & \text{ falls } k > n.
		\end{cases}
	\]
	Die Zahlen $\binom{n}{k}$ heißen \textbf{Binomialkoeffizienten}. \index{Binomialkoeffizient}
\end{definition}

\begin{lemma}
	\label{lemma:I.1.13}
	Für alle $1 \leq k \leq n$ gilt
	\[
		\binom{n}{k} = \binom{n-1}{k-1} + \binom{n-1}{k}.
	\]
	Das bedeutet, dass die Binomialkoeffizienten im so genannten Pascalschen Dreieck angeordnet werden können. \index{Pascalsches Dreieck}
	
	\begin{center}
			\begin{tabular}{ccccccccc}
			               &                &                &                & $\binom{0}{0}$ &                &                &                &  \\
			               &                &                & $\binom{1}{0}$ &                & $\binom{1}{1}$ &                &                &  \\
			               &                & $\binom{2}{0}$ &                & $\binom{2}{1}$ &                & $\binom{2}{2}$ &                &  \\
			               & $\binom{3}{0}$ &                & $\binom{3}{1}$ &                & $\binom{3}{2}$ &                & $\binom{3}{3}$ &  \\
			$\binom{4}{0}$ &                & $\binom{4}{1}$ &                & $\binom{4}{2}$ &                & $\binom{4}{3}$ &                & $\binom{4}{4}$
		\end{tabular} \hspace{1cm}
		\begin{tabular}{ccccccccc}
			  &   &   &   & 1 &   &   &   &  \\
			  &   &   & 1 &   & 1 &   &   &  \\
			  &   & 1 &   & 2 &   & 1 &   &  \\
			  & 1 &   & 3 &   & 3 &   & 1 &  \\
			1 &   & 4 &   & 6 &   & 4 &   & 1
		\end{tabular} 
	\end{center}
	
\end{lemma}

\begin{beweis}
	Ist $k = n$, so gilt
	\[
		\binom{n}{n} = \frac{n!}{n! 0!} = \frac{n!}{n!} = 1 \quad \text{und} \quad \binom{n-1}{n-1} + \binom{n-1}{n} = 1 + 0 = 1.
	\]
	In diesem Fall gilt die Formel also.
	
	Sei nun $1 \leq k < n$.
	Dann gilt
	\begin{align*}
		\binom{n-1}{k-1} + \binom{n-1}{k} &= \frac{(n-1)!}{(k-1)!(n-k)!} + \frac{(n-1)!}{k!(n-1-k)!} \\
		&= \frac{k(n-1)! + (n-k)(n-1)!}{k!(n-k)!} = \frac{n(n-1)!}{k!(n-k)!} = \frac{n!}{k!(n-k)!} = \binom{n}{k}. 
	\end{align*}
\end{beweis}

Frage: War dies ein Induktionsbeweis?

\begin{satz}[Binomische Formel]
	\label{satz:1.14}
	Für alle $x,y \in \RR$ und $n \in \NN_0$ gilt
	\[
		(x+y)^n = \sum_{k=0}^{n} \binom{n}{k} x^k y^{n-k}.
	\]
\end{satz}

\begin{beweis}[(vollständige Induktion)]
	\begin{description}
		\item[I.A.:] Sei $n=0$.
		Dann gilt $(x+y)^0 = 1$ und $\sum_{k=0}^{0} \binom{0}{k} x^k y^{0-k} = \binom{0}{0} x^0 y^0 = 1$.
		\item[I.S.:] Die Formel sei wahr für ein $n \geq 0$.
		Dann folgt
		\begin{align*}
			(x+y)^{n+1} &\stack{}{=} (x+y)^n (x+y) \stackrel{\text{Ann.}}{=} \enb{\sum_{k=0}^{n} \binom{n}{k} x^k y^{n-k}}(x+y) \\
			&\stack{}{=} \sum_{k=0}^{n} \binom{n}{k} x^{k+1}y^{n-k} + \sum_{k=0}^{n} x^k y^{n+1-k} \\
			&\stack{}{=} \sum_{k=1}^{n+1} \binom{n}{k-1} x^{k}y^{n-(k-1)} + \sum_{k=0}^{n} x^k y^{n+1-k} \\
			&\stack{}{=} \binom{n}{n} x^{n+1} y^0 + \sum_{k=1}^{n} \benb{\binom{n}{k-1} + \binom{n}{k}} x^k y^{n+1-k} + \binom{n}{0} x^0 y^{n+1} \\
			&\stack{\text{\ref{lemma:I.1.13}}}{=} \binom{n+1}{n+1} x^{n+1} y^0 + \sum_{k=1}^{n} \binom{n+1}{k} x^k y^{n+1-k} + \binom{n+1}{0} x^0 y^{n+1} \\
			&\stack{}{=} \sum_{k=0}^{n+1} \binom{n+1}{k} x^k y^{n+1-k},
		\end{align*}
		also gilt die Formel für $n+1$. 
	\end{description}
	
\end{beweis}

\cleardoubleoddemptypage