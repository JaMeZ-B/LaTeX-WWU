%!TEX root = ../LA2.tex
\section{Quotientenräume und affine Unterräume}
\label{sec:2.11}

Sei $V$ ein $K$-Vektorraum und sei $U \subseteq V$ ein Untervektorraum von $V$.
Ist dann $v_0 \in V$, so heißt $E = v_0 + U$ ein \Index{affiner Unterraum} von $V$ mit zugehörigem Untervektorraum $U$ und Aufhängepunkt $v_0$.
Wir wollen im Folgenden zeigen, dass die Menge der affinen Unterräume mit Untervektorraum $U$ wieder ein Vektorraum ist.

\begin{lemma}
	\label{lemma:11.1}
	Sei $E = v_0 + U$ ein affiner Untervektorraum von $V$.
	Ist dann $v_1 \in V$ beliebig, so sind äquivalent:
	\begin{enumerate}[(i)]
		\item $v_0 + U \cap v_1 + U \neq \emptyset$ 
		\item $v_0 + U = v_1 + U$
		\item $v_1 \in v_0 + U (\Leftrightarrow v_1 - v_0 \in U)$
	\end{enumerate}
\end{lemma}

\begin{beweis}
	\mbox{} \\[-.9cm]
	\begin{description}
		\item[(i) $\Rightarrow$ (iii):] Sei $w \in v_0 + U \cap v_1 + U$.
		Dann existiert $u_1, u_2 \in U$ mit $v_0 + u_1 = v_1 + u_2$ und dann $v_1 = v_0 + (u_1 - u_2) \in v_0 + U$.
		\item[(iii) $\Rightarrow$ (ii):] Sei $u_1 \in U$ mit $v_1 = v_0 + u_1$.
		Dann folgt $v_1 + U = (v_0 + u_1) + U = v_0 + (u_1 + U) = v_0 + U$, denn $u_1 + U = U$, da $U$ Untervektorraum von $V$.
		\item[(ii) $\Rightarrow$ (i):] trivial. \qedhere
	\end{description}
\end{beweis}

Wir sehen also, dass zwei affine Untervektorräume zum selben Untervektorraum $U$ entweder gleich oder disjunkt sind.

\begin{definition}[Quotientenraum]
	\label{def:11.2}
	Sei $V$ ein $K$-Vektorraum und sei $U$ ein Untervektorraum von $V$.
	Dann heißt
	\[
		V \diagup U := \{v + U : v \in V\} \subseteq \pot(V)
	\]
	der \Index{Quotientenraum} von $V$ nach dem Untervektorraum $U$.
\end{definition}

Beachte: Nach \autoref{lemma:11.1} sind zwei Elemente $v_1+U, v_2 +U \in V \diagup U$ gleich genau dann, wenn $v_1 - v_2 \in U$.

\begin{satz}[Quotientenabbildung]
	\label{satz:11.3}
	Sei $U \subseteq V$ ein Untervektorraum des $K$-Vektorraums $V$.
	Dann ist $V \diagup U$ ein $K$-Vektorraum mit Addition und skalarer Multiplikation definiert durch
	\begin{align*}
		(v_1 + U) + (v_2 + U) &:= (v_1 + v_2) + U, \quad v_1,v_2 \in V \\
		\lambda (v+U) &:= \lambda v + U, \quad v \in V, \lambda \in K.
	\end{align*}
	Das neutrale Element bezüglich Addition ist $0 + U = U \subseteq V$.
	Ferner gilt:
	Die kanonische Abbildung
	\begin{align*}
		q\colon V &\longrightarrow V \diagup U \\
		v &\longmapsto v+U
	\end{align*}
	ist linear und surjektiv mit $\Kern(q) = U$.
	$q$ heißt \Index{Quotientenabbildung} von $U$.
\end{satz}

\begin{beweis}
	Wir müssen zeigen, dass Addition und skalare Multiplikation auf $V \diagup U$ wohldefiniert sind, das heißt, dass sie nicht von der Wahl von $v_1,v_2,v$ abhängen.
	Seien dazu $v_1',v_2' \in V$ mit $v_1' + U = v_1 + U, v_2' + U = v_2 + U$.
	Dann existieren $u_1,u_2 \in U$ mit $v_1' = v_1 + u_1, v_2' = v_2+u_2$, und es folgt
	\[
		(v_1'-v_2')+U = (v_1+u_1+v_2+u_2)+U = (v_1+v_2) + (u_1+u_2+U) = (v_1+v_2) + U.
	\]
	Analog: Ist $v' = v+u$ für ein $u \in U$, so gilt
	\[
		\lambda v' + U = \lambda(v+u)+U = \lambda v + (\lambda u + U) = \lambda v + U.
	\]
	Damit sind Addition und skalare Multiplikation wohldefiniert.
	Die Vektorraum-Axiome folgen dann sofort aus den entsprechenden Eigenschaften in $V$.
	
	Sei nun $q$ die Quotientenabbildung von $U$.
	Aus der Definition der Addition und skalaren Multiplikation auf $V \diagup U$ folgt sofort, dass $q$ linear und surjektiv ist.
	Ist $v \in V$, so gilt
	\[
		q(v) = 0 + U \Leftrightarrow v+ U = 0+U \stackrel{\text{\ref{lemma:11.1}}}{\Leftrightarrow} v \in U,
	\]
	also folgt $\Kern(q) = U$. \qedhere
\end{beweis}

\begin{bemerkung}
	\label{bem:11.4}
	\mbox{} \\[-1.4cm]
	\begin{enumerate}[(i)]
		\item Wenn wir die Addition und skalare Multiplikation in \autoref{satz:11.3} als Addition bzw. skalare Multiplikation von Teilmengen von $V$ verstehen (was sie tatsächlich sind), so ist die Wohldefiniertheit sofort klar, denn wegen $U+U = U$ und $\lambda U = U$ für alle $\lambda \in K \setminus \setzero$ folgt
		\begin{align*}
			(v_1+U)+(v_2+U) &= v_1+v_2+U+U = v_1 + v_2 + U \\
			\lambda(v+U) &= \lambda v + \lambda U = \lambda v + U
		\end{align*}
		\item Definieren wir auf $V$ eine Relation durch
		\[
			v \sim_U w \quad :\Leftrightarrow \quad v-w \in U (\Leftrightarrow w \in v+U),
		\]
		so ist dies eine Äquivalenzrelation mit Äquivalenzklassen \index{Äquivalenzrelation}
		\[
			[v] = \{w \in V : v \sim_U w\} = v + U.
		\]
		Wir können also $V \diagup U$ auch als Quotientenraum
		\[
			V \diagup \sim_U = \{ [v] : v \in V\}
		\]
		definieren.
		In diesem Bild sind Addition und skalare Multiplikation gegeben durch
		\begin{align*}
			[v_1] + [v_2] &= [v_1 + v_2] \\
			\lambda [v] &= [\lambda v].
		\end{align*}
	\end{enumerate}
\end{bemerkung}

\begin{bemerkung}
	\label{bem:11.5}
	Ist $\wt{F}\colon V \diagup U \rightarrow W$ eine beliebige lineare Abbildung auf $V \diagup U$, so ist $F := \wt{F} \circ q \colon V \rightarrow W$ eine lineare Abbildung auf $V$ mit
	\[
		\Kern(F) = \Kern(\wt{F} \circ q) \supseteq \Kern(q) = U,
	\]
	das heißt jede lineare Abbildung von $V \diagup U$ können wir zu einer linearen Abbildung von $V$ \enquote{liften}, sodass das folgende Diagramm kommutiert:
	\[
		\begin{tikzcd}
			V \arrow[rr,"F=\wt{F} \circ q"] \arrow[rd,"q"'] & & W \\
			& V \diagup U \arrow[ru,"\wt{F}"'] &
		\end{tikzcd}
	\]
	Diese Beobachtung hat eine wichtige Umkehrung:
\end{bemerkung}

\begin{satz}[Homomorphiesatz]
	\label{satz:11.6}
	Sei $U$ ein Untervektorraum des $K$-Vektorraums $V$ und sei $F \colon V \rightarrow W$ linear.
	Dann sind äquivalent:
	\begin{enumerate}[(i)]
		\item Es existiert eine eindeutig bestimmte lineare Abbildung $\wt{F} \colon V \diagup U \rightarrow W$ mit $F = \wt{F} \circ q$
		\item Es gilt $U \subseteq \Kern(F)$.
	\end{enumerate}
	Ferner gilt:
	Ist $U = \Kern(F)$, so ist $\wt{F}$ injektiv.
\end{satz}

\begin{beweis}
	\mbox{} \\[-.9cm]
	\begin{description}
		\item[(i) $\Rightarrow$ (ii):] Das ist \autoref{bem:11.5}.
		\item[(ii) $\Rightarrow$ (i):] Wir definieren $\wt{F}\colon V \diagup U \rightarrow W$ durch $\wt{F}(q(v)) = \wt{F}(v+U) := F(v)$.
		Wir müssen zeigen, dass $\wt{F}$ wohldefiniert und linear ist.
		Es gilt dann automatisch $\wt{F} \circ q = F$ und die Eindeutigkeit ist auch klar.
		
		Sei also $v + U = v' + U$.
		Dann folgt $v' = v+u$ für ein $u \in U \subseteq \Kern(F)$ und damit
		\[
			\wt{F}(v'+U) = F(v') = F(v+u) = F(v) + F(u) = F(v) = \wt{F}(v+U).
		\]
		Also ist $\wt{F}$ wohldefiniert.
		$\wt{F}$ ist auch linear, denn mit der Linearität von $F$ gilt
		\newpage
		
		\begin{align*}
			\wt{F}((v+U)+(w+U)) &= \wt{F}(v+w+U) = F(v+w) = F(v) + F(w) = \wt{F}(v+U) + \wt{F}(w+U), \\
			\wt{F}(\lambda(v+U)) &= \wt{F}(\lambda v + U) = F(\lambda v) =\lambda F(v) = \lambda \wt{F}(v+U).		
		\end{align*}
	\end{description}
	Ist $U = \Kern(F)$, so gilt für alle $v \in V$:
	\[
		0 = \wt{F}(v+U) \Leftrightarrow 0 = F(v) \Leftrightarrow v \in U \Leftrightarrow v + U = 0 + U,
	\]
	also ist $\wt{F}$ injektiv. \qedhere
\end{beweis}