%!TEX root = ../LA2.tex
\section{Orthogonale Projektionen und Orthonormalbasen}
\label{sec:2.4}

In diesem Abschnitt ist wieder $\KK = \RR$ oder $\KK= \CC$.

\begin{definition}[Orthogonales Komplement einer Menge]
	\label{def:4.1}
	Sei $V$ ein $\KK$-Vektorraum mit Skalarprodukt $\sk{\cdot,\cdot}$.
	Ist dann $M \subseteq V$, so setzen wir \index{orthogonales Komplement}
	\[
		M^\perp := \{v \in V : \sk{v,w} = 0 \text{ für alle } w \in M\} \subseteq V.
	\]
	$M^\perp$ ist dann ein Untervektorraum von $V$, denn sind $v_1,v_2 \in M^\perp$ und $\lambda_1,\lambda_2 \in \KK$, so folgt für alle $w \in M$:
	\[
		\sk{\lambda_1 v_1 + \lambda_2 v_2, w} = \lambda_1 \cdot \Underbrace{\sk{v_1,w}}{=0} + \lambda_2 \Underbrace{\sk{v_2,w}}{=0} = 0,
	\]
	also ist $\lambda_1 v_1 + \lambda_2 v_2 \in M^\perp$.
\end{definition}

\begin{bemerkung}[Orthogonale direkte Summe]
	\label{bem:4.2}
	Ist $U \subseteq V$ ein beliebiger Untervektorraum von $V$, so gilt $U \cap U^\perp = \setzero$, denn für $u \in U \cap U^\perp$ gilt $\no{u}^2 = \sk{u,u} = 0$, also $u = 0$.
	Gilt zusätzlich $U + U^\perp = V$, so ist $V$ die direkte Summe von $U$ mit $U^\perp$.
	Da die Räume senkrecht zueinander stehen, sagen wir dann, dass $V = U \oplus U^\perp$ die \Index{orthogonale direkte Summe} von $U$ und $U^\perp$ ist.
	
	Beachte: Wir werden später sehen, dass für Untervektorräume $U \subseteq V$ mit $\dim(U) < \infty$ immer gilt, dass $U + U^\perp = V$ gilt, also $V = U \oplus U^\perp$.
\end{bemerkung}

\begin{definition}[Lot, orthogonale Projektion]
	\label{def:4.3}
	Sei $V$ ein $\KK$-Vektorraum mit $\sk{\cdot,\cdot}$ und $U \subseteq V$ ein Untervektorraum von $V$.
	Ist dann $v \in V$, so heißt $P_U(v) \in U$ das \Index{Lot} von $v$ auf $U$ (bzw. die \Index{orthogonale Projektion} von $v$ auf $U$), falls gilt:
	\[
	(v- P_U(v)) \perp U \qquad \text{bzw.} \qquad v-P_U(v) \in U^\perp.
	\]
\end{definition}

\begin{figure}[h]
	\centering
	\begin{tikzpicture}
		\draw [lightgray,fill=lightgray] (0,1) -- (3,1) -- (4,3) -- (1,3) -- cycle;
		\draw (1.5,1) node[below]{$U$};
		
		\draw [->,thick] (1,3) -- (0,1) -- (-.1,.8);
		\draw [->,thick] (1,3) -- (4.2,3);
		\draw [->,thick] (1,3) -- (1,5);
		
		\coordinate (Pv) at (2,2);
		\coordinate (v) at (2,5);		
		\coordinate (x) at (2.4,2.8);
		\coordinate (y) at (1,2);
		
		\draw (Pv) node[fill, circle, inner sep=1pt]{};
		\draw (v) node[fill, circle, inner sep=1pt]{};
		
		\draw (v) node[right]{$v$};
		\draw (Pv) node[anchor=north west]{$P_U(v)$};
		
		\draw [thin] (Pv) -- (x);
		\draw [thin] (Pv) -- (y);
		\draw [thick,dashed] (Pv) -- (v);
		
		\draw pic["$\bullet$",draw=black,angle eccentricity=.7,angle radius=.8cm]{angle=x--Pv--v};
		\draw pic["$\bullet$",draw=black,angle eccentricity=.5,angle radius=.5cm]{angle=v--Pv--y};
	\end{tikzpicture}
	\caption{Veranschaulichung der orthogonalen Projektion auf einen Untervektorraum.}
\end{figure}
\newpage
\begin{bemerkung}
	\label{bem:4.4}
	\mbox{} \\[-1.4cm]
	\begin{enumerate}[(1)]
		\item Wenn $P_U(v)$ existiert, so ist es eindeutig bestimmt, denn sind $u_1,u_2 \in U$ mit $(v-u_1) \in U^\perp$ und $(v-u_2) \in U^\perp$, so folgt
		\[
			u_2 - u_1 = (v-u_1) - (v-u_2) \in U \cap U^\perp = \setzero.
		\]
		\item Existiert $P_U(v)$ für alle $v \in V$, so folgt für alle $v \in V$:
		\[
			v = \Underbrace{P_U(v)}{\in U} + \Underbrace{(v-P_U(v))}{\in U^\perp} \in U + U^\perp.
		\]
		Damit folgt $V = U+U^\perp$ und dann $V = U \oplus U^\perp$, da $U \cap U^\perp = \setzero$, und $P_U \colon V \rightarrow U$ ist die Projektion von $V$ auf $U$ wie in \autoref{satz:I.12.5}.
		\item Existiert $P_U(v)$ für alle $v \in V$, so existiert auch $P_{U^\perp}(v)$ für alle $v \in V$, und es gilt
		\[
			P_{U^\perp}(v) = v - P_U(v).
		\]
		Denn $v - P_U(v) \in U^\perp$ und $v-(v-P_U(v)) = P_U(v) \in U$ und $U \subseteq (U^\perp)^\perp$.
		In diesem Fall gilt sogar $U = (U^\perp)^\perp$, denn ist $v \in (U^\perp)^\perp$, so gilt nach (ii) $v = u+w$ für ein $u \in U, w \in U^\perp$, und dann folgt
		\[
			0 = \sk{v,w} = \sk{u+w,w} = \Underbrace{\sk{u,w}}{\mathclap{=0\text{, da } u \in U, w \in U^\perp}} + \sk{w,w},
		\]
		also folgt $0 = \sk{w,w}$ und damit $w = 0, v = u \in U$.
	\end{enumerate}
\end{bemerkung}

Wir wollen im Folgenden zeigen, dass $P_U$ immer existiert, wenn $\dim(U) < \infty$.
Dazu benötigen wir die folgende wichtige Notation:

\begin{definition}[Orthogonalsystem, Orthonormalsystem, Orthonormalbasis]
	\label{def:4.5}
	Sei $V$ ein $\KK$-Vektorraum mit Skalarprodukt $\sk{\cdot,\cdot}$ und seien $v_1,\dots,v_l \in V$.
	Dann definieren wir:
	\begin{enumerate}[(i)]
		\item $\{v_1,\dots,v_l\}$ heißt \Index{Orthogonalsystem}, wenn $v_i \perp v_j$ für alle $1 \leq i,j \leq l$ mit $i \neq j$.
		\item $\{v_1,\dots,v_l\}$ heißt \Index{Orthonormalsystem}, falls $\{v_1,\dots,v_l\}$ ein Orthogonalsystem ist mit $\no{v_i} = 1$ für alle $1 \leq i \leq l$.
		\item $\{v_1,\dots,v_l\}$ heißt \Index{Orthonormalbasis}, falls $\{v_1,\dots,v_l\}$ ein Orthogonalsystem ist mit $\LH\{v_1,\dots,v_l\} = V$.
	\end{enumerate}
\end{definition}

\begin{definition}[Kroneckersymbol]
	\label{def:4.6}
	Ist $I \neq \emptyset$ eine Indexmenge, so definieren wir für $i,j \in I$ das \Index{Kroneckersymbol}:
	\[
		\delta_{ij} = \begin{cases}
			1, & \text{falls } i = j \\
			0, & \text{falls } i \neq j
		\end{cases}
	\]
	Dann ist $\{v_1,\dots,v_l\}$ ein Orthogonalsystem, falls für alle $i,j \in \{1,\dots,l\}$ gilt:
	\[
		\sk{v_i,v_j} = \delta_{ij} = \no{v_i}^2
	\]
\end{definition}

\begin{minipage}{.7\textwidth}
	\begin{satz}[Satz von \textsc{Pythagoras}]
		\label{satz:4.7}
		Ist $\{v_1,\dots,v_l\}$ ein Orthogonalsystem, so gilt für alle $\lambda_1,\dots,\lambda_l \in \KK$:
		\[
		\no{\sum_{i=1}^{l} \lambda_i v_i}^2 = \sum_{i=1}^{l} \abs{\lambda_i}^2 \no{v_i}^2.
		\]
	\end{satz}
\end{minipage} \hfill
\begin{minipage}{.23\textwidth}
		\centering
		\begin{tikzpicture}[>=Latex]
		\coordinate (O) at (0,0);
		\coordinate (v1) at (2.5,0);
		\coordinate (v2) at (0,1.5);
		\coordinate (v3) at (2.5,1.5);
		
		\draw [thick,->] (0,0) -- (v1) node[anchor=north west]{$v_1$};
		\draw [thick,->] (0,0) -- (v2) node[left]{$v_2$};
		\draw (1.25,0) node[below]{$\no{v_1}$};
		
		\draw [thick,->] (0,0) -- (v3);
		\draw (2.5,0.75) node[right]{$\no{v_2}$};
		
		\draw [thick,dashed] (2.5,0) -- (2.5,1.5);
		\draw pic["$\bullet$",draw=black,angle eccentricity=.5,angle radius=.5cm]{angle=v3--v1--O};
		
		\draw (1.25,0.75) node[above,rotate=30]{$\no{v_1+v_2}$};
		\end{tikzpicture}	
\end{minipage}

\vspace*{.25cm}
\begin{beweis}
	Es gilt
	\[
		\no{\sum_{i=1}^{l} \lambda_i v_i}^2 = \sk{\sum_{i=1}^{l} \lambda_i v_i,\sum_{j=1}^{l} \lambda_j v_j} = \sum_{i=1}^{l} \sum_{j=1}^{l} \lambda_i \ol{\lambda_j} \sk{v_i,v_j} = \sum_{i,j=1}^{l} \lambda_i \ol{\lambda_j} \delta_{ij} \no{v_i}^2 = \sum_{i=1}^{l} \lambda_i \ol{\lambda_i} \no{v_i}^2 \qedhere
	\]
\end{beweis}

\begin{lemma}
	\label{lemma:4.8}
	Ist $\{v_1,\dots,v_l\}$ ein Orthogonalsystem mit $v_i \neq 0$ für alle $1 \leq i \leq l$, dann sind die $v_1, \dots, v_l$ linear unabhängig.
	Insbesondere folgt, dass jede Orthonormalbasis von $V$ eine Basis von $V$ ist.
\end{lemma}

\begin{beweis}
	Seien $\lambda_1, \dots, \lambda_l \in \KK$ mit $\sum_{i=1}^{l} \lambda_i v_i = 0$, so folgt mit \autoref{satz:4.7}:
	\[
		0 = \no{\sum_{i=1}^{l} \lambda_i v_i}^2 \stackrel{\ref{satz:4.7}}{=} \sum_{i=1}^{l} \abs{\lambda_i}^2 \no{v_i}^2,
	\]
	und damit $\abs{\lambda_i}^2 \no{v_i}^2 = 0$ für alle $1 \leq i \leq l$.
	Da $v_i \neq 0$, folgt $\lambda_i = 0$ für alle $1 \leq i \leq l$. \qedhere
\end{beweis}

\begin{satz}
	\label{satz:4.9}
	Sei $V$ ein $\KK$-Vektorraum mit $\sk{\cdot,\cdot}$ und sei $\{v_1,\dots,v_l\}$ ein Orthonormalsystem in $V$.
	Ferner sei $U = \LH\{v_1,\dots,v_l\}$.
	Dann gelten:
	\begin{enumerate}[(i)]
		\item Für alle $v \in V$ existiert die orthogonale Projektion $P_U(v) \in U$ und es gilt
		\[
			P_U(v) = \sum_{i=1}^{l} \sk{v,v_i} v_i
		\]
		\item Für alle $v \in V$ gilt die \Index{Besselsche Ungleichung}
		\[
			\no{P_U(v)}^2 = \sum_{i=1}^{l} \abs{\sk{v,v_i}}^2 \leq \no{v}^2.
		\]
		\item Ist $v \in V$, so gilt $\no{v- P_U(v)} \leq \no{v-u}$ für alle $u \in U$ mit $u \neq v$, das heißt $P_U(v)$ ist das eindeutige Element in $U$, das von $v$ den kleinsten Abstand hat.
	\end{enumerate}
\end{satz}
\newpage
\begin{beweis}
	\mbox{} \\[-.9cm]
	\begin{enumerate}[(i)]
		\item Nach \autoref{bem:4.4} genügt es zu zeigen:
		Ist $u_0 = \sum_{i=1}^{l} \sk{v,v_i} v_i$, so gilt $(v-u_0) \perp U$.
		Zunächst gilt für alle $v_j$ mit $1 \leq j \leq l$:
		\begin{align*}
			&\sk{v-u_0,v_j} = \sk{v,v_j} - \sk{u_0,v_j} = \sk{v,v_j} - \sk{\sum_{i=1}^{l} \sk{v_i,v}v_i,v_j} \\
			=\quad &\sk{v,v_j} - \sum_{i=1}^{l} \sk{v_i,v} \Underbrace{\sk{v_i,v_j}}{=\delta_{ij}} = \sk{v,v_j} - \sk{v,v_j} = 0.
		\end{align*}
		Damit folgt $\sk{v-u_0,v_j} = 0$ für alle $1 \leq j \leq l$.
		Ist nun $u = \sum_{j=1}^{l} \lambda_j v_j \in \LH\{v_1,\dots,v_l\} = U$, so folgt auch
		\[
			\sk{v-u_0,u} = \sk{v-u_0,\sum_{i=1}^{l} \lambda_j v_j} = \sum_{i=1}^{l} \ol{\lambda_j} \sk{v-u_0,v_j} = 0.
		\]
		\item Nach \autoref{satz:4.7} und (i) gilt
		\[
			\no{P_U(v)}^2 = \no{\sum_{i=1}^{l} \sk{v,v_i} v_i}^2 = \sum_{i=1}^{l} \abs{\sk{v,v_i}}^2 \Underbrace{\no{v_i}^2}{=1} = \sum_{i=1}^{l} \abs{\sk{v,v_i}}^2.
		\]
		Ferner gilt, da $v - P_U(v) \perp P_U(v)$, dass
		\[
			\no{v}^2 = \no{(v-P_U(v)) + P_U(v)}^2 \stack{\ref{satz:4.7}}{=} \no{v - P_U(v)}^2 + \no{P_U(v)}^2 \geq \no{P_U(v)}^2
		\]
		Beachte: Es folgt auch $\no{v} = \no{P_U(v)} \Leftrightarrow v = P_U(v) \in U$.
		\item Ist $u \in U$ mit $u \neq P_U(v)$, so folgt $P_U(v) - u \neq 0$.
		Dann folgt mit \autoref{satz:4.7}:
		\begin{align*}
			\no{v-u}^2 &\stack{}{=} \no{(v-P_U(v)) + \Underbrace{(P_U(v) - u)}{\in U}}^2 \\
			&\stack{\ref{satz:4.7}}{=} \no{v-P_U(v)}^2 + \Underbrace{\no{P_U(v)-u}}{\neq 0 \text{ n.V.}}^2 > \no{v-P_U(v)}^2 \qedhere
		\end{align*}
	\end{enumerate}
\end{beweis}

\begin{definition}[Abstand]
	\label{def:4.10}
	Ist $V$ ein $\KK$-Vektorraum mit $\sk{\cdot,\cdot}$ und ist $\emptyset \neq M \subseteq V$, so definieren wir für ein $v \in V$ den \Index{Abstand} von $v$ zu $M$ durch
	\[
		d(v,M) := \inf \{ \no{v-u} : u \in M\} \geq 0
	\]
\end{definition}
\newpage
\begin{korollar}
	\label{kor:4.11}
	Ist $U \subseteq V$ ein Untervektorraum, sodass eine orthogonale Projektion $P_U\colon V \rightarrow U$ existiert, dann gilt für alle $v \in V$
	\[
		d(v,U) = \no{v - P_U(v)}
	\]
	und $P_U(v)$ ist eindeutig durch diese Gleichung festgelegt.
\end{korollar}

\begin{beweis}
	Das folgt aus dem Beweis von \autoref{satz:4.9}(iii). \qedhere
\end{beweis}

Ist $\{v_1,\dots,v_n\}$ eine Orthonormalbasis von $V$, so erhalten wir eine besonders schöne Darstellung der Elemente von $V$ als Linearkombination der Basiselemente:

\begin{satz}[Fourierentwicklung]
	\label{satz:4.12}
	Sei $V$ ein $\KK$-Vektorraum mit $\sk{\cdot,\cdot}$ und sei $\{v_1,\dots,v_n\}$ ein Orthonormalsystem von $V$.
	Dann sind äquivalent:
	\begin{enumerate}[(1)]
		\item $\{v_1,\dots,v_n\}$ ist eine Orthonormalbasis von $V$.
		\item Für alle $v \in V$ gilt $v = \sum_{i=1}^{n} \sk{v,v_i} v_i$ (\Index{Fourierentwicklung}).
		\item Für alle $v,w \in V$ gilt die \Index{Parsevalsche Gleichung} $\sk{v,w} = \sum_{i=1}^{n} \sk{v,v_i} \cdot \ol{\sk{w,v_i}}$.
		\item Für alle $v \in V$ gilt die \Index{Plancherel-Formel} $\no{v}^2 = \sum_{i=1}^{n} \abs{\sk{v,v_i}}^2$.
		\item Ist $v \in V$ mit $\sk{v,v_i} = 0$ für alle $1 \leq i \leq n$, so gilt $v = 0$, das heißt $\{v_1,\dots,v_n\}$ ist ein maximales Orthonormalsystem.
	\end{enumerate}
\end{satz}

\begin{beweis}
	\mbox{} \\[-.85cm]
	\begin{description}
		\item[(i) $\Rightarrow$ (ii):] Nach \autoref{lemma:4.8} folgt aus (i), dass $V = \LH\{v_1,\dots,v_n\}$, das heißt $v = \sum_{j=1}^{n} \lambda_j v_j$ für geeignete $\lambda_i \in \KK$.
		Dann folgt für alle $1 \leq i \leq n$:
		\[
			\sk{v,v_i} = \sk{\sum_{j=1}^{n} \lambda_j v_j, v_i} = \sum_{j=1}^{n} \lambda_j \cdot \Underbrace{\sk{v_j,v_i}}{=\delta_{ij}} = \lambda_i,
		\]
		also $v = \sum_{i=1}^{n} \sk{v,v_i} v_i$.
		\item[(ii) $\Rightarrow$ (iii):] Nach (ii) gilt
		\begin{align*}
			\sk{v,w} &= \sk{\sum_{i=1}^{n} \sk{v,v_i} v_i, \sum_{j=1}^{n} \sk{w,v_j} v_j} \\
			&= \sum_{i=1}^{n} \sum_{j=1}^{n} \sk{v,v_i} \ol{\sk{w,v_j}} \Underbrace{\sk{v_i,v_j}}{=\delta_{ij}} = \sum_{i=1}^{n} \sk{v,v_i} \ol{\sk{w,v_i}}.
		\end{align*}
		\item[(iii) $\Rightarrow$ (iv):] Einsetzen von $w = v$ liefert
		\[
			\no{v}^2 = \sk{v,v} = \sum_{i=1}^{n} \sk{v,v_i} \ol{\sk{v,v_i}} = \sum_{i=1}^{n} \abs{\sk{v,v_i}}^2.
		\]
		\item[(iv) $\Rightarrow$ (v):] Ist $v \in V$ mit $\sk{v,v_i} = 0$ für alle $1 \leq i \leq n$, so folgt
		\[
			\no{v}^2 = \sum_{i=1}^{n} \abs{\sk{v,v_i}}^2 = 0,
		\]
		also $v = 0$.
		\item[(v) $\Rightarrow$ (i):] Sei $U = \LH\{v_1,\dots,v_n\}$.
		Zeige: $U = V$.
		Nach \autoref{satz:4.9} existiert die orthogonale Projektion $P_U\colon V \rightarrow U$.
		Ist dann $v \in V$, so gilt $v - P_U(v) \perp U$, also insbesondere $\sk{v-P_U(v),v_i} = 0$ für alle $1 \leq i \leq l$.
		Nach (v) folgt $v - P_U(v) = 0$, also $v = P_U(v) \in U$. \qedhere
	\end{description}
\end{beweis}

\textbf{Problem:} Besitzt jeder endlich dimensionale $\KK$-Vektorraum mit $\sk{\cdot,\cdot}$ eine Orthonormalbasis?
Und wenn ja, wie kann man eine Orthonormalbasis finden?

\begin{satz}[Schmidtsches Orthonormalisierungsverfahren]
	\label{satz:4.13}
	Sei $V$ ein $\KK$-Vektorraum mit $\sk{\cdot,\cdot}$ und seien $w_1,\dots,w_l \in V$ linear unabhängige Vektoren.
	Dann existiert ein Orthonormalsystem $\{v_1,\dots,v_l\}$ in $V$, sodass für alle $1 \leq j \leq l$ gilt:
	\[
		\LH\{v_1,\dots,v_j\} = \LH\{w_1,\dots,w_j\}.
	\]
	Insbesondere folgt:
	Ist $\{w_1,\dots,w_l\}$ eine Basis von $V$ so ist $\{v_1,\dots,v_l\}$ eine Orthonormalbasis von $V$. \index{Schmidtsches Orthonormalisierungsverfahren}
\end{satz}

\begin{beweis}
	Wir konstruieren $\{v_1,\dots,v_l\}$ rekursiv wie folgt
	\begin{enumerate}[(i)]
		\item Setze $v_1 = \no{w_1}^{-1} w_1$. (geht, da $w_i \neq 0$)
		\item Sind $v_1,\dots,v_j$ bereits konstruiert mit $j < l$, so setzen wir
		\[
			\widetilde{v}_{j+1} = w_{j+1} - \Underbrace{\sum_{i=1}^{j} \sk{w_{j+1},v_i} v_i}{\mathclap{=P_{U_j}(w_{j+1}) \text{ mit } U_j = \LH\{v_1,\dots,v_j\}}}
		\]
		Da $w_{j+1} \notin \LH\{w_1,\dots,w_j\} = \LH\{v_1,\dots,v_j\}$ gilt $\widetilde{v}_{j+1} \neq 0$.
		Setze dann
		\[
			v_{j+1} := \no{\widetilde{v}_{j+1}} \widetilde{v}_{j+1}.
		\]
		Nach Induktionsvoraussetzung ist $\{v_1,\dots,v_j\}$ ein Orthonormalsystem mit $\LH\{v_1,\dots,v_j\}$ \linebreak $= \LH\{w_1,\dots,w_j\} =: U_j$.
		Wir zeigen nun, dass dann Ähnliches für $\{v_1,\dots,v_{j+1}\}$ gilt:
		\newpage
		Da $\widetilde{v}_{j+1} = w_{j+1} - P_{U_j}(w_{j+1}) \in U_j^\perp$, gilt $\sk{\widetilde{v}_{j+1},v_i} = 0$ für alle $1 \leq i \leq j$, und dann auch $\sk{v_{j+1},v_i} = 0$ für alle $1 \leq i \leq j$.
		Da $\no{v_{j+1}} = 1$ und da $\{v_1,\dots,v_j\}$ ein Orthonormalsystem ist, folgt dann $\sk{v_i,v_j} = \delta_{ij}$ für alle $1 \leq i,l \leq  j+1$.
		
		Zeige nun: $\LH\{w_1,\dots,w_{j+1}\} = \LH\{v_1,\dots,v_{j+1}\}$.
		Da $\LH\{w_1,\dots,w_j\} = \LH\{v_1,\dots,v_j\}$, gilt zunächst, das $\LH\{w_1,\dots,w_{j+1}\} = \LH\{v_1,\dots,v_j,w_{j+1}\}$.
		Nach Konstruktion gilt
		\[
			v_{j+1} = \frac{1}{\no{\tilde{v}_{j+1}}} \enb{w_{j+1} - \sum_{i=1}^{j} \sk{w_{j+1},v_i} v_i}
		\]
		und damit ist $v_{j+1}$ eine Linearkombination von $\{v_1,\dots,v_j,w_{j+1}\}$ derart, dass der Koeffizient von $w_{j+1}$ nicht $0$ ist.
		Nach dem Austauschlemma \ref{lemma:I.9.1} folgt dann
		\[
			\LH\{v_1,\dots,v_{j+1}\} = \LH\{v_1,\dots,v_j,w_{j+1}\} = \LH\{w_1,\dots,w_{j+1}\}. \qedhere
		\]
	\end{enumerate}
\end{beweis}

\begin{korollar}
	\label{kor:4.14}
	Jeder endlich dimensionale $\KK$-Vektorraum $V$ mit Skalarprodukt $\sk{\cdot,\cdot}$ besitzt eine Orthonormalbasis.
\end{korollar}

\begin{beispiel}
	\label{bsp:4.15}
	\mbox{} \\[-1.4cm]
	\begin{enumerate}[(i)]
		\item Ist $\KK^n$ versehen mit dem Standard-Skalarprodukt $\sk{x,y} = \sum_{i=1}^{n} x_i \ol{y_i}$, so ist $\{e_1,\dots,e_n\}$ eine Orthonormalbasis.
		\item Sei $\RR^2$ versehen mit dem Standard-Skalarprodukt.
		Dann ist auch
		\[
			B = \penb{\frac{1}{\sqrt{2}} \binom{1}{1}, \frac{1}{\sqrt{2}} \binom{-1}{1}}
		\]
		eine Orthonormalbasis von $\RR^2$.
		
		Allgemein gilt: Ist $\binom{a}{b} \in \RR^2$ mit $a^2 + b^2 = 1$, so ist
		\[
			\penb{\binom{a}{b}, \binom{-b}{a}}
		\]
		eine Orthonormalbasis von $\RR^2$ bezüglich des Standard-Skalarprodukts. 
		
		\begin{figure}[h]
			\centering
			\begin{tikzpicture}[scale=.75]
				\draw [very thick,->] (-2.5,0) -- (2.5,0);
				\draw [very thick,->] (0,-2.5) -- (0,2.5);
				
				\coordinate (O) at (0,0);
				\coordinate (P) at (-1.41421,1.41421);
				\coordinate (Q) at (1.41421,1.41421);
				
				\draw (O) circle (2);
				\draw (P) -- (O) -- (Q);
				\draw (O) node[fill,circle,inner sep=1pt]{};
				\draw (P) node[fill,circle,inner sep=1pt]{};
				\draw (Q) node[fill,circle,inner sep=1pt]{};
				\draw (P) node[anchor=south east]{$\binom{-b}{a}$};
				\draw (Q) node[anchor=south west]{$\binom{a}{b}$};
				\draw pic["$\bullet$",draw=black,angle eccentricity=.5,angle radius=.5cm]{angle=Q--O--P};
			\end{tikzpicture}
		\end{figure}
		\newpage
		\item Sei $V = \RR^4$ mit dem Standard-Skalarprodukt $\sk{\cdot,\cdot}$ und sei $U \subseteq V$ der von $w_1,w_2,w_3 \in \RR^4$ erzeugte Untervektorraum von $V$ mit
		\[
			w_1 = \begin{pmatrix} 1 \\ 1 \\ 0 \\ 0 \end{pmatrix}, \qquad
			w_2 = \begin{pmatrix} 0 \\ 1 \\ 1 \\ 0 \end{pmatrix}, \qquad 
			w_3 = \begin{pmatrix} 0 \\ 0 \\ 1 \\ 1 \end{pmatrix}.
		\]
		Wir wollen eine Orthonormalbasis von $U$ berechnen.
		Da $w_1,w_2,w_3$ linear unabhängig sind, setzen wir nach Schmidt
		\[
			v_1 = \frac{1}{\no{w_1}} w_1 = \frac{1}{\sqrt{2}} \begin{pmatrix} 1 \\ 1 \\ 0 \\ 0 \end{pmatrix}
		\]
		\[
			\widetilde{v}_2 = w_2 - \sk{w_2,v_1} v_1 = \frac{1}{2} \begin{pmatrix} -1 \\ 1 \\ 2 \\ 0 \end{pmatrix} \qquad \text{ und } \qquad 
			v_2 = \frac{1}{\no{\widetilde{v}_2}} \widetilde{v}_2 = \frac{1}{\sqrt{6}} \begin{pmatrix} -1 \\ 1 \\ 2 \\ 0 \end{pmatrix}
		\]
		\[
			\widetilde{v}_3 = w_3 - \sk{w_3,v_1} v_1 - \sk{w_3,v_2} v_2 = \frac{1}{3}\begin{pmatrix} 1 \\ -1 \\ 1 \\ 3 \end{pmatrix} \qquad \text{ und } \qquad 
			v_3 = \frac{1}{\no{\widetilde{v}_3}} \widetilde{v}_3 = \frac{1}{\sqrt{12}} \begin{pmatrix} 1 \\ -1 \\ 1 \\ 3 \end{pmatrix}
		\]
		Wir können nun mit Hilfe der Orthonormalbasis $\{v_1,v_2,v_3\}$ von $U$ ohne Probleme die orthogonale Projektion $P_U\colon \RR^4 \rightarrow U \subseteq \RR^4$ berechnen.
		Nach \autoref{satz:4.9} gilt
		\[
			P_U(v) = \sk{v,v_1} v_1 + \sk{v,v_2} v_2 + \sk{v,v_3} v_3.
		\]
		Ist zum Beispiel $v = e_1$, so gilt
		\[
			P_U(e_1) = \sk{e_1,v_1} v_1 + \sk{e_1,v_2} v_2 + \sk{e_1,v_3} v_3 =
			\frac{1}{4} \begin{pmatrix} 3 \\ 1 \\ -1 \\ 1 \end{pmatrix}.
		\]
		Der Abstand von $e_1$ zu $U$ ist dann gegeben durch
		\[
			d(e_1,U) = \no{e_1,P_U(e_1)} = \frac{1}{2}.
		\]
		Beachte: Setzen wir $\widetilde{v}_4 := e_1 - P_U(e_1)$ und
		\[
			v_4 = \no{\widetilde{v}_4}^{-1} v_4 = \frac{1}{2} \begin{pmatrix}
			1 \\ -1 \\ 1 \\ -1
			\end{pmatrix},
		\]
		so ist $\{v_1,\dots,v_4\}$ eine Orthonormalbasis von $\RR^4$ (filgt aus Schmidt mit $w_4 = e_1$.
	\end{enumerate}
\end{beispiel}

Als Folgerung von \autoref{bem:4.4}, \autoref{satz:4.9} und \autoref{satz:4.13} erhalten wir:
\begin{satz}
	\label{satz:4.16}
	Sei $V$ ein $\KK$-Vektorraum mit Skalarprodukt $\sk{\cdot,\cdot}$ und sei $U \subseteq V$ ein Untervektorraum mit $\dim(U) < \infty$.
	Dann gilt $V = U \oplus U^\perp$.
	Ist $\{v_1,\dots,v_l\}$ eine Orthonormalbasis von $U$, so gilt für die orthogonale Projektion $P_U \colon V \rightarrow U$:
	\[
		P_U(v) = \sum\limits_{i=1}^{l} \sk{v,v_i} v_i.
	\]
	Ferner gilt $P_{U^\perp} = \id_V - P_U$ (bzw. $P_U = \id_V - P_{U^\perp}$).
\end{satz}

\begin{anwendung}
	\label{anw:4.17}
	Sei $V = \mathcal{C}[0,1] = \{f \colon [0,1] \rightarrow \CC : f \text{ stetig}\}$ versehen mit dem Skalarprodukt
	\[
		\sk{f,g} = \int\limits_{0}^{1} f(t) \ol{g(t)}dt.
	\]
	Für alle $n \in \ZZ$ sei $e_n \in \mathcal{C}[0,1]$ mit $e_n(t) = e^{2\pi i n t}$.
	Dann ist $\{e_n : n \in \ZZ\}$ ein Orthonormalsystem in $V$, denn für alle $n,m \in \ZZ$ gilt
	\begin{align*}
		\sk{e_n,e_m} &= \int\limits_{0}^{1} e^{2 \pi i n t} \ol{e^{2\pi i m t}} dt \\
				&= \int\limits_{0}^{1} e^{2 \pi i n t} \ol{e^{-2\pi i m t}} dt \\
				&= \int\limits_{0}^{1} e^{2\pi i (n-m)t} dt \\
				&= \begin{dcases}
					1, & \text{ falls } n=m \\
					\frac{1}{2 \pi (n-m)} e^{2\pi i (n-m) t} \bigg|_0^1 = 0, & \text{ falls } n \neq m \\ 
				\end{dcases} \\
				&= \delta_{nm}.
	\end{align*}
	Sei nun $f \in \mathcal{C}[0,1]$ fest.
	Für alle $n \in \ZZ$ sei
	\[
		c_n := \sk{f,e_n} = \int\limits_{0}^{1} f(t) e^{-2\pi i n t} dt.
	\]
	Dann heißt $c_n$ der $n$-te \textbf{Fourierkoeffizient} von $f$.
	Für $N \in \NN$ sei
	\[
		S_N(f) = \sum\limits_{n=-N}^{N} c_n e_n.
	\]
	Dann heißt $S_N(f)$ das $N$-te Fourierpolynom von $f$.
	$S_N(f)$ ist die orthogonale Projektion von $f$ auf dem Raum $\mathcal{U}_N = \LH\{e_n : \abs{n} \leq N\}$ aller trigonometrischen Polynome vom Grad $\leq N$.
	Man kann dann zeigen:
	Die Folge $(S_N(t))_{N \in \NN}$ konvergiert gegen $f$ bezüglich der von $\sk{\cdot,\cdot}$ erzeugten Norm $\no{\cdot}_2$, also
	\[
		\no{f-S_N(f)}_2 = \enb{\int\limits_{0}^{1} \abs{f(t)-S_N(f)(t)}^2 dt}^{\frac{1}{2}} \xrightarrow{N\rightarrow \infty} \infty.
	\]
	Die Folge $S_N(f)$ ist die Folge der Partialsummen der \Index{Fourierreihe}
	\[
		S(f)(t) = \sum\limits_{n=-\infty}^{\infty} c_n e^{2\pi i n t}.
	\]
	
	Beachte: Da $e^{2\pi i n t} = \cos(2\pi n t) + i \cdot \sin(2 \pi n t)$ gilt
	\begin{equation}
		S_N(f)(t) = \sum\limits_{n = -N}^{N} c_n (\cos(2 \pi n t) + i \cdot \sin(2 \pi n t) \label{eq:fourier}
	\end{equation}
	Ist $f$ reellwertig, so gilt ferner
	\[
		c_n = \int\limits_{0}^{1} f(x) (\cos(2\pi n t) - i \cdot \sin(2 \pi n t)) dt = a_n - i \cdot b_n
	\]
	mit $a_n = \sk{f, \cos(2 \pi n \cdot)}, b_n = \sk{f,\sin(2\pi n \cdot)}$.
	Da $\cos(-x) = \cos(x), \sin(-x) = - \sin(x)$, folgt $a_{-n} = a_n$ und $b_{-n} = -b_n$.
	Einsetzen in \eqref{eq:fourier} liefert dann
	\begin{align*}
		S_N(f)(t) &\stack{}{=} \sum\limits_{n=-N}^{N} (a_n - i \cdot b_n) \cdot (\cos(2 \pi n t) - i \cdot \sin(2 \pi n t)) \\
		&\stack{}{=} \sum\limits_{n=-N}^{N} (a_n \cdot \cos(2\pi n t) + b_n \cdot \sin(2 \pi n t)) - i \cdot (b_n \cdot \cos(2\pi n t) + a_n \cdot \sin(2 \pi n t)) \\
		&\stack{}{=} a_0 + 2 \cdot \enb{\sum\limits_{n=1}^{} a_n \cos(2 \pi n t) + b_n \sin(2 \pi n t)}.
	\end{align*}
	Hierbei folgt die letzte Gleichheit aus den Relationen $b_0 = 0$, da $\sin(2 \pi 0 t) = 0$, $a_{-n} = a_n$ und $\cos(-x) = \cos(x)$, sowie $b_{-n} = -b_n$ und $\sin(-x) = -\sin(x)$. Wir sehen also, dass wir mit Hilfe der Fourierreihen die gegebene Funktion $f$ durch überlagerte $\cos$- und $\sin$-Schwingungen approximieren können, wobei die Koeffizienten gegeben sind durch
	\[
		a_n = \sk{f,\cos(2\pi n \cdot)}, \qquad b_n = \sk{f,\sin(2\pi n \cdot)}.
	\]
\end{anwendung}



\newpage