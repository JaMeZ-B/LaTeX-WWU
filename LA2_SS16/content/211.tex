%!TEX root = ../LA2_SS16.tex
\subsection{Quotientenräume und affine Unterräume}
\label{sec:2.11}

Sei $V$ ein $K$-Vektorraum und sei $U \subseteq V$ ein Untervektorraum von $V$.
Ist dann $v_0 \in V$, so heißt $E = v_0 + U$ ein \Index{affiner Unterraum} von $V$ mit zugehörigem Untervektorraum $U$ und Aufhängepunkt $v_0$.
Wir wollen im Folgenden zeigen, dass die Menge der affinen Unterräume mit Untervektorraum $U$ wieder ein Vektorraum ist.

\begin{lemma}
	\label{lemma:11.1}
	Sei $E = v_0 + U$ ein affiner Untervektorraum von $V$.
	Ist dann $v_1 \in V$ beliebig, so sind äquivalent:
	\begin{enumerate}[(i)]
		\item $v_0 + U \cap v_1 + U \neq \emptyset$ 
		\item $v_0 + U = v_1 + U$
		\item $v_1 \in v_0 + U (\Leftrightarrow v_1 - v_0 \in U)$
	\end{enumerate}
\end{lemma}

\begin{beweis}
	\begin{description}
		\item[(i) $\Rightarrow$ (iii):] Sei $w \in v_0 + U \cap v_1 + U$.
		Dann existiert $u_1, u_2 \in U$ mit $v_0 + u_1 = v_1 + u_2$ und dann $v_1 = v_0 + (u_1 - u_2) \in v_0 + U$.
		\item[(iii) $\Rightarrow$ (ii):] Sei $u_1 \in U$ mit $v_1 = v_0 + u_1$.
		Dann folgt $v_1 + U = (v_0 + u_1) + U = v_0 + (u_1 + U) = v_0 + U$, denn $u_1 + U = U$, da $U$ Untervektorraum von $V$.
		\item[(ii) $\Rightarrow$ (i):] trivial. 
	\end{description}
\end{beweis}

Wir sehen also, dass zwei affine Untervektorräume zum selben Untervektorraum $U$ entweder gleich oder disjunkt sind.

\begin{definition}[Quotientenraum]
	\label{def:11.2}
	Sei $V$ ein $K$-Vektorraum und sei $U$ ein Untervektorraum von $V$.
	Dann heißt
	\[
		V \diagup U := \{v + U : v \in V\} \subseteq \pot(V)
	\]
	der \Index{Quotientenraum} von $V$ nach dem Untervektorraum $U$.
\end{definition}

Beachte: Nach Lemma~\ref{lemma:11.1} sind zwei Elemente $v_1+U, v_2 +U \in V \diagup U$ gleich genau dann, wenn $v_1 - v_2 \in U$.

\begin{satz}[Quotientenabbildung]
	\label{satz:11.3}
	Sei $U \subseteq V$ ein Untervektorraum des $K$-Vektorraums $V$.
	Dann ist $V \diagup U$ ein $K$-Vektorraum mit Addition und skalarer Multiplikation definiert durch
	\begin{align*}
		(v_1 + U) + (v_2 + U) &:= (v_1 + v_2) + U, \quad v_1,v_2 \in V \\
		\lambda (v+U) &:= \lambda v + U, \quad v \in V, \lambda \in K.
	\end{align*}
	Das neutrale Element bezüglich Addition ist $0 + U = U \subseteq V$.
	Ferner gilt:
	Die kanonische Abbildung
	\begin{align*}
		q\colon V &\longrightarrow V \diagup U \\
		v &\longmapsto v+U
	\end{align*}
	ist linear und surjektiv mit $\Kern(q) = U$.
	$q$ heißt \Index{Quotientenabbildung} von $U$.
\end{satz}

\begin{beweis}
	Wir müssen zeigen, dass Addition und skalare Multiplikation auf $V \diagup U$ wohldefiniert sind, das heißt, dass sie nicht von der Wahl von $v_1,v_2,v$ abhängen.
	Seien dazu $v_1',v_2' \in V$ mit $v_1' + U = v_1 + U, v_2' + U = v_2 + U$.
	Dann existieren $u_1,u_2 \in U$ mit $v_1' = v_1 + u_1, v_2' = v_2+u_2$, und es folgt
	\[
		(v_1'-v_2')+U = (v_1+u_1+v_2+u_2)+U = (v_1+v_2) + (u_1+u_2+U) = (v_1+v_2) + U.
	\]
	Analog: Ist $v' = v+u$ für ein $u \in U$, so gilt
	\[
		\lambda v' + U = \lambda(v+u)+U = \lambda v + (\lambda u + U) = \lambda v + U.
	\]
	Damit sind Addition und skalare Multiplikation wohldefiniert.
	Die Vektorraum-Axiome folgen dann sofort aus den entsprechenden Eigenschaften in $V$.
	
	Sei nun $q$ die Quotientenabbildung von $U$.
	Aus der Definition der Addition und skalaren Multiplikation auf $V \diagup U$ folgt sofort, dass $q$ linear und surjektiv ist.
	Ist $v \in V$, so gilt
	\[
		q(v) = 0 + U \Leftrightarrow v+ U = 0+U \stackrel{\text{\ref{lemma:11.1}}}{\Leftrightarrow} v \in U,
	\]
	also folgt $\Kern(q) = U$. 
\end{beweis}

\begin{bemerkung}
	\label{bem:11.4}
	\begin{enumerate}[(i)]
		\item Wenn wir die Addition und skalare Multiplikation in Satz~\ref{satz:11.3} als Addition bzw. skalare Multiplikation von Teilmengen von $V$ verstehen (was sie tatsächlich sind), so ist die Wohldefiniertheit sofort klar, denn wegen $U+U = U$ und $\lambda U = U$ für alle $\lambda \in K \setminus \setzero$ folgt
		
		\begin{align*}
			(v_1+U)+(v_2+U) &= v_1+v_2+U+U = v_1 + v_2 + U \\
			\lambda(v+U) &= \lambda v + \lambda U = \lambda v + U
		\end{align*}
		
		\item Definieren wir auf $V$ eine Relation durch
		\[
			v \sim_U w \quad :\Leftrightarrow \quad v-w \in U (\Leftrightarrow w \in v+U),
		\]
		so ist dies eine Äquivalenzrelation mit Äquivalenzklassen \index{Äquivalenzrelation}
		\[
			[v] = \{w \in V : v \sim_U w\} = v + U.
		\]
		Wir können also $V \diagup U$ auch als Quotientenraum
		\[
			V \diagup \sim_U = \{ [v] : v \in V\}
		\]
		definieren.
		In diesem Bild sind Addition und skalare Multiplikation gegeben durch
		\begin{align*}
			[v_1] + [v_2] &= [v_1 + v_2] \\
			\lambda [v] &= [\lambda v].
		\end{align*}
	\end{enumerate}
\end{bemerkung}

\begin{bemerkung}
	\label{bem:11.5}
	Ist $\wt{F}\colon V \diagup U \rightarrow W$ eine beliebige lineare Abbildung auf $V \diagup U$, so ist $F := \wt{F} \circ q \colon V \rightarrow W$ eine lineare Abbildung auf $V$ mit
	\[
		\Kern(F) = \Kern(\wt{F} \circ q) \supseteq \Kern(q) = U,
	\]
	das heißt jede lineare Abbildung von $V \diagup U$ können wir zu einer linearen Abbildung von $V$ \enquote{liften}, sodass das folgende Diagramm kommutiert:
	\[
		\begin{tikzcd}
			V \arrow[rr,"F=\wt{F} \circ q"] \arrow[rd,"q"'] & & W \\
			& V \diagup U \arrow[ru,"\wt{F}"'] &
		\end{tikzcd}
	\]
	Diese Beobachtung hat eine wichtige Umkehrung:
\end{bemerkung}

\begin{satz}[Homomorphiesatz]
	\label{satz:11.6}
	Sei $U$ ein Untervektorraum des $K$-Vektorraums $V$ und sei $F \colon V \rightarrow W$ linear.
	Dann sind äquivalent: \index{Homomorphiesatz}
	\begin{enumerate}[(i)]
		\item Es existiert eine eindeutig bestimmte lineare Abbildung $\wt{F} \colon V \diagup U \rightarrow W$ mit $F = \wt{F} \circ q$
		\item Es gilt $U \subseteq \Kern(F)$.
	\end{enumerate}
	Ferner gilt:
	Ist $U = \Kern(F)$, so ist $\wt{F}$ injektiv.
\end{satz}

\begin{beweis}
	\begin{description}
		\item[(i) $\Rightarrow$ (ii):] Das ist Bemerkung~\ref{bem:11.5}.
		\item[(ii) $\Rightarrow$ (i):] Wir definieren $\wt{F}\colon V \diagup U \rightarrow W$ durch $\wt{F}(q(v)) = \wt{F}(v+U) := F(v)$.
		Wir müssen zeigen, dass $\wt{F}$ wohldefiniert und linear ist.
		Es gilt dann automatisch $\wt{F} \circ q = F$ und die Eindeutigkeit ist auch klar.
		
		Sei also $v + U = v' + U$.
		Dann folgt $v' = v+u$ für ein $u \in U \subseteq \Kern(F)$ und damit
		\[
			\wt{F}(v'+U) = F(v') = F(v+u) = F(v) + F(u) = F(v) = \wt{F}(v+U).
		\]
		Also ist $\wt{F}$ wohldefiniert.
		$\wt{F}$ ist auch linear, denn mit der Linearität von $F$ gilt
		 
		
		\begin{align*}
			\wt{F}((v+U)+(w+U)) &= \wt{F}(v+w+U) = F(v+w) = F(v) + F(w) = \wt{F}(v+U) + \wt{F}(w+U), \\
			\wt{F}(\lambda(v+U)) &= \wt{F}(\lambda v + U) = F(\lambda v) =\lambda F(v) = \lambda \wt{F}(v+U).		
		\end{align*}
	\end{description}
	Ist $U = \Kern(F)$, so gilt für alle $v \in V$:
	\[
		0 = \wt{F}(v+U) \Leftrightarrow 0 = F(v) \Leftrightarrow v \in U \Leftrightarrow v + U = 0 + U,
	\]
	also ist $\wt{F}$ injektiv. 
\end{beweis}

\begin{bemerkung}
	\label{bem:11.7}
	Im Allgemeinen gilt $\Kern(\wt{F}) = \Kern(F) \diagup U \subseteq V \diagup U$, wenn $F \colon V \rightarrow W$ linearen mit $U \subseteq \Kern(F)$ und $\wt{F}\colon V \diagup U \rightarrow W$ wie im Satz, denn
	\[
		0 = \wt{F}(v+U) \Leftrightarrow 0 = F(v) \Leftrightarrow v \in \Kern(F) \Leftrightarrow v + U \in \Kern(F) \diagup U.
	\]
\end{bemerkung}

\begin{korollar}[Isomorphiesatz]
	\label{kor:11.8}
	Sei $F\colon V \rightarrow W$ eine surjektive lineare Abbildung zwischen den $K$-Vektorräumen $V$ und $W$.
	Ist dann $U = \Kern(F)$, so ist $\wt{F} \colon V \diagup U \rightarrow W$ ein Isomorphismus. \index{Isomorphiesatz}
\end{korollar}

\begin{beweis}
	Nach Satz~\ref{satz:11.6} ist $\wt{F}$ injektiv, und wegen $\wt{F}(V \diagup U) = F(V) = W$ ist $\wt{F}$ auch surjektiv. 
\end{beweis}

\begin{satz}
	\label{satz:11.9}
	Sei $V$ ein $K$-Vektorraum mit $\dim(V) = n < \infty$.
	Ist dann $U \subseteq V$ ein Untervektorraum, so gilt
	\[
		\dim(V) = \dim(U) + \dim(V \diagup U)
	\]
	Genauer: Ist $\{v_1+U,\dots,v_l + U\}$ eine Basis von $V \diagup U$ und ist $\{u_1,\dots,u_m\}$ eine Basis von $U$, so ist $\{v_1,\dots,v_l,u_1,\dots,u_m\}$ eine Basis von $V$.
\end{satz}

\begin{beweis}
	Sei $q \colon V \rightarrow V \diagup U$ die Quotientenabbildung.
	Dann gilt $\Bild(q) = V \diagup U$ und $\Kern(q) = U$, also liefert die Dimensionsformel:
	\begin{align*}
		\dim(V) &= \dim(\Kern(q)) + \dim(\Bild(q)) \\
		&= \dim(U) + \dim(V\diagup U).
	\end{align*}
	 
	Zum Zusatz: Wegen der Dimensionsformel genügt es zu zeigen, dass $v_1,\dots,v_l,u_1,\dots,u_m$ linear unabhängig sind.
	Seien dazu $\mu_1,\dots,\mu_l,\lambda_1,\dots,\lambda_m \in K$ mit
	\[
		0 = \Underbrace{\sum_{i=1}^{l} \mu_i v_i}{:= v} + \Underbrace{\sum_{j=1}^{m} \lambda_j u_j}{:= u}  = v + u.
	\]
	Dann folgt $0 = q(0) = q(v+u) = q(v) = q \enb{\sum_{i=1}^{l} \mu_i v_i} = \sum_{i=1}^{l} \mu_i (v_i+U)$.
	Da $v_1+U, \dots, v_l+U$ linear unabhängig in $V \diagup U$ sind, folgt $\mu_1,\dots,\mu_l = 0$.
	Dann folgt $v = 0$ und dann auch $u = 0$, also $\sum_{j=1}^{m} \lambda_j u_j$.
	Da $u_1,\dots,u_m$ linear unabhängig sind, folgt dann auch $\lambda_1,\dots,\lambda_m = 0$.
	Also folgt $\mu_1,\dots,\mu_l,\lambda_1,\dots,\lambda_m = 0$. 
\end{beweis}

\begin{bemerkung}
	\label{bem:11.10}
	Der hier vorgestellte Homomorphiesatz (bzw. Isomorphiesatz) ist ein Prototyp für viele ähnliche Sätze für algebraische Objekte.
	Es existieren ähnliche Sätze für Gruppen, Ringe, Algebren, etc.
	Der Beweis ist eigentlich immer gleich, man muss nur aufpassen, was die richtigen \enquote{Unterräume} in den jeweiligen Situationen sind.
	Im Fall einer Gruppe $G$ sind dies \Index{Normalteiler} -- das sind Untergruppen $N \subseteq G$ mit $gN = Ng$ für alle $g \in G$.
	Im Fall eines Rings $R$ sind dies \textbf{Ideale} -- das sind Unterringe $I \subseteq R$ mit $aI, Ia \in I$ für alle $a \in R$. \index{Ideal}
\end{bemerkung}

\begin{definition}[Sequenzen]
	\label{def:11.11}
	Für alle $n \in \ZZ$ sei $V_n$ ein $K$-Vektorraum und sei $d_n \colon V_n \rightarrow V_{n-1}$ eine lineare Abbildung, sodass $d_{n-1} \circ d_n = 0$ für alle $n \in \NN$.
	Dann heißt
	\[
		\dots V_n \xrightarrow{d_n} V_{n-1} \xrightarrow{d_{n-1}} V_{n-2} \xrightarrow{d_{n-2}} V_{n-3} \dots
	\]
	eine \Index{Sequenz} (bzw. \Index{Kette}) von $K$-Vektorräumen.
	Eine Sequenz heißt \textbf{exakt}, wenn für alle $n \in \NN$ gilt: \index{Sequenz!exakt}
	\[
		\Bild(d_n) = \Kern(d_{n-1}).
	\]
	In diesem Fall induziert jedes $d_n$ einen Isomorphismus
	\begin{align*}
		\wt{d}_n \colon V_n \diagup \Kern(d_n) \xrightarrow{\simeq} \Bild(d_n)
	\end{align*}
	bzw.
	\begin{align*}
		\wt{d}_n \colon V_n \diagup \Kern(d_{n+1}) \xrightarrow{\simeq} \Bild(d_{n-1}).
	\end{align*}
	Für jede Sequenz gilt $\Bild(d_n) \subseteq \Kern(d_{n-1})$, da $d_{n-1} \circ d_n = 0$, und die Sequenz ist genau dann exakt, wenn die Quotientenräume
	\[
		H_n := \Kern(d_{n-1}) \diagup \Bild(d_n)
	\]
	alle trivial sind.
	Man nennt $H_n$ die $n$-te \Index{Homologie} der Sequenz $(V_n,d_n)$.
\end{definition}

Solche Sequenzen spielen eine ganz wichtige Rolle in der modernen Mathematik, insbesondere in der algebraischen Topologie, wo man etwa jedem topologischen Raum auf bestimmte Weise solche Sequenzen zuordnet, und die Homologien dieser Ketten dann wichtige Invarianten der topologischen Räume liefern.
Ist zum Beispiel $H_n(X) \neq H_n(Y)$ für zwei topologische Räume, so sind $X$ und $Y$ nicht homöomorph.

Wichtiger Spezialfall:
Eine \textbf{kurze Sequenz} ist eine Sequenz der Form \index{Sequenz!kurz}
\[
	0 \rightarrow U \xrightarrow{\varphi} V \xrightarrow{\psi} W \rightarrow 0,
\]
das heißt $\varphi, \psi$ sind linear mit $\psi \circ \phi = 0$.
Eine kurze Sequenz ist exakt, falls $\Kern(\varphi) = \setzero, \Bild(\Phi) = \Kern(\Psi), \Bild(\Psi) = W$.
Wir erhalten dann mit dem Isomorphiesatz via $\varphi \colon U \rightarrow \varphi(U)$, da $\varphi$ injektiv und $\psi$ surjektiv ist:
\[
	U \simeq \varphi(U) \quad \text{und} \quad V \diagup \varphi(U) \xrightarrow[\simeq]{\wt{\varphi}} W.
\]
\cleardoubleoddemptypage