%!TEX root = ../LA2.tex
\section{Diagonalisierbarkeit normaler Endomorphismen}
\label{sec:2.7}

In diesem Abschnitt wollen wir die Eigenwerttheorie benutzen, um selbstadjungierte oder unitäre Endormorphismen genauer zu untersuchen.
Beide sind \enquote{normal} im folgenden Sinne:

\begin{definition}
	\label{def:7.1}
	Sei $(V,\sk{\cdot,\cdot}$ ein endlich dimensionaler unitärer bzw. euklidischer $\KK$-Vektorraum.
	Ein Endomorphismus $F \in \End(V)$ heißt \Index{normal}, falls $F^* \circ F = F \circ F°*$.
	
	Analog heißt eine Matrix $A \in M(n \times n,\KK)$ \Index{normal}, wenn $A^*A = AA^*$ gilt.
\end{definition}

Ist $F = F^*$, so ist $F$ auch normal, da $F^* \circ F = F \circ F = F \circ F^*$.
Ist $F$ unitär bzw. orthogonal, so ist $F$ ebenfalls normal, da dann $F^* \circ F = \id_V = F \circ F^*$ gilt (analog für Matrizen).

Wir wollen im Folgenden zeigen, dass jeder normale Endomorphismus über $\CC$ diagonalisierbar ist.
Es gibt dann sogar eine Orthonormalbasis aus Eigenvektoren!
Analoges folgt dann auch für selbstadjungierte Endomorphismen über $\RR$.
Ferner werden wir die Struktur von beliebigen orthogonalen Endomorphismen bzw. Matrizen verstehen.

\begin{lemma}
	\label{lemma:7.2}
	Sei $(V,\sk{\cdot,\cdot})$ ein endlich dimensionaler unitärer $\CC$-Vektorraum und sei $F \in \End(V)$ normal.
	Dann gelten:
	\begin{enumerate}[(i)]
		\item Ist $\lambda \in \CC$ ein Eigenwert von $F$, so ist $\ol{\lambda}$ ein Eigenwert von $F^*$ und es gilt $E_\lambda(F) = E_{\ol{\lambda}}(F^*)$.
		\item Sind $\lambda_1,\lambda_2 \in \CC$ Eigenwerte von $F$ mit $\lambda_1 \neq \lambda_2$, so gilt $E_{\lambda_1}(F) \perp E_{\lambda_2}(F)$.
	\end{enumerate}
\end{lemma}

\begin{beweis}
	\mbox{} \\[-.9cm]
	\begin{enumerate}[(i)]
		\item Da $F$ normal ist, ist auch $\lambda \cdot \id -F$ normal, denn
		\[
			(\lambda \cdot \id - F)^* \circ (\lambda \cdot \id - F) = \lambda \ol{\lambda} \id - \ol{\lambda} F - \lambda F^* + F \circ F^* = \lambda \ol{\lambda} \id - \lambda F^* - \ol{\lambda} F + F^* \circ F = (\lambda \cdot \id - F) \circ (\lambda \cdot \id - F)^*.
		\]
		Nun gilt für alle $G \in \End(V)$ mit $G$ normal, dass $\Kern(G) = \Kern(G^*)$, denn:
		\begin{align*}
			v \in \Kern(G) \quad &\Leftrightarrow \quad 0 = \sk{Gv,Gv} = \sk{G^* \circ Gv,v} = \sk{G \circ G^*v,v} = \sk{G^*v,G^*v} \\
			&\Leftrightarrow \quad v \in \Kern(G^*).
		\end{align*} 
		Mit $G = \lambda \id - F$ folgt
		\[
			E_\lambda(F) = \Kern(\lambda \id - F) = \Kern(\lambda \id - F)^* = \Kern(\ol{\lambda} \id - F^*) = E_{\ol{\lambda}}(F^*).
		\]
		\item Sind $v \in E_{\lambda_1}(F), w \in E_{\lambda_2}(F) = E_{\ol{\lambda_2}}(F^*)$, so folgt
		\[
			\lambda_1 \sk{v,w} = \sk{\lambda_1v,w} = \sk{F(v),w} = \sk{v,F^*(w)} = \sk{v,\ol{\lambda_2} w} = \lambda_2 \sk{v,w},
		\]
		also $(\lambda_1 - \lambda_2)\sk{v,w} = 0$.
		Da $\lambda_1 \neq \lambda_2$, folgt $\sk{v,w} = 0$. \qedhere
	\end{enumerate}
\end{beweis}

\begin{satz}
	\label{satz:7.3}
	Sei $(V,\sk{\cdot,\cdot})$ ein endlich dimensionaler unitärer $\CC$-Vektorraum und sei $F \in \End(V)$ normal.
	Dann besitzt $V$ eine Orthonormalbasis $B = \{v_1,\dots,v_n\}$ aus Eigenvektoren von $F$.
	Insbesondere ist $F$ diagonalisierbar.
\end{satz}

\begin{beweis}
	Seien $\lambda_1,\dots,\lambda_k$ die paarweise verschiedenen Eigenwerte von $F$ (da $\KK = \CC$, existiert mindestens ein Eigenwert, da das charakteristische Polynom von $F$ mindestens eine Nullstelle besitzt).
	Seien $E_{\lambda_1}(F), \dots E_{\lambda_k}(F)$ die zugehörigen Eigenräume.
	Bestimme (etwa mit dem Schmidtschen Verfahren) eine Orthonormalbasis $\{v_{i_1},\dots v_{i_{n_i}}\}$ für $E_{\lambda_i}(F)$ für alle $1 \leq i \leq k$.
	Nach \autoref{lemma:7.2} gilt $E_{\lambda_i}(F) \perp E_{\lambda_j}(F)$ für $i \neq j$, und damit ist
	\[
		B = \{v_{11},\dots,v_{1,n_1},v_{21},\dots,v_{2,n_2},\dots,v_{k1},\dots,v_{k,n_k}\}
	\]
	ein Orthonormalsystem aus Eigenvektoren von $F$ in $V$.
	
	Zeige nun $V = \LH(B)$ (dann ist $B$ die gesuchte Orthonormalbasis).
	Sei dazu $W := \LH(B) = \bigoplus_{i=1}^k E_{\lambda_i}(V)$.
	Dann gilt $W \neq V \Leftrightarrow W^\perp \neq \setzero$.
	Wir wollen dies zu einem Widerspruch führen!
	
	Angenommen, $W^\perp \neq \setzero$.
	Dann gilt $F(W^\perp) \subseteq W^\perp$, das heißt $F \big|_{W^\perp} \in \End(W^\perp)$, denn:
	Da $E_{\lambda_i}(F) = E_{\ol{\lambda_i}}(F^*)$, folgt $F^*(E_{\lambda_i}(F)) \subseteq E_{\lambda_i}(F)$ und damit $F^*(W) \subseteq W$, da $W = \bigoplus_{i=1}^k E_{\lambda_i}(F)$.
	Dann folgt für alle $w \in W$ und $v \in W^\perp$:
	\[
		\sk{F(v),w} = \sk{v,F^*(w)} = 0,
	\]
	also $F(v) \in W^\perp$.
	
	Da $\KK= \CC$ besitzt $F \big|_{W^\perp}$ mindestens einen Eigenwert $\mu \in \CC$, und dann existiert ein Eigenvektor $v \in W^\perp \setminus \setzero$ für $\mu$.
	Da $W^\perp \subseteq V$, ist $\mu$ auch Eigenwert von $F$ und $v \in E_\mu(F)$, das heißt es existiert ein $i \in \{1,\dots,k\}$ mit $\mu = \lambda_i$ und dann $v \in E_{\lambda_i}(F) \subseteq W$.
	Damit ist $v \in W \cap W^\perp = \setzero$.
	Widerspruch zu $v \neq 0$. \qedhere
\end{beweis}

\begin{korollar}
	\label{kor:7.4}
	Sei $A \in M(n\times n,\CC)$ normal.
	Dann existiert eine unitäre Matrix $U \in \UU(n)$, sodass $U^*AU$ eine Diagonalmatrix ist.
\end{korollar}

\begin{beweis}
	Betrachte $\CC^n$ mit dem Standard-Skalarprodukt und $F_A \colon \CC^n \rightarrow \CC^n, x \mapsto Ax$.
	Dann ist $F_A$ normal und ist $B = \{v_1,\dots,v_n\}$ eine Orthonormalbasis von $\CC^n$ wie in \autoref{satz:7.3}, so ist $U = (v_1,\dots,v_n)$ invertierbar und $U^{-1} A U$ ist eine Diagonalmatrix (vgl. Lineare Algebra I, \ref{sec:I.6}).
	Nach \autoref{satz:6.9} ist $U$ unitär und $U^* = U^{-1}$. \qedhere
\end{beweis}

Wir wollen nun die reelle Situation untersuchen.
Sei also im Folgenden $(V,\sk{\cdot,\cdot})$ ein endlich dimensionaler euklidischer $\RR$-Vektorraum und sei $F \in \End(V)$ normal.
Dann ist $F$ nicht immer über $\RR$ diagonalisierbar.
Zum Beispiel ist
\[
	O(\alpha) = \begin{pmatrix}
		\cos(\alpha) & -\sin(\alpha) \\
		\sin(\alpha) & \cos(\alpha)
	\end{pmatrix}
\]
diagonalisierbar genau dann, wenn $\alpha = 0,\pi$, aber $O(\alpha)$ ist normal für alle $\alpha \in \RR$, da orthogonal.
Nach \autoref{satz:7.3} bzw. \autoref{kor:7.4} ist $O(\alpha)$ aber über $\CC$ diagonalisierbar.
Wir wollen dieses nutzen, um auch die reelle Situation zu verstehen.

\begin{definition}[Komplexifizierung]
	\label{def:7.5}
	Sei $(V,\sk{\cdot,\cdot})$ ein euklidischer $\RR$-Vektorraum.
	Wir setzen dann
	\[
		V_{\CC} := \{ v+iw : v,w \in V\}
	\]
	versehen mit der Addition und skalaren Multiplikation
	\begin{align*}
		(v_1 + iw_1) + (v_2 + iw_2) &:= (v_1+v_2) + i(w_1 + w_2) \\
		(a+ib) \cdot (v + iw) &:= (av - bw) + i (bv + aw)
	\end{align*}
	und dem komplexwertigen Skalarprodukt
	\[
		\sk{v_1+iw_1,v_2+iw_2}_{\CC} := \sk{v_1,v_2} + \sk{w_1,w_2} + i(\sk{w_1,v_2}-\sk{v_1,w_2}).
	\]
	Dann gilt:
	$(V_{\CC},\sk{\cdot,\cdot}_{\CC})$ ist ein unitärer $\CC$-Vektorraum mit $\dim_{\CC}(V_{\CC}) = \dim_{\RR}(V)$.
	Ist $F \in \End(V)$, so wird durch
	\begin{align*}
		F_{\CC}\colon V_{\CC} &\longrightarrow V_{\CC} \\
		v+iw &\longmapsto F(v) + i F(w)
	\end{align*}
	ein Endomorphismus $F_{\CC} \in \End(V_{\CC})$ definiert.
	$(V_{\CC},\sk{\cdot,\cdot}_{\CC})$ heißt die \Index{Komplexifizierung} von $(V,\sk{\cdot,\cdot})$ und $F_{\CC}$ heißt die Komplexifizierung von $F$.
\end{definition}

\begin{beispiel}
	\label{bsp:7.6}
	Ist $V = \RR^n$ mit Standard-Skalarprodukt, so ist $V_{\CC} = \CC^n$ mit Standard-Skalarprodukt, denn jeder Vektor $z \in \CC^n$ besitzt eine eindeutige Zerlegung $z = x+ iy$ mit $x,y \in \RR^n$.
	
	Ist $A \in M(n \times n,\RR)$ und $F_A\colon \RR^n \rightarrow \RR^n, x \mapsto Ax$, so ist $(F_A)_{\CC}(z) = Az$, wenn wir $A$ als komplexe Matrix auffassen.
\end{beispiel}

\begin{lemma}
	\label{lemma:7.7}
	Sei $(V,\sk{\cdot,\cdot})$ ein endlich dimensionaler euklidischer $\RR$-Vektorraum und sei $F \in \End(V)$.
	Dann sind äquivalent:
	\begin{enumerate}[(i)]
		\item $F$ ist normal (bzw. orthogonal bzw. selbstadjungiert)
		\item $F_{\CC}$ ist normal (bzw. unitär bzw. selbstadjungiert)
	\end{enumerate}
	Ferner gilt $F_{\CC}^* = (F^*)_{\CC}$.
\end{lemma}

\begin{beweis}
	Wir zeigen nur $F_{\CC}^* = (F^*)_{\CC}$.
	Die Äquivalenz der beiden Aussagen folgt dann leicht.
	Für alle $v_1+iw_1, v_2+iw_2 \in V_{\CC}$ gilt:
	\newpage
	\vspace*{-1.3cm}
	\begin{align*}
	\sk{F_{\CC}(v_1+iw_1),v_2+iw_2} &= \sk{F(v_1) + i F(w_1),v_2+iw_2} \\
	&= \sk{F(v_1),v_2} + \sk{F(w_1),w_2} + i( \sk{F(w_1),v_2} - \sk{F(v_1),w_2}) \\
	&= \sk{v_1,F^*(v_2)} + \sk{w_1,F^*(w_2)} + i(\sk{w_1,F^*(v_2)} - \sk{v_1,F^*(w_2)}) \\
	&= \sk{v_1 + iw_2,F^*(v_2) + iF^*(w_2)} \\
	&= \sk{v_1 + iw_1, (F^*)_{\CC}(v_2+iw_2)} \qedhere
	\end{align*}
\end{beweis}

\begin{bemerkung}
	\label{bem:7.8}
	Ist $(V,\sk{\cdot,\cdot})$ ein euklidischer $\RR$-Vektorraum und ist $B := \{v_1,\dots,v_n\}$ eine Orthonormalbasis von $V$, so ist $B$ auch eine Orthonormalbasis von $V_{\CC}$ (wir fassen $V$ als Teilmenge von $V_{\CC}$ auf via $v \mapsto v + i \cdot 0$).
	Denn es gilt $\sk{v_i,v_j}_{\CC} = \sk{v_i,v_j} = \delta_{ij}$ und $\dim(V_{\CC}) = \dim(V) = n$.
	
	Ist $F \in \End(V)$, so erhält man für die zugehörige Darstellungsmatrix $A^B_{F_{\CC}} = A^B_F$, denn ist $A^B_{F_{\CC}} = (\wt{a}_{ij}), A^B_F = (a_{ij})$, so gilt nach \autoref{kor:5.4}:
	\[
		\wt{a}_{ij} = \sk{F_{\CC}(v_j + i0),v_i + i0}_{\CC} = \sk{F(v_j)+i0,v_i+i0} = \sk{F(v_j),v_i} = a_{ij}.
	\]
\end{bemerkung}

Wir wollen aber nun umgekehrt aus einer Orthonormalbasis von $V_{\CC}$ aus Eigenvektoren von $F_{\CC}$ eine \enquote{schöne} Orthonormalbasis für $V$ basteln.
Dazu benutzen wir zunächst, dass jedes reelle Polynom $p(\lambda) = \sum_{k=0}^{n} a_k \lambda^k$ durch Einsetzen von $\lambda \in \CC$ auch als komplexes Polynom aufgefasst werden kann.

\begin{lemma}
	\label{lemma:7.9}
	Sei $(V,\sk{\cdot,\cdot})$ ein endlich dimensionaler euklidischer $\RR$-Vektorraum und sei $F \in \End(V)$.
	Dann gilt $\chi_{F_{\CC}}(\lambda) = \chi_F(\lambda)$ für alle $\lambda \in \CC$, das heißt das charakteristische Polynom von $F_{\CC}$ ist gleich dem charakteristischen Polynom von $F$ (fortgesetzt auf $\CC$).
\end{lemma}

\begin{beweis}
	Wähle eine Orthonormalbasis $B = \{v_1,\dots,v_n\}$ von $V$ wie in \autoref{lemma:7.7}.
	Dann folgt mit \autoref{bem:7.8}:
	$\chi_{F_{\CC}}(\lambda) = \det(\lambda E_n - A^B_{F_{\CC}}) = \det(\lambda E_n - A^B_F) = \chi_F(\lambda)$. \qedhere
\end{beweis}

\begin{lemma}
	\label{lemma:7.10}
	Sei $p(\lambda) = \sum_{k=0}^{n} a_k \lambda^k$ ein reelles Polynom.
	Dann gilt:
	Ist $\mu \in \CC$ eine $k$-fache komplexe Nullstelle von $p$, so ist auch $\ol{\mu} \in \CC$ eine $k$-fache komplexe Nullstelle von $p$.
	Damit besitzt $p$ eine Faktorisierung 
	\[
		p(\lambda) = a(\lambda-\lambda_1)^{n_1} \cdots (\lambda-\lambda_l)^{n_l}(\lambda-\mu_1)^{m_1} (\lambda-\ol{\mu}_1)^{m_1} \cdots (\lambda-\mu_k)^{m_k}(\lambda-\ol{\mu}_k)^{m_k},
	\]
	wobei $\lambda_1,\dots,\lambda_l$ die paarweise verschiedenen reellen Nullstellen mit Vielfachheiten $n_1,\dots,n_l$ und \linebreak $\mu_1,\ol{\mu}_1,\dots,\mu_l,\ol{\mu}_l$ die paarweise verschiedenen konjugiert komplexen Paare nicht-reller Nullstellen mit Vielfachheiten $m_1,\dots,m_k$ von $p$ sind.
\end{lemma}

\begin{beweis}
	Sei $\mu \in \CC$ Nullstelle von $p$.
	Dann gilt
	\[
		p(\ol{\mu}) = \sum_{k=0}^{n} a_k \ol{\mu}^k = \sum_{k=0}^{n} \ol{a_k} \ol{\mu}^k = \ol{\sum_{k=0}^{n} a_k \mu^k} = \ol{p(\mu)} = 0,
	\]
	also ist auch $\ol{\mu}$ Nullstelle von $p$.
	Ist $\mu \notin \RR$, so ist $\ol{\mu} \neq \mu$.
	Ferner gilt mit $\mu = \alpha + i \beta$:
	\[
		(\lambda-\mu)(\lambda-\ol{\mu}) = (\lambda - (\alpha-i\beta))(\lambda-(\alpha-i\beta)) = \lambda^2 + 2\alpha \lambda + (\alpha^2 + \beta^2),
	\]
	also ist $(\lambda-\mu)(\lambda-\ol{\mu})$ ein reelles Polynom.
	Damit existiert ein reelles Polynom mit $\grad(q) = \grad(p) - 2$ und $p(\lambda) = q(\lambda)(\lambda-\mu)(\lambda-\ol{\mu})$.
	Die Behauptung folgt dann per Induktion nach $\grad(p)$. \qedhere
\end{beweis}

% \newpage