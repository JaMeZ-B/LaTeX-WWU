\section{Summen und direkte Summen von Vektorräumen}

\begin{definition}[Summe von Untervektorräumen]
	\label{def:I.12.1}
	Sei $V$ ein $K$-Vektorraum und $W_1,W_2 \subseteq V$ Untervektorräume von $V$.
	Dann heißt der Untervektorraum
	\[
		W_1 + W_2 := \{w_1 + w_2 : w_1 \in W_1, w_2 \in W_2\} \subseteq V
	\]
	die \Index{Summe} von $W_1$ und $W_2$.
\end{definition}

\begin{satz}[Dimensionsformel für Untervektorräume]
	\label{satz:I.12.2}
	Sei $V$ ein $K$-Vektorraum und $W_1,W_2 \subseteq V$ endlich dimensionale Untervektorräume von $V$.
	Dann gilt
	\[
		\dim_K(W_1+W_2) = \dim_K(W_1) + \dim_K(W_2) - \dim_K(W_1 \cap W_2).
	\]
\end{satz}

\begin{definition}[Direkte Summe]
	\label{def:I.12.3}
	Sei $V$ ein $K$-Vektorraum und $W_1,W_2 \subseteq V$ Untervektorräume von $V$.
	Dann heißt $V$ die \Index{direkte Summe} von $W_1$ und $W_2$, falls gilt:
	\begin{enumerate}[(1)]
		\item	$V = W_1 + W_2$
		\item	$W_1 \cap W_2 = \setzero$
	\end{enumerate}
	Wir schreiben dann $V = W_1 \oplus W_2$.
\end{definition}

\begin{satz}
	\label{satz:I.12.4}
	Sei $V$ ein $K$-Vektorraum und $W_1,W_2 \subseteq V$ Untervektorräume von $V$.
	Dann gelten:
	\begin{enumerate}[(1)]
		\item Ist $V = W_1 \oplus W_2$ und ist $B_1$ eine Basis von $W_1$ und $B_2$ eine Basis von  $W_2$, so ist $B_1 \cup B_2$ eine Basis von $V$.
		\item Ist $V = W_1 \oplus W_2$, so ist jedes $v \in V$ eindeutig darstellbar als Summe $v = w_1 + w_2$ mit $w_1 \in W_1$ und $w_2 \in W_2$.
		\item Ist $\dim(V) = n$, so sind äquivalent:
		\begin{enumerate}[a)]
			\item $V = W_1 \oplus W_2$
			\item $V = W_1 + W_2$ und $\dim(W_1) + \dim(W_2) = n$.
			\item $W_1 \cap W_2 = \setzero$ und $\dim(W_1) + \dim(W_2) = n$.
		\end{enumerate}
	\end{enumerate}
\end{satz}

\begin{satz}[Existenz von Projektionen]
	\label{satz:I.12.5}
	Sei $V$ ein $K$-Vektorraum und $W_1,W_2 \subseteq V$ Untervektorräume mit $V = W_1 \oplus W_2$.
	Dann existieren für $i = 1,2$ eindeutige lineare Abbildungen
	\begin{align*}
		P_i\colon V &\longrightarrow W_i \\
		w_1 + w_2 &\longmapsto w_i.
	\end{align*}
	$P_1$ und $P_2$ heißen \textbf{Projektionen} auf $W_1$ und $W_2$.
	$P_1$ und $P_2$ sind als Abbildungen $V \rightarrow V$ \Index{idempotent}, das heißt es gilt $P_1 \circ P_1 = P_1$ bzw. $P_2 \circ P_2 = P_2$. \index{Projektion}
\end{satz}

\setcounter{definition}{6}
\begin{satz}
	\label{satz:I.12.7}
	Sei $P \colon V \rightarrow V$ eine idempotente lineare Abbildung.
	Dann gilt $V = \Bild(P) \oplus \Kern(P)$ und $P\colon V \rightarrow \Bild(P)$ ist die zugehörige Projektion auf den ersten Summanden.
\end{satz}

\begin{satz}[Existenz von Komplementärräumen]
	\label{saz:I.12.8}
	Sei $V$ ein endlich-dimensionaler $K$-Vektorraum und sei $W \subseteq V$ ein Untervektorraum.
	Dann existiert ein Untervektorraum $\widetilde{W} \subseteq V$ mit $V = W \oplus \widetilde{W}$.
	Der Raum $\widetilde{W}$ heißt \Index{Komplementärraum} von $W$ in $V$.
\end{satz}