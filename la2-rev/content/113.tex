%!TEX root = ../LA2_SS16.tex
\subsection{Determinanten}

\begin{definition}[Determinante]
	\label{def:I.13.1}
	Sei $K$ ein Körper und sei $n \in \NN$ fest.
	Eine \Index{Determinante} ist eine Abbildung
	\begin{align*}
		\Det_n\colon M(n \times n,K) &\longrightarrow K \\
		A &\longmapsto \Det_n(A)
	\end{align*}
	mit folgenden Eigenschaften:
	\begin{description}
		\item[(D1)] $\det_n(E_n) = 1$, das heißt $\det_n$ ist normiert.
		\item[(D2)] Besitzt $A$ zwei gleiche Spalten, so gilt $\det_n(A) = 0$, das heißt $\det_n$ ist alternierend.
		\item[(D3)] $\det_n$ ist linear in jeder Spalte, das heißt für alle $\lambda \in K$ und $a_1,\dots,a_n,a_j' \in K^n$ mit $1 \leq j \leq n$ gilt
		\[
			\Det_n(a_1,\dots,a_{j-1},\lambda a_j + a_j',a_{j+1},\dots,a_n) = \lambda\cdot \Det_n(a_1,\dots,a_j,\dots,a_n) + \Det_n(a_1,\dots,a_j',\dots,a_n).
		\]  
	\end{description}
\end{definition}

\begin{satz}[Eigenschaften von Determinanten]
	\label{satz:I.13.2}
	Sei $\det_n\colon M(n \times n,K) \rightarrow K$ wie in Definition~\ref{def:I.13.1}.
	Dann gelten:
	\begin{description}
		\item[(D4)] Entsteht $A'$ aus $A$ durch Vertauschen zweier unterschiedlicher Spalten, so gilt $\det_n(A') = -\det_n(A)$.
		\item[(D5)] Ist $A = (a_1,\dots,a_n)$ mit $a_j = 0$ für ein $1 \leq j \leq n$, so gilt $\det_n(A) = 0$.
		\item[(D6)] Für alle $i \neq j$ und $\lambda \in K$ gilt
		\[
			\Det_n(a_1,\dots,a_{j-1},a_j+\lambda a_i,a_{j+1},\dots,a_n) = \Det_n(a_1,\dots,a_n).
		\] 
	\end{description}
\end{satz}

\setcounter{satz}{4}
\begin{satz}[Mehr Eigenschaften von Determinanten]
	\label{satz:I.13.5}
	Sei $\det_n\colon M(n\times n,K)$ eine Determinante wie in Definition~\ref{def:I.13.1} und $A,B \in M(n \times n,K)$.
	Dann gelten:
	\begin{enumerate}[(i)]
		\item	$\det_n(A) \neq 0 \Leftrightarrow A \in \GL(n,K) \Leftrightarrow \Rang(A) = n \Leftrightarrow \text{ Die Spalten bzw. Zeilen von } A \text{ sind linear unabhängig.}$
		\item Es gilt $\det_n(AB) = \det_n(A) \cdot \det_n(B)$.
		\item Es gilt $\det_n(A^T) = \det_n(A)$ und $\det_n(A^{-1}) = \det_n(A)^{1}$, falls $A$ invertierbar.
	\end{enumerate}
\end{satz}

\setcounter{satz}{7}
\begin{korollar}
	\label{kor:I.13.8}
	Für jeden Körper $K$ und jedes $n \in \NN$ existiert höchstens eine Determinante $\det_n\colon M(n \times n,K)$.
\end{korollar}
\cleardoubleoddemptypage