%!TEX root = ../LA2_SS16.tex
\subsection{Mengen und Abbildungen}
\label{sec:1.2}

\begin{definition}[Menge, \textsc{Cantor}]
	\label{def:I.2.1}
	Eine \Index{Menge} $M$ ist die Zusammenfassung von wohlbestimmten und wohlunterscheidbaren Objekten unserer Anschauung oder unseres Denkens (welche als \textbf{Elemente} von $M$ genannt werden) zu einem Ganzen. \index{Element}
\end{definition}

Beachte: Dieser Mengenbegriff ist problematisch und führt zu Widersprüchen!
Für unsere Zwecke ist er aber zunächst ausreichend (mehr dazu in der Vorlesung \textit{Logische Grundlagen}).

\begin{beispiel}
	\label{bsp:I.2.2}
	\begin{enumerate}[(i)]
		\item $\NN = \{1,2,3,\dots\}$ heißt die Menge der natürlichen Zahlen.
		
		$\ZZ = \{0, \pm 1, \pm 2, \pm 3, \dots\}$ heißt die Menge der ganzen Zahlen.
		
		$\QQ = \penb{\frac{m}{n} : m \in \ZZ, n \in \NN}$ heißt die Menge der rationalen Zahlen.
		
		$\RR = $ Menge der reellen Zahlen (siehe \textit{Analysis I}).
		
		$\CC = $ Menge der komplexen Zahlen (später).
		\item  Die Variante $\RR_{\geq 0} = \{x : x \in \RR \wedge x \geq 0\} = \{x \in \RR : x \geq 0\}$ ist eine wichtige und häufig genutzte Schreibweise für Mengen, bei der wir die Elemente der Menge durch Eigenschaften festlegen.
		\item Die Menge aller Häuser in Münster.
	\end{enumerate}
\end{beispiel}

\begin{definition}[Teilmenge]
	\label{def:I.2.3}
	\begin{enumerate}[(i)]
		\item Sei $M$ eine Menge.
		Wir schreiben $x \in M$ (bzw. $x \notin M$), wenn $x$ ein (bzw. kein) Element von $M$ ist.
		\item $M = \{x,y,z,\dots\}$ bezeichnet die Menge mit den Elementen $x,y,z,\dots$.
		\item $M = \{x : x \text{ hat Eigenschaft } E\}$ bezeichnet die Menge aller Objekte, die die Eigenschaft $E$ besitzen (wie zum Beispiel $\RR_{\geq 0}$).
		\item Eine Menge $M_1$ heißt \Index{Teilmenge} der Menge $M_2$ (schreibe dann $M_1 \subseteq M_2$ oder $M_2 \subset M_2$), falls jedes Element $x$ von $M_1$ auch Element von $M_2$ ist.
		Formal:
		\[
			M_1 \subseteq M_2 \quad \Leftrightarrow \quad [\forall x \in M_1 \text{ gilt } x \in M_2]
		\]
		Es gilt:
		\[
			M_1 \subseteq M_2 \quad \Leftrightarrow \quad [M_1 \subseteq M_2 \wedge M_2 \subseteq M_1]
		\]
		Wir schreiben $M_1 \nsubseteq M_2$, falls $M_1$ nicht Teilmenge von $M_2$ ist.
	\end{enumerate}
\end{definition}

\begin{definition}[Vereinigung, Durchschnitt, Komplement]
	\label{def:I.2.4}
	Seien $M_1, M_2$ Mengen.
	Wir definieren:
	\begin{enumerate}[(i)]
		\item $M_1 \cup M_2 := \{x : x \in M_1 \vee x \in M_2\}$ heißt die \Index{Vereinigung} von $M_1$ und $M_2$.
		\item $M_1 \cap M_2 = \{x : x \in M_1 \wedge x \in M_2\}$ heißt der \Index{Durchschnitt} (oder \Index{Schnitt}) von $M_1$ und $M_2$.
		\item $M_1 \setminus M_2 = \{x : x \in M_1 \wedge x \notin M_2\}$ heißt das \Index{Komplement} von $M_2$ in $M_1$ (oft auch \Index{Differenzmenge} $M_1$ ohne $M_2$ genannt).
	\end{enumerate}
\end{definition}

Wenn $M_1$ und $M_2$ kein Element gemeinsam haben, besitzt $M_1 \cap M_2$ kein Element.
Das führt zu folgender Definition:

\begin{definition}[Leere Menge]
	\label{def:I.2.5}
	Die leere Menge $\emptyset$ (oder auch $\penb{}$) ist die Menge, die kein Element enthält.
	Es gilt $\emptyset \subseteq M$ für jede Menge $M$ (denn die Aussage \enquote{Jedes Element von $\emptyset$ ist auch Element von $M$} ist eine wahre Aussage).
\end{definition}

\begin{satz}
	\label{satz:I.2.6}
	Es seien $M_1, M_2, M_3$ Mengen.
	Dann gelten:
	\begin{enumerate}[(i)]
		\item $M_1 \cup (M_2 \cup M_3) = (M_1 \cup M_2) \cup M_3$
		\item $M_1 \cap (M_2 \cap M_3) = (M_1 \cap M_2) \cap M_3$
		\item $M_1 \cap (M_2 \cup M_3) = (M_1 \cap M_2) \cup (M_1 \cap M_3)$
		\item $M_1 \cup (M_2 \cap M_3) = (M_1 \cup M_2) \cap (M_1 \cup M_3)$
	\end{enumerate}
\end{satz}

\begin{beweis}
	Wir beweisen nur (iii) und lassen den Rest als Übungsaufgabe.
	Es gilt
	\begin{align*}
		&x \in M_1 \cap (M_2 \cup M_3) \\
		\Leftrightarrow \quad &(x \in M_1) \wedge (x \in M_2 \cup M_3) \\
		\Leftrightarrow \quad &(x \in M_1) \wedge (x \in M_2 \vee x \in M_3) \\
		\xLeftrightarrow{\ref{def:I.1.2}} \quad &[(x \in M_1) \wedge (x \in M_2)] \vee [(x \in M_1) \wedge (x \in M_3)] \\
		\Leftrightarrow \quad &x \in (M_1 \cap M_2) \cup (M_1 \cap M_3).
	\end{align*}
\end{beweis}

\begin{definition}
	\label{def:I.2.7}
	Sei $I$ eine beliebige Indexmenge und für jedes $i \in I$ sei $M_i$ eine Menge.
	Wir setzen:
	
	\begin{enumerate}[(i)]
		\item $\bigcup\limits_{i \in I} M_i = \{x : \exists i \in I \text{ mit } x \in M_i\}$
		\item $\bigcap\limits_{i \in I} M_i = \{x : \forall i \in I \text{ gilt } x \in M_i\}$
	\end{enumerate}
\end{definition}

\begin{satz}[Regel von \textsc{de Morgan}]
	\label{satz:I.2.8}
	Sei $M$ eine Menge und seien $M_i$ Mengen für alle $i \in I$.
	Dann gelten: \index{de Morgansche Regeln}
	\begin{enumerate}[(i)]
		\item $M \setminus \enb{\bigcup\limits_{i\in I} M_i} = \bigcap\limits_{i \in I} (M \setminus M_i)$
		\item $M \setminus \enb{\bigcap\limits_{i\in I} M_i} = \bigcup\limits_{i \in I} (M \setminus M_i)$
	\end{enumerate}
\end{satz}

\begin{beweis}
	Der Beweis von (i) benutzt unsere Verneinungsregeln:
	\begin{align*}
			&x \in M \setminus \enb{\bigcup\limits_{i \in I} M_i} \\
		\Leftrightarrow \quad &(x \in M) \wedge \enb{x \notin \bigcup\limits_{i \in I} M_i} \\
		\Leftrightarrow \quad &(x \in M) \wedge \neg(\exists i \in I \text{ mit } x \in M_i) \\
		\xLeftrightarrow{\ref{lemma:I.1.6}} \quad &(x \in M) \wedge \forall i \in I \text{ gilt } x \notin M_i \\
		\Leftrightarrow \quad &\forall i \in I \text{ gilt } x \in M \wedge x \in M_i \\
		\Leftrightarrow \quad &\forall i \in I \text{ gilt } x \in M \setminus M_i \\
		\Leftrightarrow \quad &x \in \bigcap\limits_{i \in I} (M \setminus M_i).
	\end{align*}
	Der Beweis von (ii) geht sehr ähnlich.
\end{beweis}

Wir wollen nun Abbildungen zwischen Mengen betrachten:

\begin{definition}[Abbildung, Funktion, Urbildraum, Bildraum]
	\label{def:I.2.9}
	Es seien $X,Y$ Mengen.
	Eine \Index{Abbildung} (oder \Index{Funktion}) $f\colon X \rightarrow Y$ ist eine Vorschrift, die jedem $x \in X$ genau ein $f(x) \in Y$ zuordnet.
	$X$ heißt dann \Index{Urbildraum} (oder \Index{Definitionsbereich}) von $f$.
	Die Menge $Y$ heißt der \Index{Bildraum} (oder \Index{Zielraum}) von $f$.
\end{definition}
\newpage
\textbf{Beispiele:}
\begin{enumerate}[a)]
	\item $f\colon \NN \rightarrow \NN, f(n) = n+1$
	\item $f \colon \RR \rightarrow \RR, f(x) = x^2$
	\item $f \colon \RR \rightarrow [0,\infty), f(x) = x^2$
	\item $f \colon \RR \setminus \setzero, f(x) = \frac{1}{x}$
	\item Ist $\setzero \neq Y$ eine Menge, so heißt $\id_X \colon X \rightarrow X, \id_X(x) = x$ die \Index{Identität} (oder identische Abbildung) auf $X$.
\end{enumerate}

Beachte: Die Abbildungen in b) und c) sind verschieden, da sie verschiedene Bildräume haben!
Außerdem: $f \colon \RR \rightarrow \RR, f(x) = \frac{1}{x}$ ergibt keinen Sinn, da $\frac{1}{0}$ nicht definiert ist. Aber:

\[
	f\colon \RR \rightarrow \RR, f(x) = \begin{cases}
		\frac{1}{x}, & \text{falls } x \neq 0 \\
		0, & \text{falls } x = 0
	\end{cases}
\]
ist in Ordnung.

Wir benutzen auch oft die Schreibweisen
\[
	f \colon X \rightarrow Y, x \mapsto f(x) \qquad \text{bzw.} \qquad X \rightarrow Y, x \mapsto f(x).
\]

Ist $X \neq \emptyset$, $Y = \emptyset$, so kann es keine Abbildung $f \colon X \rightarrow Y$ geben.

\begin{definition}[Komposition]
	\label{def:I.2.10}
	Seien $f \colon X \rightarrow Y, g \colon Y \rightarrow Z$ zwei Abbildungen.
	Dann ist die \Index{Komposition} (Hintereinanderausführung) $g \circ f \colon X \rightarrow Z$ definiert durch $g \circ f(x) = g(f(x))$ für alle $x \in X$.
\end{definition}

\begin{definition}[Bild, Urbild]
	\label{def:I.2.11}
	Sei $f \colon X \rightarrow Y$ eine Abbildung.
	Dann definieren wir:
	\begin{enumerate}[(i)]
		\item Ist $A \subseteq X$, so heißt $f(A) := \{y \in Y : \exists x \in A \text{ mit } f(x) = y \} = \{f(x) : x \in A\} \subseteq Y$ das \Index{Bild} von $A$ unter $f$.
		\item Ist $B \subseteq Y$, so heißt $f^{-1}(B) := \{x \in X : f(x) \in B\} \subseteq X$ das \Index{Urbild} von $V$ unter $f$.
	\end{enumerate}
\end{definition}
\newpage
Beispiel: Sei $f \colon \RR \rightarrow \RR, f(x) = x^2$. Dann ist
\begin{itemize}
	\item $f(\RR) = [0,\infty)$
	\item $f([0,1]) = [0,1]$
	\item $f^{-1}([-1,1]) = f^{-1}([0,1]) = [-1,1]$
	\item $f^{-1}((-\infty,0]) = \setzero$
	\item $f^{-1}((-\infty,-1]) = \emptyset$
	\item \dots
\end{itemize}

Sehr wichtig sind auch die folgenden Begriffe:

\begin{definition}[surjektiv, injektiv, bijektiv]
	\label{def:I.2.12}
	Sei $f \colon X \rightarrow Y$ eine Abbildung.
	Wir definieren:
	\begin{enumerate}[(i)]
		\item $f$ heißt \Index{surjektiv}, falls $f(X) = Y$ (bzw. $\forall y \in Y \ \exists x \in X \text{ mit } f(x) = y$).
		\item $f$ heißt \Index{injektiv}, falls für alle $x_1,x_2 \in X$ gilt: $x_1 \neq x_2 \Rightarrow f(x_1) \neq f(x_2)$ (bzw. $f(x_1) = f(x_2) \Rightarrow x_1 = x_2$).
		\item $f$ heißt \Index{bijektiv}, falls $f$ injektiv und surjektiv ist (bzw. für alle $y \in Y$ existiert \textit{genau} ein $x \in X$ mit $f(x) = y$).
	\end{enumerate}
\end{definition}

Bijektive Abbildung sind genau die Abbildungen, die eine Umkehrabbildung besitzen:

\begin{satz}[Umkehrabbildung]
	\label{satz:I.2.13}
	Sei $f \colon X \rightarrow Y$ eine Abbildung.
	Dann sind äquivalent:
	\begin{enumerate}[(i)]
		\item $f$ ist bijektiv.
		\item Es existiert eine Abbildung $g \colon Y \rightarrow X$ mit $g(f(x))=x$ für alle $x \in X$ und $f(g(y)) = y$ für alle $y \in Y$ (bzw. kurz: $g \circ f = \id_X$ und $f \circ g = \id_Y$).
	\end{enumerate}
	Die Abbildung $g$ in (ii) ist eindeutig bestimmt und heißt die \Index{Umkehrabbildung} (oder \Index{inverse Abbildung}) von $f$.
	Wir schreiben dann $g =: f^{-1}$.	
\end{satz}

\begin{beweis}
	\begin{description}
		\item[(i) $\Rightarrow$ (ii):] Nach Definition gilt: $f$ ist bijektiv genau dann, wenn zu jedem $y \in Y$ genau ein $x_y \in X$ existiert mit $f(x_y) = y$.
		Setze $g(y) = x_y$.
		Dann folgt $f(g(y)) = y$ für alle $y \in Y$, und für alle $x \in X$ gilt $x = x_{f(x)} = g(f(x))$.
		\item[(ii) $\Rightarrow$ (i):] Sei $g \colon Y \rightarrow X$ eine Abbildung mit $g \circ f = \id_X, f \circ g = \id_Y$.
		Dann ist $f$ injektiv, denn sind $x_1,x_2 \in X$ mit $f(x_1) = f(x_2)$, so folgt $x_1 = g(f(x_1)) = g(f(x_2)) = x_2$.
		$f$ ist auch surjektiv, da für alle $y \in Y$ gilt: $f(g(y))= y$.
		\item[Eindeutigkeit:] Sei $\wt{g} \colon Y \rightarrow X$ eine weitere Funktion wie in (ii).
		Dann folgt für alle $y \in Y$:
		Ist $x \in X$ mit $f(x) = y$ (existiert, da $f$ surjektiv), so gilt
		\[
			\wt{g}(y) = \wt{g}(f(x)) \stackrel{\text{(ii)}}{=} x \stackrel{\text{(ii)}}{=} g(f(x)) = g(y),
		\]
		also $\wt{g} = g$.
	\end{description}	
\end{beweis}

Achtung: Der Begriff $f^{-1} \colon Y \rightarrow X$ für die Umkehrfunktion einer bijektiven Abbildung $f \colon X \rightarrow Y$ führt oft zu Missverständnissen in Bezug auf das Urbild $f^{-1}(B)$ einer Teilmenge $B \subseteq Y$.

Beachte: $f^{-1}(B)$ für $B \subseteq Y$ ist für \textit{jede} Abbildung $f \colon X \rightarrow Y$ definiert, aber $f^{-1}(y)$ für $y \in Y$ nur dann, wenn $f$ bijektiv ist!
Wir unterscheiden auch immer streng zwischen 
\[
	f^{-1}(y) \in X \qquad \text{und} \qquad f^{-1}(\{y\}) = \{x \in X : f(x) = y\} \subseteq X
\]

\begin{lemma}
	\label{lemma:I.2.14}
	Seien $f \colon X \rightarrow Y, g \colon Y \rightarrow Z$ Abbildungen.
	Dann gelten:
	Sind $f$ und $g$ injektiv (bzw. surjektiv, bzw. bijektiv), so auch $f \circ g$.
\end{lemma}

\begin{beweis}
	Übungsaufgabe.
\end{beweis}

\begin{beispiel}[Transpositionsabbildung]
	\label{bsp:I.2.15}
	Sei $n \in \NN$ und für $1 \leq i,j \leq n$ sei
	\[
		\tau_{ij}\colon \{1,\dots,n\} \rightarrow \{1,\dots,n\}
	\]
	die Abbildung, die $i$ auf $j$, $j$ auf $i$ und alle anderen $k \in \{1,\dots,n\}$ auf $k$ abbildet (das heißt, $\tau_{ij}$ vertauscht $i$ und $j$ und hält alle anderen Elemente fest).
	Dann ist $\tau_{ij}$ bijektiv.
	Es gilt $\tau_{ii} = \id_{\{1,\dots,n\}}$.
	$\tau_{ij}$ heißt \Index{Transpositionsabbildung}.
\end{beispiel}

\begin{lemma}
	\label{lemma:I.2.16}
	Seien $n,m \in \NN$.
	Dann gelten:
	\begin{enumerate}[(i)]
		\item Existiert eine injektive Abbildung $f \colon \{1,\dots,n\} \rightarrow \{1,\dots,m\}$, so gilt $n \leq m$.
		\item Existiert eine surjektive Abbildung $f \colon \{1,\dots,n\} \rightarrow \{1,\dots,m\}$, so gilt $m \leq n$.
		\item Existiert eine bijektive Abbildung $f \colon \{1,\dots,n\} \rightarrow \{1,\dots,m\}$, so gilt $m = n$.
	\end{enumerate}
\end{lemma}

\begin{beweis}[vollständige Induktion]
	Der Fall $n=1$ ist klar.
	Sei also die Aussage wahr für ein $n \in \NN$, und sei $f\colon \{1, \dots, n+1\} \rightarrow \{1,\dots,m\}$ eine injektive Abbildung.
	Sei $k = f(n+1) \in \{1,\dots,m\}$.
	Dann ist auch $\wt{f}\colon \tau_{km} \circ f \colon \{1,\dots,n\} \rightarrow \{1,\dots,m\}$ injektiv nach Lemma~\ref{lemma:I.2.14} mit $\wt{f}(n+1)=m$.
	Sei $g \colon \{1,\dots,n\} \rightarrow \{1,\dots,m-1\}$ definiert durch $g(l) = \wt{f}(l)$.
	Dann ist auch $g$ injektiv, da $\wt{f}$ injektiv.
	Nach Induktionsannahme gilt $n \leq m-1$, also dann auch $n+1 \leq m$.
	
	Damit ist (i) bewiesen.
	Wir lassen den Beweis von (ii) als Übungsaufgabe.
	(iii) folgt aus (i) und (ii).
\end{beweis}

\begin{definition}[endliche Menge]
	\label{def:I.2.17}
	Eine Menge $M$ heißt \index{endlich}, falls $M = \emptyset$ oder falls ein $n \in \NN$ und eine bijektive Abbildung $f \colon \{1,\dots,n\} \rightarrow M$ existieren.
	In diesem Fall ist $n$ eindeutig bestimmt und wir setzen $\abs{M} := n$.
	Wir nennen $\abs{M}$ die \Index{Mächtigkeit} (Anzahl der Elemente) von $M$.
	Ferner setzen wir $\abs{\emptyset} = 0$ und $\abs{M} = \infty$, falls $M$ nicht endlich ist.
\end{definition}

\begin{beweis}[Eindeutigkeit von $n$]
	Sind $n,m \in \NN$ und $f_1\colon \{1,\dots,n\} \rightarrow M$ und $f_2 \colon \{1,\dots,m\} \rightarrow M$ bijektiv, so ist auch
	\[
		f_2^{-1} \circ f_1 \colon \{1,\dots,n\} \rightarrow \{1,\dots,m\}
	\]
	bijektiv.
	Nach Lemma~\ref{lemma:I.2.16} (iii) folgt $n = m$.
\end{beweis} 
\cleardoubleoddemptypage