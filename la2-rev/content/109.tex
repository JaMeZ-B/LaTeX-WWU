%!TEX root = ../LA2.tex
\section{Berechnung von Basen}

\begin{lemma}[Austauschlemma]
	\label{lemma:I.9.1}
	Sei $\{v_1,\dots,v_l\}$ ein Erzeugendensystem des $K$-Vektorraums $V$ und sei $i \in \{1,\dots,l\}$.
	Dann gilt:
	Ist $w = \sum_{j=1}^{l} \lambda_j v_j$ mit $\lambda_i \neq 0$, so ist auch $\{v_1,\dots,v_{i-1},w,v_{i+1},\dots,v_l\}$ ein Erzeugendensystem von $V$.
\end{lemma}

\begin{satz}[Basisergänzungssatz]
	\label{satz:I.9.2}
	Sei $V$ ein $K$-Vektorraum und sei $\{v_1,\dots,v_n\}$ eine Basis von $V$.
	Sind dann $w_1,\dots,w_r \in V$ linear unabhängig, so gilt $r \leq n$ und es existiert eine Umnummerierung von $\{v_1,\dots,v_n\}$, sodass nach dem Umsortieren das System $\{w_1,\dots,w_r,v_{r+1},\dots,v_n\}$ eine Basis von $V$ ist.
\end{satz}

\begin{satz}
	\label{satz:I.9.3}
	Sei $V$ ein $K$-Vektorraum mit $\dim_K(V) = n$ und seien $B := \{v_1,\dots,v_n\} \subseteq V$.
	Dann sind äquivalent:
	\begin{enumerate}[(i)]
		\item $B$ ist Basis von $V$.
		\item $B$ ist linear unabhängig.
		\item $\LH(B) = V$.
	\end{enumerate}
\end{satz}

\begin{satz}
	\label{satz:I.9.4}
	Sei $W \subseteq K^n$ ein Untervektorraum und $w_1,\dots,w_l$ ein Erzeugendensystem von $W$.
	Dann gilt:
	Entsteht das System $\{\widetilde{w_1},\dots,\widetilde{w_l}\}$ aus $\{w_1,\dots,w_l\}$ durch folgende Umformungen, so ist auch $\{\widetilde{w_1},\dots,\widetilde{w_l}\}$ ein Erzeugendensystem von $W$:
	\begin{enumerate}[(i)]
		\item Vertauschen der Reihenfolge.
		\item Multiplikation eines Vektors mit $\lambda \in K \setminus \setzero$.
		\item Addition des $\lambda$-fachen des $i$-ten Vektors auf den $j$-ten Vektor.
	\end{enumerate}
	Das bedeutet:
	Schreiben wir $w_1,\dots,w_k$ als Spalten in eine Matrix und führen elementare Spaltenumformungen (vgl. \autoref{def:I.3.9}) durch, bilden die resultierenden Nichtnullspalten ein Erzeugendensystem von $W$.
	Bringt man die Matrix auf Spaltenstufenform (vgl. \autoref{satz:I.3.11}), erhält man dadurch eine Basis von $W$.
\end{satz}

\begin{definition}[Transponierte Matrix]
	\label{def:I.9.5}
	Sei $A = (a_{ij})_{ij} \in M(m \times n,K)$, dann ist die \Index{transponierte Matrix} $A^T := (a_{ij}^T)_{ij} \in M(n \times m,K)$ von $A$ definiert durch
	\[
		a_{ij}^T := a_{ji},
	\]
	das heißt $A^T$ entsteht aus $A$ durch "spiegeln an der Diagonalen".
	Es gilt $(A^T)^T = A$, $(A+B)^T = A^T + B^T$, $(\lambda A)^T = \lambda A^T$ und $(AB)^T = B^T A^T$.
\end{definition}
\newpage