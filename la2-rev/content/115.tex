%!TEX root = ../LA2_SS16.tex
\subsection{Die \textsc{Vandermonde}-Determinante und Polynome}

\begin{satz}[\textsc{Vandermonde}-Determinante]
	\label{satz:I.15.1}
	Seien $x_0,\dots,x_n \in K$ und $B := (x_{i-1}^{j-1})_{1 \leq i,j \leq n-1} \in M((n+1) \times (n+1),K)$.
	Dann gilt \Index{Vandermonde-Determinante}
	\[
		\Det_{n+1}(B) = \prod_{0 \leq i<j \leq n} (x_j-x_i).
	\]
	Insbesondere ist $B$ invertierbar, wenn die $x_i$ paarweise verschieden sind.
\end{satz}

\setcounter{satz}{2}
\begin{korollar}
	\label{kor:I.15.3}
	Sei $K$ ein Körper, $x_0, \dots, x_n \in K$ paarweise verschieden und $y_0, \dots, y_n \in K$.
	Dann existiert genau ein Polynom $p \in K[T]$ mit $\grad(p) \leq n$ und $p(x_i)=y_i$ für alle $0 \leq i \leq n$.
\end{korollar}

\begin{korollar}
	\label{kor:I.15.4}
	Sei $K$ ein Körper und $p \in K[T]$ mit $\grad(p) = n \in \NN_0$.
	Dann besitzt $p$ höchstens $n$ paarweise verschiedene Nullstellen in $K$.
\end{korollar}

\setcounter{satz}{5}
\begin{satz}
	\label{satz:I.15.6}
	Sei $K$ ein Körper mit unendlich vielen Elementen und $\mathcal{P}_K := \{p \colon K \rightarrow K[T]\} \subseteq \Abb(K,K)$ der Vektorraum aller \textbf{Polynomfunktionen} auf $K$ (vgl. Beispiel~\ref{bsp:I.5.2}(3)). \index{Polynomfunktion}
	Dann ist die Abbildung
	\begin{align*}
		\Phi \colon K[T] &\longrightarrow \mathcal{P}_K \\
		p(T) &\longmapsto p
	\end{align*}
	ein Isomorphismus von $K$-Vektorräumen.
\end{satz}
\cleardoubleoddemptypage