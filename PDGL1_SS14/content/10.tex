% % % % % % 03. Juni
\section{Variationsrechnung und nichtlineare Gleichungen}
\label{sec:para10}
	Das Lösen einer PDGL ist oft gleichbedeutend mit der Minimierung von Energie. In der Physik sind partielle Differentialgleichungen oft eine Konsequenz aus dem Prinzip der minimalen Energie. \\
	Wir betrachten folgende Situation:
	\begin{itemize}
		\item \Index{Lagrange-Dichte} $L\colon \RR^n \times \RR \times \Omega \rightarrow \RR, (p,z,x) \mapsto L(p,z,x)$ (zur Vereinfachung als glatt vorausgesetzt, Ableitungen: $L_p,L_z,L_x$)
		\item Energie $E[u] = \int_\Omega L(\nabla u(x),u(x),x)dx$, definiert für (schwach) differenzierbare Funktionen $u\colon \Omega \rightarrow \RR$.
		\item \Index{Gâteaux-Differential} von $E$ in Richtung $v\colon \Omega \rightarrow \RR$:
		\[ \der_u E[u](v) = \frac{d}{dt} E[u+tv] = \int_{\Omega} L_p(\nabla u,u(x),x) \cdot \nabla v + L_z(\nabla u, u, x) v dx \]
	\end{itemize}
	Betrachte das Minimierungsproblem
	\[ \min_{u \colon \Omega \rightarrow \RR} E[u] \text{ in Abhängigkeit von } u = g \text{ auf } \der \Omega \]
	Wir nehmen an, es besitzt einen glatten Minimierer $u^*$. Dann ist 0 ein Minimierer von $t \mapsto E[u^*+tv]$ für jedes $v \in C_c^\infty(\Omega)$, d.h. $\der_u E[u^*](v) = 0$. Partielle Integration liefert
	\[ 0 = \int_{\Omega} v(-\diver L_p(\nabla u^*,u^*,x) + L_z(\nabla u^*,u^*,x)) dx \]
	für alle $v \in C_c^\infty (\Omega)$, d.h. $u^*$ löst die PDGL
	\[ 0 = -\div L_p(\nabla u(x),u(x),x) + L_z(\nabla u(x),u(x),x) \text{ auf } \Omega \quad \text{ mit } u = g \text{ auf } \der \Omega \]
	
\begin{bsp} \label{bsp_75}
\marginnote{[75]} \begin{tabular}{cccc}
Langrange-Dichte $L(p,z,x)$ & Energie $E[u]$ & PDGL &  \\ 
$\frac{1}{2} |p|^2$ & $\int_\Omega \frac{1}{2} |\nabla u|^2 dx$ & $\Delta u = 0$ & (Laplace-Gleichung) \\ 
$\frac{1}{2} p^T A(x)p - zf(x)$ & $\int_\Omega \frac{1}{2} \nabla u^T A \nabla u - fu dx$ & $-\diver(A\nabla u) = f$ & (allgemeine Poisson-Gleichung) \\ 
$\frac{1}{2} |p|^2 - F(z)$ & $\int_\Omega \frac{1}{2} |\nabla u|^2 - F(u) dx$ & $-\Delta u = F'(u)$ & (nichtlineare Poisson-Gleichung) \\ 
$\sqrt{1+|p|^2}$ & $\int_\Omega \sqrt{1+ |\nabla u|^2} dx$ & $\diver \frac{\nabla u}{\sqrt{1+ |\nabla u|^2}} = 0$ & Minimalflächengleichung \\ 
\end{tabular}
\end{bsp}

\mbox{} \\
Wir führen nun ein paar Hilfsmittel aus der Funktionalanalysis ein und beweisen dann die Existenz von Minimierern (und damit von Lösungen der Differentialgleichungen) für eine umfassende Klasse nichtlinearer Energien.

\begin{defn}[Dualraum] \label{def_76}
	Der \Index{Dualraum} $X'$ \marginnote{[76]} eines Banachraums $X$ ist der Raum der beschränkten linearen Funktionale $f\colon X \rightarrow \RR$ auf $X$.
\end{defn}

\begin{defn}[schwache Konvergenz, schwach-*-Konvergenz] \label{def_77}
	Eine Folge $x_k \in X$ \bet{konvergiert schwach} gegen $x \in X$, falls $f(x_k) \rightarrow f(x)$ für alle $f \in X'$. Schreibe: $x_k \weak x$. \marginnote{[77]} \index{schwache Konvergenz}\\
	Eine Folge $f_k \in X'$ \bet{konvergiert schwach-*} gegen $f \in X'$, falls $f_k(x) \rightarrow f(x)$ für alle $x \in X$. Schreibe: $f_k \weaks f$. \index{schwach-*-Konvergenz} \\
	Offensichtlich folgt aus der Konvergenz in $X$ bzw. $X'$ auch die schwache Konvergenz bzw. die schwach-*-Konvergenz.
\end{defn}
	
\begin{thm}[Schwach-*-Kompaktheit] \label{thm_78}
	Die Einheitskugel (und damit jede abgeschlossene Teilmenge)\marginnote{[78]} eines separablen Banachraums $X$ ist schwach-*-folgenkompakt, d.h. jede Folge enthält eine konvergente Teilfolge.
\end{thm}

\minisec{Beweis}
	siehe z.B. Alt, \glqq Lineare Funktionalanalysis\grqq, S. 229.
	
\begin{defn}[Reflexivität] \label{def_79}
	Ein Banachrau $X$ heißt \Index{reflexiv}, falls er isometrisch isomorph ist zu seinem Bidualraum $(X')'$. \marginnote{[79]} \\
	Hinweis: Nach dem vorigen Satz besitzen beschränkte Folgen in $X$ eine schwach konvergente Teilfolge, falls $X$ separabel ist.
\end{defn}

\begin{thm}[Sobolevräume sind reflexiv] \label{thm_80}
	Sei $\Omega$ offen, beschränkt und mit Lipschitzrand sowie $k \in \NN_0$. Für $p \in (1,\infty)$ ist $W^{k,p}(\Omega)$ separabel und reflexiv. \marginnote{[80]}
\end{thm}
	
\minisec{Beweis}
	siehe z.B. Alt, \glqq Lineare Funktionalanalysis\grqq, S. 234.
	
% % % % % 6. Juni
\begin{bem} \label{bem_81}
	Mit der Reflexivität von $L^p(\Omega)$ für $p \in (1,\infty)$ \marginnote{6. Jun. \\ \ [81]} und der Hölder-Ungleichung ist leicht zu sehen, dass $L^p(\Omega) \subseteq (L^{p^*}(\Omega))'$. Gleichermaßen erkennt man $L^{p^*}(\Omega) \subseteq (L^p(\Omega))'$. Tatsächlich folgt mit dem Satz von Radon-Nikodým:
	\[ (L^p(\Omega))' = L^{p^*}(\Omega) \]
\end{bem}
	
\begin{thm} \label{thm_82}
	Sei \marginnote{[82]} \begin{itemize}
	\item $L(p,z,x)$ konvex in $p$
	\item $L(p,z,x)$ halbstetig von unten in $z$
	\item $L(p,z,x) \geq \alpha |p|^q - \beta$ für gewisse $\alpha, \beta > 0, q \in (1,\infty)$.
	\end{itemize}
	Falls $g \in W^{1,q}(\Omega)$, dann hat $E$ einen Minimierer in $\{u \in W^{1,q}(\Omega) : u = g \text{ auf } \der \Omega \}$.
\end{thm}
	
\minisec{Beweis}
	\glqq Direkte Methode der Variationsrechnung\grqq \begin{enumerate}[1.]
		\item Weder gilt $E \equiv \infty$, noch ist $E$ nach unten unbeschränkt.
		\item Betrachte eine minimierende Folge $u_k$ mit $E[u_k] \rightarrow \inf_u E[u]$ monoton.
		\item Zeige die Kompaktheit der Folge, d.h. die Existenz einer (in einem gewissen Sinn) konvergenten Teilfolge $u_k \rightarrow u^*$ für ein $u^*$. \\
		Es ist $E[u] \geq \alpha ||\nabla u||_{L^q(\Omega)}^q - \beta |\Omega|$, daher existiert ein $C > 0$ mit $||\nabla u_k||_{L^q(\Omega)} \leq C$ für alle $k$. Mit der Poincaré-Ungleichung folgt $||u_k||_{W^{1,q}(\Omega)} \leq C$ für eine (andere) Konstante $C \geq 0$. Da $W^{1,q}(\Omega)$ reflexiv ist, konvergiert eine Teilfolge $u_k$ schwach gegen ein $u^* \in W^{1,q}(\Omega)$.
		\item Zeige mit $u_k$, dass $E$ halbstetig von unten ist, d.h. $E[u] \leq \liminf_{k \rightarrow \infty} E[u_k]$. \\
		Aufgrund der kompakten Einbettung $W^{1,q}(\Omega) \hookrightarrow L^q(\Omega)$ gilt $u_k \rightarrow u^*$ im starken Sinne in $L^q(\Omega)$ für eine Teilfolge und demach sogar punktweise fast überall nach dem Entfernen einer anderen Teilfolge. Mit dem Satz von Egoroff finden wir sogar für jedes $\varepsilon > 0$ eine messbare Menge $\Omega_\varepsilon \subseteq \Omega$ mit $|\Omega \setminus \Omega_\varepsilon| < \varepsilon$ und $u_k \rightarrow u^*$ gleichmäßig in $\Omega_\varepsilon$. \\
		O.B.d.A. sei $L \geq 0$. Da $L$ konvex in $p$, folgt
		\[ \liminf_{k \rightarrow \infty} \int_{\Omega} L(\nabla u_k,u_k,x) dx \geq \liminf_{k \rightarrow \infty} \enbrace*{ \int_{\Omega} L(\nabla u^*,u_k,x) dx + \int_{\Omega} L_p(\nabla u^*,u_k,x) \cdot (\nabla u_k - \nabla u^*) dx }. \]
		Damit haben wir
		\begin{equation}
		\begin{aligned}
			\liminf_{k \rightarrow \infty} \int_{\Omega} L(\nabla u^*,u_k,x) dx \geq \ &\liminf_{k \rightarrow \infty} \int_{\Omega} \inf_{j \geq k} L(\nabla u^*,u_j,x) + \beta dx - \beta |\Omega| \\ \notag
			\overset{(A)}{=} \ &\int_{\Omega} \liminf_{k \rightarrow \infty} L(\nabla u^*,u_k,x)+ \beta dx - \beta |\Omega| \overset{(B)}{\geq} \int_{\Omega} L(\nabla u^*,u^*,x)dx = E[u^*]
		\end{aligned}
		\end{equation}
		wobei wir bei (A) den Satz von der monotonen Kovergenz (Beppo Levi) bzw. das Lemma von Fatou und bei (B) die Halbstetigkeit von unten von $L$ benutzen. Weiter folgt:
		\begin{equation}
		\begin{aligned}
			&\int_\Omega L_p(\nabla u^*,u_k,x) \cdot (\nabla u_k,-\nabla u^*) dx \\ \notag
			\geq \ &\int_{\Omega_\varepsilon} L_p(\nabla u^*,u^*,x) \cdot (\nabla u_k,\nabla u^*) dx + \int_{\Omega_\varepsilon} (L_p(\nabla u^*,u_k,x) - L_p(\nabla u^*,u^*,x)) \cdot (\nabla u_k - \nabla u^*) dx
		\end{aligned}
		\end{equation}
		Der erste Term konvergiert wegen der schwachen Konvergenz von $u_k$ gegen 0 und der zweite wegen der Hölder-Ungleichung und der gleichmäßigen Konvergenz. Mit $\varepsilon \rightarrow 0$ folgt die Behauptung. \qed
	\end{enumerate}
\newpage